% Generated by Sphinx.
\def\sphinxdocclass{report}
\documentclass[letterpaper,10pt,english]{sphinxmanual}
\usepackage[utf8]{inputenc}
\DeclareUnicodeCharacter{00A0}{\nobreakspace}
\usepackage{cmap}
\usepackage[T1]{fontenc}
\usepackage{babel}
\usepackage{times}

    \usepackage[Sonny]{fncychap}

\usepackage{longtable}
\usepackage{sphinx}
\usepackage{multirow}
\usepackage{eqparbox}
\usepackage{amsfonts}

\addto\captionsenglish{\renewcommand{\figurename}{Fig. }}
\addto\captionsenglish{\renewcommand{\tablename}{Table }}
\SetupFloatingEnvironment{literal-block}{name=Listing }


    \usepackage{changepage}
    \usepackage{graphicx}
    \usepackage{hyperref}

    \makeatletter
    \renewcommand{\maketitle}{%
      \begin{titlepage}%
        \let\footnotesize\small
        \let\footnoterule\relax
        \ifsphinxpdfoutput
          \begingroup
          % This \def is required to deal with multi-line authors; it
          % changes \\ to ', ' (comma-space), making it pass muster for
          % generating document info in the PDF file.
          \def\\{, }
          \pdfinfo{
            /Author (\@author)
            /Title (\@title)
          }
          \endgroup
        \fi
        \changepage{1in}{}{1in}{0.5in}{}{-0.5in}{}{}{}
        \begin{flushright}%
          \includegraphics[scale=0.8]{icon}\par%
          \vskip 3em%
          {\rm\Huge\py@HeaderFamily \@title \par}%
          {\em\LARGE\py@HeaderFamily \py@release\releaseinfo \par}
          \vfill
          {\large\py@HeaderFamily \@author \par}
          \vfill
          \includegraphics[scale=0.75]{nistlogo}\par
          %{\py@authoraddress \par}
          %\vfill
          {%\large
           %\vfill
          }%
        \end{flushright}%\par
        \@thanks
      \end{titlepage}%
      %\clearpage%
      %\changepage{}{}{}{}{}{}{}{}{}
      \vspace*{\fill}
      {\py@authoraddress \par}
      \clearpage
      \vfill
      \input LICENSE
      \clearpage
      \rule{\textwidth}{0.1pt}
      \changepage{}{}{}{}{}{}{}{}{}
      \input DISCLAIMER
      \clearpage
      \changepage{}{}{}{}{}{}{}{}{}
      \input DEVELOPERS
      \clearpage
      \setcounter{footnote}{0}%
      \let\thanks\relax\let\maketitle\relax
      %\gdef\@thanks{}\gdef\@author{}\gdef\@title{}
    }
    \makeatother
    
    \authoraddress{\Large Questions/Suggestions\\
    \\
    \small Subscribe to our mailing list: email \href{mailto:mosaic-request@nist.gov?subject=subscribe}{mosaic-request@nist.gov} with subject \textit{subscribe}\\
    Once subscribed, send messages by emailing \href{mailto:mosaic@nist.gov}{mosaic@nist.gov}.\\
    The list archive can be found at: \href[pdfnewwindow=true]{http://news.gmane.org/gmane.comp.python.mosaic}{http://news.gmane.org/gmane.comp.python.mosaic}\\
    \\
    \\
    Report problems with MOSAIC using the issue tracker on GitHub\\
    \href{https://github.com/usnistgov/mosaic/issues}{https://github.com/usnistgov/mosaic/issues}\\
    \\
    \\
    \\
    }

    \ChNameVar{\fontsize{18}{20}\usefont{OT1}{phv}{m}{n}\selectfont} 
    \ChNumVar{\fontsize{24}{28}\usefont{OT1}{ptm}{m}{n}\selectfont} 
    \ChTitleVar{\huge\bfseries\rm}
    \ChRuleWidth{1pt}


\title{MOSAIC Manual}
\date{May 25, 2016}
\release{1.3b2}
\author{`}
\newcommand{\sphinxlogo}{}
\renewcommand{\releasename}{Release}
\setcounter{tocdepth}{1}
\makeindex

\makeatletter
\def\PYG@reset{\let\PYG@it=\relax \let\PYG@bf=\relax%
    \let\PYG@ul=\relax \let\PYG@tc=\relax%
    \let\PYG@bc=\relax \let\PYG@ff=\relax}
\def\PYG@tok#1{\csname PYG@tok@#1\endcsname}
\def\PYG@toks#1+{\ifx\relax#1\empty\else%
    \PYG@tok{#1}\expandafter\PYG@toks\fi}
\def\PYG@do#1{\PYG@bc{\PYG@tc{\PYG@ul{%
    \PYG@it{\PYG@bf{\PYG@ff{#1}}}}}}}
\def\PYG#1#2{\PYG@reset\PYG@toks#1+\relax+\PYG@do{#2}}

\expandafter\def\csname PYG@tok@gd\endcsname{\def\PYG@tc##1{\textcolor[rgb]{0.63,0.00,0.00}{##1}}}
\expandafter\def\csname PYG@tok@gu\endcsname{\let\PYG@bf=\textbf\def\PYG@tc##1{\textcolor[rgb]{0.50,0.00,0.50}{##1}}}
\expandafter\def\csname PYG@tok@gt\endcsname{\def\PYG@tc##1{\textcolor[rgb]{0.00,0.27,0.87}{##1}}}
\expandafter\def\csname PYG@tok@gs\endcsname{\let\PYG@bf=\textbf}
\expandafter\def\csname PYG@tok@gr\endcsname{\def\PYG@tc##1{\textcolor[rgb]{1.00,0.00,0.00}{##1}}}
\expandafter\def\csname PYG@tok@cm\endcsname{\let\PYG@it=\textit\def\PYG@tc##1{\textcolor[rgb]{0.25,0.50,0.56}{##1}}}
\expandafter\def\csname PYG@tok@vg\endcsname{\def\PYG@tc##1{\textcolor[rgb]{0.73,0.38,0.84}{##1}}}
\expandafter\def\csname PYG@tok@vi\endcsname{\def\PYG@tc##1{\textcolor[rgb]{0.73,0.38,0.84}{##1}}}
\expandafter\def\csname PYG@tok@mh\endcsname{\def\PYG@tc##1{\textcolor[rgb]{0.13,0.50,0.31}{##1}}}
\expandafter\def\csname PYG@tok@cs\endcsname{\def\PYG@tc##1{\textcolor[rgb]{0.25,0.50,0.56}{##1}}\def\PYG@bc##1{\setlength{\fboxsep}{0pt}\colorbox[rgb]{1.00,0.94,0.94}{\strut ##1}}}
\expandafter\def\csname PYG@tok@ge\endcsname{\let\PYG@it=\textit}
\expandafter\def\csname PYG@tok@vc\endcsname{\def\PYG@tc##1{\textcolor[rgb]{0.73,0.38,0.84}{##1}}}
\expandafter\def\csname PYG@tok@il\endcsname{\def\PYG@tc##1{\textcolor[rgb]{0.13,0.50,0.31}{##1}}}
\expandafter\def\csname PYG@tok@go\endcsname{\def\PYG@tc##1{\textcolor[rgb]{0.20,0.20,0.20}{##1}}}
\expandafter\def\csname PYG@tok@cp\endcsname{\def\PYG@tc##1{\textcolor[rgb]{0.00,0.44,0.13}{##1}}}
\expandafter\def\csname PYG@tok@gi\endcsname{\def\PYG@tc##1{\textcolor[rgb]{0.00,0.63,0.00}{##1}}}
\expandafter\def\csname PYG@tok@gh\endcsname{\let\PYG@bf=\textbf\def\PYG@tc##1{\textcolor[rgb]{0.00,0.00,0.50}{##1}}}
\expandafter\def\csname PYG@tok@ni\endcsname{\let\PYG@bf=\textbf\def\PYG@tc##1{\textcolor[rgb]{0.84,0.33,0.22}{##1}}}
\expandafter\def\csname PYG@tok@nl\endcsname{\let\PYG@bf=\textbf\def\PYG@tc##1{\textcolor[rgb]{0.00,0.13,0.44}{##1}}}
\expandafter\def\csname PYG@tok@nn\endcsname{\let\PYG@bf=\textbf\def\PYG@tc##1{\textcolor[rgb]{0.05,0.52,0.71}{##1}}}
\expandafter\def\csname PYG@tok@no\endcsname{\def\PYG@tc##1{\textcolor[rgb]{0.38,0.68,0.84}{##1}}}
\expandafter\def\csname PYG@tok@na\endcsname{\def\PYG@tc##1{\textcolor[rgb]{0.25,0.44,0.63}{##1}}}
\expandafter\def\csname PYG@tok@nb\endcsname{\def\PYG@tc##1{\textcolor[rgb]{0.00,0.44,0.13}{##1}}}
\expandafter\def\csname PYG@tok@nc\endcsname{\let\PYG@bf=\textbf\def\PYG@tc##1{\textcolor[rgb]{0.05,0.52,0.71}{##1}}}
\expandafter\def\csname PYG@tok@nd\endcsname{\let\PYG@bf=\textbf\def\PYG@tc##1{\textcolor[rgb]{0.33,0.33,0.33}{##1}}}
\expandafter\def\csname PYG@tok@ne\endcsname{\def\PYG@tc##1{\textcolor[rgb]{0.00,0.44,0.13}{##1}}}
\expandafter\def\csname PYG@tok@nf\endcsname{\def\PYG@tc##1{\textcolor[rgb]{0.02,0.16,0.49}{##1}}}
\expandafter\def\csname PYG@tok@si\endcsname{\let\PYG@it=\textit\def\PYG@tc##1{\textcolor[rgb]{0.44,0.63,0.82}{##1}}}
\expandafter\def\csname PYG@tok@s2\endcsname{\def\PYG@tc##1{\textcolor[rgb]{0.25,0.44,0.63}{##1}}}
\expandafter\def\csname PYG@tok@nt\endcsname{\let\PYG@bf=\textbf\def\PYG@tc##1{\textcolor[rgb]{0.02,0.16,0.45}{##1}}}
\expandafter\def\csname PYG@tok@nv\endcsname{\def\PYG@tc##1{\textcolor[rgb]{0.73,0.38,0.84}{##1}}}
\expandafter\def\csname PYG@tok@s1\endcsname{\def\PYG@tc##1{\textcolor[rgb]{0.25,0.44,0.63}{##1}}}
\expandafter\def\csname PYG@tok@ch\endcsname{\let\PYG@it=\textit\def\PYG@tc##1{\textcolor[rgb]{0.25,0.50,0.56}{##1}}}
\expandafter\def\csname PYG@tok@m\endcsname{\def\PYG@tc##1{\textcolor[rgb]{0.13,0.50,0.31}{##1}}}
\expandafter\def\csname PYG@tok@gp\endcsname{\let\PYG@bf=\textbf\def\PYG@tc##1{\textcolor[rgb]{0.78,0.36,0.04}{##1}}}
\expandafter\def\csname PYG@tok@sh\endcsname{\def\PYG@tc##1{\textcolor[rgb]{0.25,0.44,0.63}{##1}}}
\expandafter\def\csname PYG@tok@ow\endcsname{\let\PYG@bf=\textbf\def\PYG@tc##1{\textcolor[rgb]{0.00,0.44,0.13}{##1}}}
\expandafter\def\csname PYG@tok@sx\endcsname{\def\PYG@tc##1{\textcolor[rgb]{0.78,0.36,0.04}{##1}}}
\expandafter\def\csname PYG@tok@bp\endcsname{\def\PYG@tc##1{\textcolor[rgb]{0.00,0.44,0.13}{##1}}}
\expandafter\def\csname PYG@tok@c1\endcsname{\let\PYG@it=\textit\def\PYG@tc##1{\textcolor[rgb]{0.25,0.50,0.56}{##1}}}
\expandafter\def\csname PYG@tok@o\endcsname{\def\PYG@tc##1{\textcolor[rgb]{0.40,0.40,0.40}{##1}}}
\expandafter\def\csname PYG@tok@kc\endcsname{\let\PYG@bf=\textbf\def\PYG@tc##1{\textcolor[rgb]{0.00,0.44,0.13}{##1}}}
\expandafter\def\csname PYG@tok@c\endcsname{\let\PYG@it=\textit\def\PYG@tc##1{\textcolor[rgb]{0.25,0.50,0.56}{##1}}}
\expandafter\def\csname PYG@tok@mf\endcsname{\def\PYG@tc##1{\textcolor[rgb]{0.13,0.50,0.31}{##1}}}
\expandafter\def\csname PYG@tok@err\endcsname{\def\PYG@bc##1{\setlength{\fboxsep}{0pt}\fcolorbox[rgb]{1.00,0.00,0.00}{1,1,1}{\strut ##1}}}
\expandafter\def\csname PYG@tok@mb\endcsname{\def\PYG@tc##1{\textcolor[rgb]{0.13,0.50,0.31}{##1}}}
\expandafter\def\csname PYG@tok@ss\endcsname{\def\PYG@tc##1{\textcolor[rgb]{0.32,0.47,0.09}{##1}}}
\expandafter\def\csname PYG@tok@sr\endcsname{\def\PYG@tc##1{\textcolor[rgb]{0.14,0.33,0.53}{##1}}}
\expandafter\def\csname PYG@tok@mo\endcsname{\def\PYG@tc##1{\textcolor[rgb]{0.13,0.50,0.31}{##1}}}
\expandafter\def\csname PYG@tok@kd\endcsname{\let\PYG@bf=\textbf\def\PYG@tc##1{\textcolor[rgb]{0.00,0.44,0.13}{##1}}}
\expandafter\def\csname PYG@tok@mi\endcsname{\def\PYG@tc##1{\textcolor[rgb]{0.13,0.50,0.31}{##1}}}
\expandafter\def\csname PYG@tok@kn\endcsname{\let\PYG@bf=\textbf\def\PYG@tc##1{\textcolor[rgb]{0.00,0.44,0.13}{##1}}}
\expandafter\def\csname PYG@tok@cpf\endcsname{\let\PYG@it=\textit\def\PYG@tc##1{\textcolor[rgb]{0.25,0.50,0.56}{##1}}}
\expandafter\def\csname PYG@tok@kr\endcsname{\let\PYG@bf=\textbf\def\PYG@tc##1{\textcolor[rgb]{0.00,0.44,0.13}{##1}}}
\expandafter\def\csname PYG@tok@s\endcsname{\def\PYG@tc##1{\textcolor[rgb]{0.25,0.44,0.63}{##1}}}
\expandafter\def\csname PYG@tok@kp\endcsname{\def\PYG@tc##1{\textcolor[rgb]{0.00,0.44,0.13}{##1}}}
\expandafter\def\csname PYG@tok@w\endcsname{\def\PYG@tc##1{\textcolor[rgb]{0.73,0.73,0.73}{##1}}}
\expandafter\def\csname PYG@tok@kt\endcsname{\def\PYG@tc##1{\textcolor[rgb]{0.56,0.13,0.00}{##1}}}
\expandafter\def\csname PYG@tok@sc\endcsname{\def\PYG@tc##1{\textcolor[rgb]{0.25,0.44,0.63}{##1}}}
\expandafter\def\csname PYG@tok@sb\endcsname{\def\PYG@tc##1{\textcolor[rgb]{0.25,0.44,0.63}{##1}}}
\expandafter\def\csname PYG@tok@k\endcsname{\let\PYG@bf=\textbf\def\PYG@tc##1{\textcolor[rgb]{0.00,0.44,0.13}{##1}}}
\expandafter\def\csname PYG@tok@se\endcsname{\let\PYG@bf=\textbf\def\PYG@tc##1{\textcolor[rgb]{0.25,0.44,0.63}{##1}}}
\expandafter\def\csname PYG@tok@sd\endcsname{\let\PYG@it=\textit\def\PYG@tc##1{\textcolor[rgb]{0.25,0.44,0.63}{##1}}}

\def\PYGZbs{\char`\\}
\def\PYGZus{\char`\_}
\def\PYGZob{\char`\{}
\def\PYGZcb{\char`\}}
\def\PYGZca{\char`\^}
\def\PYGZam{\char`\&}
\def\PYGZlt{\char`\<}
\def\PYGZgt{\char`\>}
\def\PYGZsh{\char`\#}
\def\PYGZpc{\char`\%}
\def\PYGZdl{\char`\$}
\def\PYGZhy{\char`\-}
\def\PYGZsq{\char`\'}
\def\PYGZdq{\char`\"}
\def\PYGZti{\char`\~}
% for compatibility with earlier versions
\def\PYGZat{@}
\def\PYGZlb{[}
\def\PYGZrb{]}
\makeatother

\renewcommand\PYGZsq{\textquotesingle}

\begin{document}

\maketitle
\tableofcontents
\phantomsection\label{index::doc}



\chapter{Introduction}
\label{doc/Introduction:introduction}\label{doc/Introduction:introduction-page}\label{doc/Introduction::doc}\label{doc/Introduction:projname-projlongname}
\emph{MOSAIC} is a modular toolbox for analyzing data from single molecule experiments. Primarily developed to analyze data from nanopore experiments \phantomsection\label{doc/Introduction:id1}{\hyperref[doc/References:reiner\string-2012bg]{\emph{{[}RBR+12{]}}}}, \emph{MOSAIC} can analyze any data that fit the form \phantomsection\label{doc/Introduction:id2}{\hyperref[doc/References:balijepalli\string-2014ft]{\emph{{[}BEC+14{]}}}}:
\begin{gather}
\begin{split}i(t)=i_0 + \sum_{j=1}^{N} a_j\left(1-e^{-\left(t-\mu_j\right)/\tau_j}\right) H\left(t-\mu_j\right)\end{split}\notag
\end{gather}
The above functional form, which represents the response to a step change from one state to another is ubiqutous in many disciplines. By fitting individual state changes to the equation above, \emph{MOSAIC} is able to automatically identify the states corresponding to each change. Moreover this approach allows us to accurately characterize transient events before they asymptotically approach a steady state. In nanopore applications, this has resulted in a 20-fold improvement in the number of states identified per unit time \phantomsection\label{doc/Introduction:id3}{\hyperref[doc/References:balijepalli\string-2014ft]{\emph{{[}BEC+14{]}}}}.

\emph{MOSAIC} offers tremendous flexibility in how it can be used. Nanopore data can be analyzed and visualized using the {\hyperref[doc/GraphicalInterface:gui\string-page]{\emph{MOSAIC GUI}}} (GUI), which is available as a stand-alone application (\href{http://usnistgov.github.io/mosaic/platforms.html}{download binaries}). This is a convenient way for most users to analyze nanopore data. Advanced users can write their own Python scripts to include \emph{MOSAIC} in their analysis workflow (see {\hyperref[doc/ScriptingandAdvancedFeatures:scripting\string-page]{\emph{Scripting and Advanced Features}}}). Finally, because \emph{MOSAIC} was designed from the start using object oriented design, developers can easily extend it by combining existing classes to define new functionality or writing their own classes (see {\hyperref[doc/Extend:extend\string-page]{\emph{Extend MOSAIC}}}).


\chapter{Data Processing Algorithms in \emph{MOSAIC}}
\label{doc/Algorithms:data-processing-algorithms-in-projname}\label{doc/Algorithms::doc}\label{doc/Algorithms:algorithms-page}\label{doc/Algorithms:mosaic-page-on-github}
There are three primary algorithms available in \emph{MOSAIC} to process time-series data from single-molecule nanopore experiments. Fitting-based approaches are outlined in the {\hyperref[doc/Introduction:introduction\string-page]{\emph{Introduction}}}, are implemented in \emph{MOSAIC} using two separate algorithms, i) StepResponseAnalysis is used for events that exhibit a single state, and ii) MultistateAnalysis for \emph{N}-state events. In addition, the CUSUM algorithm is available for \emph{N}-state events.


\section{ADEPT 2-State}
\label{doc/Algorithms:stepresponse-page}\label{doc/Algorithms:adept-2-state}
This algorithm limits the generalized algorithm for state-detection \phantomsection\label{doc/Algorithms:id1}{\hyperref[doc/References:balijepalli\string-2014ft]{\emph{{[}BEC+14{]}}}} to cases with a single state as seen in the figure below. This simplified approach speeds up the analysis considerably and is appropriate to use for many applications, for example the detection of PEG, small molecules, DNA homopolymers, etc. The {\hyperref[api\string-doc/mosaic.processing:mosaic.adept2State.adept2State]{\emph{\code{adept2State}}}} class uses a simplified form of the expression for the ionic current across a nanopore as shown below. Settings that control the fit are defined through the settings file and are described in more detail in the {\hyperref[doc/settingsFile:algorithm\string-settings\string-sec]{\emph{Optimizing Settings}}} section. This functional form is fit to a time-series from a single event to recover optimal parameters for the mdoel.
\begin{gather}
\begin{split}i(t)=i_0 + a \left[ \left(e^{-(t+\mu_1)/\tau} -1\right) H\left(t-\mu_1\right)  + \left(1- e^{-(t+\mu_2)/\tau} \right)H\left(t-\mu_2\right) \right]\end{split}\notag
\end{gather}
This simplification speeds up the analysis for two state events like the PEG event in the figure below. The figure shows the results of the fit (or meta-data) superimposed on the time-series of a single event.
\begin{figure}[htbp]
\centering

\includegraphics[width=0.500\linewidth]{{StepResponse}.png}
\end{figure}


\subsection{Algorithm Settings}
\label{doc/Algorithms:algorithm-settings}\begin{quote}

Analyze an event that is characteristic of PEG blockades. This method includes system
information in the analysis, specifically the filtering effects (throught the RC constant)
of either amplifiers or the membrane/nanopore complex. The analysis generates several
parameters that are stored as metadata including:
\begin{enumerate}
\item {} 
Blockade depth: the ratio of the open channel current to the blocked current

\item {} 
Residence time: the time the molecule spends inside the pore

\item {} 
Tau: the RC constant of the response to a step input (e.g. the entry or exit of the molecule into or out of the nanopore).

\end{enumerate}
\begin{quote}\begin{description}
\item[{Keyword Args}] \leavevmode\begin{description}
\item[{In addition to {\hyperref[api\string-doc/mosaic.meta:mosaic.metaEventProcessor.metaEventProcessor]{\emph{\code{metaEventProcessor}}}} args,}] \leavevmode\begin{itemize}
\item {} 
\emph{FitTol} :            Tolerance value for the least squares algorithm that controls the convergence of the fit (Default: \emph{1e-7}).

\item {} 
\emph{FitIters} :          Maximum number of iterations before terminating the fit (Default: \emph{50000}).

\item {} 
\emph{LinkRCConst} :       When True, the RC constants associated with each state in the fit function are varied together. (Default: \emph{True})

\end{itemize}

\end{description}

\item[{Errors}] \leavevmode
When an event cannot be analyzed, the blockade depth, residence time and rise time are set to -1.

\end{description}\end{quote}
\end{quote}


\subsection{Metadata Output}
\label{doc/Algorithms:metadata-output}
Meta-data for individual events generated by {\hyperref[api\string-doc/mosaic.processing:mosaic.adept2State.adept2State]{\emph{\code{adept2State}}}} can be queried using \href{http://www.sqlite.org/}{SQLite} as described in the {\hyperref[doc/DatabaseStructure:database\string-page]{\emph{Database Structure and Query Syntax}}} section. A list of meta-data stored by the step response algorithm is given below.

\begin{tabulary}{\linewidth}{p{4cm}p{4cm}p{6cm}}
\hline
\textsf{\relax 
\textbf{Column Name}
} & \textsf{\relax 
\textbf{Column Type}
} & \textsf{\relax 
\textbf{Description}
}\\
\hline
recIDX

ProcessingStatus

OpenChCurrent

BlockedCurrent

EventStart

EventEnd

BlockDepth

ResTime

RCConstant

AbsEventStart

ReducedChiSquared

TimeSeries
 & 
INTEGER

TEXT

REAL

REAL

REAL

REAL

REAL

REAL

REAL

REAL

REAL

REAL\_LIST
 & 
Record index.

Status of the analysis.

Open channel current in pA.

Blocked state current in pA.

Event start in ms.

Event end in ms.

BlockedCurrent/OpenChCurrent.

EventEnd-EventStart in ms.

System RC constant in ms.

Global event start time in ms.

Reduced Chi-squared of fit.

(OPTIONAL) Event time-series.
\\
\hline\end{tabulary}



\section{ADEPT}
\label{doc/Algorithms:multistate-page}\label{doc/Algorithms:adept}
The multistate algorithm implements the  general case for identifying states in nanopore data \phantomsection\label{doc/Algorithms:id2}{\hyperref[doc/References:balijepalli\string-2014ft]{\emph{{[}BEC+14{]}}}}. The general form of the equation used in this algorithm is shown below, where \emph{N} is the number of states. This functional form is fit to a time-series from a single event to recover optimal parameters for the mdoel.
\begin{gather}
\begin{split}i(t)=i_0 + \sum_{j=1}^{N} a_j\left(1-e^{-\left(t-\mu_j\right)/\tau_j}\right) H\left(t-\mu_j\right)\end{split}\notag
\end{gather}
Settings that control the fit are defined through the settings file and are described in more detail in the {\hyperref[doc/settingsFile:algorithm\string-settings\string-sec]{\emph{Optimizing Settings}}} section. Upon successfully fitting the model to an event, {\hyperref[api\string-doc/mosaic.processing:mosaic.adept.adept]{\emph{\code{adept}}}} generates meta-data the describes the individual states in the event. A representative example of one such event is shown in the figure below.
\begin{figure}[htbp]
\centering

\includegraphics[width=0.500\linewidth]{{Multistate}.png}
\end{figure}


\subsection{Algorithm Settings}
\label{doc/Algorithms:id3}\begin{quote}

Analyze a multi-step event that contains two or more states. This method includes system
information in the analysis, specifically the filtering effects (through the RC constant)
of either amplifiers or the membrane/nanopore complex. The analysis generates several
parameters that are stored as metadata including:
\begin{enumerate}
\item {} 
Blockade depth: the ratio of the open channel current to the blocked current

\item {} 
Residence time: the time the molecule spends inside the pore

\item {} 
Tau: the RC constant  of the response to a step input (e.g. the entry or exit of the molecule into or out of the nanopore).

\end{enumerate}
\begin{quote}\begin{description}
\item[{Keyword Args}] \leavevmode\begin{description}
\item[{In addition to {\hyperref[api\string-doc/mosaic.meta:mosaic.metaEventProcessor.metaEventProcessor]{\emph{\code{metaEventProcessor}}}} args,}] \leavevmode\begin{itemize}
\item {} 
\emph{StepSize} :                  The multiple of the standard deviations considered significant to dtecting an event (default: 3.0).

\item {} 
\emph{MinStateLength} :    minimum number of data points required to assign a state within an event (default: 4)

\item {} 
\emph{MaxEventLength} :    maximum length (in data points) of events that will be processed (default: 10000)

\item {} 
\emph{FitTol} :                    fit tolerance for convergence (default: 1.e-7)

\item {} 
\emph{FitIters} :                  maximum fit iterations (default: 5000)

\item {} 
\emph{LinkRCConst} :       When True, the RC constants associated with each state in the fit function are varied together. (Default: \emph{True})

\end{itemize}

\end{description}

\item[{Errors}] \leavevmode
When an event cannot be analyzed, all metadata are set to -1.

\end{description}\end{quote}
\end{quote}


\subsection{Metadata Output}
\label{doc/Algorithms:id4}
The {\hyperref[api\string-doc/mosaic.processing:mosaic.adept.adept]{\emph{\code{adept}}}} algorithm outputs meta-data that characterizes every processed event. Similar to the \DUspan{xref,std,std-ref}{stepresponse-page} algorithm, this information is stored in a \href{http://www.sqlite.org/}{SQLite} database and is available for further processing (see {\hyperref[doc/DatabaseStructure:database\string-page]{\emph{Database Structure and Query Syntax}}}). Notably, the data output by {\hyperref[api\string-doc/mosaic.processing:mosaic.adept.adept]{\emph{\code{adept}}}} differs from {\hyperref[api\string-doc/mosaic.processing:mosaic.adept2State.adept2State]{\emph{\code{adept2State}}}} in one important way. Because the number of states (\emph{NStates}) detected in each event is not pre-determined, key meta-data (e.g. \emph{BlockDepth}, \emph{EventDelay}, etc.) are stored as arrays of real numbers with length equal to \emph{NStates}.

\begin{tabulary}{\linewidth}{p{4cm}p{4cm}p{8cm}}
\hline
\textsf{\relax 
\textbf{Column Name}
} & \textsf{\relax 
\textbf{Column Type}
} & \textsf{\relax 
\textbf{Description}
}\\
\hline
recIDX

ProcessingStatus

OpenChCurrent

NStates

CurrentStep

BlockDepth

EventStart

EventEnd

EventDelay

ResTime

RCConstant

AbsEventStart

ReducedChiSquared

TimeSeries
 & 
INTEGER

TEXT

REAL

INTEGER

REAL\_LIST

REAL\_LIST

REAL

REAL

REAL\_LIST

REAL

REAL

REAL

REAL

REAL\_LIST
 & 
Record index.

Status of the analysis.

Open channel current in pA.

Number of detected states.

Blocked current steps in pA.

BlockedCurrent/OpenChCurrent for each state.

Event start in ms.

Event end in ms.

Start time of each state in ms.

EventEnd-EventStart in ms.

System RC constant in ms.

Global event start time in ms.

Reduced Chi-squared of fit.

(OPTIONAL) Event time-series.
\\
\hline\end{tabulary}



\section{CUSUM+}
\label{doc/Algorithms:cusumlevel-page}\label{doc/Algorithms:cusum}
The CUSUM algorithm (used by OpenNanopore for example) \phantomsection\label{doc/Algorithms:id5}{\hyperref[doc/References:raillon\string-2012is]{\emph{{[}RGG+12{]}}}} is available in \emph{MOSAIC}. In contrast with other algorithms available in \emph{MOSAIC}, this approach does not leverage system information in the analysis. This however results in a faster estimation of single- and multi-level events, compared with \DUspan{xref,std,std-ref}{stepresponse-page} and \DUspan{xref,std,std-ref}{multistate-page}. You can read about the CUSUM algorithm \href{http://pubs.rsc.org/en/Content/ArticleLanding/2012/NR/c2nr30951c\#!divAbstract}{here}.

Some known issues with CUSUM:
\begin{enumerate}
\item {} 
If the duration of a sub-event is shorter than a five RC constants, the averaging will underestimate the extent of the current change. For longer events, CUSUM should achieve very similar output to the fitting employed elsewhere in \emph{MOSAIC}.

\item {} 
CUSUM assumes an instantaneous transition between current states. As a result, if the RC rise time of the system is large, CUSUM can trigger and detect intermediate states. This can usually be mitigated by optimizing the algorithm sensitivity settings.

\item {} 
If an event is very long, CUSUM will detect a state transistion even if there is no real change, leading to an artificially high number of states. This is a consequence of false positives from using a statistical t-test. In some cases this can be mitigated by reducing the sensitivity.

\end{enumerate}

Settings that control the algorithm are defined through the settings file, as described the {\hyperref[doc/settingsFile:algorithm\string-settings\string-sec]{\emph{Optimizing Settings}}} section. Upon successfully analyzing an event, {\hyperref[api\string-doc/mosaic.processing:mosaic.cusumPlus.cusumPlus]{\emph{\code{cusumPlus}}}} generates meta-data the describes the individual states in the event. A representative example of one such event is shown in the figure below.
\begin{figure}[htbp]
\centering

\includegraphics[width=0.500\linewidth]{{CUSUM}.png}
\end{figure}


\subsection{Metadata Output}
\label{doc/Algorithms:id6}
The {\hyperref[api\string-doc/mosaic.processing:mosaic.cusumPlus.cusumPlus]{\emph{\code{cusumPlus}}}} algorithm outputs meta-data that characterizes every processed event. Similar to the \DUspan{xref,std,std-ref}{multistate-page} algorithm, this information is stored in a \href{http://www.sqlite.org/}{SQLite} database and is available for further processing (see {\hyperref[doc/DatabaseStructure:database\string-page]{\emph{Database Structure and Query Syntax}}}).

\begin{tabulary}{\linewidth}{p{4cm}p{4cm}p{8cm}}
\hline
\textsf{\relax 
\textbf{Column Name}
} & \textsf{\relax 
\textbf{Column Type}
} & \textsf{\relax 
\textbf{Description}
}\\
\hline
recIDX

ProcessingStatus

OpenChCurrent

NStates

CurrentStep

BlockDepth

EventStart

EventEnd

EventDelay

ResTime

AbsEventStart

TimeSeries
 & 
INTEGER

TEXT

REAL

INTEGER

REAL\_LIST

REAL\_LIST

REAL

REAL

REAL\_LIST

REAL

REAL

REAL\_LIST
 & 
Record index.

Status of the analysis.

Open channel current in pA.

Number of detected states.

Blocked current steps in pA.

BlockedCurrent/OpenChCurrent for each state.

Event start in ms.

Event end in ms.

Start time of each state in ms.

EventEnd-EventStart in ms.

Global event start time in ms.

(OPTIONAL) Event time-series.
\\
\hline\end{tabulary}



\chapter{Getting Started}
\label{doc/GettingStarted:id29}\label{doc/GettingStarted:gettingstarted}\label{doc/GettingStarted:getting-started}\label{doc/GettingStarted::doc}

\section{Binary Installation}
\label{doc/GettingStarted:binary-installation}
\emph{MOSAIC} is available as a pre-compiled binary for Windows and Mac OS X (\href{http://usnistgov.github.io/mosaic/platforms.html}{download binaries}). \emph{MOSAIC} binaries do not need special installation. Under \textbf{Mac OS X} open the the downloaded disk image and drag the \emph{MOSAIC} executable to the Applications folder. Under \textbf{Windows}, unzip downloaded zip file and move the \emph{MOSAIC} executable to your hard disk.

\begin{notice}{note}{Note:}
\emph{MOSAIC} binaries are 64-bit. If you need 32-bit support, please build \emph{MOSAIC} from source as described in the {\hyperref[doc/GettingStarted:sourceinstall]{\emph{Source Installation}}} section.
\end{notice}


\section{Source Installation}
\label{doc/GettingStarted:source-installation}\label{doc/GettingStarted:sourceinstall}

\subsection{Install \emph{MOSAIC} on Mac OS X}
\label{doc/GettingStarted:install-projname-on-mac-os-x}\label{doc/GettingStarted:installosx}
In the following guide, we provide step-by-step instructions on setting up and running \emph{MOSAIC} on OS X. To simplify the isntallation, we use \href{http://brew.sh/}{Homebrew} to install some required dependencies. \href{http://brew.sh/}{Homebrew}
requires Apple command line tools, but will directly prompt you to install it on set up.

\textbf{1. Installing Homebrew}

First we will install Homebrew, a useful package manager, to help install some of the dependencies required by \emph{MOSAIC}. You will need administrator access for this step. In the OS X Terminal, run the following command:

\begin{Verbatim}[commandchars=\\\{\}]
\PYG{g+gp}{\PYGZdl{}}  ruby \PYGZhy{}e \PYG{l+s+s2}{\PYGZdq{}}\PYG{k}{\PYGZdl{}(}curl \PYGZhy{}fsSL https://raw.githubusercontent.com/Homebrew/install/master/install\PYG{k}{)}\PYG{l+s+s2}{\PYGZdq{}}
\end{Verbatim}

Note, if the Apple command line tools are not installed, \href{http://brew.sh/}{Homebrew} will prompt you do so during installation.

\begin{notice}{hint}{Hint:}
To test if \href{http://brew.sh/}{Homebrew} is properly installed, run the following in the terminal: \code{brew doctor}
\end{notice}

To ensure that \href{http://brew.sh/}{Homebrew} is set up correctly, add the \href{http://brew.sh/}{Homebrew} directory to \textasciitilde{}/.bash\_profile. This can be done using the following command:

\begin{Verbatim}[commandchars=\\\{\}]
\PYG{g+gp}{\PYGZdl{}}  \PYG{n+nb}{echo} \PYG{l+s+s1}{\PYGZsq{}export PATH=\PYGZdq{}/usr/local/bin:\PYGZdl{}PATH\PYGZdq{}\PYGZsq{}} \PYGZgt{}\PYGZgt{} \PYGZti{}/.bash\PYGZus{}profile
\end{Verbatim}

\begin{notice}{hint}{Hint:}
If you don't have a .bash\_profile file in your home directory, you can create one manually using a text editor.
\end{notice}

Restart the terminal to update your shell.

\textbf{2. Installing brewed Python and other neccessary packages}

\emph{MOSAIC} is written in \href{http://www.python.org/}{Python} 2.7+ and utilizes a number of different packages and utilities. In the following we'll install a number of these (specifically, python, gcc, gfortran, qt, and pyQt4). With homebrew this is easy to do in one line! Run the following in the terminal:

\begin{Verbatim}[commandchars=\\\{\}]
\PYG{g+gp}{\PYGZdl{}}  brew install python gcc gfortran qt pyqt
\end{Verbatim}

At this point, it is a good idea to update the PYTHONPATH environment variable in \textasciitilde{}/.bash\_profile:
\phantomsection\label{doc/GettingStarted:setpythonpathosx}
\begin{Verbatim}[commandchars=\\\{\}]
\PYG{g+gp}{\PYGZdl{}}  \PYG{n+nb}{export} \PYG{n+nv}{PYTHONPATH}\PYG{o}{=}\PYGZdl{}PYTHONPATH:/usr/local/lib/python2.7/site\PYGZhy{}packages
\end{Verbatim}

\textbf{3. (Optional) Install and Setup Virtual Environment}

It is generally a good practice to run \emph{MOSAIC} from within a dedicated virtual environment. This minimizes conflicts with other installed programs. While we highly recommend this approach, it is not required to run \emph{MOSAIC}. If you prefer to skip this, move on to the next step now.

To setup a virtual environment, we need two different packages: \emph{virtualenv}, which creates the virtual environments, and \emph{virtualenvwrapper}, a wrapper for \emph{virtualenv} that simplifies set up and use.

To install these and set up the virtual enviroment wrapper, run the following in a shell:

\begin{Verbatim}[commandchars=\\\{\}]
\PYG{g+gp}{\PYGZdl{}}  pip install virtualenv virtualenvwrapper
\end{Verbatim}

\begin{notice}{hint}{Hint:}
Under Ubuntu, you may need install virtualenv and virtualenvwrapper as root. Simply prefix the command above with sudo.
\end{notice}

If you would like virtualenvwrapper to be available each time you open a new terminal window, add the line below to  \textasciitilde{}/.bash\_profile on OS X or \textasciitilde{}/.bashrc on Linux.

\begin{Verbatim}[commandchars=\\\{\}]
\PYG{g+go}{source /usr/local/bin/virtualenvwrapper.sh}
\end{Verbatim}

\begin{notice}{hint}{Hint:}
Depending on the process used to install \emph{virtualenv}, the path to virtualenvwrapper.sh may vary. Find the approporiate path by running \code{\$ find /usr -name virtualenvwrapper.sh}. Adjust the line in your .bash\_profile or .bashrc script accordingly.
\end{notice}

Open a new shell to make the new virtual environment available. Now we are ready to create a virtual environment.  You can choose any name for your virtual environment, here we name it \emph{MOSAIC}:

\begin{Verbatim}[commandchars=\\\{\}]
\PYG{g+gp}{\PYGZdl{}}  mkvirtualenv \PYGZhy{}p \PYGZlt{}path to python\PYGZgt{}/python MOSAIC
\end{Verbatim}

\begin{notice}{hint}{Hint:}
We explicitly specify the \href{http://www.python.org/}{Python} installation to use. This is not mandatory, but is useful if you have multiple \href{http://www.python.org/}{Python} installations on your computer. The \emph{\textless{}path to python\textgreater{}} may vary according to the specific version of python you wish to use. In most cases, this will be either \emph{/usr/local/bin/} or \emph{/usr/bin}
\end{notice}

\textbf{4. Installing MOSAIC}

\textbf{Install using Setuptools}

The command-line version of \emph{MOSAIC} can be installed using \emph{pip} as shown below. Any additional dependencies required by \emph{MOSAIC} will be installed automatically.

\begin{Verbatim}[commandchars=\\\{\}]
\PYG{g+go}{pip install mosaic\PYGZhy{}nist}
\end{Verbatim}

\begin{notice}{note}{Note:}
Installing the graphical interface requires one to install \emph{MOSAIC} from the source distribution as outlined below.
\end{notice}

\textbf{Install from a Downloaded Source Distribution}

First we need to obtain the \emph{MOSAIC} source code. For analyzing publication data, we recommend downloading the latest stable version of the source code (\href{http://usnistgov.github.io/mosaic/platforms.html}{download source}). Alternatively, the latest development version can be downloaded from the \href{https://github.com/usnistgov/mosaic}{MOSAIC page on Github}. Here we will show you how to set up \emph{MOSAIC} from the latest stable release:
\begin{enumerate}
\item {} 
Download the latest release (\href{http://usnistgov.github.io/mosaic/platforms.html}{download source})

\item {} 
Create a directory for the project source. In this case we will create a directory called MOSAIC, located in \code{\textasciitilde{}/projects/}, where `\textasciitilde{}' is your home directory.

\end{enumerate}

\begin{Verbatim}[commandchars=\\\{\}]
\PYG{g+gp}{\PYGZdl{}} mkdir \PYGZti{}/projects/MOSAIC
\end{Verbatim}
\begin{enumerate}
\setcounter{enumi}{2}
\item {} 
Navigate to the directory:

\end{enumerate}

\begin{Verbatim}[commandchars=\\\{\}]
\PYG{g+gp}{\PYGZdl{}} \PYG{n+nb}{cd} \PYGZti{}/projects/MOSAIC
\end{Verbatim}
\begin{enumerate}
\setcounter{enumi}{3}
\item {} 
Extract the source into this folder.

\item {} 
Make sure you are working in the virtual environment we set up in the previous step by typing:

\end{enumerate}

\begin{Verbatim}[commandchars=\\\{\}]
\PYG{g+gp}{\PYGZdl{}} workon MOSAIC
\end{Verbatim}

\begin{notice}{note}{Note:}
You will notice that (\emph{MOSAIC}) now appears in front of the \$ prompt in your shell. This inidicates that the virtual environment is active. We have employed this notation to indicate commands that should be run from inside the virtual environment.
\end{notice}
\begin{enumerate}
\setcounter{enumi}{5}
\item {} 
\emph{MOSAIC} and its dependencies are built using setuptools. Navigate to \textasciitilde{}/projects/MOSAIC/ and run the following:

\end{enumerate}

\begin{Verbatim}[commandchars=\\\{\}]
\PYG{g+gp}{(MOSAIC)\PYGZdl{}} python setup.py mosaic\PYGZus{}deps
\end{Verbatim}
\begin{enumerate}
\setcounter{enumi}{6}
\item {} 
Finally, add the installation directory (\textasciitilde{}/projects/MOSAIC as set up previously) to your \emph{PYTHONPATH} as shown below. This addition can be made permanent by adding the line below to your \emph{.bash\_profile} (OS X) or \emph{.bashrc} (Ubuntu) script.

\end{enumerate}

\begin{Verbatim}[commandchars=\\\{\}]
\PYG{g+gp}{(MOSAIC)\PYGZdl{}} \PYG{n+nb}{export} \PYG{n+nv}{PYTHONPATH}\PYG{o}{=}\PYGZdl{}PYTHONPATH:\PYGZti{}/projects/MOSAIC
\end{Verbatim}


\subsection{Install \emph{MOSAIC} on Ubuntu(14.04)}
\label{doc/GettingStarted:installubuntu}\label{doc/GettingStarted:install-projname-on-ubuntu-14-04}
\emph{MOSAIC} can be run under Ubuntu using a procedure very similar to \DUspan{xref,std,std-ref}{installosx}.

\textbf{1. Prerequisites}

Several prerequisites must be installed prior to building \emph{MOSAIC} dependencies. This is easily accomplished in Ubuntu using the \emph{aptitude} package manager.

\begin{notice}{hint}{Hint:}
\emph{superuser} privileges are needed when installing \emph{MOSAIC} prerequisites.
\end{notice}

\begin{Verbatim}[commandchars=\\\{\}]
\PYG{g+gp}{\PYGZdl{}}  sudo apt\PYGZhy{}get install python python\PYGZhy{}dev python\PYGZhy{}pip python\PYGZhy{}qt4
\PYG{g+go}{   pkg\PYGZhy{}config freetype* gfortran liblapack\PYGZhy{}dev libblas\PYGZhy{}dev}
\end{Verbatim}
\phantomsection\label{doc/GettingStarted:setpythonpathubuntu}
Next add the following to \textasciitilde{}/.bashrc

\begin{Verbatim}[commandchars=\\\{\}]
\PYG{g+go}{export PYTHONPATH=/usr/lib/python2.7/dist\PYGZhy{}packages}
\end{Verbatim}

\textbf{2. (Optional) Install and Setup Virtual Environment}

It is generally a good practice to run \emph{MOSAIC} from within a dedicated virtual environment. This minimizes conflicts with other installed programs. While we highly recommend this approach, it is not required to run \emph{MOSAIC}. If you prefer to skip this, move on to the next step now.

To setup a virtual environment, we need two different packages: \emph{virtualenv}, which creates the virtual environments, and \emph{virtualenvwrapper}, a wrapper for \emph{virtualenv} that simplifies set up and use.

To install these and set up the virtual enviroment wrapper, run the following in a shell:

\begin{Verbatim}[commandchars=\\\{\}]
\PYG{g+gp}{\PYGZdl{}}  pip install virtualenv virtualenvwrapper
\end{Verbatim}

\begin{notice}{hint}{Hint:}
Under Ubuntu, you may need install virtualenv and virtualenvwrapper as root. Simply prefix the command above with sudo.
\end{notice}

If you would like virtualenvwrapper to be available each time you open a new terminal window, add the line below to  \textasciitilde{}/.bash\_profile on OS X or \textasciitilde{}/.bashrc on Linux.

\begin{Verbatim}[commandchars=\\\{\}]
\PYG{g+go}{source /usr/local/bin/virtualenvwrapper.sh}
\end{Verbatim}

\begin{notice}{hint}{Hint:}
Depending on the process used to install \emph{virtualenv}, the path to virtualenvwrapper.sh may vary. Find the approporiate path by running \code{\$ find /usr -name virtualenvwrapper.sh}. Adjust the line in your .bash\_profile or .bashrc script accordingly.
\end{notice}

Open a new shell to make the new virtual environment available. Now we are ready to create a virtual environment.  You can choose any name for your virtual environment, here we name it \emph{MOSAIC}:

\begin{Verbatim}[commandchars=\\\{\}]
\PYG{g+gp}{\PYGZdl{}}  mkvirtualenv \PYGZhy{}p \PYGZlt{}path to python\PYGZgt{}/python MOSAIC
\end{Verbatim}

\begin{notice}{hint}{Hint:}
We explicitly specify the \href{http://www.python.org/}{Python} installation to use. This is not mandatory, but is useful if you have multiple \href{http://www.python.org/}{Python} installations on your computer. The \emph{\textless{}path to python\textgreater{}} may vary according to the specific version of python you wish to use. In most cases, this will be either \emph{/usr/local/bin/} or \emph{/usr/bin}
\end{notice}

\textbf{3. Installing MOSAIC}

\textbf{Install using Setuptools}

The command-line version of \emph{MOSAIC} can be installed using \emph{pip} as shown below. Any additional dependencies required by \emph{MOSAIC} will be installed automatically.

\begin{Verbatim}[commandchars=\\\{\}]
\PYG{g+go}{pip install mosaic\PYGZhy{}nist}
\end{Verbatim}

\begin{notice}{note}{Note:}
Installing the graphical interface requires one to install \emph{MOSAIC} from the source distribution as outlined below.
\end{notice}

\textbf{Install from a Downloaded Source Distribution}

First we need to obtain the \emph{MOSAIC} source code. For analyzing publication data, we recommend downloading the latest stable version of the source code (\href{http://usnistgov.github.io/mosaic/platforms.html}{download source}). Alternatively, the latest development version can be downloaded from the \href{https://github.com/usnistgov/mosaic}{MOSAIC page on Github}. Here we will show you how to set up \emph{MOSAIC} from the latest stable release:
\begin{enumerate}
\item {} 
Download the latest release (\href{http://usnistgov.github.io/mosaic/platforms.html}{download source})

\item {} 
Create a directory for the project source. In this case we will create a directory called MOSAIC, located in \code{\textasciitilde{}/projects/}, where `\textasciitilde{}' is your home directory.

\end{enumerate}

\begin{Verbatim}[commandchars=\\\{\}]
\PYG{g+gp}{\PYGZdl{}} mkdir \PYGZti{}/projects/MOSAIC
\end{Verbatim}
\begin{enumerate}
\setcounter{enumi}{2}
\item {} 
Navigate to the directory:

\end{enumerate}

\begin{Verbatim}[commandchars=\\\{\}]
\PYG{g+gp}{\PYGZdl{}} \PYG{n+nb}{cd} \PYGZti{}/projects/MOSAIC
\end{Verbatim}
\begin{enumerate}
\setcounter{enumi}{3}
\item {} 
Extract the source into this folder.

\item {} 
Make sure you are working in the virtual environment we set up in the previous step by typing:

\end{enumerate}

\begin{Verbatim}[commandchars=\\\{\}]
\PYG{g+gp}{\PYGZdl{}} workon MOSAIC
\end{Verbatim}

\begin{notice}{note}{Note:}
You will notice that (\emph{MOSAIC}) now appears in front of the \$ prompt in your shell. This inidicates that the virtual environment is active. We have employed this notation to indicate commands that should be run from inside the virtual environment.
\end{notice}
\begin{enumerate}
\setcounter{enumi}{5}
\item {} 
\emph{MOSAIC} and its dependencies are built using setuptools. Navigate to \textasciitilde{}/projects/MOSAIC/ and run the following:

\end{enumerate}

\begin{Verbatim}[commandchars=\\\{\}]
\PYG{g+gp}{(MOSAIC)\PYGZdl{}} python setup.py mosaic\PYGZus{}deps
\end{Verbatim}
\begin{enumerate}
\setcounter{enumi}{6}
\item {} 
Finally, add the installation directory (\textasciitilde{}/projects/MOSAIC as set up previously) to your \emph{PYTHONPATH} as shown below. This addition can be made permanent by adding the line below to your \emph{.bash\_profile} (OS X) or \emph{.bashrc} (Ubuntu) script.

\end{enumerate}

\begin{Verbatim}[commandchars=\\\{\}]
\PYG{g+gp}{(MOSAIC)\PYGZdl{}} \PYG{n+nb}{export} \PYG{n+nv}{PYTHONPATH}\PYG{o}{=}\PYGZdl{}PYTHONPATH:\PYGZti{}/projects/MOSAIC
\end{Verbatim}


\chapter{\emph{MOSAIC} GUI}
\label{doc/GraphicalInterface:projname-gui}\label{doc/GraphicalInterface:gui-page}\label{doc/GraphicalInterface::doc}\label{doc/GraphicalInterface:mosaic-page-on-github}
\emph{MOSAIC}`s GUI interface is designed to allow you to easily setup and run an analysis and to analyze the results of prior trials via a graphical interface; it contains the most commonly used features of \emph{MOSAIC}.  The GUI contains modular panels for setting up an analysis, running it, and analyzing the results. Here we give you a brief overview of the graphical interface and its basic use. You can learn more in the \DUspan{xref,std,std-ref}{examples-page} section.

\textbf{Opening the GUI}

If you installed \emph{MOSAIC} from a precomiled binary, you can open the GUI by double clicking the \emph{MOSAIC} icon. Alternatively, if you compiled \emph{MOSAIC} from source code, you can run the GUI from the terminal window -- navigate to the installation directory and type:

\begin{Verbatim}[commandchars=\\\{\}]
\PYG{g+go}{python runMOSAIC}
\end{Verbatim}

\begin{notice}{hint}{Hint:}
Having trobule getting the GUI to start? Frequently, this arises because your PYTHONPATH environment variable is set up incorrectly. To fix this error, first type \code{echo \$PYTHONPATH} in the terminal. If you don't see the path to the \emph{MOSAIC} installation in \emph{PYTHONPATH}, consult the operating-system specific instructions (\DUspan{xref,std,std-ref}{OSX} or \DUspan{xref,std,std-ref}{Ubuntu}) to help resolve this issue.
\end{notice}


\section{Interface Overview}
\label{doc/GraphicalInterface:interface-overview}\begin{figure}[htbp]
\centering
\capstart

\includegraphics[width=0.850\linewidth]{{mainpanels-panelLabel}.png}
\caption{Primary panels in \emph{MOSAIC}: (A)Analysis Setup (B) Trajectory Viewer (C) Live Blockade Depth Histogram (D) Live Analysis Statistics (E) Event Viewer.}\end{figure}

The main interface consists of five panels which we go over in detail later in this document. Briefly, these are:
\begin{enumerate}
\item {} 
\emph{Analysis Setup}: This panel is used to set up the analysis parameters.

\item {} 
\emph{Trajectory Viewer}: This panel shows a snippet of the ionic current time-series and an all points histogram, used to set the baseline and threshold parameters found in {\hyperref[doc/GraphicalInterface:analysis\string-setup]{\emph{Panel A: Analysis Setup}}}.

\item {} 
\emph{Blockade Depth Histogram}: Once the data processing has started, this panel shows a live blockade depth histogram; a query can be defined to restrict the histogram to data which fulfills a user-defined criteria.

\item {} 
\emph{Analysis Statistics}: Displays live statistics about the data processed.

\item {} 
\emph{Event Viewer}: Displays the partitioned events and their fit. This panel is active only if ``Write Events to Disk'' is enabled in the {\hyperref[doc/GraphicalInterface:analysis\string-setup]{\emph{Analysis Setup.}}}

\end{enumerate}


\section{Panels A \& B: Analysis Setup and Trajectory Viewer}
\label{doc/GraphicalInterface:panels-a-b-analysis-setup-and-trajectory-viewer}\begin{figure}[htbp]
\centering
\capstart

\includegraphics[width=0.700\linewidth]{{panelab}.png}
\caption{\emph{Overview of Panels A \& B}: (A) Analysis setup panel (B) Trajectory viewer panel}\end{figure}


\subsection{Panel A: Analysis Setup}
\label{doc/GraphicalInterface:analysis-setup}\label{doc/GraphicalInterface:panel-a-analysis-setup}
\textbf{1. Data Settings}
\begin{itemize}
\item {} 
\textbf{Path}: Allows user to set the directory containing files to analyze. Click the ''...'' icon to navigate to the directory.

\item {} 
\textbf{File Type}: The GUI is natively compatible with either ABF or QDF Files, this field is automatically populated based on the files in the directory you've chosen.The \textbf{Rfb} and \textbf{Cfb} parameters are needed to correctly analyze QDF files (see {\hyperref[api\string-doc/mosaic.traj:module\string-mosaic.qdfTrajIO]{\emph{\code{qdfTrajIO}}}} for more information)

\item {} 
\textbf{Rfb \& Cfb}: \emph{MOSAIC} supports the QUB QDF file format used by the \href{http://electronicbio.com}{Electronic Biosciences} Nanopatch system. Two additional parameters, the feedback resistance (Rfb) in Ohms and capacitance (Cfb) in Farads are required to appropriately convert the measurements to ionic current.

\item {} 
\textbf{Start} and \textbf{End}: These parameters allow you to analyze a range of your data. Choose the starting and ending times if you'd like to analyze a small time segement of your data. If this is left blank, all data will be analyzed.

\item {} 
\textbf{DC Offset}: If your measurement contains a systematic bias, it can be manually corrected by entering the DC offset here.

\end{itemize}

\textbf{2. Baseline Current Detection}
\begin{itemize}
\item {} 
\(\mu\): Mean baseline current, in picoamperes (pA). This is shown schematically in the trajectory viewer (see Label \#8). When \emph{Auto} is selected, this will be greyed out and labeled \code{\textless{}auto\textgreater{}}

\item {} 
\(\sigma\): Noise level (in pA). This is expected noise level of your baseline. Typically one would set this to the measured RMS noise of the open channel state at the cutoff frequency. When \emph{Auto} is selected, this will be greyed out and labeled \code{\textless{}auto\textgreater{}}.

\item {} 
\textbf{Auto}: Checking this box enables automatic dectection of the mean baseline current (\(\mu\)) and noise level (\(\sigma\)). When auto is enabled, the values chosen by the software will be displayed in the trajectory viewer panel (see Label \#10)

\item {} 
\textbf{Block Size}: Controls the amount of data examined to determine the baseline. This also controls the amount of data shown in the trajectory viewer.

\end{itemize}

\textbf{3. Event Partition Control}

This panel is used to set the current threhold used for event detection
\begin{itemize}
\item {} 
\textbf{Algorithm}: Currently,the only event partitioning algorithm enabled is \emph{CurrentThreshold}.

\item {} 
\textbf{Threshold}: This is used to set the minimum current threshold used to partition events with the \emph{CurrentThreshold} algorithm.

\end{itemize}

\textbf{4.  Event Processing Setup}

\textbf{Event Processing Algorithm}: The GUI supports two event processing algorithms, i) \emph{StepResponseAnalysis} and ii) \emph{MultiStateAnalysis}. \emph{StepResponseAnalysis} is the default analysis, and should be used with data sets with unimodal events. For events with mutliple states or steps the \emph{MultiStateAnalysis} algorithm, which is capable of automatically analyzing events with \emph{N} states, should be used. Note that \emph{StepResponseAnalysis} is a restricted case of \emph{MultiStateAnalysis} and is more computationally efficient to run if you have unimodal (or single states) data.
\begin{itemize}
\item {} 
\textbf{Write Events to Disk}: When this box is checked, the data points for each partition events are written to the \href{http://www.sqlite.org/}{SQLite} database. When this is checked it is possible to view the individual fits of each  in the \emph{Event Fits} panel.

\end{itemize}

\begin{notice}{hint}{Hint:}
When \emph{Write Evens to Disk} is checked, your database can become extremely large! This is because \emph{MOSAIC} is effectively writing most of your time-series to the database. Note that the fit parameters are \emph{always} written to the database.
\end{notice}
\begin{itemize}
\item {} 
\textbf{Parallel Processing} and \textbf{Processors}: Parallel processing can be enabled by checking this box. This box will be greyed out if the python module \href{http://zeromq.github.io/pyzmq/}{ZeroMQ} is not installed. The \emph{Processors} box allows you to select the number of processors used in the analysis. It is important to note that the GUI will occupy one processor, so choosing 3 processors will actually use a total of 4 processors.

\end{itemize}

\textbf{5. Plot Results and Advanced Settings}
\begin{itemize}
\item {} 
\textbf{Event Fits}: Checking this box will show the events viewer (Panel E). This can also be accessed from the file menu \code{View\textgreater{}Plots\textgreater{}Event Fits}. If \emph{Write Events to Disk} is not enabled this checkbox will be greyed out.

\item {} 
\textbf{Blockade Depth Histogram}: Checking this box will show the blockade depth histogram (Panel C). This can also be accessed through the file menu \code{View\textgreater{}Plots\textgreater{}Blockade Depth Histogram}.

\item {} 
\textbf{Advanced Settings}: This opens a dialog window to manually edit settings not otherwise accessible in the GUI. See the {\hyperref[doc/settingsFile:settings\string-page]{\emph{Settings File}}} section for further details.

\end{itemize}


\subsection{Panel B: Trajectory Viewer}
\label{doc/GraphicalInterface:trajectory-viewer}\label{doc/GraphicalInterface:panel-b-trajectory-viewer}
This panel shows a segment of the data time series. The file currently being displayed is shown at the top of the window. If data from multiple files are loaded, the last filename is displayed. The length of time displayed in the window is controlled by \emph{BlockSize} in Panel A (see \#2).

\textbf{6.  Time Series (Trajectory)}
\begin{itemize}
\item {} 
This plot shows the ionic current time series, of length \emph{BlockSize}. Other features in the panel (such as histogram, denoising, etc.) only utilize the data in the window for their calculations.

\end{itemize}

\textbf{7.  All Points Histogram}
\begin{itemize}
\item {} 
This shows a histogram of the time series data shown in \#6.

\end{itemize}

\textbf{8. Dashed line indicates mean baseline current}

\textbf{9. Detection threshold level indicated by solid red line}

\textbf{10. Navigation, Denoising, and Statistics}
\begin{itemize}
\item {} 
\textbf{Navigation Tools:} Tools to navigate the plot window are shown below the time-series plot. These can be applied to either the trajectory or all points histogram plots. The arrow bar on the bottom right of the trajectory viewer can be used to advance to the next data block.

\item {} 
\textbf{Denoising} Wavelet denoising can be activated by clicking , the denoising level is enabled here, the level of denoising can be varied between 1 and 5.

\end{itemize}

\begin{notice}{warning}{Warning:}
Wavelet-based denoising is currently an experimental feature and should be used with caution.
\end{notice}
\begin{itemize}
\item {} 
\textbf{Baseline Statistics}: The mean baseline current, standard deviation, and the threshold used for event detection (specified as a multiple of the standard deviation in parenthesis) correspond to the settings in the main window. If the baseline current detection is set to \emph{auto} these values will update as each data segment is examined. The size of this segment is determined by the \emph{Block Size} setting. In the figure above, the \emph{Block Size} is set to 0.5 s.

\end{itemize}


\section{Panels C,D, \& E: Blockade Depth Histogram, Statistics, and Event Viewer}
\label{doc/GraphicalInterface:panels-c-d-e-blockade-depth-histogram-statistics-and-event-viewer}

\subsection{Panel C: Blockade Depth Histogram}
\label{doc/GraphicalInterface:panel-c-blockade-depth-histogram}\label{doc/GraphicalInterface:blockade-depth-hist}\begin{figure}[htbp]
\centering
\capstart

\includegraphics[width=0.500\linewidth]{{bd_hist}.png}
\caption{Blockade depth histogram}\end{figure}

This window shows the blockade depth histogram calculated from the meta-data output by \emph{MOSAIC}.
\begin{itemize}
\item {} 
\textbf{Filter}: The data displayed in the histogram can be restricted to events that fulfill specific user-defined criteria. For instance, the default filter \code{ResTime \textgreater{} 0.025} only includes events longer than 0.025 ms (or 25 \(\mu s\)). The GUI uses a \href{http://en.wikipedia.org/wiki/SQL}{SQL} \emph{select} statement to restrict the events included in the histogram. The text in the \emph{Filter} field represents the part of the query after the \emph{where} clause, and allows the user to use standard \href{http://en.wikipedia.org/wiki/SQL}{SQL} syntax to narrow the results in the plot. See the \DUspan{xref,std,std-ref}{working-with-sqlite-sec} section for details on \href{http://en.wikipedia.org/wiki/SQL}{SQL} syntax.

\item {} 
\textbf{Bins}: The number of bins in the histogram are defined here. By default, 500 bins are used, but the user can change this necessary.

\item {} 
\textbf{Detect Peaks}: Checking \emph{Detect Peaks} enables a wavelet-based peak detection algorithm. The wavelet level slider controls the sensitivity of the peak detection. Sliding it to the right will decrease the number of peaks picked up. The peaks detected are represented with red dots. Mousing over the detected peaks cause the coordinates of the peak to be displayed in the lower right hand corner of the window. The detected peaks can also be exported to a CSV file from the file menu \code{File\textgreater{}Save Histogram}.

\end{itemize}


\subsection{Panel D: Statistics}
\label{doc/GraphicalInterface:panel-d-statistics}\label{doc/GraphicalInterface:statistics}\begin{figure}[htbp]
\centering
\capstart

\includegraphics[width=0.400\linewidth]{{statistics}.png}
\caption{Live statistics window}\end{figure}

The \emph{Statistics Window} is displayed when a new analysis is started and displays:
\begin{itemize}
\item {} 
\textbf{Events Processed}: The number of events processed.

\item {} 
\textbf{Processing Error}: The processing error rate (i.e. the percentage of events for which fit has failed).

\item {} 
\textbf{Capture Rate}: An estimate of the mean capture rate.

\item {} 
\textbf{Analysis Time}: The amount of data processed (in seconds).

\end{itemize}


\subsection{Panel E: Event Viewer}
\label{doc/GraphicalInterface:event-viewer}\label{doc/GraphicalInterface:panel-e-event-viewer}\begin{figure}[htbp]
\centering
\capstart

\includegraphics[width=0.500\linewidth]{{event_viewerpng}.png}
\caption{Event viewer window}\end{figure}

If \emph{Write to Disk} is enabled, this panel allows you to view the first 10,000 events processed. This is useful to ensure the quality of the analysis and to debug potential problems with the settings.


\subsection{Console Log}
\label{doc/GraphicalInterface:console-log}\begin{figure}[htbp]
\centering
\capstart

\includegraphics[width=0.500\linewidth]{{console}.png}
\caption{Console log window}\end{figure}

When processing is complete, this panel displays a log of the analysis. This log contains useful information such as the analysis settings, the number of events fit, baseline drift, open channel conducatance, etc. This file is written to the database and can be accessed later.


\subsection{Advanced Settings}
\label{doc/GraphicalInterface:advanced-settings}\begin{figure}[htbp]
\centering
\capstart

\includegraphics[width=0.500\linewidth]{{advanced-settings}.png}
\caption{Advanced settings window}\end{figure}

This dialog allows you to manually edit advanced settings for uncommon use cases not natively accessible from within the GUI. Further information can be found in the {\hyperref[doc/settingsFile:settings\string-page]{\emph{Settings File}}}.


\chapter{Settings File}
\label{doc/settingsFile:settings-page}\label{doc/settingsFile:settings-file}\label{doc/settingsFile::doc}\label{doc/settingsFile:mosaic-page-on-github}
\emph{MOSAIC} stores its settings in the \href{http://json.org/}{JSON} format. When using the graphical interface, a settings file is generated automatically upon starting an analysis, or by clicking \emph{Save Settings} in the \emph{File menu}  (see {\hyperref[doc/GraphicalInterface:gui\string-page]{\emph{MOSAIC GUI}}}).


\section{Settings Layout}
\label{doc/settingsFile:settings-layout}
\href{http://json.org/}{JSON} is a human readable file format that consists of key-value pairs separated by sections. Each section in a JSON object consists of a section name and a list of string key-value pairs.

\begin{Verbatim}[commandchars=\\\{\}]
\PYG{p}{\PYGZob{}}
        \PYG{l+s+s2}{\PYGZdq{}\PYGZlt{}section name\PYGZgt{}\PYGZdq{}} \PYG{o}{:} \PYG{p}{\PYGZob{}}
                \PYG{l+s+s2}{\PYGZdq{}key1\PYGZdq{}} \PYG{o}{:} \PYG{l+s+s2}{\PYGZdq{}value1\PYGZdq{}}\PYG{p}{,}
                \PYG{l+s+s2}{\PYGZdq{}key2\PYGZdq{}} \PYG{o}{:} \PYG{l+s+s2}{\PYGZdq{}value2\PYGZdq{}}\PYG{p}{,}
                \PYG{p}{...}
        \PYG{p}{\PYGZcb{}}

\PYG{p}{\PYGZcb{}}
\end{Verbatim}

\emph{MOSAIC} settings define a new section for each class, with key-value pairs corresponding to class attributes that are set upon initialization. This is illustrated below for the adept2State class. The \emph{adept2State} section in the settings file holds parameters corresponding to the {\hyperref[api\string-doc/mosaic.processing:module\string-mosaic.adept2State]{\emph{\code{adept2State}}}} class. Note that that the section name in the settings file is identical to the corresponding class name. Three parameters are then defined within the section that control the behavior of the class.

\begin{Verbatim}[commandchars=\\\{\}]
\PYG{p}{\PYGZob{}}
        \PYG{l+s+s2}{\PYGZdq{}adept2State\PYGZdq{}} \PYG{o}{:} \PYG{p}{\PYGZob{}}
                \PYG{l+s+s2}{\PYGZdq{}FitTol\PYGZdq{}}                        \PYG{o}{:} \PYG{l+s+s2}{\PYGZdq{}1.e\PYGZhy{}7\PYGZdq{}}\PYG{p}{,}
                \PYG{l+s+s2}{\PYGZdq{}FitIters\PYGZdq{}}                      \PYG{o}{:} \PYG{l+s+s2}{\PYGZdq{}50000\PYGZdq{}}\PYG{p}{,}
                \PYG{l+s+s2}{\PYGZdq{}BlockRejectRatio\PYGZdq{}}              \PYG{o}{:} \PYG{l+s+s2}{\PYGZdq{}0.9\PYGZdq{}}
        \PYG{p}{\PYGZcb{}}
\PYG{p}{\PYGZcb{}}
\end{Verbatim}

Finally, {\hyperref[api\string-doc/mosaic.processing:module\string-mosaic.adept2State]{\emph{\code{adept2State}}}} is initialized by defining class attributes corresponding to the key-value pairs in the settings file.

\begin{Verbatim}[commandchars=\\\{\}]
\PYG{k}{try}\PYG{p}{:}
                \PYG{n+nb+bp}{self}\PYG{o}{.}\PYG{n}{FitTol}\PYG{o}{=}\PYG{n+nb}{float}\PYG{p}{(}\PYG{n+nb+bp}{self}\PYG{o}{.}\PYG{n}{settingsDict}\PYG{o}{.}\PYG{n}{pop}\PYG{p}{(}\PYG{l+s+s2}{\PYGZdq{}}\PYG{l+s+s2}{FitTol}\PYG{l+s+s2}{\PYGZdq{}}\PYG{p}{,} \PYG{l+m+mf}{1.e\PYGZhy{}7}\PYG{p}{)}\PYG{p}{)}
                \PYG{n+nb+bp}{self}\PYG{o}{.}\PYG{n}{FitIters}\PYG{o}{=}\PYG{n+nb}{int}\PYG{p}{(}\PYG{n+nb+bp}{self}\PYG{o}{.}\PYG{n}{settingsDict}\PYG{o}{.}\PYG{n}{pop}\PYG{p}{(}\PYG{l+s+s2}{\PYGZdq{}}\PYG{l+s+s2}{FitIters}\PYG{l+s+s2}{\PYGZdq{}}\PYG{p}{,} \PYG{l+m+mi}{5000}\PYG{p}{)}\PYG{p}{)}

                \PYG{n+nb+bp}{self}\PYG{o}{.}\PYG{n}{BlockRejectRatio}\PYG{o}{=}\PYG{n+nb}{float}\PYG{p}{(}\PYG{n+nb+bp}{self}\PYG{o}{.}\PYG{n}{settingsDict}\PYG{o}{.}\PYG{n}{pop}\PYG{p}{(}\PYG{l+s+s2}{\PYGZdq{}}\PYG{l+s+s2}{BlockRejectRatio}\PYG{l+s+s2}{\PYGZdq{}}\PYG{p}{,} \PYG{l+m+mf}{0.8}\PYG{p}{)}\PYG{p}{)}

\PYG{k}{except} \PYG{n+ne}{ValueError} \PYG{k}{as} \PYG{n}{err}\PYG{p}{:}
        \PYG{k}{raise} \PYG{n}{commonExceptions}\PYG{o}{.}\PYG{n}{SettingsTypeError}\PYG{p}{(} \PYG{n}{err} \PYG{p}{)}
\end{Verbatim}


\section{Trajectory Settings}
\label{doc/settingsFile:trajectory-settings}

\subsection{Common Settings (\texttt{metaTrajIO})}
\label{doc/settingsFile:common-settings-metatrajio}\begin{quote}

\begin{notice}{warning}{Warning:}
This metaclass must be sub-classed. All abstract methods within this metaclass must be implemented.
\end{notice}

Initialize a TrajIO object. The object can load all the data in a directory,
N files from a directory or from an explicit list of filenames. In addition
to the arguments defined below, implementations of this meta class may require
the definition of additional arguments. See the documentation of those classes
for what those may be. For example, the qdfTrajIO implementation of metaTrajIO also requires
the feedback resistance (Rfb) and feedback capacitance (Cfb) to be passed at initialization.
\begin{quote}\begin{description}
\item[{Parameters}] \leavevmode\begin{itemize}
\item {} 
\emph{dirname} :           all files from a directory (`\textless{}full path to data directory\textgreater{}')

\item {} 
\emph{nfiles} :            if requesting N files (in addition to dirname) from a specified directory

\item {} 
\emph{fnames} :            explicit list of filenames ({[}file1, file2,...{]}). This argument cannot be used in conjuction with dirname/nfiles. The filter argument is ignored when used in combination with fnames.

\item {} 
\emph{filter} :            `\textless{}wildcard filter\textgreater{}' (optional, filter is `*' if not specified)

\item {} 
\emph{start} :             Data start point in seconds.

\item {} 
\emph{end} :                       Data end point in seconds.

\item {} 
\emph{datafilter} :        Handle to the algorithm to use to filter the data. If no algorithm is specified, datafilter     is None and no filtering is performed.

\item {} 
\emph{dcOffset} :          Subtract a DC offset from the ionic current data.

\end{itemize}

\item[{Properties}] \leavevmode\begin{itemize}
\item {} 
\emph{FsHz} :                                      sampling frequency in Hz. If the data was decimated, this property will hold the sampling frequency after decimation.

\item {} 
\emph{LastFileProcessed} :         return the data file that was last processed.

\item {} 
\emph{ElapsedTimeSeconds} :        return the analysis time in sec.

\end{itemize}

\item[{Errors}] \leavevmode\begin{itemize}
\item {} 
\emph{IncompatibleArgumentsError} :        when conflicting arguments are used.

\item {} 
\emph{EmptyDataPipeError} :                        when out of data.

\item {} 
\emph{FileNotFoundError} :                         when data files do not exist in the specified path.

\item {} 
\emph{InsufficientArgumentsError} :        when incompatible arguments are passed

\end{itemize}

\end{description}\end{quote}
\end{quote}


\subsection{QDF Files (\texttt{qdfTrajIO})}
\label{doc/settingsFile:qdf-files-qdftrajio}\begin{quote}

Use the readqdf module from EBS to read individual QDF files.

In addition to {\hyperref[api\string-doc/mosaic.meta:mosaic.metaTrajIO.metaTrajIO]{\emph{\code{metaTrajIO}}}} args, check if the
feedback resistance (\code{Rfb}) and feedback capacitance (\code{Cfb})
are defined to convert qdf binary data into pA.

A typical settings section to read QDF files is shown below. Note, that
the values for \code{Rfb} and \code{Cfb} are specific to the amplifier used.

\begin{Verbatim}[commandchars=\\\{\}]
\PYG{l+s+s2}{\PYGZdq{}qdfTrajIO\PYGZdq{}}\PYG{o}{:} \PYG{p}{\PYGZob{}}
\PYG{l+s+s2}{\PYGZdq{}Rfb\PYGZdq{}}                           \PYG{o}{:} \PYG{l+m+mf}{9.1}\PYG{n+nx}{e}\PYG{o}{+}\PYG{l+m+mi}{12}\PYG{p}{,}
\PYG{l+s+s2}{\PYGZdq{}Cfb\PYGZdq{}}                           \PYG{o}{:} \PYG{l+m+mf}{1.07}\PYG{n+nx}{e}\PYG{o}{\PYGZhy{}}\PYG{l+m+mi}{12}\PYG{p}{,}
\PYG{l+s+s2}{\PYGZdq{}dcOffset\PYGZdq{}}                      \PYG{o}{:} \PYG{l+m+mf}{0.0}\PYG{p}{,}
\PYG{l+s+s2}{\PYGZdq{}filter\PYGZdq{}}                        \PYG{o}{:} \PYG{l+s+s2}{\PYGZdq{}*.qdf\PYGZdq{}}\PYG{p}{,}
\PYG{l+s+s2}{\PYGZdq{}start\PYGZdq{}}                         \PYG{o}{:} \PYG{l+m+mf}{0.0}
\PYG{p}{\PYGZcb{}}
\end{Verbatim}
\begin{quote}\begin{description}
\item[{Parameters}] \leavevmode\begin{description}
\item[{In addition to metaTrajIO.\_\_init\_\_ args,}] \leavevmode\begin{itemize}
\item {} 
\emph{Rfb} :               feedback resistance of amplifier

\item {} 
\emph{Cfb} :               feedback capacitance of amplifier

\item {} 
\emph{format} :    `V' for voltage or `pA' for current. Default is `V'

\end{itemize}

\end{description}

\item[{Returns}] \leavevmode
None

\item[{Errors}] \leavevmode\begin{itemize}
\item {} 
\emph{InsufficientArgumentsError} : if the mandatory arguments \code{Rfb} and \code{Cfb} are not set.

\end{itemize}

\end{description}\end{quote}
\end{quote}


\subsection{ABF Files (\texttt{abfTrajIO})}
\label{doc/settingsFile:abf-files-abftrajio}\begin{quote}

Read ABF1 and ABF2 file formats. Currently, only
gap-free mode and single channel recordings are supported.

A typical settings section to read ABF files is shown below.

\begin{Verbatim}[commandchars=\\\{\}]
\PYG{l+s+s2}{\PYGZdq{}abfTrajIO\PYGZdq{}} \PYG{o}{:} \PYG{p}{\PYGZob{}}
\PYG{l+s+s2}{\PYGZdq{}filter\PYGZdq{}}                        \PYG{o}{:} \PYG{l+s+s2}{\PYGZdq{}*.abf\PYGZdq{}}\PYG{p}{,}
\PYG{l+s+s2}{\PYGZdq{}start\PYGZdq{}}                         \PYG{o}{:} \PYG{l+m+mf}{0.0}\PYG{p}{,}
\PYG{l+s+s2}{\PYGZdq{}dcOffset\PYGZdq{}}                      \PYG{o}{:} \PYG{l+m+mf}{0.0}
\PYG{p}{\PYGZcb{}}
\end{Verbatim}
\begin{quote}\begin{description}
\item[{Parameters}] \leavevmode\begin{description}
\item[{In addition to {\hyperref[api\string-doc/mosaic.meta:mosaic.metaTrajIO.metaTrajIO]{\emph{\code{metaTrajIO}}}} args,}] \leavevmode
None

\end{description}

\end{description}\end{quote}
\end{quote}


\subsection{Binary Files (\texttt{binTrajIO})}
\label{doc/settingsFile:binary-files-bintrajio}\begin{quote}

Read a file that contains interleaved binary data, ordered by column. Only a single
column that holds ionic current data is read. The current in pA
is returned after scaling by the amplifier scale factor (\code{AmplifierScale}) and
removing any offsets (\code{AmplifierOffset}) if provided.
\begin{quote}\begin{description}
\item[{Usage and Assumptions}] \leavevmode
Binary data is interleaved by column. For three columns (\emph{a}, \emph{b}, and \emph{c}) and \emph{N} rows,
binary data is assumed to be of the form:
\begin{quote}

{[} a\_1, b\_1, c\_1, a\_2, b\_2, c\_2, ... ... ..., a\_N, b\_N, c\_N {]}
\end{quote}

The column layout is specified with the \code{ColumnTypes} parameter, which accepts a list of tuples.
For the example above, if column \textbf{a} is the ionic current in a 64-bit floating point format,
column \textbf{b} is the ionic current representation in 16-bit integer format and column \textbf{c} is
an index in 16-bit integer format, the \code{ColumnTypes} paramter is a list with three
tuples, one for each column, as shown below:
\begin{quote}

{[}(`curr\_pA', `float64'), (`AD\_V', `int16'), (`index', `int16'){]}
\end{quote}

The first element of each tuple is an arbitrary text label and the second element is
a valid \href{http://docs.scipy.org/doc/numpy/user/basics.types.html}{Numpy type}.

Finally, the \code{IonicCurrentColumn} parameter holds the name (text label defined above) of the
column that holds the ionic current time-series. Note that if an integer column is selected,
the \code{AmplifierScale} and \code{AmplifierOffset} parameters can be used to convert the voltage from
the A/D to a current.

Assuming that we use a floating point representation of the ionic current, and
a sampling rate of 50 kHz, a settings section that will read the binary file format
defined above is:

\begin{Verbatim}[commandchars=\\\{\}]
\PYG{l+s+s2}{\PYGZdq{}binTrajIO\PYGZdq{}}\PYG{o}{:} \PYG{p}{\PYGZob{}}
        \PYG{l+s+s2}{\PYGZdq{}AmplifierScale\PYGZdq{}} \PYG{o}{:} \PYG{l+s+s2}{\PYGZdq{}1\PYGZdq{}}\PYG{p}{,}
        \PYG{l+s+s2}{\PYGZdq{}AmplifierOffset\PYGZdq{}} \PYG{o}{:} \PYG{l+s+s2}{\PYGZdq{}0\PYGZdq{}}\PYG{p}{,}
        \PYG{l+s+s2}{\PYGZdq{}SamplingFrequency\PYGZdq{}} \PYG{o}{:} \PYG{l+s+s2}{\PYGZdq{}50000\PYGZdq{}}\PYG{p}{,}
        \PYG{l+s+s2}{\PYGZdq{}ColumnTypes\PYGZdq{}} \PYG{o}{:} \PYG{l+s+s2}{\PYGZdq{}[(\PYGZsq{}curr\PYGZus{}pA\PYGZsq{}, \PYGZsq{}float64\PYGZsq{}), (\PYGZsq{}AD\PYGZus{}V\PYGZsq{}, \PYGZsq{}int16\PYGZsq{}), (\PYGZsq{}index\PYGZsq{}, \PYGZsq{}int16\PYGZsq{})]\PYGZdq{}}\PYG{p}{,}
        \PYG{l+s+s2}{\PYGZdq{}IonicCurrentColumn\PYGZdq{}} \PYG{o}{:} \PYG{l+s+s2}{\PYGZdq{}curr\PYGZus{}pA\PYGZdq{}}\PYG{p}{,}
        \PYG{l+s+s2}{\PYGZdq{}dcOffset\PYGZdq{}}\PYG{o}{:} \PYG{l+s+s2}{\PYGZdq{}0.0\PYGZdq{}}\PYG{p}{,}
        \PYG{l+s+s2}{\PYGZdq{}filter\PYGZdq{}}\PYG{o}{:} \PYG{l+s+s2}{\PYGZdq{}*.bin\PYGZdq{}}\PYG{p}{,}
        \PYG{l+s+s2}{\PYGZdq{}start\PYGZdq{}}\PYG{o}{:} \PYG{l+s+s2}{\PYGZdq{}0.0\PYGZdq{}}\PYG{p}{,}
        \PYG{l+s+s2}{\PYGZdq{}HeaderOffset\PYGZdq{}}\PYG{o}{:} \PYG{l+m+mi}{0}
\PYG{p}{\PYGZcb{}}
\end{Verbatim}

\item[{Settings Examples}] \leavevmode
Read 16-bit signed integers (big endian) with a 512 byte header offset. Set the amplifier scale to 400 pA, sampling rate to 200 kHz.
\begin{quote}

\begin{Verbatim}[commandchars=\\\{\}]
\PYG{l+s+s2}{\PYGZdq{}binTrajIO\PYGZdq{}}\PYG{o}{:} \PYG{p}{\PYGZob{}}
        \PYG{l+s+s2}{\PYGZdq{}AmplifierOffset\PYGZdq{}}\PYG{o}{:} \PYG{l+s+s2}{\PYGZdq{}0.0\PYGZdq{}}\PYG{p}{,}
        \PYG{l+s+s2}{\PYGZdq{}SamplingFrequency\PYGZdq{}}\PYG{o}{:} \PYG{l+m+mi}{200000}\PYG{p}{,}
        \PYG{l+s+s2}{\PYGZdq{}AmplifierScale\PYGZdq{}}\PYG{o}{:} \PYG{l+s+s2}{\PYGZdq{}400./2**16\PYGZdq{}}\PYG{p}{,}
        \PYG{l+s+s2}{\PYGZdq{}ColumnTypes\PYGZdq{}}\PYG{o}{:} \PYG{l+s+s2}{\PYGZdq{}[(\PYGZsq{}curr\PYGZus{}pA\PYGZsq{}, \PYGZsq{}\PYGZgt{}i2\PYGZsq{})]\PYGZdq{}}\PYG{p}{,}
        \PYG{l+s+s2}{\PYGZdq{}dcOffset\PYGZdq{}}\PYG{o}{:} \PYG{l+m+mf}{0.0}\PYG{p}{,}
        \PYG{l+s+s2}{\PYGZdq{}filter\PYGZdq{}}\PYG{o}{:} \PYG{l+s+s2}{\PYGZdq{}*.dat\PYGZdq{}}\PYG{p}{,}
        \PYG{l+s+s2}{\PYGZdq{}start\PYGZdq{}}\PYG{o}{:} \PYG{l+m+mf}{0.0}\PYG{p}{,}
        \PYG{l+s+s2}{\PYGZdq{}HeaderOffset\PYGZdq{}}\PYG{o}{:} \PYG{l+m+mi}{512}\PYG{p}{,}
        \PYG{l+s+s2}{\PYGZdq{}IonicCurrentColumn\PYGZdq{}}\PYG{o}{:} \PYG{l+s+s2}{\PYGZdq{}curr\PYGZus{}pA\PYGZdq{}}
\PYG{p}{\PYGZcb{}}
\end{Verbatim}
\end{quote}

Read a two-column file: 64-bit floating point and 64-bit integers, and no header offset. Set the amplifier scale to 1 and sampling rate to 200 kHz.
\begin{quote}

\begin{Verbatim}[commandchars=\\\{\}]
\PYG{l+s+s2}{\PYGZdq{}binTrajIO\PYGZdq{}}\PYG{o}{:} \PYG{p}{\PYGZob{}}
        \PYG{l+s+s2}{\PYGZdq{}AmplifierOffset\PYGZdq{}}\PYG{o}{:} \PYG{l+s+s2}{\PYGZdq{}0.0\PYGZdq{}}\PYG{p}{,}
        \PYG{l+s+s2}{\PYGZdq{}SamplingFrequency\PYGZdq{}}\PYG{o}{:} \PYG{l+m+mi}{200000}\PYG{p}{,}
        \PYG{l+s+s2}{\PYGZdq{}AmplifierScale\PYGZdq{}}\PYG{o}{:} \PYG{l+s+s2}{\PYGZdq{}1.0\PYGZdq{}}\PYG{p}{,}
        \PYG{l+s+s2}{\PYGZdq{}ColumnTypes\PYGZdq{}} \PYG{o}{:} \PYG{l+s+s2}{\PYGZdq{}[(\PYGZsq{}curr\PYGZus{}pA\PYGZsq{}, \PYGZsq{}float64\PYGZsq{}), (\PYGZsq{}AD\PYGZus{}V\PYGZsq{}, \PYGZsq{}int64\PYGZsq{})]\PYGZdq{}}\PYG{p}{,}
        \PYG{l+s+s2}{\PYGZdq{}dcOffset\PYGZdq{}}\PYG{o}{:} \PYG{l+m+mf}{0.0}\PYG{p}{,}
        \PYG{l+s+s2}{\PYGZdq{}filter\PYGZdq{}}\PYG{o}{:} \PYG{l+s+s2}{\PYGZdq{}*.bin\PYGZdq{}}\PYG{p}{,}
        \PYG{l+s+s2}{\PYGZdq{}start\PYGZdq{}}\PYG{o}{:} \PYG{l+m+mf}{0.0}\PYG{p}{,}
        \PYG{l+s+s2}{\PYGZdq{}HeaderOffset\PYGZdq{}}\PYG{o}{:} \PYG{l+m+mi}{0}\PYG{p}{,}
        \PYG{l+s+s2}{\PYGZdq{}IonicCurrentColumn\PYGZdq{}}\PYG{o}{:} \PYG{l+s+s2}{\PYGZdq{}curr\PYGZus{}pA\PYGZdq{}}
\PYG{p}{\PYGZcb{}}
\end{Verbatim}
\end{quote}

\item[{Parameters}] \leavevmode\begin{description}
\item[{In addition to {\hyperref[api\string-doc/mosaic.meta:mosaic.metaTrajIO.metaTrajIO]{\emph{\code{metaTrajIO}}}} args,}] \leavevmode\begin{itemize}
\item {} 
\emph{AmplifierScale} :            Full scale of amplifier (pA/2\textasciicircum{}nbits) that varies with the gain (default: 1.0).

\item {} 
\emph{AmplifierOffset} :           Current offset in the recorded data in pA  (default: 0.0).

\item {} 
\emph{SamplingFrequency} : Sampling rate of data in the file in Hz.

\item {} 
\emph{HeaderOffset} :              Ignore first \emph{n} bytes of the file for header (default: 0 bytes).

\item {} 
\emph{ColumnTypes} :       A list of tuples with column names and types (see \href{http://docs.scipy.org/doc/numpy/user/basics.types.html}{Numpy types}). Note only integer and floating point numbers are supported.

\item {} 
\emph{IonicCurrentColumn} : Column name that holds ionic current data.

\end{itemize}

\end{description}

\item[{Returns}] \leavevmode
None

\item[{Errors}] \leavevmode
None

\end{description}\end{quote}
\end{quote}


\section{Optimizing Settings}
\label{doc/settingsFile:algorithm-settings-sec}\label{doc/settingsFile:optimizing-settings}
\emph{MOSAIC} classes are controlled through the \href{http://json.org/}{JSON} settings files as defined above. In most cases, running \emph{MOSAIC} through the GUI (see {\hyperref[doc/GraphicalInterface:gui\string-page]{\emph{MOSAIC GUI}}}) should generate satisfactory results. However, settings can be further optimized either by editing a file named \code{.settings} stored within the data directory, or by clicking on the \code{Advanced Settings} check-box in the {\hyperref[doc/GraphicalInterface:analysis\string-setup]{\emph{Panel A: Analysis Setup}}} section of the GUI.


\subsection{Initial Event Detection (\texttt{eventSegment})}
\label{doc/settingsFile:initial-event-detection-eventsegment}\label{doc/settingsFile:eventseg-settings-sec}
The first step when analyzing an ionic-current time series is to perform a quick partition to identify events. This is accomplished by overriding the \code{eventPartition} class. Currently, the only implementation of event partitioning is the \code{eventSegment} algorithm. This algorithm uses a thresholding technique to detect the start and end of an event. When an event is detected the ionic current time-series associated with that event is passed to a processing algorithm for fitting. Settings that can be passed to \code{eventSegment} are given below followed by their descriptions.

\begin{Verbatim}[commandchars=\\\{\}]
\PYG{l+s+s2}{\PYGZdq{}eventSegment\PYGZdq{}} \PYG{o}{:} \PYG{p}{\PYGZob{}}
        \PYG{l+s+s2}{\PYGZdq{}blockSizeSec\PYGZdq{}}                  \PYG{o}{:} \PYG{l+s+s2}{\PYGZdq{}0.5\PYGZdq{}}\PYG{p}{,}
        \PYG{l+s+s2}{\PYGZdq{}eventPad\PYGZdq{}}                      \PYG{o}{:} \PYG{l+s+s2}{\PYGZdq{}50\PYGZdq{}}\PYG{p}{,}
        \PYG{l+s+s2}{\PYGZdq{}minEventLength\PYGZdq{}}                \PYG{o}{:} \PYG{l+s+s2}{\PYGZdq{}5\PYGZdq{}}\PYG{p}{,}
        \PYG{l+s+s2}{\PYGZdq{}eventThreshold\PYGZdq{}}                \PYG{o}{:} \PYG{l+s+s2}{\PYGZdq{}6.0\PYGZdq{}}\PYG{p}{,}
        \PYG{l+s+s2}{\PYGZdq{}driftThreshold\PYGZdq{}}                \PYG{o}{:} \PYG{l+s+s2}{\PYGZdq{}999.0\PYGZdq{}}\PYG{p}{,}
        \PYG{l+s+s2}{\PYGZdq{}maxDriftRate\PYGZdq{}}                  \PYG{o}{:} \PYG{l+s+s2}{\PYGZdq{}999.0\PYGZdq{}}\PYG{p}{,}
        \PYG{l+s+s2}{\PYGZdq{}meanOpenCurr\PYGZdq{}}                  \PYG{o}{:} \PYG{l+s+s2}{\PYGZdq{}\PYGZhy{}1\PYGZdq{}}\PYG{p}{,}
        \PYG{l+s+s2}{\PYGZdq{}sdOpenCurr\PYGZdq{}}                    \PYG{o}{:} \PYG{l+s+s2}{\PYGZdq{}\PYGZhy{}1\PYGZdq{}}\PYG{p}{,}
        \PYG{l+s+s2}{\PYGZdq{}slopeOpenCurr\PYGZdq{}}                 \PYG{o}{:} \PYG{l+s+s2}{\PYGZdq{}\PYGZhy{}1\PYGZdq{}}\PYG{p}{,}
        \PYG{l+s+s2}{\PYGZdq{}writeEventTS\PYGZdq{}}                  \PYG{o}{:} \PYG{l+s+s2}{\PYGZdq{}1\PYGZdq{}}\PYG{p}{,}
        \PYG{l+s+s2}{\PYGZdq{}parallelProc\PYGZdq{}}                  \PYG{o}{:} \PYG{l+s+s2}{\PYGZdq{}0\PYGZdq{}}\PYG{p}{,}
        \PYG{l+s+s2}{\PYGZdq{}reserveNCPU\PYGZdq{}}                   \PYG{o}{:} \PYG{l+s+s2}{\PYGZdq{}2\PYGZdq{}}
\PYG{p}{\PYGZcb{}}
\end{Verbatim}

\begin{tabulary}{\linewidth}{p{4cm}p{12cm}}
\hline
\textsf{\relax 
\textbf{Setting}
} & \textsf{\relax 
\textbf{Description}
}\\
\hline
blockSizeSec

eventPad

minEventLength

eventThreshold

meanOpenCurr

sdOpenCurr

slopeOpenCurr

driftThreshold

maxDriftRate

writeEventTS

parallelProc

reserveNCPU
 & 
Time-series length (in sec) for block operations.

Pad an event with the specified number of points.

Discard events with fewer than the specfied points.

Event detection threshold.

Set the mean open channel current (i0) in pA. -1 computes i0 automatically.

Set the open channel std. dev. in pA. -1 computes SD automatically.

Set the open channel drift in pA/ms. -1 automatically computes the slope.

Aborts the analysis when the open channel drift exceeds the specified value.

Aborts the analysis when the open channel slope exceeds the specified value (pA/ms).

Write the event time-series to the output database.

Enable parallel processing.

Use N-reserveNCPU for parallel processing.
\\
\hline\end{tabulary}



\subsection{Two-State Identification  (\texttt{adept2State})}
\label{doc/settingsFile:two-state-identification-adept2state}\label{doc/settingsFile:stepresp-settings-sec}
Once the time-series is partitioned, individual events are processed by a processing algorithm. For simple event patterns (e.g. homopolymers of DNA, PEG, etc.), one can use the \DUspan{xref,std,std-ref}{stepresponse-page} algorithm. Settings that can be passed to this algorithm are below, followed by their descriptions. For a vast majority of cases, the settings below can be used without modification.

\begin{Verbatim}[commandchars=\\\{\}]
\PYG{l+s+s2}{\PYGZdq{}adept2State\PYGZdq{}} \PYG{o}{:} \PYG{p}{\PYGZob{}}
        \PYG{l+s+s2}{\PYGZdq{}FitTol\PYGZdq{}}                        \PYG{o}{:} \PYG{l+s+s2}{\PYGZdq{}1.e\PYGZhy{}7\PYGZdq{}}\PYG{p}{,}
        \PYG{l+s+s2}{\PYGZdq{}FitIters\PYGZdq{}}                      \PYG{o}{:} \PYG{l+s+s2}{\PYGZdq{}50000\PYGZdq{}}
\PYG{p}{\PYGZcb{}}
\end{Verbatim}


\subsection{Multi-State Identification  (\texttt{adept})}
\label{doc/settingsFile:multi-state-identification-adept}
For more complex signals with multiple states, the \DUspan{xref,std,std-ref}{multistate-page} algorithm yields better results. The settings passed to this algorithm (described below) are largely similar to {\hyperref[doc/settingsFile:stepresp\string-settings\string-sec]{\emph{Two-State Identification  (adept2State)}}}.

\begin{Verbatim}[commandchars=\\\{\}]
\PYG{l+s+s2}{\PYGZdq{}adept\PYGZdq{}} \PYG{o}{:} \PYG{p}{\PYGZob{}}
        \PYG{l+s+s2}{\PYGZdq{}FitTol\PYGZdq{}}                        \PYG{o}{:} \PYG{l+s+s2}{\PYGZdq{}1.e\PYGZhy{}7\PYGZdq{}}\PYG{p}{,}
        \PYG{l+s+s2}{\PYGZdq{}FitIters\PYGZdq{}}                      \PYG{o}{:} \PYG{l+s+s2}{\PYGZdq{}50000\PYGZdq{}}\PYG{p}{,}
        \PYG{l+s+s2}{\PYGZdq{}InitThreshold\PYGZdq{}}                 \PYG{o}{:} \PYG{l+s+s2}{\PYGZdq{}3.0\PYGZdq{}}
\PYG{p}{\PYGZcb{}}
\end{Verbatim}

\begin{notice}{hint}{Hint:}
The parameter \code{InitThreshold} is used for preliminary state identification within multi-state events. As a rule of thumb, this value should be set to roughly half that of \code{eventThreshold} in the {\hyperref[doc/settingsFile:eventseg\string-settings\string-sec]{\emph{Initial Event Detection (eventSegment)}}} section. However, the final value may be adjusted further for optimal results.
\end{notice}


\section{Default Settings}
\label{doc/settingsFile:default-settings}
\begin{Verbatim}[commandchars=\\\{\}]
\PYG{p}{\PYGZob{}}
        \PYG{l+s+s2}{\PYGZdq{}eventSegment\PYGZdq{}} \PYG{o}{:} \PYG{p}{\PYGZob{}}
                \PYG{l+s+s2}{\PYGZdq{}blockSizeSec\PYGZdq{}}                  \PYG{o}{:} \PYG{l+s+s2}{\PYGZdq{}0.5\PYGZdq{}}\PYG{p}{,}
                \PYG{l+s+s2}{\PYGZdq{}eventPad\PYGZdq{}}                              \PYG{o}{:} \PYG{l+s+s2}{\PYGZdq{}50\PYGZdq{}}\PYG{p}{,}
                \PYG{l+s+s2}{\PYGZdq{}minEventLength\PYGZdq{}}                \PYG{o}{:} \PYG{l+s+s2}{\PYGZdq{}5\PYGZdq{}}\PYG{p}{,}
                \PYG{l+s+s2}{\PYGZdq{}eventThreshold\PYGZdq{}}                \PYG{o}{:} \PYG{l+s+s2}{\PYGZdq{}6.0\PYGZdq{}}\PYG{p}{,}
                \PYG{l+s+s2}{\PYGZdq{}driftThreshold\PYGZdq{}}                \PYG{o}{:} \PYG{l+s+s2}{\PYGZdq{}999.0\PYGZdq{}}\PYG{p}{,}
                \PYG{l+s+s2}{\PYGZdq{}maxDriftRate\PYGZdq{}}                  \PYG{o}{:} \PYG{l+s+s2}{\PYGZdq{}999.0\PYGZdq{}}\PYG{p}{,}
                \PYG{l+s+s2}{\PYGZdq{}meanOpenCurr\PYGZdq{}}                  \PYG{o}{:} \PYG{l+s+s2}{\PYGZdq{}\PYGZhy{}1\PYGZdq{}}\PYG{p}{,}
                \PYG{l+s+s2}{\PYGZdq{}sdOpenCurr\PYGZdq{}}                    \PYG{o}{:} \PYG{l+s+s2}{\PYGZdq{}\PYGZhy{}1\PYGZdq{}}\PYG{p}{,}
                \PYG{l+s+s2}{\PYGZdq{}slopeOpenCurr\PYGZdq{}}                 \PYG{o}{:} \PYG{l+s+s2}{\PYGZdq{}\PYGZhy{}1\PYGZdq{}}\PYG{p}{,}
                \PYG{l+s+s2}{\PYGZdq{}writeEventTS\PYGZdq{}}                  \PYG{o}{:} \PYG{l+s+s2}{\PYGZdq{}1\PYGZdq{}}\PYG{p}{,}
                \PYG{l+s+s2}{\PYGZdq{}parallelProc\PYGZdq{}}                  \PYG{o}{:} \PYG{l+s+s2}{\PYGZdq{}0\PYGZdq{}}\PYG{p}{,}
                \PYG{l+s+s2}{\PYGZdq{}reserveNCPU\PYGZdq{}}                   \PYG{o}{:} \PYG{l+s+s2}{\PYGZdq{}2\PYGZdq{}}
        \PYG{p}{\PYGZcb{}}\PYG{p}{,}
        \PYG{l+s+s2}{\PYGZdq{}singleStepEvent\PYGZdq{}} \PYG{o}{:} \PYG{p}{\PYGZob{}}
                \PYG{l+s+s2}{\PYGZdq{}binSize\PYGZdq{}}                               \PYG{o}{:} \PYG{l+s+s2}{\PYGZdq{}1.0\PYGZdq{}}\PYG{p}{,}
                \PYG{l+s+s2}{\PYGZdq{}histPad\PYGZdq{}}                               \PYG{o}{:} \PYG{l+s+s2}{\PYGZdq{}10\PYGZdq{}}\PYG{p}{,}
                \PYG{l+s+s2}{\PYGZdq{}maxFitIters\PYGZdq{}}                   \PYG{o}{:} \PYG{l+s+s2}{\PYGZdq{}5000\PYGZdq{}}\PYG{p}{,}
                \PYG{l+s+s2}{\PYGZdq{}a12Ratio\PYGZdq{}}                              \PYG{o}{:} \PYG{l+s+s2}{\PYGZdq{}1.e4\PYGZdq{}}\PYG{p}{,}
                \PYG{l+s+s2}{\PYGZdq{}minEvntTime\PYGZdq{}}                   \PYG{o}{:} \PYG{l+s+s2}{\PYGZdq{}10.e\PYGZhy{}6\PYGZdq{}}\PYG{p}{,}
                \PYG{l+s+s2}{\PYGZdq{}minDataPad\PYGZdq{}}                    \PYG{o}{:} \PYG{l+s+s2}{\PYGZdq{}75\PYGZdq{}}
        \PYG{p}{\PYGZcb{}}\PYG{p}{,}
        \PYG{l+s+s2}{\PYGZdq{}adept2State\PYGZdq{}} \PYG{o}{:} \PYG{p}{\PYGZob{}}
                \PYG{l+s+s2}{\PYGZdq{}FitTol\PYGZdq{}}                                \PYG{o}{:} \PYG{l+s+s2}{\PYGZdq{}1.e\PYGZhy{}7\PYGZdq{}}\PYG{p}{,}
                \PYG{l+s+s2}{\PYGZdq{}FitIters\PYGZdq{}}                              \PYG{o}{:} \PYG{l+s+s2}{\PYGZdq{}50000\PYGZdq{}}
        \PYG{p}{\PYGZcb{}}\PYG{p}{,}
        \PYG{l+s+s2}{\PYGZdq{}adept\PYGZdq{}} \PYG{o}{:} \PYG{p}{\PYGZob{}}
                \PYG{l+s+s2}{\PYGZdq{}FitTol\PYGZdq{}}                \PYG{o}{:} \PYG{l+s+s2}{\PYGZdq{}1.e\PYGZhy{}7\PYGZdq{}}\PYG{p}{,}
                \PYG{l+s+s2}{\PYGZdq{}FitIters\PYGZdq{}}              \PYG{o}{:} \PYG{l+s+s2}{\PYGZdq{}50000\PYGZdq{}}\PYG{p}{,}
                \PYG{l+s+s2}{\PYGZdq{}InitThreshold\PYGZdq{}} \PYG{o}{:} \PYG{l+s+s2}{\PYGZdq{}3.0\PYGZdq{}}
     \PYG{p}{\PYGZcb{}}\PYG{p}{,}
     \PYG{l+s+s2}{\PYGZdq{}cusumPlus\PYGZdq{}}\PYG{o}{:} \PYG{p}{\PYGZob{}}
                        \PYG{l+s+s2}{\PYGZdq{}StepSize\PYGZdq{}}              \PYG{o}{:} \PYG{l+m+mf}{3.0}\PYG{p}{,}
                        \PYG{l+s+s2}{\PYGZdq{}Threshold\PYGZdq{}}             \PYG{o}{:} \PYG{l+m+mf}{3.0}
\PYG{p}{\PYGZcb{}}\PYG{p}{,}
        \PYG{l+s+s2}{\PYGZdq{}besselLowpassFilter\PYGZdq{}} \PYG{o}{:} \PYG{p}{\PYGZob{}}
                \PYG{l+s+s2}{\PYGZdq{}filterOrder\PYGZdq{}}                   \PYG{o}{:} \PYG{l+s+s2}{\PYGZdq{}6\PYGZdq{}}\PYG{p}{,}
                \PYG{l+s+s2}{\PYGZdq{}filterCutoff\PYGZdq{}}                  \PYG{o}{:} \PYG{l+s+s2}{\PYGZdq{}10000\PYGZdq{}}\PYG{p}{,}
                \PYG{l+s+s2}{\PYGZdq{}decimate\PYGZdq{}}                              \PYG{o}{:} \PYG{l+s+s2}{\PYGZdq{}1\PYGZdq{}}
        \PYG{p}{\PYGZcb{}}\PYG{p}{,}
        \PYG{l+s+s2}{\PYGZdq{}waveletDenoiseFilter\PYGZdq{}} \PYG{o}{:} \PYG{p}{\PYGZob{}}
                \PYG{l+s+s2}{\PYGZdq{}wavelet\PYGZdq{}}                               \PYG{o}{:} \PYG{l+s+s2}{\PYGZdq{}sym5\PYGZdq{}}\PYG{p}{,}
                \PYG{l+s+s2}{\PYGZdq{}level\PYGZdq{}}                                 \PYG{o}{:} \PYG{l+s+s2}{\PYGZdq{}5\PYGZdq{}}\PYG{p}{,}
                \PYG{l+s+s2}{\PYGZdq{}thresholdType\PYGZdq{}}                 \PYG{o}{:} \PYG{l+s+s2}{\PYGZdq{}soft\PYGZdq{}}\PYG{p}{,}
                \PYG{l+s+s2}{\PYGZdq{}thresholdSubType\PYGZdq{}}              \PYG{o}{:} \PYG{l+s+s2}{\PYGZdq{}sqtwolog\PYGZdq{}}
        \PYG{p}{\PYGZcb{}}\PYG{p}{,}
        \PYG{l+s+s2}{\PYGZdq{}abfTrajIO\PYGZdq{}} \PYG{o}{:} \PYG{p}{\PYGZob{}}
                \PYG{l+s+s2}{\PYGZdq{}filter\PYGZdq{}}                                \PYG{o}{:} \PYG{l+s+s2}{\PYGZdq{}*.abf\PYGZdq{}}\PYG{p}{,}
                \PYG{l+s+s2}{\PYGZdq{}start\PYGZdq{}}                                 \PYG{o}{:} \PYG{l+m+mf}{0.0}\PYG{p}{,}
                \PYG{l+s+s2}{\PYGZdq{}dcOffset\PYGZdq{}}                              \PYG{o}{:} \PYG{l+m+mf}{0.0}
        \PYG{p}{\PYGZcb{}}\PYG{p}{,}
        \PYG{l+s+s2}{\PYGZdq{}qdfTrajIO\PYGZdq{}}\PYG{o}{:} \PYG{p}{\PYGZob{}}
                \PYG{l+s+s2}{\PYGZdq{}Rfb\PYGZdq{}}\PYG{o}{:} \PYG{l+m+mf}{9.1}\PYG{n+nx}{e}\PYG{o}{+}\PYG{l+m+mi}{12}\PYG{p}{,}
                \PYG{l+s+s2}{\PYGZdq{}Cfb\PYGZdq{}}\PYG{o}{:} \PYG{l+m+mf}{1.07}\PYG{n+nx}{e}\PYG{o}{\PYGZhy{}}\PYG{l+m+mi}{12}\PYG{p}{,}
                \PYG{l+s+s2}{\PYGZdq{}dcOffset\PYGZdq{}}\PYG{o}{:} \PYG{l+m+mf}{0.0}\PYG{p}{,}
                \PYG{l+s+s2}{\PYGZdq{}filter\PYGZdq{}}\PYG{o}{:} \PYG{l+s+s2}{\PYGZdq{}*.qdf\PYGZdq{}}\PYG{p}{,}
                \PYG{l+s+s2}{\PYGZdq{}start\PYGZdq{}}\PYG{o}{:} \PYG{l+m+mf}{0.0}
        \PYG{p}{\PYGZcb{}}\PYG{p}{,}
        \PYG{l+s+s2}{\PYGZdq{}binTrajIO\PYGZdq{}}\PYG{o}{:} \PYG{p}{\PYGZob{}}
                \PYG{l+s+s2}{\PYGZdq{}AmplifierScale\PYGZdq{}}\PYG{o}{:} \PYG{l+s+s2}{\PYGZdq{}1.0\PYGZdq{}}\PYG{p}{,}
                \PYG{l+s+s2}{\PYGZdq{}AmplifierOffset\PYGZdq{}}\PYG{o}{:} \PYG{l+s+s2}{\PYGZdq{}0.0\PYGZdq{}}\PYG{p}{,}
                \PYG{l+s+s2}{\PYGZdq{}SamplingFrequency\PYGZdq{}}\PYG{o}{:} \PYG{l+s+s2}{\PYGZdq{}50000\PYGZdq{}}\PYG{p}{,}
                \PYG{l+s+s2}{\PYGZdq{}HeaderOffset\PYGZdq{}}\PYG{o}{:} \PYG{l+s+s2}{\PYGZdq{}0\PYGZdq{}}\PYG{p}{,}
                \PYG{l+s+s2}{\PYGZdq{}ColumnTypes\PYGZdq{}}\PYG{o}{:} \PYG{l+s+s2}{\PYGZdq{}[(\PYGZsq{}curr\PYGZus{}pA\PYGZsq{}, \PYGZsq{}float64\PYGZsq{})]\PYGZdq{}}\PYG{p}{,}
                \PYG{l+s+s2}{\PYGZdq{}IonicCurrentColumn\PYGZdq{}} \PYG{o}{:} \PYG{l+s+s2}{\PYGZdq{}curr\PYGZus{}pA\PYGZdq{}}\PYG{p}{,}
                \PYG{l+s+s2}{\PYGZdq{}dcOffset\PYGZdq{}}\PYG{o}{:} \PYG{l+s+s2}{\PYGZdq{}0.0\PYGZdq{}}\PYG{p}{,}
                \PYG{l+s+s2}{\PYGZdq{}filter\PYGZdq{}}\PYG{o}{:} \PYG{l+s+s2}{\PYGZdq{}*.bin\PYGZdq{}}\PYG{p}{,}
                \PYG{l+s+s2}{\PYGZdq{}start\PYGZdq{}}\PYG{o}{:} \PYG{l+s+s2}{\PYGZdq{}0.0\PYGZdq{}}
        \PYG{p}{\PYGZcb{}}
\PYG{p}{\PYGZcb{}}
\end{Verbatim}


\chapter{Database Structure and Query Syntax}
\label{doc/DatabaseStructure:database-page}\label{doc/DatabaseStructure:database-structure-and-query-syntax}\label{doc/DatabaseStructure::doc}\label{doc/DatabaseStructure:mosaic-page-on-github}
\emph{MOSAIC} stores the output of an analysis in a \href{http://www.sqlite.org/}{SQLite} database. Database files are stored in the same directory as the data being processed. Each analysis creates a new database file named \emph{eventMD-\textless{}date\textgreater{}-\textless{}time\textgreater{}.sqlite}, where \emph{\textless{}date\textgreater{}} is the date the analysis was performed (e.g. 20140929 for Sep 29, 2014) and \emph{\textless{}time\textgreater{}} is the analysis start time (e.g. 112937 for 11:29:37 AM).

\href{http://www.sqlite.org/}{SQLite} databases store data in tables similar to spreadsheets, where each table is analogous to a sheet in an Excel spreadsheet. Databased generated by \emph{MOSAIC} can be inspected using a database viewer, for example the open source \href{http://sqlitebrowser.org/}{DB browser for SQLite}.  \emph{MOSAIC} outputs databases with multiple tables as seen from the figure below. Databases output by \emph{MOSAIC} contain four tables: i) \emph{analysisinfo} contains general information about the anlysis such as the data path, analysis algorithm etc., ii) \emph{analysissettings} contains a \href{http://json.org/}{JSON} formatted string with the analysis settings, iii) \emph{metadata} holds the output of the analysis, and iv) \emph{metadata\_t} lists the data types for each column in \emph{metadata}. Two tables most relevant to the analysis (\emph{metadata} and \emph{analysissettings}) are discussed in detail below.
\begin{figure}[htbp]
\centering

\includegraphics[width=0.750\linewidth]{{sqlstructFig1}.png}
\end{figure}


\section{Metadata Table}
\label{doc/DatabaseStructure:metadata-table}\label{doc/DatabaseStructure:metadata-table-sec}
The \emph{metadata} table contains the primary output of the analysis. \emph{MOSAIC} processes individual blockade events from a time-series of ionic current. The parameters describing each event (or metadata) are stored in individual rows of the \emph{metadata} table in the database file. The column names describe the metadata and are unique to the processing algorithm used. For example, the column names for the \DUspan{xref,std,std-ref}{stepresponse-page} algorithm are shown below. The column names for \DUspan{xref,std,std-ref}{multistate-page} differ from this list.

\begin{Verbatim}[commandchars=\\\{\}]
\PYG{p}{\PYGZob{}}
    \PYG{n+nx}{ProcessingStatus}\PYG{p}{,}
    \PYG{n+nx}{OpenChCurrent}\PYG{p}{,}
    \PYG{n+nx}{BlockedCurrent}\PYG{p}{,}
    \PYG{n+nx}{EventStart}\PYG{p}{,}
    \PYG{n+nx}{EventEnd}\PYG{p}{,}
    \PYG{n+nx}{BlockDepth}\PYG{p}{,}
    \PYG{n+nx}{ResTime}\PYG{p}{,}
    \PYG{n+nx}{RiseTime}\PYG{p}{,}
    \PYG{n+nx}{AbsEventStart}\PYG{p}{,}
    \PYG{n+nx}{RedChiSq}\PYG{p}{,}
    \PYG{n+nx}{TimeSeries}
\PYG{p}{\PYGZcb{}}
\end{Verbatim}

Note that the column names can be used in constructing queries passed to SQLite, and is described in more detail in the \DUspan{xref,std,std-ref}{working-with-sqlite-sec} section and the {\hyperref[doc/ScriptingandAdvancedFeatures:scripting\string-page]{\emph{Scripting and Advanced Features}}} section. The first example SQL query below returns the \emph{BlockDepth} column (ratio of \emph{BlockedCurrent} to \emph{OpenChCurrent}). One can imagine assembling more complex queries for example restricting the results to events whose residence time is greater than 0.2 ms as seen from the second example query below.

\begin{Verbatim}[commandchars=\\\{\}]
\PYG{k}{select} \PYG{n}{BlockDepth} \PYG{k}{from} \PYG{n}{metadata} \PYG{k}{where} \PYG{n}{ProcessingStatus}\PYG{o}{=}\PYG{l+s+s1}{\PYGZsq{}normal\PYGZsq{}}

\PYG{k}{select} \PYG{n}{BlockDepth} \PYG{k}{from} \PYG{n}{metadata} \PYG{k}{where} \PYG{n}{ProcessingStatus}\PYG{o}{=}\PYG{l+s+s1}{\PYGZsq{}normal\PYGZsq{}} \PYG{k}{and} \PYG{n}{ResTime} \PYG{o}{\PYGZgt{}} \PYG{l+m+mi}{0}\PYG{p}{.}\PYG{l+m+mi}{2}
\end{Verbatim}

A typical \emph{metadata} table for the \DUspan{xref,std,std-ref}{stepresponse-page} algorithm is shown below. The \emph{ProcessingStatus} column is a text field that should read \emph{normal} if the fit for a particular event was successful. If a failure occurred the corresponding error code (e.g. \emph{eInvalidFitParameters}) is stored and all other columns (except \emph{TimeSeries}) are set to -1. If event time-series storage was requested, then the \emph{TimeSeries} column will store the ionic current data for that entry in binary format.
\begin{figure}[htbp]
\centering

\includegraphics[width=0.750\linewidth]{{sqlstructFig2}.png}
\end{figure}


\section{Analysis Settings Table}
\label{doc/DatabaseStructure:analysis-settings-table}\label{doc/DatabaseStructure:settings-table-sec}
The \emph{analysissettings} table contains a single text entry that stores the settings file for the analysis. This allows any database opened with the \emph{MOSAIC} GUI to retrieve settings that correspond to the analysis results in the file. As seen from the figure below, the settings file is in the \href{http://json.org/}{JSON} format as described in the {\hyperref[doc/settingsFile:settings\string-page]{\emph{Settings File}}} documentation.
\begin{figure}[htbp]
\centering

\includegraphics[width=0.750\linewidth]{{sqlstructFig3}.png}
\end{figure}


\section{Work with SQLite}
\label{doc/DatabaseStructure:work-with-sqlite}\label{doc/DatabaseStructure:working-with-sqlite-sec}
\emph{MOSAIC} stores the output of an analysis in a SQLite database as described in the {\hyperref[doc/DatabaseStructure:database\string-page]{\emph{Database Structure and Query Syntax}}} section. Interacting with the data through the \href{http://en.wikipedia.org/wiki/SQL}{Structured Query Language (SQL)} is a flexible approach to further analyze or plot the output. Here we provide a few detailed examples of the common ways in which the output of \emph{MOSAIC} can be queried for further processing. While this section is not a comprehensive SQL tutorial, it provides common use cases to allow you to get started.

One way to retrieve data from a \href{http://www.sqlite.org/}{SQLite} database is to use the \emph{select} command. In its simplest form, a \emph{select} query can return the entire contents of a table using the syntax below. The statement below selects all columns \emph{(select *)} from the table specified by \emph{\textless{}tablename\textgreater{}}.

\begin{Verbatim}[commandchars=\\\{\}]
\PYG{k}{select} \PYG{o}{*} \PYG{k}{from} \PYG{o}{\PYGZlt{}}\PYG{n}{tablename}\PYG{o}{\PYGZgt{}}
\end{Verbatim}

The power of SQL lies in its ability to restrict results to match specific criteria. This is accomplished with the \emph{where} clause described next. SQL queries can be very fast event for large databases. It is often desirable to only include events that were successfully fit in a plot or other analysis. All \DUspan{xref,std,std-ref}{eventprocess-page} algorithms implemented in \emph{MOSAIC} store a \emph{ProcessingStatus} column in the output database. This enables one to easily query events that were successfully processed. This is easily accomplished with the query below, which returns all columns for events that were successfully processed (\emph{ProcessingStatus=normal}).

\begin{Verbatim}[commandchars=\\\{\}]
\PYG{k}{select} \PYG{o}{*} \PYG{k}{from} \PYG{n}{metadata} \PYG{k}{where} \PYG{n}{ProcessingStatus}\PYG{o}{=}\PYG{l+s+s1}{\PYGZsq{}normal\PYGZsq{}}
\end{Verbatim}

It is not always necessary to retrieve every column for events that fit a certain criteria. For example, \DUspan{xref,std,std-ref}{gui-blockdepth-sec} in the GUI displays a histogram of the blockade depths that match a user specified criteria. This is accomplished within the GUI by a query similar to the one shown below. There are two important differences between the query below and previous examples: i) by replacing * with \emph{BlockDepth}, we only retrieve the \emph{BlockDepth} column for events that meet the criteria specified after the \emph{where} clause, and ii) selection criteria specified after where can be compound statements or even nested as seen in the examples below.

\begin{Verbatim}[commandchars=\\\{\}]
\PYG{k}{select} \PYG{n}{BlockDepth} \PYG{k}{from} \PYG{n}{metadata} \PYG{k}{where} \PYG{n}{ProcessingStatus}\PYG{o}{=}\PYG{l+s+s1}{\PYGZsq{}normal\PYGZsq{}} \PYG{k}{and} \PYG{n}{ResTime} \PYG{o}{\PYGZgt{}} \PYG{l+m+mi}{0}\PYG{p}{.}\PYG{l+m+mi}{2}
\end{Verbatim}

\begin{Verbatim}[commandchars=\\\{\}]
\PYG{k}{select} \PYG{n}{BlockDepth} \PYG{k}{from} \PYG{n}{metadata} \PYG{k}{where} \PYG{n}{ProcessingStatus}\PYG{o}{=}\PYG{l+s+s1}{\PYGZsq{}normal\PYGZsq{}} \PYG{k}{and} \PYG{n}{ResTime} \PYG{o}{\PYGZgt{}} \PYG{l+m+mi}{0}\PYG{p}{.}\PYG{l+m+mi}{2}
\PYG{k}{and} \PYG{n}{BlockDepth} \PYG{k}{between} \PYG{l+m+mi}{0}\PYG{p}{.}\PYG{l+m+mi}{1} \PYG{k}{and} \PYG{l+m+mi}{0}\PYG{p}{.}\PYG{l+m+mi}{5}
\end{Verbatim}

Multiple columns can be retrieved from a table by providing a comma separated list of column names after the \emph{select} clause. As in previous cases, only events that meet a specified criteria are returned. The results can be ordered using \emph{order}. In this example we sort the results in ascending order by the \emph{AbsEventStart} column.

\begin{Verbatim}[commandchars=\\\{\}]
\PYG{k}{select} \PYG{n}{BlockDepth}\PYG{p}{,} \PYG{n}{ResTime}\PYG{p}{,} \PYG{n}{AbsEventStart} \PYG{k}{from} \PYG{n}{metadata} \PYG{k}{where} \PYG{n}{ProcessingStatus}\PYG{o}{=}\PYG{l+s+s1}{\PYGZsq{}normal\PYGZsq{}}
\PYG{k}{order} \PYG{k}{by} \PYG{n}{AbsEventStart} \PYG{k}{ASC}
\end{Verbatim}

Finally, SQL allows the number of results returned to be limited using the \emph{limit} clause. In this example, we limit the query results to the first 500 rows that meet our criteria.

\begin{Verbatim}[commandchars=\\\{\}]
\PYG{k}{select} \PYG{n}{AbsEventStart} \PYG{k}{from} \PYG{n}{metadata} \PYG{k}{where} \PYG{n}{ProcessingStatus}\PYG{o}{=}\PYG{l+s+s1}{\PYGZsq{}normal\PYGZsq{}}
\PYG{k}{order} \PYG{k}{by} \PYG{n}{AbsEventStart} \PYG{k}{ASC} \PYG{k}{limit} \PYG{l+m+mi}{500}
\end{Verbatim}


\section{Export to CSV}
\label{doc/DatabaseStructure:export-to-csv}\begin{figure}[htbp]
\centering

\includegraphics[width=0.250\linewidth]{{dbExport}.png}
\end{figure}
\begin{figure}[htbp]
\centering

\includegraphics[width=0.250\linewidth]{{dbExportSQL}.png}
\end{figure}


\chapter{Scripting and Advanced Features}
\label{doc/ScriptingandAdvancedFeatures:scripting-page}\label{doc/ScriptingandAdvancedFeatures::doc}\label{doc/ScriptingandAdvancedFeatures:scripting-and-advanced-features}\label{doc/ScriptingandAdvancedFeatures:mosaic-page-on-github}
The analysis can be run from the command line by setting up a \href{http://www.python.org/}{Python} script. Scripting allows one to build additional analysis tools on top of \emph{MOSAIC}. The first step is to import \emph{MOSAIC} as shown below.

\begin{Verbatim}[commandchars=\\\{\}]
\PYG{k+kn}{import} \PYG{n+nn}{mosaic}
\end{Verbatim}

Alternatively, one can import sub-modules of \emph{MOSAIC} directly into a script to access other parts of the system as shown below.

\begin{Verbatim}[commandchars=\\\{\}]
\PYG{k+kn}{import} \PYG{n+nn}{mosaic.qdfTrajIO} \PYG{k+kn}{as} \PYG{n+nn}{qdf}
\PYG{k+kn}{import} \PYG{n+nn}{mosaic.abfTrajIO} \PYG{k+kn}{as} \PYG{n+nn}{abf}

\PYG{k+kn}{import} \PYG{n+nn}{mosaic.SingleChannelAnalysis}
\PYG{k+kn}{import} \PYG{n+nn}{mosaic.eventSegment} \PYG{k+kn}{as} \PYG{n+nn}{es}
\PYG{k+kn}{import} \PYG{n+nn}{mosaic.adept2State}
\PYG{k+kn}{import} \PYG{n+nn}{mosaic.besselLowpassFilter} \PYG{k+kn}{as} \PYG{n+nn}{bessel}
\end{Verbatim}


\section{Import Data and Run an Analysis}
\label{doc/ScriptingandAdvancedFeatures:import-data-and-run-an-analysis}
Once the required modules are imported, a basic analysis can be run with the code snippet below. The top-level object that is used to configure and run a new analysis is {\hyperref[api\string-doc/mosaic:mosaic.SingleChannelAnalysis.SingleChannelAnalysis]{\emph{\code{SingleChannelAnalysis}}}}, which takes five arguments: i) the path to the data directory, ii) a handle to a \emph{TrajIO} object that reads in data (e.g. {\hyperref[api\string-doc/mosaic.traj:mosaic.abfTrajIO.abfTrajIO]{\emph{\code{abfTrajIO}}}}), iii) a handle to a data filtering algorithm (e.g. {\hyperref[api\string-doc/mosaic.filter:mosaic.besselLowpassFilter.besselLowpassFilter]{\emph{\code{besselLowpassFilter}}}} or \emph{None} for no filtering), iv) a handle to a partitioning algorithm (e.g. \code{eventSegment}) that partitions the data and v) a handle to a processing algorithm (e.g. {\hyperref[api\string-doc/mosaic.processing:mosaic.adept2State.adept2State]{\emph{\code{adept2State}}}}) that processes individual blockade events.

\begin{Verbatim}[commandchars=\\\{\}]
\PYG{c+c1}{\PYGZsh{} Process all ABF files in a directory}
\PYG{n}{analysisObj}\PYG{o}{=}\PYG{n}{mosaic}\PYG{o}{.}\PYG{n}{SingleChannelAnalysis}\PYG{o}{.}\PYG{n}{SingleChannelAnalysis}\PYG{p}{(}
            \PYG{l+s+s1}{\PYGZsq{}}\PYG{l+s+s1}{\PYGZti{}/ReferenceData/abfSet1}\PYG{l+s+s1}{\PYGZsq{}}\PYG{p}{,}
            \PYG{n}{abf}\PYG{o}{.}\PYG{n}{abfTrajIO}\PYG{p}{,}
            \PYG{n+nb+bp}{None}\PYG{p}{,}
            \PYG{n}{es}\PYG{o}{.}\PYG{n}{eventSegment}\PYG{p}{,}
            \PYG{n}{moasaic}\PYG{o}{.}\PYG{n}{adept2State}\PYG{o}{.}\PYG{n}{adept2State}
        \PYG{p}{)}
\end{Verbatim}

The analysis is started by calling the \emph{Run()} function.

\begin{Verbatim}[commandchars=\\\{\}]
\PYG{n}{analysisObj}\PYG{o}{.}\PYG{n}{Run}\PYG{p}{(}\PYG{p}{)}
\end{Verbatim}

The code listing above analyzes all ABF files in the specified directory. Handles to trajectory I/O, data filtering, event partitioning and event processing are controlled with their corresponding sections in the {\hyperref[doc/settingsFile:settings\string-page]{\emph{Settings File}}}. Default settings used to read ABF files are shown below.

\begin{Verbatim}[commandchars=\\\{\}]
\PYG{l+s+s2}{\PYGZdq{}abfTrajIO\PYGZdq{}} \PYG{o}{:} \PYG{p}{\PYGZob{}}
    \PYG{l+s+s2}{\PYGZdq{}filter\PYGZdq{}}            \PYG{o}{:} \PYG{l+s+s2}{\PYGZdq{}*.abf\PYGZdq{}}\PYG{p}{,}
    \PYG{l+s+s2}{\PYGZdq{}start\PYGZdq{}}             \PYG{o}{:} \PYG{l+m+mf}{0.0}\PYG{p}{,}
    \PYG{l+s+s2}{\PYGZdq{}dcOffset\PYGZdq{}}          \PYG{o}{:} \PYG{l+m+mf}{0.0}
\PYG{p}{\PYGZcb{}}
\end{Verbatim}

\emph{MOSAIC} also supports the QUB QDF file format used by the \href{http://electronicbio.com}{Electronic Biosciences} Nanopatch system. This is accomplished by replacing {\hyperref[api\string-doc/mosaic.traj:mosaic.abfTrajIO.abfTrajIO]{\emph{\code{abfTrajIO}}}} in the previous example with {\hyperref[api\string-doc/mosaic.traj:mosaic.qdfTrajIO.qdfTrajIO]{\emph{\code{qdfTrajIO}}}}.  Settings for QDF files require two additional parameters to be specified in the settings file, the feedback resistance (Rfb) in Ohms and capacitance (Cfb) in Farads as described in the {\hyperref[doc/apidocs:api\string-docs\string-page]{\emph{API Documentation}}}. A sample section of the settings file to read QDF files, followed by Python code required to run an anlysis, is shown below.

\begin{Verbatim}[commandchars=\\\{\}]
\PYG{l+s+s2}{\PYGZdq{}qdfTrajIO\PYGZdq{}}\PYG{o}{:} \PYG{p}{\PYGZob{}}
        \PYG{l+s+s2}{\PYGZdq{}Rfb\PYGZdq{}}               \PYG{o}{:} \PYG{l+m+mf}{9.1}\PYG{n+nx}{e}\PYG{o}{+}\PYG{l+m+mi}{12}\PYG{p}{,}
        \PYG{l+s+s2}{\PYGZdq{}Cfb\PYGZdq{}}               \PYG{o}{:} \PYG{l+m+mf}{1.07}\PYG{n+nx}{e}\PYG{o}{\PYGZhy{}}\PYG{l+m+mi}{12}\PYG{p}{,}
        \PYG{l+s+s2}{\PYGZdq{}dcOffset\PYGZdq{}}          \PYG{o}{:} \PYG{l+m+mf}{0.0}\PYG{p}{,}
        \PYG{l+s+s2}{\PYGZdq{}filter\PYGZdq{}}            \PYG{o}{:} \PYG{l+s+s2}{\PYGZdq{}*.qdf\PYGZdq{}}\PYG{p}{,}
        \PYG{l+s+s2}{\PYGZdq{}start\PYGZdq{}}             \PYG{o}{:} \PYG{l+m+mf}{0.0}
    \PYG{p}{\PYGZcb{}}
\end{Verbatim}

\begin{Verbatim}[commandchars=\\\{\}]
\PYG{c+c1}{\PYGZsh{} Process all QDF files in a directory}
\PYG{n}{mosaic}\PYG{o}{.}\PYG{n}{SingleChannelAnalysis}\PYG{o}{.}\PYG{n}{SingleChannelAnalysis}\PYG{p}{(}
            \PYG{l+s+s1}{\PYGZsq{}}\PYG{l+s+s1}{\PYGZti{}/ReferenceData/qdfSet1}\PYG{l+s+s1}{\PYGZsq{}}\PYG{p}{,}
            \PYG{n}{qdf}\PYG{o}{.}\PYG{n}{qdfTrajIO}\PYG{p}{,}
            \PYG{n+nb+bp}{None}\PYG{p}{,}
            \PYG{n}{es}\PYG{o}{.}\PYG{n}{eventSegment}\PYG{p}{,}
            \PYG{n}{mosaic}\PYG{o}{.}\PYG{n}{adept2State}\PYG{o}{.}\PYG{n}{adept2State}
        \PYG{p}{)}\PYG{o}{.}\PYG{n}{Run}\PYG{p}{(}\PYG{p}{)}
\end{Verbatim}

Upon completion the analysis writes a log file to the directory containing the data. The log file summarizes the conditions under which the analysis were run, the settings used and timing information.

\begin{Verbatim}[commandchars=\\\{\}]
Start time: 2014\PYGZhy{}10\PYGZhy{}05 11:53 AM

[Status]
    Segment trajectory: ***USER STOP***
    Process events: ***NORMAL***


[Summary]
    Baseline open channel conductance:
        Mean    = 136.0 pA
        SD  = 5.5 pA
        Slope   = 0.0 pA/s

    Event segment stats:
        Events detected = 11306

        Open channel drift (max) = 0.0 * SD
        Open channel drift rate (min/max) = (\PYGZhy{}2.77/3.0) pA/s


[Settings]
    Trajectory I/O settings:
        Files processed = 27
        Data path = \PYGZti{}/ReferenceData/qdfSet1
        File format = qdf
        Sampling frequency = 500.0 kHz

        Feedback resistance = 9.1 GOhm
        Feedback capacitance = 1.07 pF

    Event segment settings:
        Window size for block operations = 0.5 s
        Event padding = 50 points
        Min. event rejection length = 5 points
        Event trigger threshold = 2.36363636364 * SD

        Drift error threshold = 999.0 * SD
        Drift rate error threshold = 999.0 pA/s


    Event processing settings:
        Algorithm = adept2State

        Max. iterations  = 50000
        Fit tolerance (rel. err in leastsq)  = 1e\PYGZhy{}07
        Blockade Depth Rejection = 0.9



[Output]
    Output path = \PYGZti{}/ReferenceData/qdfSet1
    Event characterization data = \PYGZti{}/ReferenceData/qdfSet1/eventMD\PYGZhy{}20141005\PYGZhy{}115324.sqlite
    Event time\PYGZhy{}series = ***enabled***
    Log file = eventProcessing.log

[Timing]
    Segment trajectory = 98.03 s
    Process events = 0.0 s

    Total = 98.03 s
    Time per event = 8.67 ms
\end{Verbatim}


\subsection{Filter Data}
\label{doc/ScriptingandAdvancedFeatures:scripting-filter-sec}\label{doc/ScriptingandAdvancedFeatures:filter-data}
\begin{Verbatim}[commandchars=\\\{\}]
\PYG{c+c1}{\PYGZsh{} Filter data with a Bessel filter before processing}
\PYG{n}{mosaic}\PYG{o}{.}\PYG{n}{SingleChannelAnalysis}\PYG{o}{.}\PYG{n}{SingleChannelAnalysis}\PYG{p}{(}
            \PYG{l+s+s1}{\PYGZsq{}}\PYG{l+s+s1}{\PYGZti{}/ReferenceData/abfSet1}\PYG{l+s+s1}{\PYGZsq{}}\PYG{p}{,}
            \PYG{n}{abf}\PYG{o}{.}\PYG{n}{abfTrajIO}\PYG{p}{,}
            \PYG{n}{bessel}\PYG{o}{.}\PYG{n}{besselLowpassFilter}\PYG{p}{,}
            \PYG{n}{es}\PYG{o}{.}\PYG{n}{eventSegment}\PYG{p}{,}
            \PYG{n}{mosaic}\PYG{o}{.}\PYG{n}{adept2State}\PYG{o}{.}\PYG{n}{adept2State}
        \PYG{p}{)}\PYG{o}{.}\PYG{n}{Run}\PYG{p}{(}\PYG{p}{)}
\end{Verbatim}

\emph{MOSAIC} supports filtering data prior to analysis. This is achieved by passing the \emph{dataFilterHnd} argument to the {\hyperref[api\string-doc/mosaic:mosaic.SingleChannelAnalysis.SingleChannelAnalysis]{\emph{\code{SingleChannelAnalysis}}}} object. In the code above, the ABF data is filtered using a {\hyperref[api\string-doc/mosaic.filter:mosaic.besselLowpassFilter.besselLowpassFilter]{\emph{\code{besselLowpassFilter}}}}. Parameters for the filter are defined within the settings file as described in the {\hyperref[doc/settingsFile:settings\string-page]{\emph{Settings File}}} section.

\begin{Verbatim}[commandchars=\\\{\}]
\PYG{l+s+s2}{\PYGZdq{}besselLowpassFilter\PYGZdq{}} \PYG{o}{:} \PYG{p}{\PYGZob{}}
    \PYG{l+s+s2}{\PYGZdq{}filterOrder\PYGZdq{}}    \PYG{o}{:} \PYG{l+s+s2}{\PYGZdq{}6\PYGZdq{}}\PYG{p}{,}
    \PYG{l+s+s2}{\PYGZdq{}filterCutoff\PYGZdq{}}   \PYG{o}{:} \PYG{l+s+s2}{\PYGZdq{}10000\PYGZdq{}}\PYG{p}{,}
    \PYG{l+s+s2}{\PYGZdq{}decimate\PYGZdq{}}       \PYG{o}{:} \PYG{l+s+s2}{\PYGZdq{}1\PYGZdq{}}
\PYG{p}{\PYGZcb{}}
\end{Verbatim}

A similar approach can be used to filter data using a {\hyperref[api\string-doc/mosaic.filter:mosaic.waveletDenoiseFilter.waveletDenoiseFilter]{\emph{\code{waveletDenoiseFilter}}}} or a tap delay line ({\hyperref[api\string-doc/mosaic.filter:mosaic.convolutionFilter.convolutionFilter]{\emph{\code{convolutionFilter}}}}). Additional filters can be easily added to \emph{MOSAIC} as described in {\hyperref[doc/Extend:extend\string-page]{\emph{Extend MOSAIC}}}.


\section{Advanced Scripting}
\label{doc/ScriptingandAdvancedFeatures:advanced-scripting}
Scripting with \href{http://www.python.org/}{Python} allows transforming the output of the \emph{MOSAIC} further to generate plots, perform additional analysis or extend functionality. Moreover, individual components of the \emph{MOSAIC} module, which forms the back end code executed in the data processing pipeline, can be used for specific tasks. In this section, we highlight a few typical use cases.

\textbf{Plot the Ionic Current Time-Series}

\begin{Verbatim}[commandchars=\\\{\}]
\PYG{k+kn}{import} \PYG{n+nn}{mosaic.abfTrajIO} \PYG{k+kn}{as} \PYG{n+nn}{abf}
\PYG{k+kn}{import} \PYG{n+nn}{matplotlib.pyplot} \PYG{k+kn}{as} \PYG{n+nn}{plt}
\PYG{k+kn}{import} \PYG{n+nn}{numpy} \PYG{k+kn}{as} \PYG{n+nn}{np}

\PYG{n}{abfDat}\PYG{o}{=}\PYG{n}{abf}\PYG{o}{.}\PYG{n}{abfTrajIO}\PYG{p}{(}\PYG{n}{dirname}\PYG{o}{=}\PYG{l+s+s1}{\PYGZsq{}}\PYG{l+s+s1}{\PYGZti{}/abfSet1/}\PYG{l+s+s1}{\PYGZsq{}}\PYG{p}{,} \PYG{n+nb}{filter}\PYG{o}{=}\PYG{l+s+s1}{\PYGZsq{}}\PYG{l+s+s1}{*.abf}\PYG{l+s+s1}{\PYGZsq{}}\PYG{p}{)}
\PYG{n}{plt}\PYG{o}{.}\PYG{n}{plot}\PYG{p}{(} \PYG{n}{np}\PYG{o}{.}\PYG{n}{arange}\PYG{p}{(}\PYG{l+m+mi}{0}\PYG{p}{,}\PYG{l+m+mi}{1}\PYG{p}{,}\PYG{l+m+mi}{1}\PYG{o}{/}\PYG{l+m+mf}{500000.}\PYG{p}{)}\PYG{p}{,} \PYG{n}{b}\PYG{o}{.}\PYG{n}{popdata}\PYG{p}{(}\PYG{l+m+mi}{500000}\PYG{p}{)}\PYG{p}{,} \PYG{l+s+s1}{\PYGZsq{}}\PYG{l+s+s1}{b.}\PYG{l+s+s1}{\PYGZsq{}}\PYG{p}{,} \PYG{n}{markersize}\PYG{o}{=}\PYG{l+m+mi}{2} \PYG{p}{)}
\PYG{n}{plt}\PYG{o}{.}\PYG{n}{xlabel}\PYG{p}{(}\PYG{l+s+s2}{\PYGZdq{}}\PYG{l+s+s2}{t (s)}\PYG{l+s+s2}{\PYGZdq{}}\PYG{p}{,} \PYG{n}{fontsize}\PYG{o}{=}\PYG{l+m+mi}{14}\PYG{p}{)}
\PYG{n}{plt}\PYG{o}{.}\PYG{n}{ylabel}\PYG{p}{(}\PYG{l+s+s2}{\PYGZdq{}}\PYG{l+s+s2}{\PYGZhy{}i (pA)}\PYG{l+s+s2}{\PYGZdq{}}\PYG{p}{,} \PYG{n}{fontsize}\PYG{o}{=}\PYG{l+m+mi}{14}\PYG{p}{)}
\PYG{n}{plt}\PYG{o}{.}\PYG{n}{show}\PYG{p}{(}\PYG{p}{)}
\end{Verbatim}

It is useful to visualize time-series data to highlight unique characteristics of a sample. For example the sample code above was used to load 1 second of monodisperse PEG28 data, sampled at 500 kHz. The data was read using a {\hyperref[api\string-doc/mosaic.traj:mosaic.abfTrajIO.abfTrajIO]{\emph{\code{abfTrajIO}}}} object similar to the examples above. The {\hyperref[api\string-doc/mosaic.meta:mosaic.metaTrajIO.metaTrajIO.popdata]{\emph{\code{popdata()}}}} command was used to take 500k data points (or 1 second) and then plot a time-series using \href{http://matplotlib.org/}{matplotlib} (see figure below). Calling {\hyperref[api\string-doc/mosaic.meta:mosaic.metaTrajIO.metaTrajIO.popdata]{\emph{\code{popdata()}}}} again will return the next \emph{n} points.
\begin{figure}[htbp]
\centering

\includegraphics[width=0.500\linewidth]{{advancedFig2}.png}
\end{figure}

We have packaged time-series plotting into an easy to use module {\hyperref[api\string-doc/mosaicscripts:module\string-mosaicscripts.plots.timeseries]{\emph{\code{timeseries}}}}. Run interactive examples in an IPython notebook: {\color{red}\bfseries{}\textbar{}timeseries\textbar{}}

\textbf{Estimate the Channel Gating Duration}

Scripting can be used to obtain statistics from the raw time-series. In the code snippet below, we estimate the amount of time a channel spends in a gated state by combining modules defined within \emph{MOSAIC}. The analysis is performed in blocks for efficiency. We first define a Python function that takes multiple arguments including  \emph{TrajIO} object, the threshold at which we want to define the gated state in pA (gatingcurrentpa), the block size in seconds (blocksz), the total time of the time-series being processed in seconds (totaltime) and the sampling rate of the data in Hz (fshz). The function then calculates the number of blocks in which the channel was in a gated state and returns the time spent in that state in seconds.

\begin{Verbatim}[commandchars=\\\{\}]
\PYG{k+kn}{import} \PYG{n+nn}{mosaic.abfTrajIO} \PYG{k+kn}{as} \PYG{n+nn}{abf}
\PYG{k+kn}{import} \PYG{n+nn}{numpy} \PYG{k+kn}{as} \PYG{n+nn}{np}

\PYG{k}{def} \PYG{n+nf}{estimateGatingDuration}\PYG{p}{(} \PYG{n}{trajioobj}\PYG{p}{,} \PYG{n}{gatingcurrentpa}\PYG{p}{,} \PYG{n}{blocksz}\PYG{p}{,} \PYG{n}{totaltime}\PYG{p}{,} \PYG{n}{fshz} \PYG{p}{)}\PYG{p}{:}
    \PYG{n}{npts} \PYG{o}{=} \PYG{n+nb}{int}\PYG{p}{(}\PYG{p}{(}\PYG{n}{fshz}\PYG{p}{)}\PYG{o}{*}\PYG{n}{blocksz}\PYG{p}{)}
    \PYG{n}{nblk} \PYG{o}{=} \PYG{n+nb}{int}\PYG{p}{(}\PYG{n}{totaltime}\PYG{o}{/}\PYG{n}{blocksz}\PYG{p}{)}\PYG{o}{\PYGZhy{}}\PYG{l+m+mi}{1}

    \PYG{c+c1}{\PYGZsh{} Iterate over the blocks of data and check if the channel was in a gated state.}
    \PYG{c+c1}{\PYGZsh{} The code below returns the mean ionic current of blocks that are below the gating}
    \PYG{c+c1}{\PYGZsh{} threshold (gatingcurrentpa)}
    \PYG{n}{gEvents} \PYG{o}{=} \PYG{n+nb}{filter}\PYG{p}{(}  \PYG{k}{lambda} \PYG{n}{x}\PYG{p}{:}\PYG{n}{x}\PYG{o}{\PYGZlt{}}\PYG{n+nb}{float}\PYG{p}{(}\PYG{n}{gatingcurrentpa}\PYG{p}{)}\PYG{p}{,}
                       \PYG{p}{[} \PYG{n}{np}\PYG{o}{.}\PYG{n}{mean}\PYG{p}{(}\PYG{n}{trajioobj}\PYG{o}{.}\PYG{n}{popdata}\PYG{p}{(}\PYG{n}{npts}\PYG{p}{)}\PYG{p}{)} \PYG{k}{for} \PYG{n}{i} \PYG{o+ow}{in} \PYG{n+nb}{range}\PYG{p}{(}\PYG{n}{nblk}\PYG{p}{)} \PYG{p}{]}\PYG{p}{)}

    \PYG{k}{return} \PYG{n+nb}{len}\PYG{p}{(}\PYG{n}{gEvents}\PYG{p}{)}\PYG{o}{*}\PYG{n}{blocksz}

\PYG{n}{abfObj}\PYG{o}{=}\PYG{n}{abf}\PYG{o}{.}\PYG{n}{abfTrajIO}\PYG{p}{(}\PYG{n}{dirname}\PYG{o}{=}\PYG{l+s+s1}{\PYGZsq{}}\PYG{l+s+s1}{\PYGZti{}/abfSet1}\PYG{l+s+s1}{\PYGZsq{}}\PYG{p}{,}\PYG{n+nb}{filter}\PYG{o}{=}\PYG{l+s+s1}{\PYGZsq{}}\PYG{l+s+s1}{*.abf}\PYG{l+s+s1}{\PYGZsq{}}\PYG{p}{)}
\PYG{k}{print} \PYG{n}{estimateGatingDuration}\PYG{p}{(} \PYG{n}{abfObj}\PYG{p}{,} \PYG{l+m+mf}{20.}\PYG{p}{,} \PYG{l+m+mf}{0.25}\PYG{p}{,} \PYG{l+m+mi}{100}\PYG{p}{,} \PYG{n}{abfObj}\PYG{o}{.}\PYG{n}{FsHz} \PYG{p}{)}
\end{Verbatim}

\textbf{Plot the Output of an Analysis}

This final example shows how one can use \emph{MOSAIC} to process an ionic current time-series and then build a custom script that further analyses and plots the results. This example uses single-molecule mass spectrometry (SMMS) data {[}Robertson:2007jo{]}, described in more detail in the \DUspan{xref,std,std-ref}{smms-sec} section .

In the code below, we first process all the ABF files in a specified directory similar to the examples in previous sections. Upon completion of the analysis, the results are stored in a \href{http://www.sqlite.org/}{SQLite} database, which can be then queried using the structured query language (\href{http://en.wikipedia.org/wiki/SQL}{SQL}).

\begin{Verbatim}[commandchars=\\\{\}]
\PYG{k+kn}{import} \PYG{n+nn}{mosaic.qdfTrajIO} \PYG{k+kn}{as} \PYG{n+nn}{qdf}
\PYG{k+kn}{import} \PYG{n+nn}{mosaic.abfTrajIO} \PYG{k+kn}{as} \PYG{n+nn}{abf}

\PYG{k+kn}{import} \PYG{n+nn}{mosaic.SingleChannelAnalysis}
\PYG{k+kn}{import} \PYG{n+nn}{mosaic.eventSegment} \PYG{k+kn}{as} \PYG{n+nn}{es}
\PYG{k+kn}{import} \PYG{n+nn}{mosaic.adept2State}

\PYG{k+kn}{import} \PYG{n+nn}{glob}
\PYG{k+kn}{import} \PYG{n+nn}{pylab} \PYG{k+kn}{as} \PYG{n+nn}{pl}
\PYG{k+kn}{import} \PYG{n+nn}{numpy} \PYG{k+kn}{as} \PYG{n+nn}{np}
\PYG{k+kn}{import} \PYG{n+nn}{mosaic.sqlite3MDIO} \PYG{k+kn}{as} \PYG{n+nn}{sql}

\PYG{c+c1}{\PYGZsh{} Process all ABF files in a directory}
\PYG{n}{mosaic}\PYG{o}{.}\PYG{n}{SingleChannelAnalysis}\PYG{o}{.}\PYG{n}{SingleChannelAnalysis}\PYG{p}{(}
            \PYG{l+s+s1}{\PYGZsq{}}\PYG{l+s+s1}{\PYGZti{}/ReferenceData/abfSet1}\PYG{l+s+s1}{\PYGZsq{}}\PYG{p}{,}
            \PYG{n}{abf}\PYG{o}{.}\PYG{n}{abfTrajIO}\PYG{p}{,}
            \PYG{n+nb+bp}{None}\PYG{p}{,}
            \PYG{n}{es}\PYG{o}{.}\PYG{n}{eventSegment}\PYG{p}{,}
            \PYG{n}{mosaic}\PYG{o}{.}\PYG{n}{adept2State}\PYG{o}{.}\PYG{n}{adept2State}
        \PYG{p}{)}\PYG{o}{.}\PYG{n}{Run}\PYG{p}{(}\PYG{p}{)}


\PYG{c+c1}{\PYGZsh{} Load the results of the analysis}
\PYG{n}{s}\PYG{o}{=}\PYG{n}{sql}\PYG{o}{.}\PYG{n}{sqlite3MDIO}\PYG{p}{(}\PYG{p}{)}
\PYG{n}{s}\PYG{o}{.}\PYG{n}{openDB}\PYG{p}{(}\PYG{n}{glob}\PYG{o}{.}\PYG{n}{glob}\PYG{p}{(}\PYG{l+s+s2}{\PYGZdq{}}\PYG{l+s+s2}{\PYGZti{}/ReferenceData/abfSet1/*sqlite}\PYG{l+s+s2}{\PYGZdq{}}\PYG{p}{)}\PYG{p}{[}\PYG{o}{\PYGZhy{}}\PYG{l+m+mi}{1}\PYG{p}{]}\PYG{p}{)}

\PYG{c+c1}{\PYGZsh{} We first set up a string that holds the query to retrieve the analysis results. Note that \PYGZob{}col\PYGZcb{}}
\PYG{c+c1}{\PYGZsh{} will be replaced with the name of the database column when we run the query below.}
\PYG{n}{q} \PYG{o}{=} \PYG{l+s+s2}{\PYGZdq{}}\PYG{l+s+s2}{select \PYGZob{}col\PYGZcb{} from metadata where ProcessingStatus=}\PYG{l+s+s2}{\PYGZsq{}}\PYG{l+s+s2}{normal}\PYG{l+s+s2}{\PYGZsq{}}\PYG{l+s+s2}{ and ResTime \PYGZgt{} 0.2 }\PYG{l+s+se}{\PYGZbs{}}
\PYG{l+s+s2}{     and BlockDepth between 0.15 and 0.55}\PYG{l+s+s2}{\PYGZdq{}}

\PYG{c+c1}{\PYGZsh{} Now we run two separate queries \PYGZhy{} the first returns the blockade depth}
\PYG{c+c1}{\PYGZsh{} and the second returns the residence time. Note that we simply take the query}
\PYG{c+c1}{\PYGZsh{} string \PYGZsq{}q\PYGZsq{} above and replace \PYGZob{}col\PYGZcb{} with the column name.}
\PYG{n}{x}\PYG{o}{=}\PYG{n}{np}\PYG{o}{.}\PYG{n}{hstack}\PYG{p}{(} \PYG{n}{s}\PYG{o}{.}\PYG{n}{queryDB}\PYG{p}{(} \PYG{n}{q}\PYG{o}{.}\PYG{n}{format}\PYG{p}{(}\PYG{n}{col}\PYG{o}{=}\PYG{l+s+s1}{\PYGZsq{}}\PYG{l+s+s1}{BlockDepth}\PYG{l+s+s1}{\PYGZsq{}}\PYG{p}{)} \PYG{p}{)} \PYG{p}{)}
\PYG{n}{y}\PYG{o}{=}\PYG{n}{np}\PYG{o}{.}\PYG{n}{hstack}\PYG{p}{(} \PYG{n}{s}\PYG{o}{.}\PYG{n}{queryDB}\PYG{p}{(} \PYG{n}{q}\PYG{o}{.}\PYG{n}{format}\PYG{p}{(}\PYG{n}{col}\PYG{o}{=}\PYG{l+s+s1}{\PYGZsq{}}\PYG{l+s+s1}{ResTime}\PYG{l+s+s1}{\PYGZsq{}}\PYG{p}{)} \PYG{p}{)} \PYG{p}{)}

\PYG{c+c1}{\PYGZsh{} Use matplotlib to plot the results with 2 views:}
\PYG{c+c1}{\PYGZsh{} i)  a 1D histogram of blockade depths and}
\PYG{c+c1}{\PYGZsh{} ii) a 2D histogram of the residence times vs. blockade depth}
\PYG{n}{fig} \PYG{o}{=} \PYG{n}{pl}\PYG{o}{.}\PYG{n}{gcf}\PYG{p}{(}\PYG{p}{)}
\PYG{n}{fig}\PYG{o}{.}\PYG{n}{canvas}\PYG{o}{.}\PYG{n}{set\PYGZus{}window\PYGZus{}title}\PYG{p}{(}\PYG{l+s+s1}{\PYGZsq{}}\PYG{l+s+s1}{Residence Time vs. Blockade Depth}\PYG{l+s+s1}{\PYGZsq{}}\PYG{p}{)}

\PYG{n}{pl}\PYG{o}{.}\PYG{n}{subplot}\PYG{p}{(}\PYG{l+m+mi}{2}\PYG{p}{,} \PYG{l+m+mi}{1}\PYG{p}{,} \PYG{l+m+mi}{1}\PYG{p}{)}
\PYG{n}{pl}\PYG{o}{.}\PYG{n}{hist}\PYG{p}{(}\PYG{n}{x}\PYG{p}{,} \PYG{n}{bins}\PYG{o}{=}\PYG{l+m+mi}{500}\PYG{p}{,} \PYG{n}{histtype}\PYG{o}{=}\PYG{l+s+s1}{\PYGZsq{}}\PYG{l+s+s1}{step}\PYG{l+s+s1}{\PYGZsq{}}\PYG{p}{,} \PYG{n}{rwidth}\PYG{o}{=}\PYG{l+m+mf}{0.1}\PYG{p}{)}
\PYG{n}{pl}\PYG{o}{.}\PYG{n}{xticks}\PYG{p}{(}\PYG{p}{(}\PYG{p}{)}\PYG{p}{)}
\PYG{n}{pl}\PYG{o}{.}\PYG{n}{ylabel}\PYG{p}{(}\PYG{l+s+s2}{\PYGZdq{}}\PYG{l+s+s2}{Counts}\PYG{l+s+s2}{\PYGZdq{}}\PYG{p}{,} \PYG{n}{fontsize}\PYG{o}{=}\PYG{l+m+mi}{14}\PYG{p}{)}

\PYG{n}{pl}\PYG{o}{.}\PYG{n}{subplot}\PYG{p}{(}\PYG{l+m+mi}{2}\PYG{p}{,} \PYG{l+m+mi}{1}\PYG{p}{,} \PYG{l+m+mi}{2}\PYG{p}{)}
\PYG{n}{pl}\PYG{o}{.}\PYG{n}{hist2d}\PYG{p}{(}\PYG{n}{x}\PYG{p}{,}\PYG{n}{y}\PYG{p}{,} \PYG{n}{bins}\PYG{o}{=}\PYG{l+m+mi}{500}\PYG{p}{)}

\PYG{n}{pl}\PYG{o}{.}\PYG{n}{xlabel}\PYG{p}{(}\PYG{l+s+s2}{\PYGZdq{}}\PYG{l+s+s2}{Blockade Depth}\PYG{l+s+s2}{\PYGZdq{}}\PYG{p}{,} \PYG{n}{fontsize}\PYG{o}{=}\PYG{l+m+mi}{14}\PYG{p}{)}
\PYG{n}{pl}\PYG{o}{.}\PYG{n}{ylabel}\PYG{p}{(}\PYG{l+s+s2}{\PYGZdq{}}\PYG{l+s+s2}{Residence Time (ms)}\PYG{l+s+s2}{\PYGZdq{}}\PYG{p}{,} \PYG{n}{fontsize}\PYG{o}{=}\PYG{l+m+mi}{14}\PYG{p}{)}
\PYG{n}{pl}\PYG{o}{.}\PYG{n}{ylim}\PYG{p}{(}\PYG{p}{[}\PYG{l+m+mf}{0.2}\PYG{p}{,} \PYG{l+m+mi}{20}\PYG{p}{]}\PYG{p}{)}

\PYG{n}{pl}\PYG{o}{.}\PYG{n}{show}\PYG{p}{(}\PYG{p}{)}
\end{Verbatim}

Running the code above generates a two pane plot using \href{http://matplotlib.org/}{matplotlib}. The top pane contains a histogram of the blockade depth, while the bottom pane plots a 2D histogram of residence time vs. blockade depth.
\begin{figure}[htbp]
\centering

\includegraphics[width=0.500\linewidth]{{advancedFig3}.png}
\end{figure}


\chapter{Extend \emph{MOSAIC}}
\label{doc/Extend:extend-projname}\label{doc/Extend:extend-page}\label{doc/Extend::doc}\label{doc/Extend:mosaic-page-on-github}
\emph{MOSAIC} was designed from the start using object oriented tools, which makes it easy to extend. {\hyperref[api\string-doc/mosaic.meta:api\string-metaclass\string-page]{\emph{Meta-Classes}}} define interfaces to five key parts of \emph{MOSAIC}: time-series IO ({\hyperref[api\string-doc/mosaic.meta:mosaic.metaTrajIO.metaTrajIO]{\emph{\code{metaTrajIO}}}}), time-series filtering ({\hyperref[api\string-doc/mosaic.meta:mosaic.metaIOFilter.metaIOFilter]{\emph{\code{metaIOFilter}}}}), analysis output ({\hyperref[api\string-doc/mosaic.meta:mosaic.metaMDIO.metaMDIO]{\emph{\code{metaMDIO}}}}), event partition and segmenting ({\hyperref[api\string-doc/mosaic.meta:mosaic.metaEventPartition.metaEventPartition]{\emph{\code{metaEventPartition}}}}), and event processing ({\hyperref[api\string-doc/mosaic.meta:mosaic.metaEventProcessor.metaEventProcessor]{\emph{\code{metaEventProcessor}}}}). Sub-classing any of these meta classes and implementing their  interface functions allows one to extend \emph{MOSAIC} while maintaining compatibility with other parts of the program. We highlight these capabilities via two examples. In the first example, we show how one can extend {\hyperref[api\string-doc/mosaic.meta:mosaic.metaTrajIO.metaTrajIO]{\emph{\code{metaTrajIO}}}} to read arbitrary binary files. In the second example, we implement a new top-level class that converts files to the comma separated value (CSV) format.


\section{Read Arbitrary Binary Data Files}
\label{doc/Extend:read-arbitrary-binary-data-files}
In this first example, we implement a class that can read an arbitrary binary data file and make its data available via the interface functions in {\hyperref[api\string-doc/mosaic.meta:mosaic.metaTrajIO.metaTrajIO]{\emph{\code{metaTrajIO}}}}. This allows the newly implemented binary data to be used across \emph{MOSAIC}. A complete listing of the code used in this example ({\hyperref[api\string-doc/mosaic.traj:mosaic.binTrajIO.binTrajIO]{\emph{\code{binTrajIO}}}}) is available in the API documentation.

The new binary IO class is implemented by sub-classing {\hyperref[api\string-doc/mosaic.meta:mosaic.metaTrajIO.metaTrajIO]{\emph{\code{metaTrajIO}}}} as shown in the listing below.

\begin{Verbatim}[commandchars=\\\{\}]
\PYG{k}{class} \PYG{n+nc}{binTrajIO}\PYG{p}{(}\PYG{n}{metaTrajIO}\PYG{o}{.}\PYG{n}{metaTrajIO}\PYG{p}{)}\PYG{p}{:}
\end{Verbatim}

Next, we must fully implement the {\hyperref[api\string-doc/mosaic.meta:mosaic.metaTrajIO.metaTrajIO]{\emph{\code{metaTrajIO}}}} interface functions ({\hyperref[api\string-doc/mosaic.meta:mosaic.metaTrajIO.metaTrajIO._init]{\emph{\code{\_init()}}}}, {\hyperref[api\string-doc/mosaic.meta:mosaic.metaTrajIO.metaTrajIO.readdata]{\emph{\code{readdata()}}}} and {\hyperref[api\string-doc/mosaic.meta:mosaic.metaTrajIO.metaTrajIO._formatsettings]{\emph{\code{\_formatsettings()}}}}). Note that the arguments of each function must match their corresponding base-class versions. For example the {\hyperref[api\string-doc/mosaic.meta:mosaic.metaTrajIO.metaTrajIO._init]{\emph{\code{\_init()}}}} function only accepts keyword arguments and is defined as shown below.

\begin{Verbatim}[commandchars=\\\{\}]
\PYG{k}{def} \PYG{n+nf}{\PYGZus{}init}\PYG{p}{(}\PYG{n+nb+bp}{self}\PYG{p}{,} \PYG{o}{*}\PYG{o}{*}\PYG{n}{kwargs}\PYG{p}{)}\PYG{p}{:}
\end{Verbatim}

The {\hyperref[api\string-doc/mosaic.meta:mosaic.metaTrajIO.metaTrajIO._init]{\emph{\code{\_init()}}}} function checks the arguments passed to \emph{kwargs} and raises an exception if they are not defined.

\begin{Verbatim}[commandchars=\\\{\}]
        \PYG{k}{if} \PYG{o+ow}{not} \PYG{n+nb}{hasattr}\PYG{p}{(}\PYG{n+nb+bp}{self}\PYG{p}{,} \PYG{l+s+s1}{\PYGZsq{}}\PYG{l+s+s1}{SamplingFrequency}\PYG{l+s+s1}{\PYGZsq{}}\PYG{p}{)}\PYG{p}{:}
                \PYG{k}{raise} \PYG{n}{metaTrajIO}\PYG{o}{.}\PYG{n}{InsufficientArgumentsError}\PYG{p}{(}\PYG{l+s+s2}{\PYGZdq{}}\PYG{l+s+s2}{\PYGZob{}0\PYGZcb{} requires the sampling rate in Hz to be defined.}\PYG{l+s+s2}{\PYGZdq{}}\PYG{o}{.}\PYG{n}{format}\PYG{p}{(}\PYG{n+nb}{type}\PYG{p}{(}\PYG{n+nb+bp}{self}\PYG{p}{)}\PYG{o}{.}\PYG{n}{\PYGZus{}\PYGZus{}name\PYGZus{}\PYGZus{}}\PYG{p}{)}\PYG{p}{)}
        \PYG{k}{if} \PYG{o+ow}{not} \PYG{n+nb}{hasattr}\PYG{p}{(}\PYG{n+nb+bp}{self}\PYG{p}{,} \PYG{l+s+s1}{\PYGZsq{}}\PYG{l+s+s1}{PythonStructCode}\PYG{l+s+s1}{\PYGZsq{}}\PYG{p}{)}\PYG{p}{:}
                \PYG{k}{raise} \PYG{n}{metaTrajIO}\PYG{o}{.}\PYG{n}{InsufficientArgumentsError}\PYG{p}{(}\PYG{l+s+s2}{\PYGZdq{}}\PYG{l+s+s2}{\PYGZob{}0\PYGZcb{} requires the Python struct code to be defined.}\PYG{l+s+s2}{\PYGZdq{}}\PYG{o}{.}\PYG{n}{format}\PYG{p}{(}\PYG{n+nb}{type}\PYG{p}{(}\PYG{n+nb+bp}{self}\PYG{p}{)}\PYG{o}{.}\PYG{n}{\PYGZus{}\PYGZus{}name\PYGZus{}\PYGZus{}}\PYG{p}{)}\PYG{p}{)}
\end{Verbatim}

Next we define the {\hyperref[api\string-doc/mosaic.meta:mosaic.metaTrajIO.metaTrajIO.readdata]{\emph{\code{readdata()}}}} function that reads in the data and stores the results in a numpy array. This array is then passed back to the calling function.

\begin{Verbatim}[commandchars=\\\{\}]
     \PYG{k}{def} \PYG{n+nf}{readdata}\PYG{p}{(}\PYG{n+nb+bp}{self}\PYG{p}{,} \PYG{n}{fname}\PYG{p}{)}\PYG{p}{:}

             \PYG{n}{tempdata}\PYG{o}{=}\PYG{n}{np}\PYG{o}{.}\PYG{n}{array}\PYG{p}{(}\PYG{p}{[}\PYG{p}{]}\PYG{p}{)}
             \PYG{c+c1}{\PYGZsh{} Read binary data and add it to the data pipe}
             \PYG{k}{for} \PYG{n}{f} \PYG{o+ow}{in} \PYG{n}{fname}\PYG{p}{:}
                     \PYG{n}{tempdata}\PYG{o}{=}\PYG{n}{np}\PYG{o}{.}\PYG{n}{hstack}\PYG{p}{(}\PYG{p}{(} \PYG{n}{tempdata}\PYG{p}{,} \PYG{n+nb+bp}{self}\PYG{o}{.}\PYG{n}{readBinaryFile}\PYG{p}{(}\PYG{n}{f}\PYG{p}{)} \PYG{p}{)}\PYG{p}{)}

             \PYG{k}{return} \PYG{n}{tempdata}
\end{Verbatim}

Finally, we implement the {\hyperref[api\string-doc/mosaic.meta:mosaic.metaTrajIO.metaTrajIO._formatsettings]{\emph{\code{\_formatsettings()}}}} that returns a formatted string of the settings used to read in binary data.

\begin{Verbatim}[commandchars=\\\{\}]
\PYG{k}{def} \PYG{n+nf}{\PYGZus{}formatsettings}\PYG{p}{(}\PYG{n+nb+bp}{self}\PYG{p}{)}\PYG{p}{:}
        \PYG{l+s+sd}{\PYGZdq{}\PYGZdq{}\PYGZdq{}}
\PYG{l+s+sd}{                Return a formatted string of settings for display}
\PYG{l+s+sd}{        \PYGZdq{}\PYGZdq{}\PYGZdq{}}
        \PYG{n}{fmtstr}\PYG{o}{=}\PYG{l+s+s2}{\PYGZdq{}}\PYG{l+s+s2}{\PYGZdq{}}

        \PYG{n}{fmtstr}\PYG{o}{+}\PYG{o}{=}\PYG{l+s+s1}{\PYGZsq{}}\PYG{l+s+se}{\PYGZbs{}n}\PYG{l+s+se}{\PYGZbs{}t}\PYG{l+s+se}{\PYGZbs{}t}\PYG{l+s+s1}{Amplifier scale = \PYGZob{}0\PYGZcb{} pA}\PYG{l+s+se}{\PYGZbs{}n}\PYG{l+s+s1}{\PYGZsq{}}\PYG{o}{.}\PYG{n}{format}\PYG{p}{(}\PYG{n+nb+bp}{self}\PYG{o}{.}\PYG{n}{AmplifierScale}\PYG{p}{)}
        \PYG{n}{fmtstr}\PYG{o}{+}\PYG{o}{=}\PYG{l+s+s1}{\PYGZsq{}}\PYG{l+s+se}{\PYGZbs{}t}\PYG{l+s+se}{\PYGZbs{}t}\PYG{l+s+s1}{Amplifier offset = \PYGZob{}0\PYGZcb{} pA}\PYG{l+s+se}{\PYGZbs{}n}\PYG{l+s+s1}{\PYGZsq{}}\PYG{o}{.}\PYG{n}{format}\PYG{p}{(}\PYG{n+nb+bp}{self}\PYG{o}{.}\PYG{n}{AmplifierOffset}\PYG{p}{)}
        \PYG{n}{fmtstr}\PYG{o}{+}\PYG{o}{=}\PYG{l+s+s1}{\PYGZsq{}}\PYG{l+s+se}{\PYGZbs{}t}\PYG{l+s+se}{\PYGZbs{}t}\PYG{l+s+s1}{Header offset = \PYGZob{}0\PYGZcb{} bytes}\PYG{l+s+se}{\PYGZbs{}n}\PYG{l+s+s1}{\PYGZsq{}}\PYG{o}{.}\PYG{n}{format}\PYG{p}{(}\PYG{n+nb+bp}{self}\PYG{o}{.}\PYG{n}{HeaderOffset}\PYG{p}{)}
        \PYG{n}{fmtstr}\PYG{o}{+}\PYG{o}{=}\PYG{l+s+s1}{\PYGZsq{}}\PYG{l+s+se}{\PYGZbs{}t}\PYG{l+s+se}{\PYGZbs{}t}\PYG{l+s+s1}{Data type code = }\PYG{l+s+se}{\PYGZbs{}\PYGZsq{}}\PYG{l+s+s1}{\PYGZob{}0\PYGZcb{}}\PYG{l+s+se}{\PYGZbs{}\PYGZsq{}}\PYG{l+s+se}{\PYGZbs{}n}\PYG{l+s+s1}{\PYGZsq{}}\PYG{o}{.}\PYG{n}{format}\PYG{p}{(}\PYG{n+nb+bp}{self}\PYG{o}{.}\PYG{n}{PythonStructCode}\PYG{p}{)}

        \PYG{k}{return} \PYG{n}{fmtstr}
\end{Verbatim}

The newly defined {\hyperref[api\string-doc/mosaic.traj:mosaic.binTrajIO.binTrajIO]{\emph{\code{binTrajIO}}}} class can then be used as shown below and in {\hyperref[doc/ScriptingandAdvancedFeatures:scripting\string-page]{\emph{Scripting and Advanced Features}}}.

\begin{Verbatim}[commandchars=\\\{\}]
\PYG{c+c1}{\PYGZsh{} Process all binary files in a directory}
\PYG{n}{mosaic}\PYG{o}{.}\PYG{n}{SingleChannelAnalysis}\PYG{o}{.}\PYG{n}{SingleChannelAnalysis}\PYG{p}{(}
            \PYG{l+s+s2}{\PYGZdq{}}\PYG{l+s+s2}{\PYGZti{}/RefData/binSet1/}\PYG{l+s+s2}{\PYGZdq{}}\PYG{p}{,}
            \PYG{n+nb}{bin}\PYG{o}{.}\PYG{n}{binTrajIO}\PYG{p}{,}
            \PYG{n+nb+bp}{None}\PYG{p}{,}
            \PYG{n}{es}\PYG{o}{.}\PYG{n}{eventSegment}\PYG{p}{,}
            \PYG{n}{mosaic}\PYG{o}{.}\PYG{n}{adept2State}\PYG{o}{.}\PYG{n}{adept2State}
        \PYG{p}{)}\PYG{o}{.}\PYG{n}{Run}\PYG{p}{(}\PYG{p}{)}
\end{Verbatim}

Similar to other \emph{TrajIO} objects, parameters for {\hyperref[api\string-doc/mosaic.traj:mosaic.binTrajIO.binTrajIO]{\emph{\code{binTrajIO}}}} are obtained from the settings file when used with {\hyperref[api\string-doc/mosaic:mosaic.SingleChannelAnalysis.SingleChannelAnalysis]{\emph{\code{SingleChannelAnalysis}}}}. Example settings for {\hyperref[api\string-doc/mosaic.traj:mosaic.binTrajIO.binTrajIO]{\emph{\code{binTrajIO}}}} that read 16-bit intgers from a binary data file, assuming 50 \emph{kHz} sampling, are shown below.

\begin{Verbatim}[commandchars=\\\{\}]
\PYG{l+s+s2}{\PYGZdq{}binTrajIO\PYGZdq{}} \PYG{o}{:} \PYG{p}{\PYGZob{}}
        \PYG{l+s+s2}{\PYGZdq{}filter\PYGZdq{}}                \PYG{o}{:} \PYG{l+s+s2}{\PYGZdq{}*bin\PYGZdq{}}\PYG{p}{,}
        \PYG{l+s+s2}{\PYGZdq{}AmplifierScale\PYGZdq{}}        \PYG{o}{:} \PYG{l+s+s2}{\PYGZdq{}1.0\PYGZdq{}}\PYG{p}{,}
        \PYG{l+s+s2}{\PYGZdq{}AmplifierOffset\PYGZdq{}}       \PYG{o}{:} \PYG{l+s+s2}{\PYGZdq{}0.0\PYGZdq{}}\PYG{p}{,}
        \PYG{l+s+s2}{\PYGZdq{}SamplingFrequency\PYGZdq{}}     \PYG{o}{:} \PYG{l+s+s2}{\PYGZdq{}50000\PYGZdq{}}\PYG{p}{,}
        \PYG{l+s+s2}{\PYGZdq{}HeaderOffset\PYGZdq{}}          \PYG{o}{:} \PYG{l+s+s2}{\PYGZdq{}0\PYGZdq{}}\PYG{p}{,}
        \PYG{l+s+s2}{\PYGZdq{}PythonStructCode\PYGZdq{}}      \PYG{o}{:} \PYG{l+s+s2}{\PYGZdq{}\PYGZsq{}h\PYGZsq{}\PYGZdq{}}
\PYG{p}{\PYGZcb{}}
\end{Verbatim}


\section{Define Top-Level Functionality}
\label{doc/Extend:define-top-level-functionality}
New functionality can be added to \emph{MOSAIC} by combining other parts of the code. One way of accomplishing this is by defining new top-level functionality as shown in the following example. We define a new class that converts data from one of the supported data formats to comma separated text files (CSV). A complete listing of the {\hyperref[api\string-doc/mosaic:mosaic.ConvertToCSV.ConvertToCSV]{\emph{\code{ConvertToCSV}}}} class in this example is available in the API documentation.

The \emph{\_\_init\_\_} function of {\hyperref[api\string-doc/mosaic:mosaic.ConvertToCSV.ConvertToCSV]{\emph{\code{ConvertToCSV}}}} class accepts two arguments: a trajIO object and the location to save the converted files. If the output directory is not specified, the data is saved in the same folder as the input data. The data conversion is performed by the {\hyperref[api\string-doc/mosaic:mosaic.ConvertToCSV.ConvertToCSV.Convert]{\emph{\code{Convert()}}}} function, which saves the data in blocks controlled by the \emph{blockSize} parameter. {\hyperref[api\string-doc/mosaic:mosaic.ConvertToCSV.ConvertToCSV.Convert]{\emph{\code{Convert()}}}} saves each block to a new CSV file, named with the filename of the input data followed by an integer number (see the API documentation for {\hyperref[api\string-doc/mosaic:mosaic.ConvertToCSV.ConvertToCSV._filename]{\emph{\code{\_filename()}}}} for additional details).

\begin{Verbatim}[commandchars=\\\{\}]
\PYG{k}{class} \PYG{n+nc}{ConvertToCSV}\PYG{p}{(}\PYG{n+nb}{object}\PYG{p}{)}\PYG{p}{:}
        \PYG{k}{def} \PYG{n+nf}{\PYGZus{}\PYGZus{}init\PYGZus{}\PYGZus{}}\PYG{p}{(}\PYG{n+nb+bp}{self}\PYG{p}{,} \PYG{n}{trajDataObj}\PYG{p}{,} \PYG{n}{outdir}\PYG{o}{=}\PYG{n+nb+bp}{None}\PYG{p}{)}\PYG{p}{:}
                \PYG{n+nb+bp}{self}\PYG{o}{.}\PYG{n}{trajDataObj}\PYG{o}{=}\PYG{n}{trajDataObj}
                \PYG{n+nb+bp}{self}\PYG{o}{.}\PYG{n}{datPath}\PYG{o}{=}\PYG{n}{trajDataObj}\PYG{o}{.}\PYG{n}{datPath}

                \PYG{c+c1}{\PYGZsh{} If outdir is None, save the CSV files to the same directory as the data.}
                \PYG{k}{if} \PYG{n}{outdir}\PYG{o}{==}\PYG{n+nb+bp}{None}\PYG{p}{:}
                        \PYG{n+nb+bp}{self}\PYG{o}{.}\PYG{n}{outDir}\PYG{o}{=}\PYG{n+nb+bp}{self}\PYG{o}{.}\PYG{n}{datPath}
                \PYG{k}{else}\PYG{p}{:}
                        \PYG{n+nb+bp}{self}\PYG{o}{.}\PYG{n}{outDir}\PYG{o}{=}\PYG{n}{outdir}

                \PYG{n+nb+bp}{self}\PYG{o}{.}\PYG{n}{filePrefix}\PYG{o}{=}\PYG{n+nb+bp}{None}
                \PYG{n+nb+bp}{self}\PYG{o}{.}\PYG{n}{\PYGZus{}creategenerator}\PYG{p}{(}\PYG{p}{)}

        \PYG{k}{def} \PYG{n+nf}{Convert}\PYG{p}{(}\PYG{n+nb+bp}{self}\PYG{p}{,} \PYG{n}{blockSize}\PYG{p}{)}\PYG{p}{:}
                \PYG{n}{data}\PYG{o}{=}\PYG{n}{numpy}\PYG{o}{.}\PYG{n}{array}\PYG{p}{(}\PYG{p}{[}\PYG{p}{]}\PYG{p}{,} \PYG{n}{dtype}\PYG{o}{=}\PYG{n}{numpy}\PYG{o}{.}\PYG{n}{float64}\PYG{p}{)}

                \PYG{k}{try}\PYG{p}{:}
                        \PYG{k}{while}\PYG{p}{(}\PYG{n+nb+bp}{True}\PYG{p}{)}\PYG{p}{:}
                                \PYG{p}{(}\PYG{n+nb+bp}{self}\PYG{o}{.}\PYG{n}{trajDataObj}\PYG{o}{.}\PYG{n}{popdata}\PYG{p}{(}\PYG{n}{blockSize}\PYG{p}{)}\PYG{p}{)}\PYG{o}{.}\PYG{n}{tofile}\PYG{p}{(}
                                                \PYG{n+nb+bp}{self}\PYG{o}{.}\PYG{n}{\PYGZus{}filename}\PYG{p}{(}\PYG{p}{)}\PYG{p}{,}
                                                \PYG{n}{sep}\PYG{o}{=}\PYG{l+s+s1}{\PYGZsq{}}\PYG{l+s+s1}{,}\PYG{l+s+s1}{\PYGZsq{}}
                                        \PYG{p}{)}
                \PYG{k}{except} \PYG{n}{EmptyDataPipeError}\PYG{p}{:}
                        \PYG{k}{pass}
\end{Verbatim}

The {\hyperref[api\string-doc/mosaic:mosaic.ConvertToCSV.ConvertToCSV]{\emph{\code{ConvertToCSV}}}} class can now be used with any trajIO object as seen below.

\begin{Verbatim}[commandchars=\\\{\}]
\PYG{n}{ConvertToCSV}\PYG{p}{(} \PYG{n}{abfTrajIO}\PYG{p}{(}\PYG{n}{dirname}\PYG{o}{=}\PYG{l+s+s2}{\PYGZdq{}}\PYG{l+s+s2}{\PYGZti{}/RefData/abfSet1/}\PYG{l+s+s2}{\PYGZdq{}}\PYG{p}{,} \PYG{n+nb}{filter}\PYG{o}{=}\PYG{l+s+s2}{\PYGZdq{}}\PYG{l+s+s2}{*abf}\PYG{l+s+s2}{\PYGZdq{}}\PYG{p}{)} \PYG{p}{)}\PYG{o}{.}\PYG{n}{Convert}\PYG{p}{(}
                        \PYG{n}{blockSize}\PYG{o}{=}\PYG{l+m+mi}{50000}\PYG{p}{)}

\PYG{n}{ConvertToCSV}\PYG{p}{(} \PYG{n}{qdfTrajIO}\PYG{p}{(}\PYG{n}{dirname}\PYG{o}{=}\PYG{l+s+s2}{\PYGZdq{}}\PYG{l+s+s2}{\PYGZti{}/RefData/qdfSet1/}\PYG{l+s+s2}{\PYGZdq{}}\PYG{p}{,} \PYG{n+nb}{filter}\PYG{o}{=}\PYG{l+s+s2}{\PYGZdq{}}\PYG{l+s+s2}{*qdf}\PYG{l+s+s2}{\PYGZdq{}}\PYG{p}{,} \PYG{n}{Rfb}\PYG{o}{=}\PYG{l+s+s2}{\PYGZdq{}}\PYG{l+s+s2}{2.1E+9}\PYG{l+s+s2}{\PYGZdq{}}\PYG{p}{,}
                        \PYG{n}{Cfb}\PYG{o}{=}\PYG{l+s+s2}{\PYGZdq{}}\PYG{l+s+s2}{1.16E\PYGZhy{}12}\PYG{l+s+s2}{\PYGZdq{}}\PYG{p}{)} \PYG{p}{)}\PYG{o}{.}\PYG{n}{Convert}\PYG{p}{(}\PYG{n}{blockSize}\PYG{o}{=}\PYG{l+m+mi}{50000}\PYG{p}{)}

\PYG{n}{ConvertToCSV}\PYG{p}{(} \PYG{n}{binTrajIO}\PYG{p}{(}\PYG{n}{dirname}\PYG{o}{=}\PYG{l+s+s2}{\PYGZdq{}}\PYG{l+s+s2}{\PYGZti{}/RefData/binSet1/}\PYG{l+s+s2}{\PYGZdq{}}\PYG{p}{,} \PYG{n+nb}{filter}\PYG{o}{=}\PYG{l+s+s2}{\PYGZdq{}}\PYG{l+s+s2}{*bin}\PYG{l+s+s2}{\PYGZdq{}}\PYG{p}{,} \PYG{n}{AmplifierScale}\PYG{o}{=}\PYG{l+m+mf}{1.0}\PYG{p}{,}
                        \PYG{n}{AmplifierOffset}\PYG{o}{=}\PYG{l+m+mf}{0.0}\PYG{p}{,} \PYG{n}{SamplingFrequency}\PYG{o}{=}\PYG{l+m+mi}{50000}\PYG{p}{,} \PYG{n}{HeaderOffset}\PYG{o}{=}\PYG{l+m+mi}{0}\PYG{p}{,}
                        \PYG{n}{PythonStructCode}\PYG{o}{=}\PYG{l+s+s1}{\PYGZsq{}}\PYG{l+s+s1}{h}\PYG{l+s+s1}{\PYGZsq{}}\PYG{p}{)} \PYG{p}{)}\PYG{o}{.}\PYG{n}{Convert}\PYG{p}{(}\PYG{n}{blockSize}\PYG{o}{=}\PYG{l+m+mi}{50000}\PYG{p}{)}
\end{Verbatim}

Since {\hyperref[api\string-doc/mosaic:mosaic.ConvertToCSV.ConvertToCSV]{\emph{\code{ConvertToCSV}}}} accepts a trajIO object, we can apply a lowpass filter to the data before converting it to the CSV format. This is accomplished by passing the \emph{datafilter} option to the trajIO object as described in the {\hyperref[doc/ScriptingandAdvancedFeatures:scripting\string-filter\string-sec]{\emph{Filter Data}}} section. In the example below, we convert ABF files to the CSV format after applying a lowpass Bessel filter to the data.

\begin{Verbatim}[commandchars=\\\{\}]
\PYG{n}{ConvertToCSV}\PYG{p}{(} \PYG{n}{abfTrajIO}\PYG{p}{(}\PYG{n}{dirname}\PYG{o}{=}\PYG{l+s+s2}{\PYGZdq{}}\PYG{l+s+s2}{\PYGZti{}/RefData/abfSet1/}\PYG{l+s+s2}{\PYGZdq{}}\PYG{p}{,} \PYG{n+nb}{filter}\PYG{o}{=}\PYG{l+s+s2}{\PYGZdq{}}\PYG{l+s+s2}{*abf}\PYG{l+s+s2}{\PYGZdq{}}\PYG{p}{,}
                \PYG{n}{datafilter}\PYG{o}{=}\PYG{n}{mosaic}\PYG{o}{.}\PYG{n}{besselFilter}
        \PYG{p}{)} \PYG{p}{)}\PYG{o}{.}\PYG{n}{Convert}\PYG{p}{(}\PYG{n}{blockSize}\PYG{o}{=}\PYG{l+m+mi}{50000}\PYG{p}{)}
\end{Verbatim}

Finally, the {\hyperref[api\string-doc/mosaic:mosaic.ConvertToCSV.ConvertToCSV]{\emph{\code{ConvertToCSV}}}} class can be further extended to output arbitrary binary files in place of CSV by the simple extension shown below.

\begin{Verbatim}[commandchars=\\\{\}]
\PYG{l+s+sd}{\PYGZdq{}\PYGZdq{}\PYGZdq{}}
\PYG{l+s+sd}{        Extend the MOSAIC ConvertToCSV class to export arbitrary binary files.}

\PYG{l+s+sd}{        :Created:       02/25/2015}
\PYG{l+s+sd}{        :Author:        Arvind Balijepalli \PYGZlt{}arvind.balijepalli@nist.gov\PYGZgt{}}
\PYG{l+s+sd}{        :ChangeLog:}
\PYG{l+s+sd}{        .. line\PYGZhy{}block::}
\PYG{l+s+sd}{                02/25/15        AB      Initial version}
\PYG{l+s+sd}{\PYGZdq{}\PYGZdq{}\PYGZdq{}}
\PYG{k+kn}{import} \PYG{n+nn}{mosaic.ConvertToCSV} \PYG{k+kn}{as} \PYG{n+nn}{conv}
\PYG{k+kn}{import} \PYG{n+nn}{mosaic.binTrajIO} \PYG{k+kn}{as} \PYG{n+nn}{bin}
\PYG{k+kn}{import} \PYG{n+nn}{mosaic.settings} \PYG{k+kn}{as} \PYG{n+nn}{sett}
\PYG{k+kn}{import} \PYG{n+nn}{numpy} \PYG{k+kn}{as} \PYG{n+nn}{np}

\PYG{k+kn}{from} \PYG{n+nn}{mosaic.metaTrajIO} \PYG{k+kn}{import} \PYG{n}{EmptyDataPipeError}

\PYG{k}{class} \PYG{n+nc}{ConvertToBin}\PYG{p}{(}\PYG{n}{conv}\PYG{o}{.}\PYG{n}{ConvertToCSV}\PYG{p}{)}\PYG{p}{:}
        \PYG{k}{def} \PYG{n+nf}{Convert}\PYG{p}{(}\PYG{n+nb+bp}{self}\PYG{p}{,} \PYG{n}{blockSize}\PYG{p}{,} \PYG{n}{binType}\PYG{p}{)}\PYG{p}{:}
                \PYG{l+s+sd}{\PYGZdq{}\PYGZdq{}\PYGZdq{}}
\PYG{l+s+sd}{                        Start converting data}

\PYG{l+s+sd}{                        :Parameters:}
\PYG{l+s+sd}{                                \PYGZhy{} {}`blockSize{}`   : number of data points to convert.}
\PYG{l+s+sd}{                                \PYGZhy{} {}`binType{}`     : Numpy binary type.}
\PYG{l+s+sd}{                \PYGZdq{}\PYGZdq{}\PYGZdq{}}
                \PYG{k}{try}\PYG{p}{:}
                        \PYG{k}{while}\PYG{p}{(}\PYG{n+nb+bp}{True}\PYG{p}{)}\PYG{p}{:}
                                \PYG{n}{np}\PYG{o}{.}\PYG{n}{array}\PYG{p}{(} \PYG{n+nb+bp}{self}\PYG{o}{.}\PYG{n}{trajDataObj}\PYG{o}{.}\PYG{n}{popdata}\PYG{p}{(}\PYG{n}{blockSize}\PYG{p}{)}\PYG{p}{,} \PYG{n}{dtype}\PYG{o}{=}\PYG{n}{binType} \PYG{p}{)}\PYG{o}{.}\PYG{n}{tofile}\PYG{p}{(}\PYG{n+nb+bp}{self}\PYG{o}{.}\PYG{n}{\PYGZus{}filename}\PYG{p}{(}\PYG{p}{)}\PYG{p}{)}
                \PYG{k}{except} \PYG{n}{EmptyDataPipeError}\PYG{p}{:}
                        \PYG{k}{pass}


\PYG{k}{if} \PYG{n}{\PYGZus{}\PYGZus{}name\PYGZus{}\PYGZus{}} \PYG{o}{==} \PYG{l+s+s1}{\PYGZsq{}}\PYG{l+s+s1}{\PYGZus{}\PYGZus{}main\PYGZus{}\PYGZus{}}\PYG{l+s+s1}{\PYGZsq{}}\PYG{p}{:}
        \PYG{n}{s}\PYG{o}{=}\PYG{p}{\PYGZob{}}
        \PYG{l+s+s2}{\PYGZdq{}}\PYG{l+s+s2}{AmplifierOffset}\PYG{l+s+s2}{\PYGZdq{}}\PYG{p}{:} \PYG{l+m+mf}{0.0}\PYG{p}{,}
        \PYG{l+s+s2}{\PYGZdq{}}\PYG{l+s+s2}{SamplingFrequency}\PYG{l+s+s2}{\PYGZdq{}}\PYG{p}{:} \PYG{l+m+mi}{250000}\PYG{p}{,}
        \PYG{l+s+s2}{\PYGZdq{}}\PYG{l+s+s2}{AmplifierScale}\PYG{l+s+s2}{\PYGZdq{}}\PYG{p}{:} \PYG{l+s+s2}{\PYGZdq{}}\PYG{l+s+s2}{1.0}\PYG{l+s+s2}{\PYGZdq{}}\PYG{p}{,}
        \PYG{l+s+s2}{\PYGZdq{}}\PYG{l+s+s2}{ColumnTypes}\PYG{l+s+s2}{\PYGZdq{}}\PYG{p}{:} \PYG{l+s+s2}{\PYGZdq{}}\PYG{l+s+s2}{[(}\PYG{l+s+s2}{\PYGZsq{}}\PYG{l+s+s2}{curr\PYGZus{}pA}\PYG{l+s+s2}{\PYGZsq{}}\PYG{l+s+s2}{, }\PYG{l+s+s2}{\PYGZsq{}}\PYG{l+s+s2}{\PYGZgt{}f8}\PYG{l+s+s2}{\PYGZsq{}}\PYG{l+s+s2}{), (}\PYG{l+s+s2}{\PYGZsq{}}\PYG{l+s+s2}{volts}\PYG{l+s+s2}{\PYGZsq{}}\PYG{l+s+s2}{, }\PYG{l+s+s2}{\PYGZsq{}}\PYG{l+s+s2}{\PYGZgt{}f8}\PYG{l+s+s2}{\PYGZsq{}}\PYG{l+s+s2}{)]}\PYG{l+s+s2}{\PYGZdq{}}\PYG{p}{,}
        \PYG{l+s+s2}{\PYGZdq{}}\PYG{l+s+s2}{dcOffset}\PYG{l+s+s2}{\PYGZdq{}}\PYG{p}{:} \PYG{l+m+mf}{0.0}\PYG{p}{,}
        \PYG{l+s+s2}{\PYGZdq{}}\PYG{l+s+s2}{filter}\PYG{l+s+s2}{\PYGZdq{}}\PYG{p}{:} \PYG{l+s+s2}{\PYGZdq{}}\PYG{l+s+s2}{*.bin}\PYG{l+s+s2}{\PYGZdq{}}\PYG{p}{,}
        \PYG{l+s+s2}{\PYGZdq{}}\PYG{l+s+s2}{start}\PYG{l+s+s2}{\PYGZdq{}}\PYG{p}{:} \PYG{l+m+mf}{0.0}\PYG{p}{,}
        \PYG{l+s+s2}{\PYGZdq{}}\PYG{l+s+s2}{HeaderOffset}\PYG{l+s+s2}{\PYGZdq{}}\PYG{p}{:} \PYG{l+m+mi}{0}\PYG{p}{,}
        \PYG{l+s+s2}{\PYGZdq{}}\PYG{l+s+s2}{IonicCurrentColumn}\PYG{l+s+s2}{\PYGZdq{}}\PYG{p}{:} \PYG{l+s+s2}{\PYGZdq{}}\PYG{l+s+s2}{curr\PYGZus{}pA}\PYG{l+s+s2}{\PYGZdq{}}
    \PYG{p}{\PYGZcb{}}
        \PYG{n}{ConvertToBin}\PYG{p}{(}
                        \PYG{n+nb}{bin}\PYG{o}{.}\PYG{n}{binTrajIO}\PYG{p}{(}\PYG{n}{dirname}\PYG{o}{=}\PYG{l+s+s2}{\PYGZdq{}}\PYG{l+s+s2}{.}\PYG{l+s+s2}{\PYGZdq{}}\PYG{p}{,} \PYG{o}{*}\PYG{o}{*}\PYG{n}{s} \PYG{p}{)}\PYG{p}{,}
                        \PYG{n}{outdir}\PYG{o}{=}\PYG{l+s+s2}{\PYGZdq{}}\PYG{l+s+s2}{convert}\PYG{l+s+s2}{\PYGZdq{}}\PYG{p}{,}
                        \PYG{n}{extension}\PYG{o}{=}\PYG{l+s+s2}{\PYGZdq{}}\PYG{l+s+s2}{bin}\PYG{l+s+s2}{\PYGZdq{}}
                        \PYG{p}{)}\PYG{o}{.}\PYG{n}{Convert}\PYG{p}{(}\PYG{n}{blockSize}\PYG{o}{=}\PYG{l+m+mi}{10000000}\PYG{p}{,} \PYG{n}{binType}\PYG{o}{=}\PYG{l+s+s1}{\PYGZsq{}}\PYG{l+s+s1}{f4}\PYG{l+s+s1}{\PYGZsq{}}\PYG{p}{)}
\end{Verbatim}


\chapter{Publication Quality Figures}
\label{doc/PublicationFigures:plotting-page}\label{doc/PublicationFigures::doc}\label{doc/PublicationFigures:publication-quality-figures}
We provide packaged functions for publication quality plots using \href{http://www.python.org/}{Python} and \href{http://matplotlib.org/}{matplotlib}. Example plots using these modules are below.


\section{Timeseries Plots}
\label{doc/PublicationFigures:timeseries-plots}
\emph{Generate publication time-series plots using the
mosaicscripts.plots.timeseries module.}

\begin{Verbatim}[commandchars=\\\{\}]
:Created:   11/19/2015
:Author:    Arvind Balijepalli \PYGZlt{}arvind.balijepalli@nist.gov\PYGZgt{}
:License:   See LICENSE.TXT
:ChangeLog:
    12/12/15    AB  Generalized plot function to allow different data types
    11/20/15    AB  Initial version
\end{Verbatim}

\begin{Verbatim}[commandchars=\\\{\}]
\PYG{k+kn}{import} \PYG{n+nn}{mosaicscripts.plots.timeseries} \PYG{k+kn}{as} \PYG{n+nn}{ts}
\end{Verbatim}

Plots are generated using the \code{mosaicscripts.plots.timeseries} module.
See the timeseries module for
additional details.

Basic usage to plot 1 second of ionic current vs. time is shown below.
The \code{plotopts} argument is used to style the curve in the plot.

\begin{Verbatim}[commandchars=\\\{\}]
\PYG{n}{ts}\PYG{o}{.}\PYG{n}{PlotTimeseries}\PYG{p}{(}
    \PYG{l+s+s2}{\PYGZdq{}}\PYG{l+s+s2}{../data/}\PYG{l+s+s2}{\PYGZdq{}}\PYG{p}{,}
    \PYG{l+s+s2}{\PYGZdq{}}\PYG{l+s+s2}{abf}\PYG{l+s+s2}{\PYGZdq{}}\PYG{p}{,}
    \PYG{l+m+mf}{5.0}\PYG{p}{,}
    \PYG{l+m+mf}{6.0}\PYG{p}{,}
    \PYG{l+m+mi}{50000}\PYG{p}{,}
    \PYG{n}{labels}\PYG{o}{=}\PYG{p}{[}\PYG{l+s+s2}{\PYGZdq{}}\PYG{l+s+s2}{t (s)}\PYG{l+s+s2}{\PYGZdq{}}\PYG{p}{,} \PYG{l+s+s2}{\PYGZdq{}}\PYG{l+s+s2}{\PYGZhy{}i (pA)}\PYG{l+s+s2}{\PYGZdq{}}\PYG{p}{]}\PYG{p}{,}
    \PYG{n}{axes}\PYG{o}{=}\PYG{n+nb+bp}{True}\PYG{p}{,}
    \PYG{n}{polarity}\PYG{o}{=}\PYG{o}{\PYGZhy{}}\PYG{l+m+mi}{1}\PYG{p}{,}
    \PYG{n}{plotopts}\PYG{o}{=}\PYG{p}{\PYGZob{}}
        \PYG{l+s+s1}{\PYGZsq{}}\PYG{l+s+s1}{color}\PYG{l+s+s1}{\PYGZsq{}} \PYG{p}{:} \PYG{l+s+s1}{\PYGZsq{}}\PYG{l+s+s1}{\PYGZsh{}3F50A0}\PYG{l+s+s1}{\PYGZsq{}}\PYG{p}{,}
        \PYG{l+s+s1}{\PYGZsq{}}\PYG{l+s+s1}{marker}\PYG{l+s+s1}{\PYGZsq{}} \PYG{p}{:} \PYG{l+s+s1}{\PYGZsq{}}\PYG{l+s+s1}{.}\PYG{l+s+s1}{\PYGZsq{}}\PYG{p}{,}
        \PYG{l+s+s1}{\PYGZsq{}}\PYG{l+s+s1}{markersize}\PYG{l+s+s1}{\PYGZsq{}} \PYG{p}{:} \PYG{l+m+mf}{0.2}
    \PYG{p}{\PYGZcb{}}
\PYG{p}{)}
\end{Verbatim}

\includegraphics{{TimeseriesPlots_3_0}.png}

Plotting other data types is also straightforward. The next example
demonstrates plotting a time-series in the QUB data format (QDF). Note
that the conversion to voltage to current is performed with the \code{Rfb}
and \code{Cfb} parameters, passed to the plotting function as a dictionary.

\begin{Verbatim}[commandchars=\\\{\}]
\PYG{n}{ts}\PYG{o}{.}\PYG{n}{PlotTimeseries}\PYG{p}{(}
    \PYG{l+s+s2}{\PYGZdq{}}\PYG{l+s+s2}{../data/}\PYG{l+s+s2}{\PYGZdq{}}\PYG{p}{,}
    \PYG{l+s+s2}{\PYGZdq{}}\PYG{l+s+s2}{qdf}\PYG{l+s+s2}{\PYGZdq{}}\PYG{p}{,}
    \PYG{l+m+mf}{0.25}\PYG{p}{,}
    \PYG{l+m+mf}{0.75}\PYG{p}{,}
    \PYG{l+m+mi}{500000}\PYG{p}{,}
    \PYG{n}{labels}\PYG{o}{=}\PYG{p}{[}\PYG{l+s+s2}{\PYGZdq{}}\PYG{l+s+s2}{t (s)}\PYG{l+s+s2}{\PYGZdq{}}\PYG{p}{,} \PYG{l+s+s2}{\PYGZdq{}}\PYG{l+s+s2}{\PYGZhy{}i (pA)}\PYG{l+s+s2}{\PYGZdq{}}\PYG{p}{]}\PYG{p}{,}
    \PYG{n}{axes}\PYG{o}{=}\PYG{n+nb+bp}{True}\PYG{p}{,}
    \PYG{n}{polarity}\PYG{o}{=}\PYG{l+m+mi}{1}\PYG{p}{,}
    \PYG{n}{plotopts}\PYG{o}{=}\PYG{p}{\PYGZob{}}
        \PYG{l+s+s1}{\PYGZsq{}}\PYG{l+s+s1}{color}\PYG{l+s+s1}{\PYGZsq{}} \PYG{p}{:} \PYG{l+s+s1}{\PYGZsq{}}\PYG{l+s+s1}{\PYGZsh{}3F50A0}\PYG{l+s+s1}{\PYGZsq{}}\PYG{p}{,}
        \PYG{l+s+s1}{\PYGZsq{}}\PYG{l+s+s1}{marker}\PYG{l+s+s1}{\PYGZsq{}} \PYG{p}{:} \PYG{l+s+s1}{\PYGZsq{}}\PYG{l+s+s1}{.}\PYG{l+s+s1}{\PYGZsq{}}\PYG{p}{,}
        \PYG{l+s+s1}{\PYGZsq{}}\PYG{l+s+s1}{markersize}\PYG{l+s+s1}{\PYGZsq{}} \PYG{p}{:} \PYG{l+m+mf}{0.2}
    \PYG{p}{\PYGZcb{}}\PYG{p}{,}
    \PYG{n}{data\PYGZus{}args}\PYG{o}{=}\PYG{p}{\PYGZob{}}\PYG{l+s+s2}{\PYGZdq{}}\PYG{l+s+s2}{Rfb}\PYG{l+s+s2}{\PYGZdq{}} \PYG{p}{:} \PYG{l+m+mf}{9.1e9}\PYG{p}{,} \PYG{l+s+s2}{\PYGZdq{}}\PYG{l+s+s2}{Cfb}\PYG{l+s+s2}{\PYGZdq{}}\PYG{p}{:} \PYG{l+m+mf}{1.07e\PYGZhy{}12}\PYG{p}{\PYGZcb{}}
\PYG{p}{)}
\end{Verbatim}

\includegraphics{{TimeseriesPlots_5_0}.png}

Segments of timeseries can be highlighted for emphasis or to show
specific features. Three blockade events are plotted using different
colors in the example below.

\begin{Verbatim}[commandchars=\\\{\}]
\PYG{n}{ts}\PYG{o}{.}\PYG{n}{PlotTimeseries}\PYG{p}{(}
    \PYG{l+s+s2}{\PYGZdq{}}\PYG{l+s+s2}{../data/}\PYG{l+s+s2}{\PYGZdq{}}\PYG{p}{,}
    \PYG{l+s+s2}{\PYGZdq{}}\PYG{l+s+s2}{abf}\PYG{l+s+s2}{\PYGZdq{}}\PYG{p}{,}
    \PYG{l+m+mf}{5.0}\PYG{p}{,}
    \PYG{l+m+mf}{6.0}\PYG{p}{,}
    \PYG{l+m+mi}{50000}\PYG{p}{,}
    \PYG{n}{labels}\PYG{o}{=}\PYG{p}{[}\PYG{l+s+s2}{\PYGZdq{}}\PYG{l+s+s2}{t (s)}\PYG{l+s+s2}{\PYGZdq{}}\PYG{p}{,} \PYG{l+s+s2}{\PYGZdq{}}\PYG{l+s+s2}{\PYGZhy{}i (pA)}\PYG{l+s+s2}{\PYGZdq{}}\PYG{p}{]}\PYG{p}{,}
    \PYG{n}{axes}\PYG{o}{=}\PYG{n+nb+bp}{True}\PYG{p}{,}
    \PYG{n}{polarity}\PYG{o}{=}\PYG{o}{\PYGZhy{}}\PYG{l+m+mi}{1}\PYG{p}{,}
    \PYG{n}{plotopts}\PYG{o}{=}\PYG{p}{\PYGZob{}}
        \PYG{l+s+s1}{\PYGZsq{}}\PYG{l+s+s1}{color}\PYG{l+s+s1}{\PYGZsq{}} \PYG{p}{:} \PYG{l+s+s1}{\PYGZsq{}}\PYG{l+s+s1}{gray}\PYG{l+s+s1}{\PYGZsq{}}\PYG{p}{,}
        \PYG{l+s+s1}{\PYGZsq{}}\PYG{l+s+s1}{marker}\PYG{l+s+s1}{\PYGZsq{}} \PYG{p}{:} \PYG{l+s+s1}{\PYGZsq{}}\PYG{l+s+s1}{.}\PYG{l+s+s1}{\PYGZsq{}}\PYG{p}{,}
        \PYG{l+s+s1}{\PYGZsq{}}\PYG{l+s+s1}{markersize}\PYG{l+s+s1}{\PYGZsq{}} \PYG{p}{:} \PYG{l+m+mf}{0.2}
    \PYG{p}{\PYGZcb{}}\PYG{p}{,}
    \PYG{n}{highlights}\PYG{o}{=}\PYG{p}{[}
        \PYG{p}{[}\PYG{p}{[}\PYG{l+m+mf}{0.282}\PYG{p}{,} \PYG{l+m+mf}{0.293}\PYG{p}{]}\PYG{p}{,} \PYG{p}{\PYGZob{}}\PYG{l+s+s1}{\PYGZsq{}}\PYG{l+s+s1}{color}\PYG{l+s+s1}{\PYGZsq{}} \PYG{p}{:} \PYG{l+s+s1}{\PYGZsq{}}\PYG{l+s+s1}{\PYGZsh{}3F50A0}\PYG{l+s+s1}{\PYGZsq{}}\PYG{p}{,} \PYG{l+s+s1}{\PYGZsq{}}\PYG{l+s+s1}{marker}\PYG{l+s+s1}{\PYGZsq{}} \PYG{p}{:} \PYG{l+s+s1}{\PYGZsq{}}\PYG{l+s+s1}{.}\PYG{l+s+s1}{\PYGZsq{}}\PYG{p}{,} \PYG{l+s+s1}{\PYGZsq{}}\PYG{l+s+s1}{markersize}\PYG{l+s+s1}{\PYGZsq{}} \PYG{p}{:} \PYG{l+m+mf}{0.1}\PYG{p}{\PYGZcb{}}\PYG{p}{]}\PYG{p}{,}
        \PYG{p}{[}\PYG{p}{[}\PYG{l+m+mf}{0.584}\PYG{p}{,} \PYG{l+m+mf}{0.597}\PYG{p}{]}\PYG{p}{,} \PYG{p}{\PYGZob{}}\PYG{l+s+s1}{\PYGZsq{}}\PYG{l+s+s1}{color}\PYG{l+s+s1}{\PYGZsq{}} \PYG{p}{:} \PYG{l+s+s1}{\PYGZsq{}}\PYG{l+s+s1}{\PYGZsh{}D42324}\PYG{l+s+s1}{\PYGZsq{}}\PYG{p}{,} \PYG{l+s+s1}{\PYGZsq{}}\PYG{l+s+s1}{marker}\PYG{l+s+s1}{\PYGZsq{}} \PYG{p}{:} \PYG{l+s+s1}{\PYGZsq{}}\PYG{l+s+s1}{.}\PYG{l+s+s1}{\PYGZsq{}}\PYG{p}{,} \PYG{l+s+s1}{\PYGZsq{}}\PYG{l+s+s1}{markersize}\PYG{l+s+s1}{\PYGZsq{}} \PYG{p}{:} \PYG{l+m+mf}{0.1}\PYG{p}{\PYGZcb{}}\PYG{p}{]}\PYG{p}{,}
        \PYG{p}{[}\PYG{p}{[}\PYG{l+m+mf}{0.685}\PYG{p}{,} \PYG{l+m+mf}{0.695}\PYG{p}{]}\PYG{p}{,} \PYG{p}{\PYGZob{}}\PYG{l+s+s1}{\PYGZsq{}}\PYG{l+s+s1}{color}\PYG{l+s+s1}{\PYGZsq{}} \PYG{p}{:} \PYG{l+s+s1}{\PYGZsq{}}\PYG{l+s+s1}{\PYGZsh{}EB751A}\PYG{l+s+s1}{\PYGZsq{}}\PYG{p}{,} \PYG{l+s+s1}{\PYGZsq{}}\PYG{l+s+s1}{marker}\PYG{l+s+s1}{\PYGZsq{}} \PYG{p}{:} \PYG{l+s+s1}{\PYGZsq{}}\PYG{l+s+s1}{.}\PYG{l+s+s1}{\PYGZsq{}}\PYG{p}{,} \PYG{l+s+s1}{\PYGZsq{}}\PYG{l+s+s1}{markersize}\PYG{l+s+s1}{\PYGZsq{}} \PYG{p}{:} \PYG{l+m+mf}{0.1}\PYG{p}{\PYGZcb{}}\PYG{p}{]}
    \PYG{p}{]}\PYG{p}{,}
    \PYG{n}{figname}\PYG{o}{=}\PYG{l+s+s2}{\PYGZdq{}}\PYG{l+s+s2}{timeseries.png}\PYG{l+s+s2}{\PYGZdq{}}
\PYG{p}{)}
\end{Verbatim}

\includegraphics{{TimeseriesPlots_7_0}.png}


\section{Histogram Plots}
\label{doc/PublicationFigures:histogram-plots}
\emph{Generate publication quality histogram plots using the
mosaicscripts.plot.histogram module.}

\begin{Verbatim}[commandchars=\\\{\}]
:Created:   12/14/2015
:Author:    Arvind Balijepalli \PYGZlt{}arvind.balijepalli@nist.gov\PYGZgt{}
:License:   See LICENSE.TXT
:ChangeLog:
    01/09/16         AB    Added a plot overlay example.
    12/14/15        AB  Initial version
\end{Verbatim}

\begin{Verbatim}[commandchars=\\\{\}]
\PYG{k+kn}{import} \PYG{n+nn}{numpy} \PYG{k+kn}{as} \PYG{n+nn}{np}
\PYG{k+kn}{from} \PYG{n+nn}{scipy.optimize} \PYG{k+kn}{import} \PYG{n}{curve\PYGZus{}fit}
\end{Verbatim}

\begin{Verbatim}[commandchars=\\\{\}]
\PYG{k+kn}{import} \PYG{n+nn}{mosaicscripts.plots.histogram} \PYG{k+kn}{as} \PYG{n+nn}{histogram}
\PYG{k+kn}{from} \PYG{n+nn}{mosaic.utilities.sqlQuery} \PYG{k+kn}{import} \PYG{n}{query}
\end{Verbatim}

\begin{Verbatim}[commandchars=\\\{\}]
\PYG{n}{q}\PYG{o}{=}\PYG{l+s+s2}{\PYGZdq{}}\PYG{l+s+s2}{select BlockDepth from metadata where ProcessingStatus=}\PYG{l+s+s2}{\PYGZsq{}}\PYG{l+s+s2}{normal}\PYG{l+s+s2}{\PYGZsq{}}\PYG{l+s+s2}{ and ResTime \PYGZgt{} 0.25 and BlockDepth between 0.3 and 0.4}\PYG{l+s+s2}{\PYGZdq{}}
\end{Verbatim}


\subsection{Basic Histogram Plots}
\label{doc/PublicationFigures:basic-histogram-plots}
Plots are generated using the
\code{mosaicscripts.plots.histogram.histogram\_plot()} function. See the
histogram module for
additional details.

\begin{Verbatim}[commandchars=\\\{\}]
\PYG{n}{histogram}\PYG{o}{.}\PYG{n}{histogram\PYGZus{}plot}\PYG{p}{(}
                            \PYG{n}{query}\PYG{p}{(}\PYG{l+s+s2}{\PYGZdq{}}\PYG{l+s+s2}{../data/eventMD\PYGZhy{}P28\PYGZhy{}bin.sqlite}\PYG{l+s+s2}{\PYGZdq{}}\PYG{p}{,} \PYG{n}{q}\PYG{p}{)}\PYG{p}{,}
                            \PYG{l+m+mi}{100}\PYG{p}{,}
                            \PYG{p}{(}\PYG{l+m+mf}{0.3}\PYG{p}{,} \PYG{l+m+mf}{0.4}\PYG{p}{)}\PYG{p}{,}
                            \PYG{n}{xticks}\PYG{o}{=} \PYG{p}{(}\PYG{l+m+mf}{0.3}\PYG{p}{,}\PYG{l+m+mf}{0.35}\PYG{p}{,}\PYG{l+m+mf}{0.4}\PYG{p}{)}\PYG{p}{,}
                            \PYG{n}{yticks}\PYG{o}{=}\PYG{p}{(}\PYG{l+m+mi}{0}\PYG{p}{,}\PYG{l+m+mi}{500}\PYG{p}{,}\PYG{l+m+mi}{1000}\PYG{p}{,}\PYG{l+m+mi}{1500}\PYG{p}{)}\PYG{p}{,}
                            \PYG{n}{xlabel}\PYG{o}{=}\PYG{l+s+s2}{r\PYGZdq{}}\PYG{l+s+s2}{\PYGZlt{}i\PYGZgt{}/\PYGZlt{}i\PYGZdl{}\PYGZus{}0\PYGZdl{}\PYGZgt{}}\PYG{l+s+s2}{\PYGZdq{}}\PYG{p}{,}
                            \PYG{n}{ylabel}\PYG{o}{=}\PYG{l+s+s2}{r\PYGZdq{}}\PYG{l+s+s2}{Counts}\PYG{l+s+s2}{\PYGZdq{}}
                    \PYG{p}{)}
\end{Verbatim}

\includegraphics{{HistogramPlots_5_0}.png}

To plot the probability density, supply the argument \code{density}=True
as shown below.

\begin{Verbatim}[commandchars=\\\{\}]
\PYG{n}{histogram}\PYG{o}{.}\PYG{n}{histogram\PYGZus{}plot}\PYG{p}{(}
                            \PYG{n}{query}\PYG{p}{(}\PYG{l+s+s2}{\PYGZdq{}}\PYG{l+s+s2}{../data/eventMD\PYGZhy{}P28\PYGZhy{}bin.sqlite}\PYG{l+s+s2}{\PYGZdq{}}\PYG{p}{,} \PYG{n}{q}\PYG{p}{)}\PYG{p}{,}
                            \PYG{l+m+mi}{100}\PYG{p}{,}
                            \PYG{p}{(}\PYG{l+m+mf}{0.3}\PYG{p}{,} \PYG{l+m+mf}{0.4}\PYG{p}{)}\PYG{p}{,}
                            \PYG{n}{xticks}\PYG{o}{=} \PYG{p}{(}\PYG{l+m+mf}{0.3}\PYG{p}{,}\PYG{l+m+mf}{0.35}\PYG{p}{,}\PYG{l+m+mf}{0.4}\PYG{p}{)}\PYG{p}{,}
                            \PYG{n}{yticks}\PYG{o}{=}\PYG{p}{(}\PYG{l+m+mi}{0}\PYG{p}{,}\PYG{l+m+mi}{50}\PYG{p}{,}\PYG{l+m+mi}{100}\PYG{p}{)}\PYG{p}{,}
                            \PYG{n}{xlabel}\PYG{o}{=}\PYG{l+s+s2}{r\PYGZdq{}}\PYG{l+s+s2}{\PYGZlt{}i\PYGZgt{}/\PYGZlt{}i\PYGZdl{}\PYGZus{}0\PYGZdl{}\PYGZgt{}}\PYG{l+s+s2}{\PYGZdq{}}\PYG{p}{,}
                            \PYG{n}{ylabel}\PYG{o}{=}\PYG{l+s+s2}{r\PYGZdq{}}\PYG{l+s+s2}{Density}\PYG{l+s+s2}{\PYGZdq{}}\PYG{p}{,}
                \PYG{n}{density}\PYG{o}{=}\PYG{n+nb+bp}{True}
                    \PYG{p}{)}
\end{Verbatim}

\includegraphics{{HistogramPlots_7_0}.png}


\subsection{Custom Styles}
\label{doc/PublicationFigures:custom-styles}
The fill transperancy can be controlled with the \code{fill\_alpha}
argument. When set to \code{1}, it results in a filled plot as seen below.
To turn off filling, simply set \code{fill\_alpha}=0

\begin{Verbatim}[commandchars=\\\{\}]
\PYG{n}{histogram}\PYG{o}{.}\PYG{n}{histogram\PYGZus{}plot}\PYG{p}{(}
                            \PYG{n}{query}\PYG{p}{(}\PYG{l+s+s2}{\PYGZdq{}}\PYG{l+s+s2}{../data/eventMD\PYGZhy{}P28\PYGZhy{}bin.sqlite}\PYG{l+s+s2}{\PYGZdq{}}\PYG{p}{,} \PYG{n}{q}\PYG{p}{)}\PYG{p}{,}
                            \PYG{l+m+mi}{100}\PYG{p}{,}
                            \PYG{p}{(}\PYG{l+m+mf}{0.3}\PYG{p}{,} \PYG{l+m+mf}{0.4}\PYG{p}{)}\PYG{p}{,}
                            \PYG{n}{xticks}\PYG{o}{=} \PYG{p}{(}\PYG{l+m+mf}{0.3}\PYG{p}{,}\PYG{l+m+mf}{0.35}\PYG{p}{,}\PYG{l+m+mf}{0.4}\PYG{p}{)}\PYG{p}{,}
                            \PYG{n}{yticks}\PYG{o}{=}\PYG{p}{(}\PYG{l+m+mi}{0}\PYG{p}{,}\PYG{l+m+mi}{500}\PYG{p}{,}\PYG{l+m+mi}{1000}\PYG{p}{,}\PYG{l+m+mi}{1500}\PYG{p}{)}\PYG{p}{,}
                            \PYG{n}{xlabel}\PYG{o}{=}\PYG{l+s+s2}{r\PYGZdq{}}\PYG{l+s+s2}{\PYGZlt{}i\PYGZgt{}/\PYGZlt{}i\PYGZdl{}\PYGZus{}0\PYGZdl{}\PYGZgt{}}\PYG{l+s+s2}{\PYGZdq{}}\PYG{p}{,}
                            \PYG{n}{ylabel}\PYG{o}{=}\PYG{l+s+s2}{r\PYGZdq{}}\PYG{l+s+s2}{Counts}\PYG{l+s+s2}{\PYGZdq{}}\PYG{p}{,}
                \PYG{n}{fill\PYGZus{}alpha}\PYG{o}{=}\PYG{l+m+mi}{1}
                    \PYG{p}{)}
\end{Verbatim}

\includegraphics{{HistogramPlots_9_0}.png}

Matplotlib plotting directies can be supplied to \code{histogram\_plot()}
using the \code{advanced\_opts} argument. See the \href{http://matplotlib.org/api/pyplot\_api.html\#matplotlib.pyplot.plot}{Matplotlib plot
documentation}
for additional details. In the example below, the plot linewidth is set
to 1.5 points.

\begin{Verbatim}[commandchars=\\\{\}]
\PYG{n}{histogram}\PYG{o}{.}\PYG{n}{histogram\PYGZus{}plot}\PYG{p}{(}
                            \PYG{n}{query}\PYG{p}{(}\PYG{l+s+s2}{\PYGZdq{}}\PYG{l+s+s2}{../data/eventMD\PYGZhy{}P28\PYGZhy{}bin.sqlite}\PYG{l+s+s2}{\PYGZdq{}}\PYG{p}{,} \PYG{n}{q}\PYG{p}{)}\PYG{p}{,}
                            \PYG{l+m+mi}{100}\PYG{p}{,}
                            \PYG{p}{(}\PYG{l+m+mf}{0.3}\PYG{p}{,} \PYG{l+m+mf}{0.4}\PYG{p}{)}\PYG{p}{,}
                            \PYG{n}{xticks}\PYG{o}{=} \PYG{p}{(}\PYG{l+m+mf}{0.3}\PYG{p}{,}\PYG{l+m+mf}{0.35}\PYG{p}{,}\PYG{l+m+mf}{0.4}\PYG{p}{)}\PYG{p}{,}
                            \PYG{n}{yticks}\PYG{o}{=}\PYG{p}{(}\PYG{l+m+mi}{0}\PYG{p}{,}\PYG{l+m+mi}{500}\PYG{p}{,}\PYG{l+m+mi}{1000}\PYG{p}{,}\PYG{l+m+mi}{1500}\PYG{p}{)}\PYG{p}{,}
                            \PYG{n}{xlabel}\PYG{o}{=}\PYG{l+s+s2}{r\PYGZdq{}}\PYG{l+s+s2}{\PYGZlt{}i\PYGZgt{}/\PYGZlt{}i\PYGZdl{}\PYGZus{}0\PYGZdl{}\PYGZgt{}}\PYG{l+s+s2}{\PYGZdq{}}\PYG{p}{,}
                            \PYG{n}{ylabel}\PYG{o}{=}\PYG{l+s+s2}{r\PYGZdq{}}\PYG{l+s+s2}{Counts}\PYG{l+s+s2}{\PYGZdq{}}\PYG{p}{,}
                \PYG{n}{color}\PYG{o}{=}\PYG{l+s+s1}{\PYGZsq{}}\PYG{l+s+s1}{purple}\PYG{l+s+s1}{\PYGZsq{}}\PYG{p}{,}
                \PYG{n}{dpi}\PYG{o}{=}\PYG{l+m+mi}{600}\PYG{p}{,}
                \PYG{n}{fill\PYGZus{}alpha}\PYG{o}{=}\PYG{l+m+mf}{0.15}\PYG{p}{,}
                \PYG{n}{advanced\PYGZus{}opts}\PYG{o}{=}\PYG{p}{\PYGZob{}}\PYG{l+s+s1}{\PYGZsq{}}\PYG{l+s+s1}{linewidth}\PYG{l+s+s1}{\PYGZsq{}}\PYG{p}{:} \PYG{l+m+mf}{1.5}\PYG{p}{\PYGZcb{}}
                    \PYG{p}{)}
\end{Verbatim}

\includegraphics{{HistogramPlots_11_0}.png}

The example below shows more advanced styling. Circular markers can be
placed at the center of each bin using the \href{http://matplotlib.org/api/pyplot\_api.html\#matplotlib.pyplot.plot}{Matplotlib marker
keywords}.

Finally, images can be saved by supplying the \code{figname} argument as
seen in the example below. Optionally, the figure resolution can be set
with the \code{dpi} argument.

\begin{Verbatim}[commandchars=\\\{\}]
\PYG{n}{histogram}\PYG{o}{.}\PYG{n}{histogram\PYGZus{}plot}\PYG{p}{(}
                            \PYG{n}{query}\PYG{p}{(}\PYG{l+s+s2}{\PYGZdq{}}\PYG{l+s+s2}{../data/eventMD\PYGZhy{}P28\PYGZhy{}bin.sqlite}\PYG{l+s+s2}{\PYGZdq{}}\PYG{p}{,} \PYG{n}{q}\PYG{p}{)}\PYG{p}{,}
                            \PYG{l+m+mi}{100}\PYG{p}{,}
                            \PYG{p}{(}\PYG{l+m+mf}{0.3}\PYG{p}{,} \PYG{l+m+mf}{0.4}\PYG{p}{)}\PYG{p}{,}
                            \PYG{n}{xticks}\PYG{o}{=} \PYG{p}{(}\PYG{l+m+mf}{0.3}\PYG{p}{,}\PYG{l+m+mf}{0.35}\PYG{p}{,}\PYG{l+m+mf}{0.4}\PYG{p}{)}\PYG{p}{,}
                            \PYG{n}{yticks}\PYG{o}{=}\PYG{p}{(}\PYG{l+m+mi}{0}\PYG{p}{,}\PYG{l+m+mi}{500}\PYG{p}{,}\PYG{l+m+mi}{1000}\PYG{p}{,}\PYG{l+m+mi}{1500}\PYG{p}{)}\PYG{p}{,}
                            \PYG{n}{figname}\PYG{o}{=}\PYG{l+s+s2}{\PYGZdq{}}\PYG{l+s+s2}{histogram.png}\PYG{l+s+s2}{\PYGZdq{}}\PYG{p}{,}
                            \PYG{n}{xlabel}\PYG{o}{=}\PYG{l+s+s2}{r\PYGZdq{}}\PYG{l+s+s2}{\PYGZlt{}i\PYGZgt{}/\PYGZlt{}i\PYGZdl{}\PYGZus{}0\PYGZdl{}\PYGZgt{}}\PYG{l+s+s2}{\PYGZdq{}}\PYG{p}{,}
                            \PYG{n}{ylabel}\PYG{o}{=}\PYG{l+s+s2}{r\PYGZdq{}}\PYG{l+s+s2}{Counts}\PYG{l+s+s2}{\PYGZdq{}}\PYG{p}{,}
                \PYG{n}{color}\PYG{o}{=}\PYG{l+s+s1}{\PYGZsq{}}\PYG{l+s+s1}{\PYGZsh{}EB771A}\PYG{l+s+s1}{\PYGZsq{}}\PYG{p}{,}
                \PYG{n}{dpi}\PYG{o}{=}\PYG{l+m+mi}{600}\PYG{p}{,}
                \PYG{n}{fill\PYGZus{}alpha}\PYG{o}{=}\PYG{l+m+mf}{0.15}\PYG{p}{,}
                \PYG{n}{advanced\PYGZus{}opts}\PYG{o}{=}\PYG{p}{\PYGZob{}}
                            \PYG{l+s+s1}{\PYGZsq{}}\PYG{l+s+s1}{marker}\PYG{l+s+s1}{\PYGZsq{}}\PYG{p}{:} \PYG{l+s+s1}{\PYGZsq{}}\PYG{l+s+s1}{o}\PYG{l+s+s1}{\PYGZsq{}}\PYG{p}{,}
                            \PYG{l+s+s1}{\PYGZsq{}}\PYG{l+s+s1}{markersize}\PYG{l+s+s1}{\PYGZsq{}}\PYG{p}{:} \PYG{l+m+mi}{6}\PYG{p}{,}
                            \PYG{l+s+s1}{\PYGZsq{}}\PYG{l+s+s1}{markeredgecolor}\PYG{l+s+s1}{\PYGZsq{}} \PYG{p}{:} \PYG{l+s+s1}{\PYGZsq{}}\PYG{l+s+s1}{\PYGZsh{}EB771A}\PYG{l+s+s1}{\PYGZsq{}}\PYG{p}{,}
                            \PYG{l+s+s1}{\PYGZsq{}}\PYG{l+s+s1}{markeredgewidth}\PYG{l+s+s1}{\PYGZsq{}} \PYG{p}{:} \PYG{l+m+mf}{0.75}\PYG{p}{,}
                            \PYG{l+s+s1}{\PYGZsq{}}\PYG{l+s+s1}{markerfacecolor}\PYG{l+s+s1}{\PYGZsq{}}\PYG{p}{:} \PYG{l+s+s1}{\PYGZsq{}}\PYG{l+s+s1}{none}\PYG{l+s+s1}{\PYGZsq{}}\PYG{p}{,}
                            \PYG{l+s+s1}{\PYGZsq{}}\PYG{l+s+s1}{linewidth}\PYG{l+s+s1}{\PYGZsq{}}\PYG{p}{:} \PYG{l+m+mf}{1.}
                        \PYG{p}{\PYGZcb{}}
                    \PYG{p}{)}
\end{Verbatim}

\includegraphics{{HistogramPlots_13_0}.png}


\subsection{Advanced Analysis and Plot Overlays}
\label{doc/PublicationFigures:advanced-analysis-and-plot-overlays}
The \code{mosaicscripts.plots.histogram.histogram\_plot()} function allows
one to overlay additional curves on top of the histogram data. This is
useful, for example, to fit the histogram to a known functional form.
Below we describe, how to fit the histogram data to a sum of two
Gaussians.

First we must define the fit function as shown below. We sum two
Gaussians of the form:
\(a_1 exp(-(x-\mu_1)^2/2\sigma_1^2)+a_2 exp(-(x-\mu_2)^2/2\sigma_2^2)\),
where \(x\) is the independent variable, \(\mu\) is the mean of
the distribution, \(\sigma\) is the standard deviation, \(a\) is
the amplitude and the subscripts denote the peak number.

\begin{Verbatim}[commandchars=\\\{\}]
\PYG{k}{def} \PYG{n+nf}{gauss\PYGZus{}sum\PYGZus{}fit}\PYG{p}{(}\PYG{n}{x}\PYG{p}{,} \PYG{n}{a1}\PYG{p}{,} \PYG{n}{mu1}\PYG{p}{,} \PYG{n}{sigma1}\PYG{p}{,} \PYG{n}{a2}\PYG{p}{,} \PYG{n}{mu2}\PYG{p}{,} \PYG{n}{sigma2}\PYG{p}{)}\PYG{p}{:}
    \PYG{k}{return} \PYG{n}{a1}\PYG{o}{*}\PYG{n}{np}\PYG{o}{.}\PYG{n}{exp}\PYG{p}{(}\PYG{o}{\PYGZhy{}}\PYG{p}{(}\PYG{n}{x}\PYG{o}{\PYGZhy{}}\PYG{n}{mu1}\PYG{p}{)}\PYG{o}{*}\PYG{o}{*}\PYG{l+m+mi}{2}\PYG{o}{/}\PYG{p}{(}\PYG{l+m+mi}{2}\PYG{o}{*}\PYG{n}{sigma1}\PYG{o}{*}\PYG{o}{*}\PYG{l+m+mi}{2}\PYG{p}{)}\PYG{p}{)} \PYG{o}{+} \PYG{n}{a2}\PYG{o}{*}\PYG{n}{np}\PYG{o}{.}\PYG{n}{exp}\PYG{p}{(}\PYG{o}{\PYGZhy{}}\PYG{p}{(}\PYG{n}{x}\PYG{o}{\PYGZhy{}}\PYG{n}{mu2}\PYG{p}{)}\PYG{o}{*}\PYG{o}{*}\PYG{l+m+mi}{2}\PYG{o}{/}\PYG{p}{(}\PYG{l+m+mi}{2}\PYG{o}{*}\PYG{n}{sigma2}\PYG{o}{*}\PYG{o}{*}\PYG{l+m+mi}{2}\PYG{p}{)}\PYG{p}{)}
\end{Verbatim}

Next, we call the \code{histogram\_plot} function as before. Note however
there are two additional options we must provide to enable us to add the
peak fits to the plot. The first is \code{show=False}, which suppresses
plotting the histogram to allow additional plots to be added to the
figure (see the \href{http://matplotlib.org/api/pyplot\_api.html\#matplotlib.pyplot.show}{Matplotlib
documentation}
for details), and the second is \code{return\_histogram=True}, which returns
the raw histogram data that we fit to.

Next, we perform the least squares fit using the \href{http://docs.scipy.org/doc/scipy-0.16.0/reference/generated/scipy.optimize.curve\_fit.html}{Scipy
curve\_fit}
function. The optimized parameters and covariance are stored in \code{popt}
and \code{pcov} respectively.

Finally, we plot the fit function and call
\href{http://matplotlib.org/api/pyplot\_api.html\#matplotlib.pyplot.show}{show()}
to display the figure.

\begin{Verbatim}[commandchars=\\\{\}]
\PYG{n}{hist}\PYG{p}{,}\PYG{n}{bins}\PYG{o}{=}\PYG{n}{histogram}\PYG{o}{.}\PYG{n}{histogram\PYGZus{}plot}\PYG{p}{(}
                            \PYG{n}{query}\PYG{p}{(}\PYG{l+s+s2}{\PYGZdq{}}\PYG{l+s+s2}{../data/eventMD\PYGZhy{}P28\PYGZhy{}bin.sqlite}\PYG{l+s+s2}{\PYGZdq{}}\PYG{p}{,} \PYG{n}{q}\PYG{p}{)}\PYG{p}{,}
                            \PYG{l+m+mi}{75}\PYG{p}{,}
                            \PYG{p}{(}\PYG{l+m+mf}{0.3}\PYG{p}{,} \PYG{l+m+mf}{0.4}\PYG{p}{)}\PYG{p}{,}
                            \PYG{n}{xticks}\PYG{o}{=} \PYG{p}{(}\PYG{l+m+mf}{0.3}\PYG{p}{,}\PYG{l+m+mf}{0.325}\PYG{p}{,}\PYG{l+m+mf}{0.35}\PYG{p}{,}\PYG{l+m+mf}{0.375}\PYG{p}{,}\PYG{l+m+mf}{0.4}\PYG{p}{)}\PYG{p}{,}
                            \PYG{n}{yticks}\PYG{o}{=}\PYG{p}{(}\PYG{l+m+mi}{0}\PYG{p}{,}\PYG{l+m+mi}{500}\PYG{p}{,}\PYG{l+m+mi}{1000}\PYG{p}{,}\PYG{l+m+mi}{1500}\PYG{p}{)}\PYG{p}{,}
                            \PYG{n}{figname}\PYG{o}{=}\PYG{l+s+s2}{\PYGZdq{}}\PYG{l+s+s2}{histogram.png}\PYG{l+s+s2}{\PYGZdq{}}\PYG{p}{,}
                            \PYG{n}{xlabel}\PYG{o}{=}\PYG{l+s+s2}{r\PYGZdq{}}\PYG{l+s+s2}{\PYGZlt{}i\PYGZgt{}/\PYGZlt{}i\PYGZdl{}\PYGZus{}0\PYGZdl{}\PYGZgt{}}\PYG{l+s+s2}{\PYGZdq{}}\PYG{p}{,}
                            \PYG{n}{ylabel}\PYG{o}{=}\PYG{l+s+s2}{r\PYGZdq{}}\PYG{l+s+s2}{Counts}\PYG{l+s+s2}{\PYGZdq{}}\PYG{p}{,}
                            \PYG{n}{fill\PYGZus{}alpha}\PYG{o}{=}\PYG{l+m+mf}{0.}\PYG{p}{,}
                            \PYG{n}{show}\PYG{o}{=}\PYG{n+nb+bp}{False}\PYG{p}{,}
                            \PYG{n}{return\PYGZus{}histogram}\PYG{o}{=}\PYG{n+nb+bp}{True}\PYG{p}{,}
                            \PYG{n}{advanced\PYGZus{}opts}\PYG{o}{=}\PYG{p}{\PYGZob{}}
                                \PYG{l+s+s1}{\PYGZsq{}}\PYG{l+s+s1}{marker}\PYG{l+s+s1}{\PYGZsq{}}\PYG{p}{:} \PYG{l+s+s1}{\PYGZsq{}}\PYG{l+s+s1}{o}\PYG{l+s+s1}{\PYGZsq{}}\PYG{p}{,}
                                \PYG{l+s+s1}{\PYGZsq{}}\PYG{l+s+s1}{markersize}\PYG{l+s+s1}{\PYGZsq{}}\PYG{p}{:} \PYG{l+m+mi}{6}\PYG{p}{,}
                                \PYG{l+s+s1}{\PYGZsq{}}\PYG{l+s+s1}{markeredgecolor}\PYG{l+s+s1}{\PYGZsq{}} \PYG{p}{:} \PYG{l+s+s1}{\PYGZsq{}}\PYG{l+s+s1}{\PYGZsh{}002A63}\PYG{l+s+s1}{\PYGZsq{}}\PYG{p}{,}
                                \PYG{l+s+s1}{\PYGZsq{}}\PYG{l+s+s1}{markeredgewidth}\PYG{l+s+s1}{\PYGZsq{}} \PYG{p}{:} \PYG{l+m+mf}{0.75}\PYG{p}{,}
                                \PYG{l+s+s1}{\PYGZsq{}}\PYG{l+s+s1}{markerfacecolor}\PYG{l+s+s1}{\PYGZsq{}}\PYG{p}{:} \PYG{l+s+s1}{\PYGZsq{}}\PYG{l+s+s1}{none}\PYG{l+s+s1}{\PYGZsq{}}\PYG{p}{,}
                                \PYG{l+s+s1}{\PYGZsq{}}\PYG{l+s+s1}{linewidth}\PYG{l+s+s1}{\PYGZsq{}}\PYG{p}{:} \PYG{l+m+mf}{0.}
                        \PYG{p}{\PYGZcb{}}
        \PYG{p}{)}

\PYG{n}{popt}\PYG{p}{,}\PYG{n}{pcov}\PYG{o}{=}\PYG{n}{curve\PYGZus{}fit}\PYG{p}{(}\PYG{n}{gauss\PYGZus{}sum\PYGZus{}fit}\PYG{p}{,} \PYG{n}{bins}\PYG{p}{,} \PYG{n}{hist}\PYG{p}{,} \PYG{p}{[}\PYG{l+m+mi}{1200}\PYG{p}{,} \PYG{l+m+mf}{0.34}\PYG{p}{,}\PYG{l+m+mf}{0.003}\PYG{p}{,} \PYG{l+m+mi}{100}\PYG{p}{,} \PYG{l+m+mf}{0.36}\PYG{p}{,}\PYG{l+m+mf}{0.003}\PYG{p}{]}\PYG{p}{)}

\PYG{n}{xdat}\PYG{o}{=}\PYG{n}{np}\PYG{o}{.}\PYG{n}{arange}\PYG{p}{(}\PYG{l+m+mf}{0.3}\PYG{p}{,} \PYG{l+m+mf}{0.4}\PYG{p}{,}\PYG{l+m+mf}{0.0005}\PYG{p}{)}
\PYG{n}{ydat}\PYG{o}{=}\PYG{n}{gauss\PYGZus{}sum\PYGZus{}fit}\PYG{p}{(}\PYG{n}{xdat}\PYG{p}{,} \PYG{o}{*}\PYG{n}{popt}\PYG{p}{)}

\PYG{n}{histogram}\PYG{o}{.}\PYG{n}{plt}\PYG{o}{.}\PYG{n}{plot}\PYG{p}{(}\PYG{n}{xdat}\PYG{p}{,} \PYG{n}{ydat}\PYG{p}{,} \PYG{n}{color}\PYG{o}{=}\PYG{l+s+s2}{\PYGZdq{}}\PYG{l+s+s2}{\PYGZsh{}002A63}\PYG{l+s+s2}{\PYGZdq{}}\PYG{p}{)}
\PYG{n}{histogram}\PYG{o}{.}\PYG{n}{plt}\PYG{o}{.}\PYG{n}{show}\PYG{p}{(}\PYG{p}{)}
\end{Verbatim}

\includegraphics{{HistogramPlots_17_0}.png}

The \code{popt} variable holds the optimized fit parameters, stored in the
order defined by the \code{gauss\_sum\_fit} above. We can extract these
values from this list. For example, the peak positions can be retrieved
as shown below.

\begin{Verbatim}[commandchars=\\\{\}]
\PYG{n}{popt}\PYG{p}{[}\PYG{l+m+mi}{1}\PYG{p}{]}\PYG{p}{,} \PYG{n}{popt}\PYG{p}{[}\PYG{l+m+mi}{4}\PYG{p}{]}
\end{Verbatim}

\begin{Verbatim}[commandchars=\\\{\}]
\PYG{p}{(}\PYG{l+m+mf}{0.33733498827022712}\PYG{p}{,} \PYG{l+m+mf}{0.3559240351776794}\PYG{p}{)}
\end{Verbatim}


\section{Contour Plots}
\label{doc/PublicationFigures:contour-plots}
\emph{Generate publication quality contour plots using the
mosaicscripts.plot.contour module.}

\begin{Verbatim}[commandchars=\\\{\}]
:Created:   11/19/2015
:Author:    Arvind Balijepalli \PYGZlt{}arvind.balijepalli@nist.gov\PYGZgt{}
:License:   See LICENSE.TXT
:ChangeLog:
    11/19/15        AB  Initial version
\end{Verbatim}

\begin{Verbatim}[commandchars=\\\{\}]
\PYG{k+kn}{import} \PYG{n+nn}{matplotlib.pyplot} \PYG{k+kn}{as} \PYG{n+nn}{plt}
\PYG{k+kn}{import} \PYG{n+nn}{mosaicscripts.plots.contour} \PYG{k+kn}{as} \PYG{n+nn}{contour}
\PYG{k+kn}{from} \PYG{n+nn}{mosaic.utilities.sqlQuery} \PYG{k+kn}{import} \PYG{n}{query}
\end{Verbatim}

Plots are generated using the \code{mosaicscripts.plots.contour\_plot()}
function. See the contour module
for additional details.

\begin{Verbatim}[commandchars=\\\{\}]
\PYG{n}{contour}\PYG{o}{.}\PYG{n}{contour\PYGZus{}plot}\PYG{p}{(}
                                    \PYG{n}{query}\PYG{p}{(}
                                            \PYG{l+s+s2}{\PYGZdq{}}\PYG{l+s+s2}{../data/eventMD\PYGZhy{}20150404\PYGZhy{}221533\PYGZus{}MSA.sqlite}\PYG{l+s+s2}{\PYGZdq{}}\PYG{p}{,}
                                            \PYG{l+s+s2}{\PYGZdq{}}\PYG{l+s+s2}{select BlockDepth, StateResTime from metadata where ProcessingStatus=}\PYG{l+s+s2}{\PYGZsq{}}\PYG{l+s+s2}{normal}\PYG{l+s+s2}{\PYGZsq{}}\PYG{l+s+s2}{ and BlockDepth \PYGZgt{} 0 and ResTime \PYGZgt{} 0.025}\PYG{l+s+s2}{\PYGZdq{}}
                                            \PYG{p}{)}\PYG{p}{,}
                                    \PYG{n}{x\PYGZus{}range}\PYG{o}{=}\PYG{p}{[}\PYG{l+m+mf}{0.01}\PYG{p}{,} \PYG{l+m+mf}{0.26}\PYG{p}{]}\PYG{p}{,}
                                    \PYG{n}{y\PYGZus{}range}\PYG{o}{=}\PYG{p}{[}\PYG{l+m+mf}{0.02}\PYG{p}{,} \PYG{l+m+mf}{0.06}\PYG{p}{]}\PYG{p}{,}
                                    \PYG{n}{bin\PYGZus{}size}\PYG{o}{=}\PYG{l+m+mf}{0.0085}\PYG{p}{,}
                                    \PYG{n}{contours}\PYG{o}{=}\PYG{l+m+mi}{6}\PYG{p}{,}
                                    \PYG{n}{colormap}\PYG{o}{=}\PYG{n}{plt}\PYG{o}{.}\PYG{n}{get\PYGZus{}cmap}\PYG{p}{(}\PYG{l+s+s1}{\PYGZsq{}}\PYG{l+s+s1}{Purples}\PYG{l+s+s1}{\PYGZsq{}}\PYG{p}{)}\PYG{p}{,}
                                    \PYG{n}{img\PYGZus{}interpolation}\PYG{o}{=}\PYG{l+s+s1}{\PYGZsq{}}\PYG{l+s+s1}{nearest}\PYG{l+s+s1}{\PYGZsq{}}\PYG{p}{,}
                                    \PYG{n}{xticks}\PYG{o}{=}\PYG{p}{[}
                                                    \PYG{p}{(}\PYG{l+m+mf}{0.05}\PYG{p}{,} \PYG{l+s+s1}{\PYGZsq{}}\PYG{l+s+s1}{0.05}\PYG{l+s+s1}{\PYGZsq{}}\PYG{p}{)}\PYG{p}{,}
                                                    \PYG{p}{(}\PYG{l+m+mf}{0.1}\PYG{p}{,} \PYG{l+s+s1}{\PYGZsq{}}\PYG{l+s+s1}{0.1}\PYG{l+s+s1}{\PYGZsq{}}\PYG{p}{)}\PYG{p}{,}
                                                    \PYG{p}{(}\PYG{l+m+mf}{0.15}\PYG{p}{,} \PYG{l+s+s1}{\PYGZsq{}}\PYG{l+s+s1}{0.15}\PYG{l+s+s1}{\PYGZsq{}}\PYG{p}{)}\PYG{p}{,}
                                                    \PYG{p}{(}\PYG{l+m+mf}{0.2}\PYG{p}{,} \PYG{l+s+s1}{\PYGZsq{}}\PYG{l+s+s1}{0.2}\PYG{l+s+s1}{\PYGZsq{}}\PYG{p}{)}
                                                    \PYG{p}{]}\PYG{p}{,}
                                    \PYG{n}{yticks}\PYG{o}{=}\PYG{p}{[}
                                                    \PYG{p}{(}\PYG{l+m+mf}{0.025}\PYG{p}{,} \PYG{l+s+s1}{\PYGZsq{}}\PYG{l+s+s1}{25}\PYG{l+s+s1}{\PYGZsq{}}\PYG{p}{)}\PYG{p}{,}
                                                    \PYG{p}{(}\PYG{l+m+mf}{0.04}\PYG{p}{,} \PYG{l+s+s1}{\PYGZsq{}}\PYG{l+s+s1}{40}\PYG{l+s+s1}{\PYGZsq{}}\PYG{p}{)}\PYG{p}{,}
                                                    \PYG{p}{(}\PYG{l+m+mf}{0.05}\PYG{p}{,} \PYG{l+s+s1}{\PYGZsq{}}\PYG{l+s+s1}{50}\PYG{l+s+s1}{\PYGZsq{}}\PYG{p}{)}
                                                    \PYG{p}{]}\PYG{p}{,}
                                    \PYG{n}{axes\PYGZus{}type}\PYG{o}{=}\PYG{p}{[}\PYG{l+s+s1}{\PYGZsq{}}\PYG{l+s+s1}{linear}\PYG{l+s+s1}{\PYGZsq{}}\PYG{p}{,} \PYG{l+s+s1}{\PYGZsq{}}\PYG{l+s+s1}{log}\PYG{l+s+s1}{\PYGZsq{}}\PYG{p}{,} \PYG{l+s+s1}{\PYGZsq{}}\PYG{l+s+s1}{linear}\PYG{l+s+s1}{\PYGZsq{}}\PYG{p}{]}\PYG{p}{,}
                    \PYG{n}{figname}\PYG{o}{=}\PYG{l+s+s2}{\PYGZdq{}}\PYG{l+s+s2}{contour.png}\PYG{l+s+s2}{\PYGZdq{}}\PYG{p}{,}
                                    \PYG{n}{colorbar\PYGZus{}num\PYGZus{}ticks}\PYG{o}{=}\PYG{l+m+mi}{4}\PYG{p}{,}
                                    \PYG{n}{cb\PYGZus{}round\PYGZus{}digits}\PYG{o}{=}\PYG{o}{\PYGZhy{}}\PYG{l+m+mi}{1}\PYG{p}{,}
                                    \PYG{n}{min\PYGZus{}count\PYGZus{}pct}\PYG{o}{=}\PYG{l+m+mf}{0.08}\PYG{p}{,}     \PYG{c+c1}{\PYGZsh{} Set bins with \PYGZlt{} 7\PYGZpc{} of max to 0,}
                                    \PYG{n}{xlabel}\PYG{o}{=}\PYG{l+s+s2}{r\PYGZdq{}}\PYG{l+s+s2}{\PYGZdl{}\PYGZlt{}i\PYGZgt{}/\PYGZlt{}i\PYGZus{}0\PYGZgt{}\PYGZdl{}}\PYG{l+s+s2}{\PYGZdq{}}\PYG{p}{,}
                                    \PYG{n}{ylabel}\PYG{o}{=}\PYG{l+s+s2}{r\PYGZdq{}}\PYG{l+s+s2}{Residence Time (\PYGZdl{}}\PYG{l+s+s2}{\PYGZbs{}}\PYG{l+s+s2}{mu s\PYGZdl{})}\PYG{l+s+s2}{\PYGZdq{}}
                    \PYG{p}{)}
\end{Verbatim}

\includegraphics{{ContourPlots_3_0}.png}

Plot styling can be controlled with custom colormaps. Examples are found
within the \code{contour.gen\_colormap()} function. Calling this function
makes two additional colormaps (\code{mosaicBlue} and \code{mosaicOrange})
available as seen below.

\begin{Verbatim}[commandchars=\\\{\}]
\PYG{n}{contour}\PYG{o}{.}\PYG{n}{gen\PYGZus{}colormaps}\PYG{p}{(}\PYG{p}{)}
\end{Verbatim}

\textbf{Note:} The \code{colormap} argument is now uses \code{Orange1} as opposed
to \code{Purples} above.

\begin{Verbatim}[commandchars=\\\{\}]
\PYG{n}{contour}\PYG{o}{.}\PYG{n}{contour\PYGZus{}plot}\PYG{p}{(}
                                    \PYG{n}{query}\PYG{p}{(}
                                            \PYG{l+s+s2}{\PYGZdq{}}\PYG{l+s+s2}{../data/eventMD\PYGZhy{}20150404\PYGZhy{}221533\PYGZus{}MSA.sqlite}\PYG{l+s+s2}{\PYGZdq{}}\PYG{p}{,}
                                            \PYG{l+s+s2}{\PYGZdq{}}\PYG{l+s+s2}{select BlockDepth, StateResTime from metadata where ProcessingStatus=}\PYG{l+s+s2}{\PYGZsq{}}\PYG{l+s+s2}{normal}\PYG{l+s+s2}{\PYGZsq{}}\PYG{l+s+s2}{ and BlockDepth \PYGZgt{} 0 and ResTime \PYGZgt{} 0.025}\PYG{l+s+s2}{\PYGZdq{}}
                                            \PYG{p}{)}\PYG{p}{,}
                                    \PYG{n}{x\PYGZus{}range}\PYG{o}{=}\PYG{p}{[}\PYG{l+m+mf}{0.01}\PYG{p}{,} \PYG{l+m+mf}{0.26}\PYG{p}{]}\PYG{p}{,}
                                    \PYG{n}{y\PYGZus{}range}\PYG{o}{=}\PYG{p}{[}\PYG{l+m+mf}{0.02}\PYG{p}{,} \PYG{l+m+mf}{0.06}\PYG{p}{]}\PYG{p}{,}
                                    \PYG{n}{bin\PYGZus{}size}\PYG{o}{=}\PYG{l+m+mf}{0.0085}\PYG{p}{,}
                                    \PYG{n}{contours}\PYG{o}{=}\PYG{l+m+mi}{6}\PYG{p}{,}
                                    \PYG{n}{colormap}\PYG{o}{=}\PYG{n}{plt}\PYG{o}{.}\PYG{n}{get\PYGZus{}cmap}\PYG{p}{(}\PYG{l+s+s1}{\PYGZsq{}}\PYG{l+s+s1}{mosaicOrange}\PYG{l+s+s1}{\PYGZsq{}}\PYG{p}{)}\PYG{p}{,}
                                    \PYG{n}{img\PYGZus{}interpolation}\PYG{o}{=}\PYG{l+s+s1}{\PYGZsq{}}\PYG{l+s+s1}{nearest}\PYG{l+s+s1}{\PYGZsq{}}\PYG{p}{,}
                                    \PYG{n}{xticks}\PYG{o}{=}\PYG{p}{[}
                                                    \PYG{p}{(}\PYG{l+m+mf}{0.05}\PYG{p}{,} \PYG{l+s+s1}{\PYGZsq{}}\PYG{l+s+s1}{0.05}\PYG{l+s+s1}{\PYGZsq{}}\PYG{p}{)}\PYG{p}{,}
                                                    \PYG{p}{(}\PYG{l+m+mf}{0.1}\PYG{p}{,} \PYG{l+s+s1}{\PYGZsq{}}\PYG{l+s+s1}{0.1}\PYG{l+s+s1}{\PYGZsq{}}\PYG{p}{)}\PYG{p}{,}
                                                    \PYG{p}{(}\PYG{l+m+mf}{0.15}\PYG{p}{,} \PYG{l+s+s1}{\PYGZsq{}}\PYG{l+s+s1}{0.15}\PYG{l+s+s1}{\PYGZsq{}}\PYG{p}{)}\PYG{p}{,}
                                                    \PYG{p}{(}\PYG{l+m+mf}{0.2}\PYG{p}{,} \PYG{l+s+s1}{\PYGZsq{}}\PYG{l+s+s1}{0.2}\PYG{l+s+s1}{\PYGZsq{}}\PYG{p}{)}
                                                    \PYG{p}{]}\PYG{p}{,}
                                    \PYG{n}{yticks}\PYG{o}{=}\PYG{p}{[}
                                                    \PYG{p}{(}\PYG{l+m+mf}{0.025}\PYG{p}{,} \PYG{l+s+s1}{\PYGZsq{}}\PYG{l+s+s1}{25}\PYG{l+s+s1}{\PYGZsq{}}\PYG{p}{)}\PYG{p}{,}
                                                    \PYG{p}{(}\PYG{l+m+mf}{0.04}\PYG{p}{,} \PYG{l+s+s1}{\PYGZsq{}}\PYG{l+s+s1}{40}\PYG{l+s+s1}{\PYGZsq{}}\PYG{p}{)}\PYG{p}{,}
                                                    \PYG{p}{(}\PYG{l+m+mf}{0.05}\PYG{p}{,} \PYG{l+s+s1}{\PYGZsq{}}\PYG{l+s+s1}{50}\PYG{l+s+s1}{\PYGZsq{}}\PYG{p}{)}
                                                    \PYG{p}{]}\PYG{p}{,}
                                    \PYG{n}{axes\PYGZus{}type}\PYG{o}{=}\PYG{p}{[}\PYG{l+s+s1}{\PYGZsq{}}\PYG{l+s+s1}{linear}\PYG{l+s+s1}{\PYGZsq{}}\PYG{p}{,} \PYG{l+s+s1}{\PYGZsq{}}\PYG{l+s+s1}{log}\PYG{l+s+s1}{\PYGZsq{}}\PYG{p}{,}\PYG{l+s+s1}{\PYGZsq{}}\PYG{l+s+s1}{linear}\PYG{l+s+s1}{\PYGZsq{}}\PYG{p}{]}\PYG{p}{,}
                    \PYG{n}{figname}\PYG{o}{=}\PYG{l+s+s2}{\PYGZdq{}}\PYG{l+s+s2}{contour.png}\PYG{l+s+s2}{\PYGZdq{}}\PYG{p}{,}
                                    \PYG{n}{colorbar\PYGZus{}num\PYGZus{}ticks}\PYG{o}{=}\PYG{l+m+mi}{4}\PYG{p}{,}
                                    \PYG{n}{cb\PYGZus{}round\PYGZus{}digits}\PYG{o}{=}\PYG{o}{\PYGZhy{}}\PYG{l+m+mi}{1}\PYG{p}{,}
                                    \PYG{n}{min\PYGZus{}count\PYGZus{}pct}\PYG{o}{=}\PYG{l+m+mf}{0.08}\PYG{p}{,}     \PYG{c+c1}{\PYGZsh{} Set bins with \PYGZlt{} 7\PYGZpc{} of max to 0,}
                                    \PYG{n}{xlabel}\PYG{o}{=}\PYG{l+s+s2}{r\PYGZdq{}}\PYG{l+s+s2}{\PYGZdl{}\PYGZlt{}i\PYGZgt{}/\PYGZlt{}i\PYGZus{}0\PYGZgt{}\PYGZdl{}}\PYG{l+s+s2}{\PYGZdq{}}\PYG{p}{,}
                                    \PYG{n}{ylabel}\PYG{o}{=}\PYG{l+s+s2}{r\PYGZdq{}}\PYG{l+s+s2}{Residence Time (\PYGZdl{}}\PYG{l+s+s2}{\PYGZbs{}}\PYG{l+s+s2}{mu s\PYGZdl{})}\PYG{l+s+s2}{\PYGZdq{}}
                    \PYG{p}{)}
\end{Verbatim}

\includegraphics{{ContourPlots_7_0}.png}


\chapter{Advanced Analysis}
\label{doc/AdvancedAnalysis:advanced-analysis}\label{doc/AdvancedAnalysis:plotting-page}\label{doc/AdvancedAnalysis::doc}\label{doc/AdvancedAnalysis:mosaic-page-on-github}
We provide packaged functions for advanced analysis such as calculating the capture rate of molecules, determining the residene time, etc as described below.


\section{Capture Rate}
\label{doc/AdvancedAnalysis:capture-rate}
Estimate the capture rate of molecules partitioning into a nanopore.

\begin{Verbatim}[commandchars=\\\{\}]
:Created:    12/27/2015
:Author:     Arvind Balijepalli \PYGZlt{}arvind.balijepalli@nist.gov\PYGZgt{}
:License:    See LICENSE.TXT
:ChangeLog:
    12/27/15        AB    Initial version
\end{Verbatim}

\begin{Verbatim}[commandchars=\\\{\}]
\PYG{k+kn}{import} \PYG{n+nn}{numpy} \PYG{k+kn}{as} \PYG{n+nn}{np}

\PYG{k+kn}{from} \PYG{n+nn}{mosaicscripts.analysis.kinetics} \PYG{k+kn}{import} \PYG{n}{CaptureRate}
\end{Verbatim}


\subsection{Wrapper Function to Estimate the Capture Rate}
\label{doc/AdvancedAnalysis:wrapper-function-to-estimate-the-capture-rate}
The capture rate can be estimated directly by calling the
\code{CaptureRate} function in \code{mosaicscripts.analysis.kinetics}. The
function returns a list with two elements: the capture rate
(s:math:\emph{\textasciicircum{}\{-1\}}), and the standard error of the capture rate
(s:math:\emph{\textasciicircum{}\{-1\}}).

\begin{Verbatim}[commandchars=\\\{\}]
\PYG{n}{np}\PYG{o}{.}\PYG{n}{round}\PYG{p}{(}
        \PYG{n}{CaptureRate}\PYG{p}{(}
            \PYG{l+s+s2}{\PYGZdq{}}\PYG{l+s+s2}{../data/eventMD\PYGZhy{}P28\PYGZhy{}bin.sqlite}\PYG{l+s+s2}{\PYGZdq{}}\PYG{p}{,}
            \PYG{l+s+s2}{\PYGZdq{}}\PYG{l+s+s2}{select AbsEventStart from metadata where ProcessingStatus=}\PYG{l+s+s2}{\PYGZsq{}}\PYG{l+s+s2}{normal}\PYG{l+s+s2}{\PYGZsq{}}\PYG{l+s+s2}{ and ResTime \PYGZgt{} 0.02 order by AbsEventStart ASC}\PYG{l+s+s2}{\PYGZdq{}}
           \PYG{p}{)}\PYG{p}{,}
        \PYG{n}{decimals}\PYG{o}{=}\PYG{l+m+mi}{1}
    \PYG{p}{)}
\end{Verbatim}

\begin{Verbatim}[commandchars=\\\{\}]
\PYG{n}{array}\PYG{p}{(}\PYG{p}{[} \PYG{l+m+mf}{27.9}\PYG{p}{,}   \PYG{l+m+mf}{0.2}\PYG{p}{]}\PYG{p}{)}
\end{Verbatim}


\subsection{Capture Rate Details}
\label{doc/AdvancedAnalysis:capture-rate-details}
\begin{Verbatim}[commandchars=\\\{\}]
\PYG{k+kn}{from} \PYG{n+nn}{scipy.optimize} \PYG{k+kn}{import} \PYG{n}{curve\PYGZus{}fit}
\PYG{k+kn}{import} \PYG{n+nn}{matplotlib.pyplot} \PYG{k+kn}{as} \PYG{n+nn}{plt}

\PYG{k+kn}{from} \PYG{n+nn}{mosaicscripts.analysis.kinetics} \PYG{k+kn}{import} \PYG{n}{query1Col}
\PYG{k+kn}{import} \PYG{n+nn}{mosaicscripts.plots.mplformat} \PYG{k+kn}{as} \PYG{n+nn}{mplformat}
\PYG{k+kn}{from} \PYG{n+nn}{mosaic.utilities.fit\PYGZus{}funcs} \PYG{k+kn}{import} \PYG{n}{singleExponential}
\end{Verbatim}

\begin{Verbatim}[commandchars=\\\{\}]
\PYG{n}{mplformat}\PYG{o}{.}\PYG{n}{update\PYGZus{}rcParams}\PYG{p}{(}\PYG{p}{)}
\end{Verbatim}

Continue reading to dig deeper into how the capture rate is estimated
within the \code{CaptureRate} function.

The first step is to read in the start times for each event. This is
easily done with a query to the MOSAIC database as shown below. The
start times are stored in the \code{AbsEventStart} column. We limit the
events we use to estimate the capture rate to ones that were
successfully fit (\code{ProcessingStatus}='normal') and those whose
residence times (\code{ResTime}) in the pore are longer than 20
\(\mu\)s.

Finally, we sort the \code{AbsEventStart} to ensure the event start times
are in ascending order.

\begin{Verbatim}[commandchars=\\\{\}]
\PYG{n}{start\PYGZus{}times}\PYG{o}{=}\PYG{n}{query1Col}\PYG{p}{(}
        \PYG{l+s+s2}{\PYGZdq{}}\PYG{l+s+s2}{../data/eventMD\PYGZhy{}P28\PYGZhy{}bin.sqlite}\PYG{l+s+s2}{\PYGZdq{}}\PYG{p}{,}
        \PYG{l+s+s2}{\PYGZdq{}}\PYG{l+s+s2}{select AbsEventStart from metadata where ProcessingStatus=}\PYG{l+s+s2}{\PYGZsq{}}\PYG{l+s+s2}{normal}\PYG{l+s+s2}{\PYGZsq{}}\PYG{l+s+s2}{ and ResTime \PYGZgt{} 0.02 order by AbsEventStart ASC}\PYG{l+s+s2}{\PYGZdq{}}
    \PYG{p}{)}
\end{Verbatim}

Next, we calculate the arrival times, i.e. the time between the start of
successive events. This is done with the \code{Numpy} diff function. Note
that \code{AbsEventStart} is stored in milliseconds within the database.
Here, we also convert the arrival times to seconds.

\begin{Verbatim}[commandchars=\\\{\}]
\PYG{n}{arrival\PYGZus{}times}\PYG{o}{=}\PYG{n}{np}\PYG{o}{.}\PYG{n}{diff}\PYG{p}{(}\PYG{n}{start\PYGZus{}times}\PYG{p}{)}\PYG{o}{/}\PYG{l+m+mf}{1000.}
\end{Verbatim}

The partitioning of molecules into the pore is a stochastic process.
There are however a couple properties related to stochastic process that
we will leverage that makes the estimation of the capture rate more
robust. With randomly occuring events that have some mean rate, the
number of events scales linearly with time. Therefore, the distribution
of these events follows a single exponential form. We can easily test
this by calculating the probability density function (PDF) using the
\code{Numpy} histogram function. Note that the \code{density}=\code{True}
argument normalizes the histogram resulting in a PDF.

\begin{Verbatim}[commandchars=\\\{\}]
\PYG{n}{density}\PYG{p}{,}\PYG{n}{bins}\PYG{o}{=}\PYG{n}{np}\PYG{o}{.}\PYG{n}{histogram}\PYG{p}{(}\PYG{n}{arrival\PYGZus{}times}\PYG{p}{,} \PYG{n}{bins}\PYG{o}{=}\PYG{l+m+mi}{100}\PYG{p}{,} \PYG{n}{density}\PYG{o}{=}\PYG{n+nb+bp}{True}\PYG{p}{)}
\end{Verbatim}

Plot the resulting PDF with \code{Matplotlib} to verify the distribution.
Sure enough on a semilog scale, the resulting distribution appears
linear suggesting an exponential form.

\begin{Verbatim}[commandchars=\\\{\}]
\PYG{n}{plt}\PYG{o}{.}\PYG{n}{semilogy}\PYG{p}{(}
        \PYG{n}{bins}\PYG{p}{[}\PYG{p}{:}\PYG{n+nb}{len}\PYG{p}{(}\PYG{n}{density}\PYG{p}{)}\PYG{p}{]}\PYG{p}{,} \PYG{n}{density}\PYG{p}{,}
        \PYG{n}{linestyle}\PYG{o}{=}\PYG{l+s+s1}{\PYGZsq{}}\PYG{l+s+s1}{None}\PYG{l+s+s1}{\PYGZsq{}}\PYG{p}{,}
        \PYG{n}{marker}\PYG{o}{=}\PYG{l+s+s1}{\PYGZsq{}}\PYG{l+s+s1}{o}\PYG{l+s+s1}{\PYGZsq{}}\PYG{p}{,}
        \PYG{n}{markersize}\PYG{o}{=}\PYG{l+m+mi}{8}\PYG{p}{,}
        \PYG{n}{markeredgecolor}\PYG{o}{=}\PYG{l+s+s1}{\PYGZsq{}}\PYG{l+s+s1}{blue}\PYG{l+s+s1}{\PYGZsq{}}\PYG{p}{,}
        \PYG{n}{markerfacecolor}\PYG{o}{=}\PYG{l+s+s1}{\PYGZsq{}}\PYG{l+s+s1}{None}\PYG{l+s+s1}{\PYGZsq{}}
    \PYG{p}{)}
\PYG{n}{plt}\PYG{o}{.}\PYG{n}{xlim}\PYG{p}{(}\PYG{l+m+mf}{0.005}\PYG{p}{,}\PYG{l+m+mf}{0.3}\PYG{p}{)}
\PYG{n}{plt}\PYG{o}{.}\PYG{n}{ylim}\PYG{p}{(}\PYG{l+m+mf}{0.006}\PYG{p}{,}\PYG{l+m+mi}{25}\PYG{p}{)}
\PYG{n}{plt}\PYG{o}{.}\PYG{n}{xticks}\PYG{p}{(}\PYG{p}{[}\PYG{l+m+mf}{0.05}\PYG{p}{,}\PYG{l+m+mf}{0.15}\PYG{p}{,}\PYG{l+m+mf}{0.25}\PYG{p}{]}\PYG{p}{)}
\PYG{n}{plt}\PYG{o}{.}\PYG{n}{yticks}\PYG{p}{(}\PYG{p}{[}\PYG{l+m+mf}{1e\PYGZhy{}2}\PYG{p}{,}\PYG{l+m+mf}{0.1}\PYG{p}{,}\PYG{l+m+mi}{1}\PYG{p}{,}\PYG{l+m+mf}{1e1}\PYG{p}{]}\PYG{p}{)}
\PYG{n}{plt}\PYG{o}{.}\PYG{n}{axes}\PYG{p}{(}\PYG{p}{)}\PYG{o}{.}\PYG{n}{set\PYGZus{}xlabel}\PYG{p}{(}\PYG{l+s+s2}{\PYGZdq{}}\PYG{l+s+s2}{Arrival Times (s)}\PYG{l+s+s2}{\PYGZdq{}}\PYG{p}{)}
\PYG{n}{plt}\PYG{o}{.}\PYG{n}{axes}\PYG{p}{(}\PYG{p}{)}\PYG{o}{.}\PYG{n}{set\PYGZus{}ylabel}\PYG{p}{(}\PYG{l+s+s2}{\PYGZdq{}}\PYG{l+s+s2}{Density (s\PYGZdl{}\PYGZca{}\PYGZob{}\PYGZhy{}1\PYGZcb{}\PYGZdl{})}\PYG{l+s+s2}{\PYGZdq{}}\PYG{p}{)}
\PYG{n}{plt}\PYG{o}{.}\PYG{n}{show}\PYG{p}{(}\PYG{p}{)}
\end{Verbatim}

\includegraphics{{CaptureRate_15_0}.png}

Next we fit the PDF to a single exponential function of the form
\(a\ e^{-t/\tau}\), where a is a scaling factor and \(\tau\) is
the mean time of the distribution (with a rate of 1/\(\tau\)).
This is accomplished with the \code{curve\_fit} function within \code{Scipy}.

\begin{Verbatim}[commandchars=\\\{\}]
\PYG{n}{popt}\PYG{p}{,} \PYG{n}{pcov} \PYG{o}{=} \PYG{n}{curve\PYGZus{}fit}\PYG{p}{(}\PYG{n}{singleExponential}\PYG{p}{,} \PYG{n}{bins}\PYG{p}{[}\PYG{p}{:}\PYG{n+nb}{len}\PYG{p}{(}\PYG{n}{density}\PYG{p}{)}\PYG{p}{]}\PYG{p}{,} \PYG{n}{density}\PYG{p}{,} \PYG{n}{p0}\PYG{o}{=}\PYG{p}{[}\PYG{l+m+mi}{1}\PYG{p}{,} \PYG{n}{np}\PYG{o}{.}\PYG{n}{mean}\PYG{p}{(}\PYG{n}{arrival\PYGZus{}times}\PYG{p}{)}\PYG{p}{]}\PYG{p}{)}
\end{Verbatim}

We then visually check the fit, by superimposing the resulting fit
function over the PDF.

\begin{Verbatim}[commandchars=\\\{\}]
\PYG{n}{plt}\PYG{o}{.}\PYG{n}{semilogy}\PYG{p}{(}
        \PYG{n}{bins}\PYG{p}{[}\PYG{p}{:}\PYG{n+nb}{len}\PYG{p}{(}\PYG{n}{density}\PYG{p}{)}\PYG{p}{]}\PYG{p}{,} \PYG{n}{density}\PYG{p}{,}
        \PYG{n}{linestyle}\PYG{o}{=}\PYG{l+s+s1}{\PYGZsq{}}\PYG{l+s+s1}{None}\PYG{l+s+s1}{\PYGZsq{}}\PYG{p}{,}
        \PYG{n}{marker}\PYG{o}{=}\PYG{l+s+s1}{\PYGZsq{}}\PYG{l+s+s1}{o}\PYG{l+s+s1}{\PYGZsq{}}\PYG{p}{,}
        \PYG{n}{markersize}\PYG{o}{=}\PYG{l+m+mi}{8}\PYG{p}{,}
        \PYG{n}{markeredgecolor}\PYG{o}{=}\PYG{l+s+s1}{\PYGZsq{}}\PYG{l+s+s1}{blue}\PYG{l+s+s1}{\PYGZsq{}}\PYG{p}{,}
        \PYG{n}{markerfacecolor}\PYG{o}{=}\PYG{l+s+s1}{\PYGZsq{}}\PYG{l+s+s1}{None}\PYG{l+s+s1}{\PYGZsq{}}
    \PYG{p}{)}
\PYG{n}{plt}\PYG{o}{.}\PYG{n}{semilogy}\PYG{p}{(}
            \PYG{n}{np}\PYG{o}{.}\PYG{n}{arange}\PYG{p}{(}\PYG{l+m+mf}{0.001}\PYG{p}{,}\PYG{l+m+mf}{0.4}\PYG{p}{,}\PYG{l+m+mf}{0.02}\PYG{p}{)}\PYG{p}{,}
            \PYG{n}{singleExponential}\PYG{p}{(}\PYG{n}{np}\PYG{o}{.}\PYG{n}{arange}\PYG{p}{(}\PYG{l+m+mf}{0.001}\PYG{p}{,}\PYG{l+m+mf}{0.4}\PYG{p}{,}\PYG{l+m+mf}{0.02}\PYG{p}{)}\PYG{p}{,} \PYG{o}{*}\PYG{n}{popt}\PYG{p}{)}\PYG{p}{,}
            \PYG{n}{color}\PYG{o}{=}\PYG{l+s+s1}{\PYGZsq{}}\PYG{l+s+s1}{blue}\PYG{l+s+s1}{\PYGZsq{}}
        \PYG{p}{)}
\PYG{n}{plt}\PYG{o}{.}\PYG{n}{xlim}\PYG{p}{(}\PYG{l+m+mf}{0.005}\PYG{p}{,}\PYG{l+m+mf}{0.3}\PYG{p}{)}
\PYG{n}{plt}\PYG{o}{.}\PYG{n}{ylim}\PYG{p}{(}\PYG{l+m+mf}{0.006}\PYG{p}{,}\PYG{l+m+mi}{25}\PYG{p}{)}
\PYG{n}{plt}\PYG{o}{.}\PYG{n}{xticks}\PYG{p}{(}\PYG{p}{[}\PYG{l+m+mf}{0.05}\PYG{p}{,}\PYG{l+m+mf}{0.15}\PYG{p}{,}\PYG{l+m+mf}{0.25}\PYG{p}{]}\PYG{p}{)}
\PYG{n}{plt}\PYG{o}{.}\PYG{n}{yticks}\PYG{p}{(}\PYG{p}{[}\PYG{l+m+mf}{1e\PYGZhy{}2}\PYG{p}{,}\PYG{l+m+mf}{0.1}\PYG{p}{,}\PYG{l+m+mi}{1}\PYG{p}{,}\PYG{l+m+mf}{1e1}\PYG{p}{]}\PYG{p}{)}
\PYG{n}{plt}\PYG{o}{.}\PYG{n}{axes}\PYG{p}{(}\PYG{p}{)}\PYG{o}{.}\PYG{n}{set\PYGZus{}xlabel}\PYG{p}{(}\PYG{l+s+s2}{\PYGZdq{}}\PYG{l+s+s2}{Arrival Times (s)}\PYG{l+s+s2}{\PYGZdq{}}\PYG{p}{)}
\PYG{n}{plt}\PYG{o}{.}\PYG{n}{axes}\PYG{p}{(}\PYG{p}{)}\PYG{o}{.}\PYG{n}{set\PYGZus{}ylabel}\PYG{p}{(}\PYG{l+s+s2}{\PYGZdq{}}\PYG{l+s+s2}{Density (s\PYGZdl{}\PYGZca{}\PYGZob{}\PYGZhy{}1\PYGZcb{}\PYGZdl{})}\PYG{l+s+s2}{\PYGZdq{}}\PYG{p}{)}
\PYG{n}{plt}\PYG{o}{.}\PYG{n}{show}\PYG{p}{(}\PYG{p}{)}
\end{Verbatim}

\includegraphics{{CaptureRate_19_0}.png}

Finally, we can extract the capture rate (1/\(\tau\)) from the
optimal fit parameters.

\begin{Verbatim}[commandchars=\\\{\}]
\PYG{n}{np}\PYG{o}{.}\PYG{n}{round}\PYG{p}{(}\PYG{p}{[}\PYG{l+m+mi}{1}\PYG{o}{/}\PYG{n}{popt}\PYG{p}{[}\PYG{l+m+mi}{1}\PYG{p}{]}\PYG{p}{]}\PYG{p}{,} \PYG{n}{decimals}\PYG{o}{=}\PYG{l+m+mi}{1} \PYG{p}{)}
\end{Verbatim}

\begin{Verbatim}[commandchars=\\\{\}]
\PYG{n}{array}\PYG{p}{(}\PYG{p}{[} \PYG{l+m+mf}{27.9}\PYG{p}{]}\PYG{p}{)}
\end{Verbatim}


\chapter{Addons}
\label{doc/Addons:addons-page}\label{doc/Addons:addons}\label{doc/Addons::doc}\label{doc/Addons:mosaic-page-on-github}
The output of \emph{MOSAIC} is often processed further to generate plots or performe more sophisticated analysis. We facilitate this process by providing addon packages that make it easy to import the \href{http://www.sqlite.org/}{SQLite} database generated by a \emph{MOSAIC} analysis into \DUspan{xref,std,std-ref}{mathematica-addons-sec}, \DUspan{xref,std,std-ref}{matlab-addons-sec} or \DUspan{xref,std,std-ref}{igor-addons-sec}. The interfaces for these programs are described in more detail in this section.


\section{Mathematica}
\label{doc/Addons:mathematica}\label{doc/Addons:mathematica-addons-sec}

\subsection{Installation}
\label{doc/Addons:installation}
The analysis output generated by \emph{MOSAIC} can be imported into \href{http://www.wolfram.com/mathematica/}{Mathematica} for further processing. This accomplished with two packages: the low level \DUspan{xref,std,std-ref}{mathematicaMosaicutilsSec} and \DUspan{xref,std,std-ref}{mathematica-mosaicanalysis-sec}, which contains additional analysis routines. The addon package must first be installed to one of the locations in the \href{http://www.wolfram.com/mathematica/}{Mathematica} path. Alternatively, the required package files can be installed to the \emph{Applications} folder using \emph{setuptools} on Mac OS X and Linux by issuing the command below in the root folder of the \emph{MOSAIC} code. Instructions for installing the package files for Windows are available \href{http://reference.wolfram.com/language/tutorial/ConfigurationFiles.html}{here}.

\begin{Verbatim}[commandchars=\\\{\}]
\PYG{g+go}{python setup.py mosaic\PYGZus{}addons \PYGZhy{}\PYGZhy{}mathematica}
\end{Verbatim}


\subsection{MosaicUtils}
\label{doc/Addons:mosaicutils}\label{doc/Addons:mathematicamosaicutilssec}
\emph{MosaicUtils} provides low level functions to interact with a database output by \emph{MOSAIC}. \emph{MosaicUtils} can use a native Mathematica (\textbf{slower, default}) or Python (\textbf{faster, but experimental}) backend to query databases output by \emph{MOSAIC}.

\begin{notice}{note}{Note:}
To select the Python backend, please call the \emph{SetQueryBackend} function as described below. If you use a virtual environment with Python, please call the \emph{SetVirtualEnv} function after you install this addon.
\end{notice}

\textbf{SetQueryBackend}{[}\emph{backend}{]}
\begin{quote}\begin{description}
\item[{Args}] \leavevmode\begin{itemize}
\item {} 
\emph{backend} :   select the \emph{Mathematica} (default) or \emph{Python} (faster, experimental) backend to run SQLite queries.

\end{itemize}

\item[{Returns}] \leavevmode
None

\end{description}\end{quote}

\textbf{ReadQueryBackend}{[}{]}
\begin{quote}\begin{description}
\item[{Args}] \leavevmode
None

\item[{Returns}] \leavevmode\begin{itemize}
\item {} 
The \emph{backend} used to run SQLite queries.

\end{itemize}

\end{description}\end{quote}

\textbf{SetVirtualEnv}{[}\emph{virtualenv}{]}
\begin{quote}\begin{description}
\item[{Args}] \leavevmode\begin{itemize}
\item {} 
\emph{virtualenv} :        name of the virtual environment configured for use with MOSAIC

\end{itemize}

\item[{Returns}] \leavevmode
None

\end{description}\end{quote}

\textbf{PrintMDKeys}{[}\emph{dbfile}{]}

Returns a list of column headings from the \emph{metadata} table.
\begin{quote}\begin{description}
\item[{Args}] \leavevmode\begin{itemize}
\item {} 
\emph{dbfile} :    full path to the database file

\end{itemize}

\item[{Returns}] \leavevmode
A list of column names in the table \emph{metadata}.

\end{description}\end{quote}

\textbf{PrintMDTypes}{[}\emph{dbfile}{]}

Returns a list of column types from the \emph{metadata} table.
\begin{quote}\begin{description}
\item[{Args}] \leavevmode\begin{itemize}
\item {} 
\emph{dbfile} :    full path to the database file

\end{itemize}

\item[{Returns}] \leavevmode
A list of column types in the table \emph{metadata}.

\end{description}\end{quote}

\textbf{QueryDB}{[}\emph{dbfile}, \emph{query}{]}

Queries the \emph{metadata} table using the supplied SQL query.
\begin{quote}\begin{description}
\item[{Args}] \leavevmode\begin{itemize}
\item {} 
\emph{dbfile} :    full path to the database file

\item {} 
\emph{query}  :    a SQL query

\end{itemize}

\item[{Returns}] \leavevmode
A nested list of query results.

\end{description}\end{quote}

\textbf{PlotEvents}{[}\emph{dbfile}, \emph{FsKHz}, \emph{nEvents}{]}

Plot the event-time series if stored in the database (see the {\hyperref[doc/settingsFile:settings\string-page]{\emph{Settings File}}} section for details on saving time-series to the analysis output).
\begin{quote}\begin{description}
\item[{Args}] \leavevmode\begin{itemize}
\item {} 
\emph{dbfile} :    full path to the database file

\item {} 
\emph{FsKHz}  :    sampling frequency in kHz.

\item {} 
\emph{nEvents} :   (optional) limit the plot to the first \code{n} entries in the database

\end{itemize}

\item[{Returns}] \leavevmode
A dynamic object that allows the user to browse event time-series and fits.

\end{description}\end{quote}

\textbf{GetAnalysisAlgorithm}{[}\emph{db}{]}

Returns the analysis algorithm used to process the current data set.
\begin{quote}\begin{description}
\item[{Args}] \leavevmode\begin{itemize}
\item {} 
\emph{db}  :               full path to a database file

\end{itemize}

\item[{Returns}] \leavevmode
Algorithm used to analyze data.

\end{description}\end{quote}

\textbf{MosaicUtils Examples}

Once installed as described above, \emph{MosaicUtils} must be imported as shown below.

\begin{Verbatim}[commandchars=\\\{\}]
\PYG{n}{In}\PYG{p}{[}\PYG{l+m+mi}{1}\PYG{p}{]}\PYG{o}{=}\PYG{+w}{ }\PYG{o}{\PYGZlt{}}\PYG{o}{\PYGZlt{}}\PYG{n+nn}{MosaicUtils{}`}
\end{Verbatim}

\href{http://en.wikipedia.org/wiki/SQL}{SQL} queries require the exact column names when querying data from a table (see {\hyperref[doc/DatabaseStructure:database\string-page]{\emph{Database Structure and Query Syntax}}}). Column names in the \emph{metadata} table, which stores the main results from the analaysis can be retrieved using the \emph{PrintMDKeys} function as shown below. In this example, the column names returned correspond to an analysis performed using the \emph{adept2State} algorithm.

\begin{Verbatim}[commandchars=\\\{\}]
\PYG{n}{In}\PYG{p}{[}\PYG{l+m+mi}{2}\PYG{p}{]}\PYG{o}{=}\PYG{+w}{ }\PYG{n}{PrintMDKeys}\PYG{p}{[}\PYG{l+s}{\PYGZdq{}\PYGZlt{}mosaicroot\PYGZgt{}/data/eventMD\PYGZhy{}PEG29\PYGZhy{}Reference.sqlite\PYGZdq{}}\PYG{p}{]}

\PYG{n}{Out}\PYG{p}{[}\PYG{l+m+mi}{2}\PYG{p}{]}\PYG{o}{=}\PYG{+w}{ }\PYG{p}{\PYGZob{}}\PYG{l+s}{\PYGZdq{}recIDX\PYGZdq{}}\PYG{p}{,}\PYG{+w}{ }\PYG{l+s}{\PYGZdq{}ProcessingStatus\PYGZdq{}}\PYG{p}{,}\PYG{+w}{ }\PYG{l+s}{\PYGZdq{}OpenChCurrent\PYGZdq{}}\PYG{p}{,}
\PYG{+w}{        }\PYG{l+s}{\PYGZdq{}BlockedCurrent\PYGZdq{}}\PYG{p}{,}\PYG{+w}{ }\PYG{l+s}{\PYGZdq{}EventStart\PYGZdq{}}\PYG{p}{,}\PYG{+w}{ }\PYG{l+s}{\PYGZdq{}EventEnd\PYGZdq{}}\PYG{p}{,}
\PYG{+w}{        }\PYG{l+s}{\PYGZdq{}BlockDepth\PYGZdq{}}\PYG{p}{,}\PYG{+w}{ }\PYG{l+s}{\PYGZdq{}ResTime\PYGZdq{}}\PYG{p}{,}\PYG{+w}{ }\PYG{l+s}{\PYGZdq{}RCConstant\PYGZdq{}}\PYG{p}{,}\PYG{+w}{ }\PYG{l+s}{\PYGZdq{}AbsEventStart\PYGZdq{}}\PYG{p}{,}
\PYG{+w}{        }\PYG{l+s}{\PYGZdq{}ReducedChiSquared\PYGZdq{}}\PYG{p}{,}\PYG{+w}{ }\PYG{l+s}{\PYGZdq{}TimeSeries\PYGZdq{}}
\PYG{+w}{        }\PYG{p}{\PYGZcb{}}
\end{Verbatim}

The \emph{MosaicUtils} package allows the output of \emph{MOSAIC} to be queried just like from Python. This accomplished using the \emph{QueryDB} function. In the example below, we retrieve a column that returns the start time of the first 10 entries in the \emph{metadata} table that have their \emph{ProcessingStatus} set to \emph{normal}. The results are then returned in a standard list. Note that \emph{QueryDB} accepts a standard \href{http://en.wikipedia.org/wiki/SQL}{SQL} query as described in more detail in the {\hyperref[doc/DatabaseStructure:database\string-page]{\emph{Database Structure and Query Syntax}}} section.

\begin{Verbatim}[commandchars=\\\{\}]
\PYG{n}{In}\PYG{p}{[}\PYG{l+m+mi}{3}\PYG{p}{]}\PYG{o}{=}\PYG{+w}{ }\PYG{n}{QueryDB}\PYG{p}{[}
\PYG{+w}{        }\PYG{l+s}{\PYGZdq{}\PYGZlt{}mosaicroot\PYGZgt{}/data/eventMD\PYGZhy{}PEG29\PYGZhy{}Reference.sqlite\PYGZdq{}}\PYG{p}{,}
\PYG{+w}{        }\PYG{l+s}{\PYGZdq{}select AbsEventStart from metadata where ProcessingStatus=\PYGZsq{}normal\PYGZsq{} limit 10\PYGZdq{}}
\PYG{+w}{        }\PYG{p}{]}

\PYG{n}{Out}\PYG{p}{[}\PYG{l+m+mi}{3}\PYG{p}{]}\PYG{o}{=}\PYG{+w}{ }\PYG{p}{\PYGZob{}}
\PYG{+w}{        }\PYG{p}{\PYGZob{}}\PYG{l+m+mf}{1.84376}\PYG{p}{\PYGZcb{}}\PYG{p}{,}\PYG{+w}{ }\PYG{p}{\PYGZob{}}\PYG{l+m+mf}{4.54439}\PYG{p}{\PYGZcb{}}\PYG{p}{,}\PYG{+w}{ }\PYG{p}{\PYGZob{}}\PYG{l+m+mf}{5.26933}\PYG{p}{\PYGZcb{}}\PYG{p}{,}\PYG{+w}{ }\PYG{p}{\PYGZob{}}\PYG{l+m+mf}{6.01253}\PYG{p}{\PYGZcb{}}\PYG{p}{,}\PYG{+w}{ }\PYG{p}{\PYGZob{}}\PYG{l+m+mf}{6.80369}\PYG{p}{\PYGZcb{}}\PYG{p}{,}
\PYG{+w}{        }\PYG{p}{\PYGZob{}}\PYG{l+m+mf}{8.48988}\PYG{p}{\PYGZcb{}}\PYG{p}{,}\PYG{+w}{ }\PYG{p}{\PYGZob{}}\PYG{l+m+mf}{10.841}\PYG{p}{\PYGZcb{}}\PYG{p}{,}\PYG{+w}{ }\PYG{p}{\PYGZob{}}\PYG{l+m+mf}{11.2246}\PYG{p}{\PYGZcb{}}\PYG{p}{,}\PYG{+w}{ }\PYG{p}{\PYGZob{}}\PYG{l+m+mf}{13.2892}\PYG{p}{\PYGZcb{}}\PYG{p}{,}\PYG{+w}{ }\PYG{p}{\PYGZob{}}\PYG{l+m+mf}{16.3983}\PYG{p}{\PYGZcb{}}
\PYG{+w}{        }\PYG{p}{\PYGZcb{}}
\end{Verbatim}

Finally, the addon package allows us to plot individual events if time-series data was stored in the database. This is accomplished using the \emph{PlotEvents} function, and provides a convenient tool to visually inspect the output of a \emph{MOSAIC} analysis. In the example below, we inspect the events stored in the reference PEG28 data set included with \emph{MOSAIC}. \emph{PlotEvents} returns a dynamic object that allows the user to inspect all the events in a database. An event that was properly characterized by the code is plotted with \emph{blue} markers (\emph{left}). The plot is overlaid with the optimized fit function (\emph{black}) and an idealized pulse (\emph{red dashed}). Events that were not properly fit are plotted with \emph{red} markers (\emph{right}).

\begin{Verbatim}[commandchars=\\\{\}]
\PYG{n}{In}\PYG{p}{[}\PYG{l+m+mi}{4}\PYG{p}{]}\PYG{o}{=}\PYG{+w}{ }\PYG{n}{PlotEvents}\PYG{p}{[}\PYG{l+s}{\PYGZdq{}\PYGZlt{}mosaicroot\PYGZgt{}/data/eventMD\PYGZhy{}PEG29\PYGZhy{}Reference.sqlite\PYGZdq{}}\PYG{p}{,}\PYG{+w}{ }\PYG{l+m+mi}{500}\PYG{p}{]}
\end{Verbatim}
\begin{figure}[htbp]\begin{flushleft}

\includegraphics[width=0.900\linewidth]{{mathematica-PlotEvents}.png}
\end{flushleft}\end{figure}


\subsection{MosaicAnalysis}
\label{doc/Addons:mosaicanalysis}\label{doc/Addons:mathematica-mosaicanalysis-sec}
\emph{MosaicAnalysis} builds on the \emph{MosaicUtils} package and provides basic analysis functions such as estimating the capture rate of molecules partitioning into a channel, or the mean residence time. Additionally, new functionality can be created by combining the functions defined below.

\textbf{ScaledSingleExponentialFit}{[}\emph{hist}, \emph{lambda}, \emph{lambda0}{]}

Scale the histogram with the number of counts in the first bin. Fit a single exponential of the form \emph{a exp(-t/tau)} to the scaled histogram.
\begin{quote}\begin{description}
\item[{Args}] \leavevmode\begin{itemize}
\item {} 
\emph{hist}        : a histogram with format \{\{bin1, counts1\}, \{bin2, counts2\}, ..., \{binN,countsN\}\}

\item {} 
\emph{lambda}      : parameter of the distribution. This symbol must be passed from the calling function.

\item {} 
\emph{lambda0}     : initial guess for \emph{lambda}.

\end{itemize}

\end{description}\end{quote}

\textbf{PlotScaledSingleExponentialFit}{[}\emph{hist}, \emph{ftfunc}, \emph{plotopts}{]}
\begin{quote}\begin{description}
\item[{Args}] \leavevmode\begin{itemize}
\item {} 
\emph{hist}        : a histogram with format \{\{bin1, counts1\}, \{bin2, counts2\}, ..., \{binN,countsN\}\}

\item {} 
\emph{ftfunc}      : an optimized fit, defined as a virtual function.

\item {} 
\emph{plotopts}: a list of options to control the plot output.

\end{itemize}

\end{description}\end{quote}

\textbf{CaptureRate}{[}\emph{arrtimes}, \emph{stime}, \emph{etime}, \emph{nbins}, \emph{plotopts}{]}

Estimate the capture rate of molecules by a channel by analyzing the arrival times of individual molecules. The arrival times of a stochastic process follow a single exponential distribution. This function first calculate a histogram of arrival times and then fits a single exponential function to the data.
\begin{quote}\begin{description}
\item[{Args}] \leavevmode\begin{itemize}
\item {} 
\emph{arrtimes} : a list of absolute start times (\emph{AbsEventStart}) queried from a database.

\item {} 
\emph{stime}    : lower limit of the arrival times distribution

\item {} 
\emph{etime}    : upper limit of the arrival times distribution

\item {} 
\emph{nbins}        : number of bins

\item {} 
\emph{plotopts} : a list of options to control the plot output.

\end{itemize}

\item[{Returns}] \leavevmode
The mean capture rate, a plot of the underlying distribution of arrival times, the arrival times distribution and the optimized fit function.

\end{description}\end{quote}

\textbf{ArrivalTimes}{[}\emph{abseventstart}{]}

Calculate the arrival times from a list of the absolute start time of each event in a data set.
\begin{quote}\begin{description}
\item[{Args}] \leavevmode\begin{itemize}
\item {} 
\emph{abseventstart}       : a list of absolute start times (\emph{AbsEventStart}) queried from a database.

\end{itemize}

\item[{Returns}] \leavevmode
A list of arrival times.

\end{description}\end{quote}

\textbf{MosaicAnalysis Examples}

\begin{Verbatim}[commandchars=\\\{\}]
\PYG{n}{In}\PYG{p}{[}\PYG{l+m+mi}{1}\PYG{p}{]}\PYG{o}{=}\PYG{+w}{ }\PYG{o}{\PYGZlt{}}\PYG{o}{\PYGZlt{}}\PYG{n+nn}{MosaicUtils{}`}
\PYG{n}{In}\PYG{p}{[}\PYG{l+m+mi}{2}\PYG{p}{]}\PYG{o}{=}\PYG{+w}{ }\PYG{o}{\PYGZlt{}}\PYG{o}{\PYGZlt{}}\PYG{n+nn}{MosaicAnalysis{}`}
\end{Verbatim}

In the following example, we estimate the capture rate of PEG28 from the reference data set included with the \emph{MOSAIC} source. The first argument fo \emph{CaptureRate} is a list of the absolute start time of each event in the database. This data can be obtained using the query shown below. The remaining arguments to \emph{CaptureRate} define the parameters of the arrival times distribution, the lower and upper limit of the arrival times and the number of bins. The function returns the mean capture rate and standard error, a plot that shows the underlying arrival times distribution, raw data used to generate the capture rate histogram, and a pure best-fit function.

\begin{Verbatim}[commandchars=\\\{\}]
In[3]= CaptureRate[
        QueryDB[
        \PYGZdq{}\PYGZlt{}mosaicroot\PYGZgt{}/data/eventMD\PYGZhy{}PEG29\PYGZhy{}Reference.sqlite\PYGZdq{},
        \PYGZdq{}select AbsEventStart from metadata where
                ProcessingStatus=\PYGZsq{}normal\PYGZsq{} and ResTime \PYGZgt{} 0.01\PYGZdq{}
        ], 0.0, 0.05, 50
]
\end{Verbatim}
\begin{figure}[htbp]\begin{flushleft}

\includegraphics[width=0.500\linewidth]{{mathematica-CaptureRate}.png}
\end{flushleft}\end{figure}

The capture rate plot above can be formatted by supplying the optional \emph{plotopts} argument, which uses standard \href{http://www.wolfram.com/mathematica/}{Mathematica} plot options, as seen in the example below.  This is particulary helpful to customize the output of the plot,, for example for publication ready graphics.

\begin{Verbatim}[commandchars=\\\{\}]
       In[4]= CaptureRate[
               QueryDB[
               \PYGZdq{}\PYGZlt{}mosaicroot\PYGZgt{}/data/eventMD\PYGZhy{}PEG29\PYGZhy{}Reference.sqlite\PYGZdq{},
               \PYGZdq{}select AbsEventStart from metadata where
                       ProcessingStatus=\PYGZsq{}normal\PYGZsq{} and ResTime \PYGZgt{} 0.01\PYGZdq{}
               ], 0.0, 0.05, 50,
               \PYGZob{}Frame \PYGZhy{}\PYGZgt{} True, FrameLabel \PYGZhy{}\PYGZgt{} \PYGZob{}Style[\PYGZdq{}t (ms)\PYGZdq{}, 16], \PYGZdq{}\PYGZdq{}\PYGZcb{},
FrameTicks \PYGZhy{}\PYGZgt{} \PYGZob{}\PYGZob{}\PYGZob{}0.05, 0.1, 0.5, 1\PYGZcb{}, None\PYGZcb{}, \PYGZob{}\PYGZob{}0.005, 0.015, 0.025\PYGZcb{},
   None\PYGZcb{}\PYGZcb{}, FrameTicksStyle \PYGZhy{}\PYGZgt{} 14, PlotStyle \PYGZhy{}\PYGZgt{} \PYGZob{}Black, Thick\PYGZcb{},
ImageSize \PYGZhy{}\PYGZgt{} 400\PYGZcb{}
       ]
\end{Verbatim}
\begin{figure}[htbp]\begin{flushleft}

\includegraphics[width=0.500\linewidth]{{mathematica-CaptureRate-Formatted}.png}
\end{flushleft}\end{figure}


\section{Matlab}
\label{doc/Addons:matlab-addons-sec}\label{doc/Addons:id2}
The SQLite database output by MOSAIC can be further processed using \href{http://www.mathworks.com/products/matlab/}{MATLAB}.  The data can then be stored in an array in the \href{http://www.mathworks.com/products/matlab/}{MATLAB} Workspace, and then manipulated as desired.

The features, of opening, querying, and storing as an array, are made available in the \href{http://www.mathworks.com/products/matlab/}{MATLAB} script openandquery.m. The script does not use the \href{http://www.mathworks.com/products/matlab/}{MATLAB} Database Manager GUI, a part of the Database Toolbox, which requires a paid license.  Instead, an open-source alternative, \href{http://sourceforge.net/projects/mksqlite/}{mksqlite}, an interface between \href{http://www.mathworks.com/products/matlab/}{MATLAB} and \href{http://www.sqlite.org/}{SQLite} is used.

This section of the manual provides information on how to set up the mksqlite- package for use with \href{http://www.mathworks.com/products/matlab/}{MATLAB}, and how to use the openandquery.m script.

All code has been successfully tested with MATLAB 2013a, MATLAB 2014a, G++ 4.7 in Ubuntu 14.04 LTS, and Windows Visual C++ 2010.  Also, \href{http://www.sqlite.org/}{SQLite} must be installed prior to performing the following steps.


\subsection{mksqlite Documentation}
\label{doc/Addons:mksqlite-documentation}
Information about \href{http://sourceforge.net/projects/mksqlite/}{mksqlite}, such as function calls and examples, is available in the MKSQLITE: A \href{http://www.mathworks.com/products/matlab/}{MATLAB} Interface to SQLite documentation.


\subsection{Installing mksqlite in Ubuntu 14.04 LTS}
\label{doc/Addons:installing-mksqlite-in-ubuntu-14-04-lts}
Download the latest \href{http://sourceforge.net/projects/mksqlite/}{mksqlite} source files from \href{http://sourceforge.net/projects/mksqlite/files/}{SourceForge}
Unzip the files to a folder, and note the path to that folder (e.g., /home/mksqlitefolder)
Open \href{http://www.mathworks.com/products/matlab/}{MATLAB}, and change the current path to that of the mksqlite folder
In the Command Window, type \emph{buildit}, and press Enter to build mksqlite (this will run the buildit.m script).  If the MEX files do not build, one of the following two problems may be why:
i) a compiler may not be installed -- see the \href{http://www.mathworks.com/support/compilers/R2014b/index.html}{MathWorks page on Supported and Compatible Compilers} to select and install a compiler, or ii) errors are generated during compilation of mksqlite.cpp.  In the latter case, see the “How to build mksqlite MEX file mksqlite.mexa64 in Linux?” thread in the \href{http://www.mathworks.com/matlabcentral/answers/86590-how-to-build-mksqlite-mex-file-mksqlite-mexa64-in-linux}{MathWorks MATLAB Answers forum}.
If the build proceeds without errors, you will first see the notification “compiling release version of mksqlite...” in the Command Window, followed by “completed.”

\begin{notice}{note}{Note:}
GCC/G++ Version (in Linux)

You may have to install a version of GCC/G++ that is compatible with with your specific MATLAB release.  If so, check out the linked discussion thread on MATLAB Central on how to \href{http://www.mathworks.com/matlabcentral/answers/137228-setup-mex-compiler-for-r2014a-for-linux}{set up a MEX Compiler}.
\end{notice}


\subsection{Installing mksqlite in Windows 7}
\label{doc/Addons:installing-mksqlite-in-windows-7}
The installation steps are essentially the same as for Ubuntu, except a different compiler (e.g, contained in Windows SDK 7) may instead have to be installed.  If the SDK installer say it cannot proceed, quit the installation, uninstall previous instances of Microsoft Visual C++ 2010, and then install Windows SDK 7 again.


\subsection{Opening, Querying, and Closing the MOSAIC Output Database}
\label{doc/Addons:opening-querying-and-closing-the-mosaic-output-database}
The MATLAB script openandquery.m contains all of the commands to:
Open a MOSAIC database (e.g., eventMD-PEG29-Reference.sqlite)
Query the database
Save queried data elements into a structure
Close the database
Convert the structure into a multi-dimensional array, that can be easily manipulated in \href{http://www.mathworks.com/products/matlab/}{MATLAB}

Two changes must be made to the openandquery m-file by the end-user:
The path to the database file must be changed for each database you wish to access.  An example path in Linux would be /home/Data/eventMD-PEG29-Reference.sqlite, and in Windows C:\textbackslash{}Data\textbackslash{}eventMD-PEG29-Reference.sqlite.
The query string can be changed as needed.  More information about queries in available in the {\hyperref[doc/DatabaseStructure:database\string-page]{\emph{Database Structure and Query Syntax}}} section.


\subsection{Example}
\label{doc/Addons:example}
The reference database file provided with MOSAIC is \emph{eventMD-PEG29-Reference.sqlite}, located in the data folder of the source code root directory.  This database contains the results of an analysis performed using the \emph{adept2State} and consists of the data fields:

\begin{Verbatim}[commandchars=\\\{\}]
\PYG{p}{\PYGZob{}}\PYG{n+nx}{recIDX}\PYG{p}{,} \PYG{n+nx}{ProcessingStatus}\PYG{p}{,} \PYG{n+nx}{OpenChCurrent}\PYG{p}{,} \PYG{n+nx}{BlockedCurrent}\PYG{p}{,} \PYG{n+nx}{EventStart}\PYG{p}{,} \PYG{n+nx}{EventEnd}\PYG{p}{,} \PYG{n+nx}{BlockDepth}\PYG{p}{,} \PYG{n+nx}{ResTime}\PYG{p}{,} \PYG{n+nx}{RCConstant}\PYG{p}{,} \PYG{n+nx}{AbsEventStart}\PYG{p}{,} \PYG{n+nx}{ReducedChiSquared}\PYG{p}{,} \PYG{n+nx}{and} \PYG{n+nx}{TimeSeries}\PYG{p}{\PYGZcb{}}
\end{Verbatim}

In the openandquery script modify line 20 by typing in, within the quotes, the correct path to the database file.

\begin{Verbatim}[commandchars=\\\{\}]
\PYG{n}{dbname} \PYG{p}{=} \PYG{l+s}{\PYGZsq{}}\PYG{l+s}{/home/Data/eventMD\PYGZhy{}PEG29\PYGZhy{}Reference.sqlite\PYGZsq{}}\PYG{p}{;}
\end{Verbatim}

The query in line 23 is to read the names of all fields in the database.  The names, along with column ID, and data type, are stored in the structure fieldnames.  You may double-click on the variable fieldnames in the Workspace, which will open the structure for you to read the field names in which you are interested.

\begin{Verbatim}[commandchars=\\\{\}]
\PYG{n}{fieldnames} \PYG{p}{=} \PYG{n}{mksqlite}\PYG{p}{(}\PYG{l+s}{\PYGZsq{}}\PYG{l+s}{PRAGMA table\PYGZus{}info(metadata)\PYGZsq{}}\PYG{p}{)}\PYG{p}{;}
\end{Verbatim}

Next, modify line 24 to include the query.  In this example we want to select (and later manipulate) the data stored in the fields AbsEventStart and BlockDepth.  This is where mksqlite comes in: the query are arguments to the mksqlite() function.  For more information about using the mksqlite.m function check out the mksqlite documentation.

\begin{Verbatim}[commandchars=\\\{\}]
\PYG{n}{querytemp} \PYG{p}{=} \PYG{n}{mksqlite}\PYG{p}{(}\PYG{l+s}{\PYGZsq{}}\PYG{l+s}{select AbsEventStart, BlockDepth from metadata\PYGZsq{}}\PYG{p}{)}\PYG{p}{;}
\end{Verbatim}

No other changes are required.  Run the script.  The queried data are stored in the variable data, seen in the MATLAB Workspace (with value 418x2 double).  This variable is a 2-column matrix.  The first column contains all 418 data elements of the field AbsEventStart, and the second column contains all elements of the field BlockDepth. Note that the query above can be replaced with any standard SQL query as outlined in the \DUspan{xref,std,std-ref}{working-with-sqlite-sec} section.


\section{IGOR}
\label{doc/Addons:igor-addons-sec}\label{doc/Addons:id26}
Data extraction in \href{http://www.wavemetrics.com/products/igorpro/igorpro.htm}{IGOR} is a work in progress, but a number of users have found a successful route to querying the data and manipulating it in the \href{http://www.wavemetrics.com/products/igorpro/igorpro.htm}{IGOR} environment.  The installation and setup for these features requires an understanding of setup and use of ODBC drivers as well as rudimentary programming within the \href{http://www.wavemetrics.com/products/igorpro/igorpro.htm}{IGOR} environment.  To date, this has been tested on Mac OS X 10.9.  Details may vary for other systems.


\subsection{Activating SQL Database Access in IGOR}
\label{doc/Addons:activating-sql-database-access-in-igor}
Database functionality in IGOR is preloaded, but not activated for use in the standard installation of \href{http://www.wavemetrics.com/products/igorpro/dataaccess/sql.htm}{SQL.xop}.  To activate this feature follow the instructions detailed in the ``Igor Pro Folder/More Extensions/utilities/SQL Help.ihf''.  The next few steps are reproduced from the \href{http://www.wavemetrics.com/products/igorpro/igorpro.htm}{IGOR} instructions.   First, activate the  step in the activation process is open the folder, ``Igor Pro Folder/More Extensions/utilities'' and create an alias for \href{http://www.wavemetrics.com/products/igorpro/dataaccess/sql.htm}{SQL.xop}.  Then move the alias to ``Igor Pro/Igor Extensions'' or a similar folder that is in the search path for \href{http://www.wavemetrics.com/products/igorpro/igorpro.htm}{IGOR}.  It may be necessary to delete the ``alias'' text from the file name for functionality.  Restart \href{http://www.wavemetrics.com/products/igorpro/igorpro.htm}{IGOR} to activate.

\href{http://www.wavemetrics.com/products/igorpro/igorpro.htm}{IGOR} relies on an external ODBC driver for database access.  Depending on the operating system, it may be necessary to install a stand alone ODBC driver administrator package. First check your machine for the \emph{ODBC administrator.app} in the \emph{\textasciitilde{}/Applications/Utilities} folder.  If not present \href{http://support.apple.com/kb/DL895}{ODBC administrator} can be downloaded directly from the Apple support pages.  To test the functionality, it is useful to follow the \emph{Installing MySQL ODBC Driver...} instructions on the \href{http://www.wavemetrics.com/products/igorpro/igorpro.htm}{IGOR} help page.  The MySQL drivers are not necessary for functionally within \emph{MOSAIC}.

With the ODBC administrator program installed, the next step is to install the \href{http://www.ch-werner.de/sqliteodbc/}{SQLite driver for IGOR} necessary to interface with the database.  Once downloaded run the installation package in ``sqlite3-odbc-0.93.dmg'' and follow the setup instructions within the disk image.  The driver should be ready to use within \href{http://www.wavemetrics.com/products/igorpro/igorpro.htm}{IGOR}.

\begin{notice}{hint}{Hint:}
The \href{http://www.wavemetrics.com/products/igorpro/igorpro.htm}{IGOR} addon installation (described above) can be activated automatically on \emph{Mac OS X} by issuing the command \code{python setup.py mosaic\_addons -{-}igor} from the \emph{MOSAIC} root directory. Note that administrator privileges are required.
\end{notice}


\subsection{Simple Database Query in IGOR}
\label{doc/Addons:simple-database-query-in-igor}
\href{http://www.wavemetrics.com/products/igorpro/igorpro.htm}{IGOR} operates on databases with a single High Level operation command.  This one command handles the database connection, query, export of data and closing of the database in one simple function or macro.  To access this functionality, first open the procedure window and create the following function:

\begin{Verbatim}[commandchars=\\\{\}]
\PYG{n+nd}{\PYGZsh{}include} \PYGZlt{}SQLUtils\PYGZgt{}

\PYG{k+kr}{Function} QuerySQLData()

        \PYG{k+kt}{String} connectionStr= \PYG{l+s}{\PYGZdq{}DRIVER=\PYGZob{}SQLite3 Driver\PYGZcb{};DATABASE=\PYGZsq{}database path\PYGZsq{};\PYGZdq{}}
        \PYG{k+kt}{String} statement = \PYG{l+s}{\PYGZdq{}select Blockdepth, ResTime from metadata where ProcessingStatus =\PYGZsq{}normal\PYGZsq{}\PYGZdq{}}

        SQLHighLevelOp/CSTR=\PYGZob{}connectionStr, SQL\PYGZus{}Driver\PYGZus{}COMPLETE\PYGZcb{}/O/E=1 statement
\PYG{k+kr}{End}
\end{Verbatim}

Running this function will extract all normal events and create two waves containing the Blockade depth and Residence time of the events in sequence for further processing in \href{http://www.wavemetrics.com/products/igorpro/igorpro.htm}{IGOR}.  Two \href{http://www.wavemetrics.com/products/igorpro/igorpro.htm}{IGOR} functions are included in the \emph{/addon/IGOR/} folder that import the data into \href{http://www.wavemetrics.com/products/igorpro/igorpro.htm}{IGOR} waves for further processing. To open these functions to run, simply double click the file and the procedures will be opened in a new \href{http://www.wavemetrics.com/products/igorpro/igorpro.htm}{IGOR} project.  Once open, the procedure file can be compiled within \href{http://www.wavemetrics.com/products/igorpro/igorpro.htm}{IGOR} to enable the code.  A new menu ``Mosaic'' should then appear in the title bar within \href{http://www.wavemetrics.com/products/igorpro/igorpro.htm}{IGOR}.  A function ``Fetch SQL data'' will bring up a dialog box to manually enter a search string.  After entering the string and clicking continue, you will be prompted to locate the database file you wish to access.  The data will be imported into waves with the name automatically imported from the database.  \emph{Warning:} this will overwrite any existing data that is called by identical wave names.


\chapter{API Documentation}
\label{doc/apidocs:id72}\label{doc/apidocs:api-docs-page}\label{doc/apidocs::doc}\label{doc/apidocs:api-documentation}
\emph{MOSAIC} is designed using object oriented tools, which makes it easy to extend. The API documentation provides class level descriptions of the different modules that can be used in customized code. Meta-Classes (in \emph{blue} below) define interfaces to five key parts of MOSAIC: time-series IO ({\hyperref[api\string-doc/mosaic.meta:mosaic.metaTrajIO.metaTrajIO]{\emph{\code{metaTrajIO}}}}), time-series filtering ({\hyperref[api\string-doc/mosaic.meta:mosaic.metaIOFilter.metaIOFilter]{\emph{\code{metaIOFilter}}}}), analysis output ({\hyperref[api\string-doc/mosaic.meta:mosaic.metaMDIO.metaMDIO]{\emph{\code{metaMDIO}}}}), event partition and segmenting ({\hyperref[api\string-doc/mosaic.meta:mosaic.metaEventPartition.metaEventPartition]{\emph{\code{metaEventPartition}}}}), and event processing ({\hyperref[api\string-doc/mosaic.meta:mosaic.metaEventProcessor.metaEventProcessor]{\emph{\code{metaEventProcessor}}}}). Sub-classing any of these meta classes and implementing their interface functions allows one to extend \emph{MOSAIC} while maintaining compatibility with other parts of the program. The diagram below shows the class inheritence in \emph{MOSAIC}, with top-level classes in \emph{gray}.


\section{\emph{MOSAIC} Modules}
\label{doc/apidocs:projname-modules}

\subsection{Top-Level Interfaces}
\label{api-doc/mosaic:top-level-interfaces}\label{api-doc/mosaic::doc}

\subsubsection{mosaic.SingleChannelAnalysis module}
\label{api-doc/mosaic:module-mosaic.SingleChannelAnalysis}\label{api-doc/mosaic:mosaic-singlechannelanalysis-module}\index{mosaic.SingleChannelAnalysis (module)}
Top level module to run a single channel analysis.
\begin{quote}\begin{description}
\item[{Created}] \leavevmode
05/15/2014

\item[{Author}] \leavevmode
Arvind Balijepalli \textless{}\href{mailto:arvind.balijepalli@nist.gov}{arvind.balijepalli@nist.gov}\textgreater{}

\item[{License}] \leavevmode
See LICENSE.TXT

\item[{ChangeLog}] \leavevmode
\end{description}\end{quote}

\begin{DUlineblock}{0em}
\item[] 5/15/14         AB      Initial version
\end{DUlineblock}
\index{SingleChannelAnalysis (class in mosaic.SingleChannelAnalysis)}

\begin{fulllineitems}
\phantomsection\label{api-doc/mosaic:mosaic.SingleChannelAnalysis.SingleChannelAnalysis}\pysiglinewithargsret{\strong{class }\code{mosaic.SingleChannelAnalysis.}\bfcode{SingleChannelAnalysis}}{\emph{dataPath}, \emph{trajDataHnd}, \emph{dataFilterHnd}, \emph{eventPartitionHnd}, \emph{eventProcHnd}}{}
Bases: \href{http://docs.python.org/library/functions.html\#object}{\code{object}}

Run a single channel analysis. This is the entry point class for the analysis.
\begin{quote}\begin{description}
\item[{Parameters}] \leavevmode\begin{itemize}
\item {} 
\emph{dataPath} : full path to the data directory

\item {} 
\emph{trajDataHnd} : a handle to an implementation of \code{metaTrajIO}

\item {} 
\emph{dataFilterHnd} : a handle to an impementation of \code{metaIOFilter}

\item {} 
\emph{eventPartitionHnd} : a handle to a sub-class of \code{metaEventPartition}

\item {} 
\emph{eventProcHnd} : a handle to a sub-class of \code{metaEventProcessor}

\end{itemize}

\end{description}\end{quote}
\index{Run() (mosaic.SingleChannelAnalysis.SingleChannelAnalysis method)}

\begin{fulllineitems}
\phantomsection\label{api-doc/mosaic:mosaic.SingleChannelAnalysis.SingleChannelAnalysis.Run}\pysiglinewithargsret{\bfcode{Run}}{\emph{forkProcess=False}}{}
Start an analysis.
\begin{quote}\begin{description}
\item[{Parameters}] \leavevmode\begin{itemize}
\item {} 
\emph{forkProcess} : start the analysis in a separate process if \emph{True}. This option is useful when the main thread is used for other processing (e.g. GUI implementations).

\end{itemize}

\end{description}\end{quote}

\end{fulllineitems}

\index{Stop() (mosaic.SingleChannelAnalysis.SingleChannelAnalysis method)}

\begin{fulllineitems}
\phantomsection\label{api-doc/mosaic:mosaic.SingleChannelAnalysis.SingleChannelAnalysis.Stop}\pysiglinewithargsret{\bfcode{Stop}}{}{}
Stop a running analysis.

\end{fulllineitems}


\end{fulllineitems}



\subsubsection{mosaic.ConvertToCSV module}
\label{api-doc/mosaic:module-mosaic.ConvertToCSV}\label{api-doc/mosaic:mosaic-converttocsv-module}\index{mosaic.ConvertToCSV (module)}
Top level module to convert any data file readble by TrajIO objects into a comma separated value text file.
\begin{quote}\begin{description}
\item[{Created}] \leavevmode
10/13/2014

\item[{Author}] \leavevmode
Arvind Balijepalli \textless{}\href{mailto:arvind.balijepalli@nist.gov}{arvind.balijepalli@nist.gov}\textgreater{}

\item[{License}] \leavevmode
See LICENSE.TXT

\end{description}\end{quote}
\index{\_filename() (mosaic.ConvertToCSV.ConvertToCSV method)}

\begin{fulllineitems}
\phantomsection\label{api-doc/mosaic:mosaic.ConvertToCSV.ConvertToCSV._filename}\pysiglinewithargsret{\code{ConvertToCSV.}\bfcode{\_filename}}{}{}
Return a output filename that contains the data file prefix and and the block index.

\end{fulllineitems}

\index{\_creategenerator() (mosaic.ConvertToCSV.ConvertToCSV method)}

\begin{fulllineitems}
\phantomsection\label{api-doc/mosaic:mosaic.ConvertToCSV.ConvertToCSV._creategenerator}\pysiglinewithargsret{\code{ConvertToCSV.}\bfcode{\_creategenerator}}{}{}
Create a new filename generator if the file prefix has changed. 
The generator returns a filename incremented by a counter each time 
its next() function is called.

\end{fulllineitems}

\index{ConvertToCSV (class in mosaic.ConvertToCSV)}

\begin{fulllineitems}
\phantomsection\label{api-doc/mosaic:mosaic.ConvertToCSV.ConvertToCSV}\pysiglinewithargsret{\strong{class }\code{mosaic.ConvertToCSV.}\bfcode{ConvertToCSV}}{\emph{trajDataObj}, \emph{outdir=None}, \emph{extension='csv'}}{}
Bases: \href{http://docs.python.org/library/functions.html\#object}{\code{object}}

Convert data read from a sub-class of metaTrajIO to a comma separated text file
\begin{quote}\begin{description}
\item[{Parameters}] \leavevmode\begin{itemize}
\item {} 
\emph{trajDataObj} : a trajIO data object

\item {} 
\emph{outdir} : the output directory. Default is \emph{None}, which causes the output to be saved in the same directory as the input data.

\end{itemize}

\end{description}\end{quote}
\index{Convert() (mosaic.ConvertToCSV.ConvertToCSV method)}

\begin{fulllineitems}
\phantomsection\label{api-doc/mosaic:mosaic.ConvertToCSV.ConvertToCSV.Convert}\pysiglinewithargsret{\bfcode{Convert}}{\emph{blockSize}}{}
Start converting data
\begin{quote}\begin{description}
\item[{Parameters}] \leavevmode\begin{itemize}
\item {} 
\emph{blockSize} : number of data points to convert.

\end{itemize}

\end{description}\end{quote}

\end{fulllineitems}


\end{fulllineitems}



\subsection{Meta-Classes}
\label{api-doc/mosaic.meta:api-metaclass-page}\label{api-doc/mosaic.meta:meta-classes}\label{api-doc/mosaic.meta::doc}

\subsubsection{mosaic.metaEventPartition module}
\label{api-doc/mosaic.meta:mosaic-metaeventpartition-module}\index{metaEventPartition (class in mosaic.metaEventPartition)}

\begin{fulllineitems}
\phantomsection\label{api-doc/mosaic.meta:mosaic.metaEventPartition.metaEventPartition}\pysiglinewithargsret{\strong{class }\code{mosaic.metaEventPartition.}\bfcode{metaEventPartition}}{\emph{trajDataObj}, \emph{eventProcHnd}, \emph{eventPartitionSettings}, \emph{eventProcSettings}, \emph{settingsString}}{}
Bases: \href{http://docs.python.org/library/functions.html\#object}{\code{object}}

\begin{notice}{warning}{Warning:}
This metaclass must be sub-classed. All abstract methods within this metaclass must be implemented.
\end{notice}

A class to abstract partitioning individual events. Once a single 
molecule event is identified, it is handed off to to an event processor.
If parallel processing is requested, detailed event processing will commence
immediately. If not, detailed event processing is performed after the event 
partition has completed.
\begin{quote}\begin{description}
\item[{Parameters}] \leavevmode\begin{itemize}
\item {} \begin{description}
\item[{\emph{trajDataObj}}] \leavevmode{[}properly initialized object instantiated from a sub-class {]}
of metaTrajIO.

\end{description}

\item {} \begin{description}
\item[{\emph{eventProcHnd}}] \leavevmode{[}handle to a sub-class of metaEventProcessor. Objects of {]}
this class are initialized as necessary

\end{description}

\item {} 
\emph{eventPartitionSettings} :    settings dictionary for the partition algorithm.

\item {} 
\emph{eventProcSettings} :         settings dictionary for the event processing algorithm.

\item {} 
\emph{settingsString} :                    settings dictionary in JSON format

\end{itemize}

\end{description}\end{quote}

Common algorithm parameters from settings file (.settings in the data path or 
current working directory)
\begin{itemize}
\item {} 
\emph{writeEventTS} :      Write event current data to file. (default: 1, write data to file)

\item {} 
\emph{parallelProc} :      Process events in parallel. (default: 1, Yes)

\item {} 
\emph{reserveNCPU} :               Reserve the specified number of CPUs and exclude them from the parallel pool

\end{itemize}
\index{\_init() (mosaic.metaEventPartition.metaEventPartition method)}

\begin{fulllineitems}
\phantomsection\label{api-doc/mosaic.meta:mosaic.metaEventPartition.metaEventPartition._init}\pysiglinewithargsret{\bfcode{\_init}}{\emph{trajDataObj}, \emph{eventProcHnd}, \emph{eventPartitionSettings}, \emph{eventProcSettings}}{}~
\begin{notice}{important}{Important:}
\textbf{Abstract method:} This method must be implemented by a sub-class.
\end{notice}

This function is called at the end of the class constructor to perform additional initialization specific to the algorithm being implemented. The arguments to this function are identical to those passed to the class constructor.

\end{fulllineitems}

\index{\_stop() (mosaic.metaEventPartition.metaEventPartition method)}

\begin{fulllineitems}
\phantomsection\label{api-doc/mosaic.meta:mosaic.metaEventPartition.metaEventPartition._stop}\pysiglinewithargsret{\bfcode{\_stop}}{}{}~
\begin{notice}{important}{Important:}
\textbf{Abstract method:} This method must be implemented by a sub-class.
\end{notice}

Stop partitioning events froma time-series

\end{fulllineitems}

\index{\_eventsegment() (mosaic.metaEventPartition.metaEventPartition method)}

\begin{fulllineitems}
\phantomsection\label{api-doc/mosaic.meta:mosaic.metaEventPartition.metaEventPartition._eventsegment}\pysiglinewithargsret{\bfcode{\_eventsegment}}{}{}~
\begin{notice}{important}{Important:}
\textbf{Abstract method:} This method must be implemented by a sub-class.
\end{notice}

An implementation of this function should separate individual events of interest from a time-series of ionic current recordings. The data pertaining to each event is then passed               to an instance of metaEventProcessor for detailed analysis. The function will collect the results of this analysis.

\end{fulllineitems}

\index{PartitionEvents() (mosaic.metaEventPartition.metaEventPartition method)}

\begin{fulllineitems}
\phantomsection\label{api-doc/mosaic.meta:mosaic.metaEventPartition.metaEventPartition.PartitionEvents}\pysiglinewithargsret{\bfcode{PartitionEvents}}{}{}
Partition events within a time-series.

\end{fulllineitems}

\index{Stop() (mosaic.metaEventPartition.metaEventPartition method)}

\begin{fulllineitems}
\phantomsection\label{api-doc/mosaic.meta:mosaic.metaEventPartition.metaEventPartition.Stop}\pysiglinewithargsret{\bfcode{Stop}}{}{}
Stop processing data.

\end{fulllineitems}

\index{formatoutputfiles() (mosaic.metaEventPartition.metaEventPartition method)}

\begin{fulllineitems}
\phantomsection\label{api-doc/mosaic.meta:mosaic.metaEventPartition.metaEventPartition.formatoutputfiles}\pysiglinewithargsret{\bfcode{formatoutputfiles}}{}{}~
\begin{notice}{important}{Important:}
\textbf{Abstract method:} This method must be implemented by a sub-class.
\end{notice}

Return a formatted string of output files.

\end{fulllineitems}

\index{formatsettings() (mosaic.metaEventPartition.metaEventPartition method)}

\begin{fulllineitems}
\phantomsection\label{api-doc/mosaic.meta:mosaic.metaEventPartition.metaEventPartition.formatsettings}\pysiglinewithargsret{\bfcode{formatsettings}}{}{}~
\begin{notice}{important}{Important:}
\textbf{Abstract method:} This method must be implemented by a sub-class.
\end{notice}

Return a formatted string of settings for display

\end{fulllineitems}

\index{formatstats() (mosaic.metaEventPartition.metaEventPartition method)}

\begin{fulllineitems}
\phantomsection\label{api-doc/mosaic.meta:mosaic.metaEventPartition.metaEventPartition.formatstats}\pysiglinewithargsret{\bfcode{formatstats}}{}{}~
\begin{notice}{important}{Important:}
\textbf{Abstract method:} This method must be implemented by a sub-class.
\end{notice}

Return a formatted string of statistics for display

\end{fulllineitems}


\end{fulllineitems}



\subsubsection{mosaic.metaEventProcessor module}
\label{api-doc/mosaic.meta:mosaic-metaeventprocessor-module}\index{metaEventProcessor (class in mosaic.metaEventProcessor)}

\begin{fulllineitems}
\phantomsection\label{api-doc/mosaic.meta:mosaic.metaEventProcessor.metaEventProcessor}\pysiglinewithargsret{\strong{class }\code{mosaic.metaEventProcessor.}\bfcode{metaEventProcessor}}{\emph{icurr}, \emph{Fs}, \emph{**kwargs}}{}
Bases: \href{http://docs.python.org/library/functions.html\#object}{\code{object}}

\begin{notice}{warning}{Warning:}
This metaclass must be sub-classed. All abstract methods within this metaclass must be implemented.
\end{notice}

Defines the interface for specific event processing algorithms. Each event processing
algorithm must sub-class metaEventProcessor and implement the following abstract
functions:
\begin{itemize}
\item {} \begin{description}
\item[{\emph{processEvent}}] \leavevmode{[}process raw event data and populate event meta-data. Store each {]}
piece of processed event data in a class attribute starting with `md'.
For example, the blockade depth meta-data can be defined as `mdBlockadeDepth'

\end{description}

\item {} 
\emph{printMetadata} :     print meta-data set by event processing in a human readable format.

\end{itemize}
\begin{quote}\begin{description}
\item[{Parameters}] \leavevmode\begin{itemize}
\item {} 
\emph{icurr} :                             ionic current in pA

\item {} 
\emph{Fs} :                                sampling frequency in Hz

\end{itemize}

\item[{Keyword Args}] \leavevmode\begin{itemize}
\item {} 
\emph{eventstart} :                the event start point

\item {} 
\emph{eventend} :                  the event end point

\item {} 
\emph{baselinestats} :     baseline conductance statistics: a list of {[}mean, sd, slope{]} for the baseline current

\item {} 
\emph{algosettingsdict} :  settings for event processing algorithm as a dictionary

\item {} 
\emph{absdatidx} :                 index of data start. This arg can allow arrival time estimation.

\item {} 
\emph{datafilehnd} :               reference to an metaMDIO object for meta-data IO

\end{itemize}

\end{description}\end{quote}
\index{\_init() (mosaic.metaEventProcessor.metaEventProcessor method)}

\begin{fulllineitems}
\phantomsection\label{api-doc/mosaic.meta:mosaic.metaEventProcessor.metaEventProcessor._init}\pysiglinewithargsret{\bfcode{\_init}}{\emph{**kwargs}}{}~
\begin{notice}{important}{Important:}
\textbf{Abstract method:} This method must be implemented by a sub-class.
\end{notice}

\end{fulllineitems}

\index{\_mdHeadingDataType() (mosaic.metaEventProcessor.metaEventProcessor method)}

\begin{fulllineitems}
\phantomsection\label{api-doc/mosaic.meta:mosaic.metaEventProcessor.metaEventProcessor._mdHeadingDataType}\pysiglinewithargsret{\bfcode{\_mdHeadingDataType}}{}{}~
\begin{notice}{important}{Important:}
\textbf{Abstract method:} This method must be implemented by a sub-class.
\end{notice}

Return a list of meta-data tags data types.

\end{fulllineitems}

\index{\_mdHeadings() (mosaic.metaEventProcessor.metaEventProcessor method)}

\begin{fulllineitems}
\phantomsection\label{api-doc/mosaic.meta:mosaic.metaEventProcessor.metaEventProcessor._mdHeadings}\pysiglinewithargsret{\bfcode{\_mdHeadings}}{}{}~
\begin{notice}{important}{Important:}
\textbf{Abstract method:} This method must be implemented by a sub-class.
\end{notice}

Return a list of meta-data tags for display purposes.

\end{fulllineitems}

\index{\_mdList() (mosaic.metaEventProcessor.metaEventProcessor method)}

\begin{fulllineitems}
\phantomsection\label{api-doc/mosaic.meta:mosaic.metaEventProcessor.metaEventProcessor._mdList}\pysiglinewithargsret{\bfcode{\_mdList}}{}{}~
\begin{notice}{important}{Important:}
\textbf{Abstract method:} This method must be implemented by a sub-class.
\end{notice}

Return a list of meta-data set by event processing.

\end{fulllineitems}

\index{\_metaEventProcessor\_\_mdformat() (mosaic.metaEventProcessor.metaEventProcessor method)}

\begin{fulllineitems}
\phantomsection\label{api-doc/mosaic.meta:mosaic.metaEventProcessor.metaEventProcessor._metaEventProcessor__mdformat}\pysiglinewithargsret{\bfcode{\_metaEventProcessor\_\_mdformat}}{\emph{dat}}{}
Round a float to 3 decimal places. Leave ints and strings unchanged

\end{fulllineitems}

\index{\_processEvent() (mosaic.metaEventProcessor.metaEventProcessor method)}

\begin{fulllineitems}
\phantomsection\label{api-doc/mosaic.meta:mosaic.metaEventProcessor.metaEventProcessor._processEvent}\pysiglinewithargsret{\bfcode{\_processEvent}}{}{}~
\begin{notice}{important}{Important:}
\textbf{Abstract method:} This method must be implemented by a sub-class.
\end{notice}

\end{fulllineitems}

\index{mdAveragePropertiesList() (mosaic.metaEventProcessor.metaEventProcessor method)}

\begin{fulllineitems}
\phantomsection\label{api-doc/mosaic.meta:mosaic.metaEventProcessor.metaEventProcessor.mdAveragePropertiesList}\pysiglinewithargsret{\bfcode{mdAveragePropertiesList}}{}{}~
\begin{notice}{important}{Important:}
\textbf{Abstract method:} This method must be implemented by a sub-class.
\end{notice}

Return a list of meta-data properties that will be averaged 
and displayed at the end of a run. This function must be overridden
by sub-classes of metaEventProcessor. As a failsafe, an empty list
is returned.

\end{fulllineitems}

\index{mdHeadingDataType() (mosaic.metaEventProcessor.metaEventProcessor method)}

\begin{fulllineitems}
\phantomsection\label{api-doc/mosaic.meta:mosaic.metaEventProcessor.metaEventProcessor.mdHeadingDataType}\pysiglinewithargsret{\bfcode{mdHeadingDataType}}{}{}
Return a list of meta-data tags data types.

\end{fulllineitems}

\index{mdHeadings() (mosaic.metaEventProcessor.metaEventProcessor method)}

\begin{fulllineitems}
\phantomsection\label{api-doc/mosaic.meta:mosaic.metaEventProcessor.metaEventProcessor.mdHeadings}\pysiglinewithargsret{\bfcode{mdHeadings}}{}{}
Return a list of meta-data tags for display purposes.

\end{fulllineitems}

\index{processEvent() (mosaic.metaEventProcessor.metaEventProcessor method)}

\begin{fulllineitems}
\phantomsection\label{api-doc/mosaic.meta:mosaic.metaEventProcessor.metaEventProcessor.processEvent}\pysiglinewithargsret{\bfcode{processEvent}}{}{}
This is the equivalent of a pure virtual function in C++.

\end{fulllineitems}

\index{rejectEvent() (mosaic.metaEventProcessor.metaEventProcessor method)}

\begin{fulllineitems}
\phantomsection\label{api-doc/mosaic.meta:mosaic.metaEventProcessor.metaEventProcessor.rejectEvent}\pysiglinewithargsret{\bfcode{rejectEvent}}{\emph{status}}{}
Set an event as rejected if it doesn't pass tests in processing.
The status is assigned to mdProcessingStatus.

\end{fulllineitems}

\index{writeEvent() (mosaic.metaEventProcessor.metaEventProcessor method)}

\begin{fulllineitems}
\phantomsection\label{api-doc/mosaic.meta:mosaic.metaEventProcessor.metaEventProcessor.writeEvent}\pysiglinewithargsret{\bfcode{writeEvent}}{}{}
Write event meta data to a metaMDIO object.

\end{fulllineitems}


\end{fulllineitems}



\subsubsection{mosaic.metaIOFilter module}
\label{api-doc/mosaic.meta:mosaic-metaiofilter-module}\index{metaIOFilter (class in mosaic.metaIOFilter)}

\begin{fulllineitems}
\phantomsection\label{api-doc/mosaic.meta:mosaic.metaIOFilter.metaIOFilter}\pysiglinewithargsret{\strong{class }\code{mosaic.metaIOFilter.}\bfcode{metaIOFilter}}{\emph{**kwargs}}{}
Bases: \href{http://docs.python.org/library/functions.html\#object}{\code{object}}

\begin{notice}{warning}{Warning:}
This metaclass must be sub-classed. All abstract methods within this metaclass must be implemented.
\end{notice}

Defines the interface for specific filter implementations. Each filtering
algorithm must sub-class metaIOFilter and implement the following abstract
function:
\begin{itemize}
\item {} 
\emph{filterData} :        apply a filter to self.eventData

\end{itemize}
\begin{quote}\begin{description}
\item[{Parameters}] \leavevmode\begin{itemize}
\item {} 
\emph{decimate} :          sets the downsampling ratio of the filtered data (default:1, no decimation).

\end{itemize}

\item[{Properties}] \leavevmode\begin{itemize}
\item {} 
\emph{filteredData} :              list of filtered and decimated data

\item {} 
\emph{filterFs} :                  sampling frequency after filtering and decimation

\end{itemize}

\end{description}\end{quote}
\index{\_init() (mosaic.metaIOFilter.metaIOFilter method)}

\begin{fulllineitems}
\phantomsection\label{api-doc/mosaic.meta:mosaic.metaIOFilter.metaIOFilter._init}\pysiglinewithargsret{\bfcode{\_init}}{\emph{**kwargs}}{}~
\begin{notice}{important}{Important:}
\textbf{Abstract method:} This method must be implemented by a sub-class.
\end{notice}

\end{fulllineitems}

\index{filterData() (mosaic.metaIOFilter.metaIOFilter method)}

\begin{fulllineitems}
\phantomsection\label{api-doc/mosaic.meta:mosaic.metaIOFilter.metaIOFilter.filterData}\pysiglinewithargsret{\bfcode{filterData}}{\emph{icurr}, \emph{Fs}}{}~
\begin{notice}{important}{Important:}
\textbf{Abstract method:} This method must be implemented by a sub-class.
\end{notice}

This is the equivalent of a pure virtual function in C++.

Implementations of this method MUST store (1) a ref to the raw event data in self.eventData AND 
(2) the sampling frequency in self.Fs.
\begin{quote}\begin{description}
\item[{Parameters}] \leavevmode\begin{itemize}
\item {} 
\emph{icurr} :     ionic current in pA

\item {} 
\emph{Fs} :        original sampling frequency in Hz

\end{itemize}

\end{description}\end{quote}

\end{fulllineitems}

\index{filterFs (mosaic.metaIOFilter.metaIOFilter attribute)}

\begin{fulllineitems}
\phantomsection\label{api-doc/mosaic.meta:mosaic.metaIOFilter.metaIOFilter.filterFs}\pysigline{\bfcode{filterFs}}
Return the sampling frequency of filtered data.

\end{fulllineitems}

\index{filteredData (mosaic.metaIOFilter.metaIOFilter attribute)}

\begin{fulllineitems}
\phantomsection\label{api-doc/mosaic.meta:mosaic.metaIOFilter.metaIOFilter.filteredData}\pysigline{\bfcode{filteredData}}
Return filtered data

\end{fulllineitems}

\index{formatsettings() (mosaic.metaIOFilter.metaIOFilter method)}

\begin{fulllineitems}
\phantomsection\label{api-doc/mosaic.meta:mosaic.metaIOFilter.metaIOFilter.formatsettings}\pysiglinewithargsret{\bfcode{formatsettings}}{}{}~
\begin{notice}{important}{Important:}
\textbf{Abstract method:} This method must be implemented by a sub-class.
\end{notice}

Return a formatted string of filter settings

\end{fulllineitems}


\end{fulllineitems}



\subsubsection{mosaic.metaMDIO module}
\label{api-doc/mosaic.meta:mosaic-metamdio-module}\index{metaMDIO (class in mosaic.metaMDIO)}

\begin{fulllineitems}
\phantomsection\label{api-doc/mosaic.meta:mosaic.metaMDIO.metaMDIO}\pysigline{\strong{class }\code{mosaic.metaMDIO.}\bfcode{metaMDIO}}
Bases: \href{http://docs.python.org/library/functions.html\#object}{\code{object}}

\begin{notice}{warning}{Warning:}
This metaclass must be sub-classed. All abstract methods within this metaclass must be implemented.
\end{notice}

This class provides the skeleton for storing metadata
generated by algorithms. It also provides an interface to query metadata, for example in a 
SQL database.
\begin{quote}\begin{description}
\item[{Properties}] \leavevmode\begin{itemize}
\item {} 
\emph{dbColumnNames} : a list of database column names

\end{itemize}

\end{description}\end{quote}
\index{\_opendb() (mosaic.metaMDIO.metaMDIO method)}

\begin{fulllineitems}
\phantomsection\label{api-doc/mosaic.meta:mosaic.metaMDIO.metaMDIO._opendb}\pysiglinewithargsret{\bfcode{\_opendb}}{\emph{dbname}, \emph{**kwargs}}{}~
\begin{notice}{important}{Important:}
\textbf{Abstract method:} This method must be implemented by a sub-class.
\end{notice}

\end{fulllineitems}

\index{\_initdb() (mosaic.metaMDIO.metaMDIO method)}

\begin{fulllineitems}
\phantomsection\label{api-doc/mosaic.meta:mosaic.metaMDIO.metaMDIO._initdb}\pysiglinewithargsret{\bfcode{\_initdb}}{\emph{**kwargs}}{}~
\begin{notice}{important}{Important:}
\textbf{Abstract method:} This method must be implemented by a sub-class.
\end{notice}

\end{fulllineitems}

\index{\_colnames() (mosaic.metaMDIO.metaMDIO method)}

\begin{fulllineitems}
\phantomsection\label{api-doc/mosaic.meta:mosaic.metaMDIO.metaMDIO._colnames}\pysiglinewithargsret{\bfcode{\_colnames}}{\emph{table=None}}{}~
\begin{notice}{important}{Important:}
\textbf{Abstract method:} This method must be implemented by a sub-class.
\end{notice}

\end{fulllineitems}

\index{closeDB() (mosaic.metaMDIO.metaMDIO method)}

\begin{fulllineitems}
\phantomsection\label{api-doc/mosaic.meta:mosaic.metaMDIO.metaMDIO.closeDB}\pysiglinewithargsret{\bfcode{closeDB}}{}{}~
\begin{notice}{important}{Important:}
\textbf{Abstract method:} This method must be implemented by a sub-class.
\end{notice}

\end{fulllineitems}

\index{initDB() (mosaic.metaMDIO.metaMDIO method)}

\begin{fulllineitems}
\phantomsection\label{api-doc/mosaic.meta:mosaic.metaMDIO.metaMDIO.initDB}\pysiglinewithargsret{\bfcode{initDB}}{\emph{**kwargs}}{}
Initialize a new database file.
\begin{quote}\begin{description}
\item[{Parameters}] \leavevmode
\end{description}\end{quote}

The arguments passed to init change based on the method of file IO selected, in addition to 
the common args below:
\begin{itemize}
\item {} 
\emph{dbPath} :            directory to store the MD database (`\textless{}full path to data directory\textgreater{}')

\item {} 
\emph{colNames} :  list of text names for the columns in the tables

\item {} 
\emph{colNames\_t} :        list of data types for each column.

\end{itemize}

\end{fulllineitems}

\index{openDB() (mosaic.metaMDIO.metaMDIO method)}

\begin{fulllineitems}
\phantomsection\label{api-doc/mosaic.meta:mosaic.metaMDIO.metaMDIO.openDB}\pysiglinewithargsret{\bfcode{openDB}}{\emph{dbname}, \emph{**kwargs}}{}
Open an existing database file.
\begin{quote}\begin{description}
\item[{Parameters}] \leavevmode\begin{itemize}
\item {} 
\emph{dbname} :            directory to store the MD database (`\textless{}full path to data directory\textgreater{}')

\end{itemize}

\end{description}\end{quote}


\strong{See also:}


The arguments passed to init change based on the method of file IO selected, in addition to the common args.



\end{fulllineitems}

\index{queryDB() (mosaic.metaMDIO.metaMDIO method)}

\begin{fulllineitems}
\phantomsection\label{api-doc/mosaic.meta:mosaic.metaMDIO.metaMDIO.queryDB}\pysiglinewithargsret{\bfcode{queryDB}}{\emph{query}}{}~
\begin{notice}{important}{Important:}
\textbf{Abstract method:} This method must be implemented by a sub-class.
\end{notice}

Query a database. 
:Parameters:
\begin{itemize}
\item {} 
\emph{query} : query string

\end{itemize}


\strong{See also:}


See specific implementations of metaMDIO for query syntax.



\end{fulllineitems}

\index{readAnalysisInfo() (mosaic.metaMDIO.metaMDIO method)}

\begin{fulllineitems}
\phantomsection\label{api-doc/mosaic.meta:mosaic.metaMDIO.metaMDIO.readAnalysisInfo}\pysiglinewithargsret{\bfcode{readAnalysisInfo}}{}{}~
\begin{notice}{important}{Important:}
\textbf{Abstract method:} This method must be implemented by a sub-class.
\end{notice}

Read analysis information from the database.

\end{fulllineitems}

\index{readAnalysisLog() (mosaic.metaMDIO.metaMDIO method)}

\begin{fulllineitems}
\phantomsection\label{api-doc/mosaic.meta:mosaic.metaMDIO.metaMDIO.readAnalysisLog}\pysiglinewithargsret{\bfcode{readAnalysisLog}}{}{}~
\begin{notice}{important}{Important:}
\textbf{Abstract method:} This method must be implemented by a sub-class.
\end{notice}

Read the analysis log from the database.

\end{fulllineitems}

\index{readSettings() (mosaic.metaMDIO.metaMDIO method)}

\begin{fulllineitems}
\phantomsection\label{api-doc/mosaic.meta:mosaic.metaMDIO.metaMDIO.readSettings}\pysiglinewithargsret{\bfcode{readSettings}}{}{}~
\begin{notice}{important}{Important:}
\textbf{Abstract method:} This method must be implemented by a sub-class.
\end{notice}

Read JSON settings from the database.

\end{fulllineitems}

\index{writeAnalysisInfo() (mosaic.metaMDIO.metaMDIO method)}

\begin{fulllineitems}
\phantomsection\label{api-doc/mosaic.meta:mosaic.metaMDIO.metaMDIO.writeAnalysisInfo}\pysiglinewithargsret{\bfcode{writeAnalysisInfo}}{\emph{infolist}}{}~
\begin{notice}{important}{Important:}
\textbf{Abstract method:} This method must be implemented by a sub-class.
\end{notice}

Write analysis information to the database. Note that subsequent calls to this method will overwrite the analysis inoformation entry in the table.
\begin{quote}\begin{description}
\item[{Args}] \leavevmode\begin{itemize}
\item {} \begin{description}
\item[{\emph{infolist}}] \leavevmode{[}A list of strings in the following order {[} datPath, dataType, partitionAlgorithm, processingAlgorithm, filteringAlgorithm{]}.{]}
\emph{datPath}                               : full path to the data directory

\emph{dataType}                              : type of data processed (e.g. ABF, QDF, etc.)

\emph{partitionAlgorithm}    : name of partition algorithm (e.g. eventSegment)

\emph{processingAlgorithm}   : name of event processing algorithm (e.g. multStateAnalysis)

\emph{filteringAlgorithm}    : name of filtering algorithm (e.g. waveletDenoiseFilter) or None if no filtering was performed.

\end{description}

\end{itemize}

\end{description}\end{quote}

\end{fulllineitems}

\index{writeAnalysisLog() (mosaic.metaMDIO.metaMDIO method)}

\begin{fulllineitems}
\phantomsection\label{api-doc/mosaic.meta:mosaic.metaMDIO.metaMDIO.writeAnalysisLog}\pysiglinewithargsret{\bfcode{writeAnalysisLog}}{\emph{analysislog}}{}~
\begin{notice}{important}{Important:}
\textbf{Abstract method:} This method must be implemented by a sub-class.
\end{notice}

Write the analysis log string to the database. Note that subsequent calls to this method will overwrite the analysis log entry.
\begin{quote}\begin{description}
\item[{Args}] \leavevmode\begin{itemize}
\item {} 
\emph{analysislog} :       analysis log string to save

\end{itemize}

\end{description}\end{quote}

\end{fulllineitems}

\index{writeRecord() (mosaic.metaMDIO.metaMDIO method)}

\begin{fulllineitems}
\phantomsection\label{api-doc/mosaic.meta:mosaic.metaMDIO.metaMDIO.writeRecord}\pysiglinewithargsret{\bfcode{writeRecord}}{\emph{data}, \emph{table=None}}{}~
\begin{notice}{important}{Important:}
\textbf{Abstract method:} This method must be implemented by a sub-class.
\end{notice}

Write data to a specified table. By default table 
is None. In this case sub-classes should fall back 
to writing data to a default table.

\end{fulllineitems}

\index{writeSettings() (mosaic.metaMDIO.metaMDIO method)}

\begin{fulllineitems}
\phantomsection\label{api-doc/mosaic.meta:mosaic.metaMDIO.metaMDIO.writeSettings}\pysiglinewithargsret{\bfcode{writeSettings}}{\emph{settingsstring}}{}~
\begin{notice}{important}{Important:}
\textbf{Abstract method:} This method must be implemented by a sub-class.
\end{notice}

Write the settings JSON object to the database.
\begin{quote}\begin{description}
\item[{Args}] \leavevmode\begin{itemize}
\item {} 
\emph{settingsstring} : a {\color{red}\bfseries{}JSON\_} formatted settings string.

\end{itemize}

\end{description}\end{quote}

\end{fulllineitems}


\end{fulllineitems}



\subsubsection{mosaic.metaTrajIO module}
\label{api-doc/mosaic.meta:mosaic-metatrajio-module}\index{metaTrajIO (class in mosaic.metaTrajIO)}

\begin{fulllineitems}
\phantomsection\label{api-doc/mosaic.meta:mosaic.metaTrajIO.metaTrajIO}\pysiglinewithargsret{\strong{class }\code{mosaic.metaTrajIO.}\bfcode{metaTrajIO}}{\emph{**kwargs}}{}
Bases: \href{http://docs.python.org/library/functions.html\#object}{\code{object}}

\begin{notice}{warning}{Warning:}
This metaclass must be sub-classed. All abstract methods within this metaclass must be implemented.
\end{notice}

Initialize a TrajIO object. The object can load all the data in a directory,
N files from a directory or from an explicit list of filenames. In addition 
to the arguments defined below, implementations of this meta class may require 
the definition of additional arguments. See the documentation of those classes
for what those may be. For example, the qdfTrajIO implementation of metaTrajIO also requires
the feedback resistance (Rfb) and feedback capacitance (Cfb) to be passed at initialization.
\begin{quote}\begin{description}
\item[{Parameters}] \leavevmode\begin{itemize}
\item {} 
\emph{dirname} :           all files from a directory (`\textless{}full path to data directory\textgreater{}')

\item {} 
\emph{nfiles} :            if requesting N files (in addition to dirname) from a specified directory

\item {} 
\emph{fnames} :            explicit list of filenames ({[}file1, file2,...{]}). This argument cannot be used in conjuction with dirname/nfiles. The filter argument is ignored when used in combination with fnames.

\item {} 
\emph{filter} :            `\textless{}wildcard filter\textgreater{}' (optional, filter is `*' if not specified)

\item {} 
\emph{start} :             Data start point in seconds.

\item {} 
\emph{end} :                       Data end point in seconds.

\item {} 
\emph{datafilter} :        Handle to the algorithm to use to filter the data. If no algorithm is specified, datafilter     is None and no filtering is performed.

\item {} 
\emph{dcOffset} :          Subtract a DC offset from the ionic current data.

\end{itemize}

\item[{Properties}] \leavevmode\begin{itemize}
\item {} 
\emph{FsHz} :                                      sampling frequency in Hz. If the data was decimated, this property will hold the sampling frequency after decimation.

\item {} 
\emph{LastFileProcessed} :         return the data file that was last processed.

\item {} 
\emph{ElapsedTimeSeconds} :        return the analysis time in sec.

\end{itemize}

\item[{Errors}] \leavevmode\begin{itemize}
\item {} 
\emph{IncompatibleArgumentsError} :        when conflicting arguments are used.

\item {} 
\emph{EmptyDataPipeError} :                        when out of data.

\item {} 
\emph{FileNotFoundError} :                         when data files do not exist in the specified path.

\item {} 
\emph{InsufficientArgumentsError} :        when incompatible arguments are passed

\end{itemize}

\end{description}\end{quote}
\index{\_init() (mosaic.metaTrajIO.metaTrajIO method)}

\begin{fulllineitems}
\phantomsection\label{api-doc/mosaic.meta:mosaic.metaTrajIO.metaTrajIO._init}\pysiglinewithargsret{\bfcode{\_init}}{\emph{**kwargs}}{}~
\begin{notice}{important}{Important:}
\textbf{Abstract method:} This method must be implemented by a sub-class.
\end{notice}

This function is called at the end of the class constructor to perform additional initialization specific to the algorithm being implemented. The arguments to this function are identical to those passed to the class constructor.

\end{fulllineitems}

\index{\_formatsettings() (mosaic.metaTrajIO.metaTrajIO method)}

\begin{fulllineitems}
\phantomsection\label{api-doc/mosaic.meta:mosaic.metaTrajIO.metaTrajIO._formatsettings}\pysiglinewithargsret{\bfcode{\_formatsettings}}{\emph{logObject}}{}~
\begin{notice}{important}{Important:}
\textbf{Abstract method:} This method must be implemented by a sub-class.
\end{notice}

Populate \emph{logObject} with settings strings for display
\begin{quote}\begin{description}
\item[{Parameters}] \leavevmode\begin{itemize}
\item {} 
\emph{logObject} :         a object that holds logging text (see {\hyperref[api\string-doc/mosaic.misc:mosaic.utilities.mosaicLog.mosaicLog]{\emph{\code{mosaicLog}}}})

\end{itemize}

\end{description}\end{quote}

\end{fulllineitems}

\index{DataLengthSec (mosaic.metaTrajIO.metaTrajIO attribute)}

\begin{fulllineitems}
\phantomsection\label{api-doc/mosaic.meta:mosaic.metaTrajIO.metaTrajIO.DataLengthSec}\pysigline{\bfcode{DataLengthSec}}~
\begin{notice}{important}{Important:}
\textbf{Property}
\end{notice}

Return the approximate length of data that will be processed. If the data are in multiple files,
this property assumes that each file contains an equal amount of data.

\end{fulllineitems}

\index{ElapsedTimeSeconds (mosaic.metaTrajIO.metaTrajIO attribute)}

\begin{fulllineitems}
\phantomsection\label{api-doc/mosaic.meta:mosaic.metaTrajIO.metaTrajIO.ElapsedTimeSeconds}\pysigline{\bfcode{ElapsedTimeSeconds}}~
\begin{notice}{important}{Important:}
\textbf{Property}
\end{notice}

Return the elapsed time in the time-series in seconds.

\end{fulllineitems}

\index{FsHz (mosaic.metaTrajIO.metaTrajIO attribute)}

\begin{fulllineitems}
\phantomsection\label{api-doc/mosaic.meta:mosaic.metaTrajIO.metaTrajIO.FsHz}\pysigline{\bfcode{FsHz}}~
\begin{notice}{important}{Important:}
\textbf{Property}
\end{notice}

Return the sampling frequency in Hz.

\end{fulllineitems}

\index{LastFileProcessed (mosaic.metaTrajIO.metaTrajIO attribute)}

\begin{fulllineitems}
\phantomsection\label{api-doc/mosaic.meta:mosaic.metaTrajIO.metaTrajIO.LastFileProcessed}\pysigline{\bfcode{LastFileProcessed}}~
\begin{notice}{important}{Important:}
\textbf{Property}
\end{notice}

Return the last data file that was processed

\end{fulllineitems}

\index{formatsettings() (mosaic.metaTrajIO.metaTrajIO method)}

\begin{fulllineitems}
\phantomsection\label{api-doc/mosaic.meta:mosaic.metaTrajIO.metaTrajIO.formatsettings}\pysiglinewithargsret{\bfcode{formatsettings}}{}{}
Return a formatted string of settings for display

\end{fulllineitems}

\index{popdata() (mosaic.metaTrajIO.metaTrajIO method)}

\begin{fulllineitems}
\phantomsection\label{api-doc/mosaic.meta:mosaic.metaTrajIO.metaTrajIO.popdata}\pysiglinewithargsret{\bfcode{popdata}}{\emph{n}}{}
Pop data points from self.currDataPipe. This function uses recursion 
to automatically read data files when the queue length is shorter
than the requested data points. When all data files are read, an
EmptyDataPipeError is thrown.
\begin{quote}\begin{description}
\item[{Parameters}] \leavevmode\begin{itemize}
\item {} 
\emph{n} : number of requested data points

\end{itemize}

\item[{Returns}] \leavevmode\begin{itemize}
\item {} 
Numpy array with requested data

\end{itemize}

\item[{Errors}] \leavevmode\begin{itemize}
\item {} 
\emph{EmptyDataPipeError} : if the queue has fewer data points than requested.

\end{itemize}

\end{description}\end{quote}

\end{fulllineitems}

\index{popfnames() (mosaic.metaTrajIO.metaTrajIO method)}

\begin{fulllineitems}
\phantomsection\label{api-doc/mosaic.meta:mosaic.metaTrajIO.metaTrajIO.popfnames}\pysiglinewithargsret{\bfcode{popfnames}}{}{}
Pop a single filename from the start of \code{self.dataFiles}. If \code{self.dataFiles} is empty,
raise an \code{EmptyDataPipeError} error.
\begin{quote}\begin{description}
\item[{Parameters}] \leavevmode\begin{itemize}
\item {} 
None

\end{itemize}

\item[{Returns}] \leavevmode
A single filename if successful.

\item[{Errors}] \leavevmode\begin{itemize}
\item {} 
\emph{EmptyDataPipeError} : when the filename list is empty.

\end{itemize}

\end{description}\end{quote}

\end{fulllineitems}

\index{previewdata() (mosaic.metaTrajIO.metaTrajIO method)}

\begin{fulllineitems}
\phantomsection\label{api-doc/mosaic.meta:mosaic.metaTrajIO.metaTrajIO.previewdata}\pysiglinewithargsret{\bfcode{previewdata}}{\emph{n}}{}
Preview data points in self.currDataPipe. This function is identical in 
behavior to popdata, except it does not remove data point from the queue.
Like popdata, it uses recursion to automatically read data files 
when the queue length is shorter than the requested data points. When all 
data files are read, an EmptyDataPipeError is thrown.
\begin{quote}\begin{description}
\item[{Parameters}] \leavevmode
\emph{n} : number of requested data points

\item[{Returns}] \leavevmode\begin{itemize}
\item {} 
Numpy array with requested data

\end{itemize}

\item[{Errors}] \leavevmode\begin{itemize}
\item {} 
\emph{EmptyDataPipeError} : if the queue has fewer data points than requested.

\end{itemize}

\end{description}\end{quote}

\end{fulllineitems}

\index{readdata() (mosaic.metaTrajIO.metaTrajIO method)}

\begin{fulllineitems}
\phantomsection\label{api-doc/mosaic.meta:mosaic.metaTrajIO.metaTrajIO.readdata}\pysiglinewithargsret{\bfcode{readdata}}{\emph{fname}}{}~
\begin{notice}{important}{Important:}
\textbf{Abstract method:} This method must be implemented by a sub-class.
\end{notice}

Return raw data from a single data file. Set a class 
attribute Fs with the sampling frequency in Hz.
\begin{quote}\begin{description}
\item[{Parameters}] \leavevmode\begin{itemize}
\item {} 
\emph{fname} :  fileame to read

\end{itemize}

\item[{Returns}] \leavevmode
An array object that holds raw (unscaled) data from \emph{fname}

\item[{Errors}] \leavevmode
None

\end{description}\end{quote}

\end{fulllineitems}

\index{scaleData() (mosaic.metaTrajIO.metaTrajIO method)}

\begin{fulllineitems}
\phantomsection\label{api-doc/mosaic.meta:mosaic.metaTrajIO.metaTrajIO.scaleData}\pysiglinewithargsret{\bfcode{scaleData}}{\emph{data}}{}~
\begin{notice}{important}{Important:}
\textbf{Abstract method:} This optional interface method can be overridden by a sub-class to modify functionality.
\end{notice}

Scale the raw data loaded with {\hyperref[api\string-doc/mosaic.meta:mosaic.metaTrajIO.metaTrajIO.readdata]{\emph{\code{readdata()}}}}. Note this function will not necessarily receive the entire data array loaded with {\hyperref[api\string-doc/mosaic.meta:mosaic.metaTrajIO.metaTrajIO.readdata]{\emph{\code{readdata()}}}}. Transformations must be able to process partial data chunks.
\begin{quote}\begin{description}
\item[{Parameters}] \leavevmode\begin{itemize}
\item {} 
\emph{data} : partial chunk of raw data loaded using {\hyperref[api\string-doc/mosaic.meta:mosaic.metaTrajIO.metaTrajIO.readdata]{\emph{\code{readdata()}}}}.

\end{itemize}

\item[{Returns}] \leavevmode\begin{itemize}
\item {} 
Array containing scaled data.

\end{itemize}

\item[{Default Behavior}] \leavevmode\begin{itemize}
\item {} 
If not implemented by a sub-class, the default behavior is to return \code{data} to the calling function without modifications.

\end{itemize}

\item[{Example}] \leavevmode
Assuming the amplifier scale and offset values are stored in the class variables \code{AmplifierScale} and \code{AmplifierOffset}, the raw data read using {\hyperref[api\string-doc/mosaic.meta:mosaic.metaTrajIO.metaTrajIO.readdata]{\emph{\code{readdata()}}}} can be transformed by {\hyperref[api\string-doc/mosaic.meta:mosaic.metaTrajIO.metaTrajIO.scaleData]{\emph{\code{scaleData()}}}}. We can also use this function to change the array data type.

\end{description}\end{quote}

\begin{Verbatim}[commandchars=\\\{\}]
\PYG{k}{def} \PYG{n+nf}{scaleData}\PYG{p}{(}\PYG{n+nb+bp}{self}\PYG{p}{,} \PYG{n}{data}\PYG{p}{)}\PYG{p}{:}
        \PYG{k}{return} \PYG{n}{np}\PYG{o}{.}\PYG{n}{array}\PYG{p}{(}\PYG{n}{data}\PYG{o}{*}\PYG{n+nb+bp}{self}\PYG{o}{.}\PYG{n}{AmplifierScale}\PYG{o}{\PYGZhy{}}\PYG{n+nb+bp}{self}\PYG{o}{.}\PYG{n}{AmplifierOffset}\PYG{p}{,} \PYG{n}{dtype}\PYG{o}{=}\PYG{l+s+s1}{\PYGZsq{}}\PYG{l+s+s1}{f8}\PYG{l+s+s1}{\PYGZsq{}}\PYG{p}{)}
\end{Verbatim}

\end{fulllineitems}


\end{fulllineitems}



\subsection{Time-Series IO}
\label{api-doc/mosaic.traj::doc}\label{api-doc/mosaic.traj:time-series-io}

\subsubsection{mosaic.abfTrajIO module}
\label{api-doc/mosaic.traj:mosaic-abftrajio-module}\phantomsection\label{api-doc/mosaic.traj:module-mosaic.abfTrajIO}\index{mosaic.abfTrajIO (module)}
A TrajIO class that supports ABF1 and ABF2 file formats via abf/abf.py. Currently, only
gap-free mode and single channel recordings are supported.
\begin{quote}\begin{description}
\item[{Created}] \leavevmode
5/23/2013

\item[{Author}] \leavevmode
Arvind Balijepalli \textless{}\href{mailto:arvind.balijepalli@nist.gov}{arvind.balijepalli@nist.gov}\textgreater{}

\item[{License}] \leavevmode
See LICENSE.TXT

\item[{ChangeLog}] \leavevmode
\end{description}\end{quote}

\begin{DUlineblock}{0em}
\item[] 9/13/15         AB      Updated logging to use mosaicLog class
\item[] 3/28/15         AB      Updated file read code to match new metaTrajIO API.
\item[] 5/23/13         AB      Initial version
\end{DUlineblock}
\index{abfTrajIO (class in mosaic.abfTrajIO)}

\begin{fulllineitems}
\phantomsection\label{api-doc/mosaic.traj:mosaic.abfTrajIO.abfTrajIO}\pysiglinewithargsret{\strong{class }\code{mosaic.abfTrajIO.}\bfcode{abfTrajIO}}{\emph{**kwargs}}{}
Bases: {\hyperref[api\string-doc/mosaic.meta:mosaic.metaTrajIO.metaTrajIO]{\emph{\code{mosaic.metaTrajIO.metaTrajIO}}}}

Read ABF1 and ABF2 file formats. Currently, only 
gap-free mode and single channel recordings are supported.

A typical settings section to read ABF files is shown below.

\begin{Verbatim}[commandchars=\\\{\}]
\PYG{l+s+s2}{\PYGZdq{}abfTrajIO\PYGZdq{}} \PYG{o}{:} \PYG{p}{\PYGZob{}}
\PYG{l+s+s2}{\PYGZdq{}filter\PYGZdq{}}                        \PYG{o}{:} \PYG{l+s+s2}{\PYGZdq{}*.abf\PYGZdq{}}\PYG{p}{,}
\PYG{l+s+s2}{\PYGZdq{}start\PYGZdq{}}                         \PYG{o}{:} \PYG{l+m+mf}{0.0}\PYG{p}{,}
\PYG{l+s+s2}{\PYGZdq{}dcOffset\PYGZdq{}}                      \PYG{o}{:} \PYG{l+m+mf}{0.0}
\PYG{p}{\PYGZcb{}}
\end{Verbatim}
\begin{quote}\begin{description}
\item[{Parameters}] \leavevmode\begin{description}
\item[{In addition to {\hyperref[api\string-doc/mosaic.meta:mosaic.metaTrajIO.metaTrajIO]{\emph{\code{metaTrajIO}}}} args,}] \leavevmode
None

\end{description}

\end{description}\end{quote}
\index{readdata() (mosaic.abfTrajIO.abfTrajIO method)}

\begin{fulllineitems}
\phantomsection\label{api-doc/mosaic.traj:mosaic.abfTrajIO.abfTrajIO.readdata}\pysiglinewithargsret{\bfcode{readdata}}{\emph{fname}}{}
Read one or more files and append their data to the data pipeline.
Set a class attribute Fs with the sampling frequency in Hz.
\begin{quote}\begin{description}
\item[{Parameters}] \leavevmode\begin{itemize}
\item {} 
\emph{fname} :  fileame to read

\end{itemize}

\item[{Returns}] \leavevmode\begin{itemize}
\item {} 
An array object that holds raw (unscaled) data from \emph{fname}

\end{itemize}

\item[{Errors}] \leavevmode\begin{itemize}
\item {} 
\emph{SamplingRateChangedError} : if the sampling rate for any data file differs from previous

\end{itemize}

\end{description}\end{quote}

\end{fulllineitems}


\end{fulllineitems}



\subsubsection{mosaic.qdfTrajIO module}
\label{api-doc/mosaic.traj:mosaic-qdftrajio-module}\phantomsection\label{api-doc/mosaic.traj:module-mosaic.qdfTrajIO}\index{mosaic.qdfTrajIO (module)}
QDF implementation of metaTrajIO. Uses the readqdf module from EBS to 
read individual qdf files.
\begin{quote}\begin{description}
\item[{Created}] \leavevmode
7/18/2012

\item[{Author}] \leavevmode
Arvind Balijepalli \textless{}\href{mailto:arvind.balijepalli@nist.gov}{arvind.balijepalli@nist.gov}\textgreater{}

\item[{License}] \leavevmode
See LICENSE.TXT

\item[{ChangeLog}] \leavevmode
\end{description}\end{quote}

\begin{DUlineblock}{0em}
\item[] 9/13/15         AB      Updated logging to use mosaicLog class
\item[] 3/28/15         AB      Updated file read code to match new metaTrajIO API.
\item[] 7/18/12         AB      Initial version
\item[] 2/11/14         AB      Support qdf files that save the current in pA. This needs
\item[]
\begin{DUlineblock}{\DUlineblockindent}
\item[] format='pA' argument.
\end{DUlineblock}
\end{DUlineblock}
\index{qdfTrajIO (class in mosaic.qdfTrajIO)}

\begin{fulllineitems}
\phantomsection\label{api-doc/mosaic.traj:mosaic.qdfTrajIO.qdfTrajIO}\pysiglinewithargsret{\strong{class }\code{mosaic.qdfTrajIO.}\bfcode{qdfTrajIO}}{\emph{**kwargs}}{}
Bases: {\hyperref[api\string-doc/mosaic.meta:mosaic.metaTrajIO.metaTrajIO]{\emph{\code{mosaic.metaTrajIO.metaTrajIO}}}}

Use the readqdf module from EBS to read individual QDF files.

In addition to {\hyperref[api\string-doc/mosaic.meta:mosaic.metaTrajIO.metaTrajIO]{\emph{\code{metaTrajIO}}}} args, check if the 
feedback resistance (\code{Rfb}) and feedback capacitance (\code{Cfb})
are defined to convert qdf binary data into pA.

A typical settings section to read QDF files is shown below. Note, that
the values for \code{Rfb} and \code{Cfb} are specific to the amplifier used.

\begin{Verbatim}[commandchars=\\\{\}]
\PYG{l+s+s2}{\PYGZdq{}qdfTrajIO\PYGZdq{}}\PYG{o}{:} \PYG{p}{\PYGZob{}}
\PYG{l+s+s2}{\PYGZdq{}Rfb\PYGZdq{}}                           \PYG{o}{:} \PYG{l+m+mf}{9.1}\PYG{n+nx}{e}\PYG{o}{+}\PYG{l+m+mi}{12}\PYG{p}{,}
\PYG{l+s+s2}{\PYGZdq{}Cfb\PYGZdq{}}                           \PYG{o}{:} \PYG{l+m+mf}{1.07}\PYG{n+nx}{e}\PYG{o}{\PYGZhy{}}\PYG{l+m+mi}{12}\PYG{p}{,}
\PYG{l+s+s2}{\PYGZdq{}dcOffset\PYGZdq{}}                      \PYG{o}{:} \PYG{l+m+mf}{0.0}\PYG{p}{,}
\PYG{l+s+s2}{\PYGZdq{}filter\PYGZdq{}}                        \PYG{o}{:} \PYG{l+s+s2}{\PYGZdq{}*.qdf\PYGZdq{}}\PYG{p}{,}
\PYG{l+s+s2}{\PYGZdq{}start\PYGZdq{}}                         \PYG{o}{:} \PYG{l+m+mf}{0.0}
\PYG{p}{\PYGZcb{}}
\end{Verbatim}
\begin{quote}\begin{description}
\item[{Parameters}] \leavevmode\begin{description}
\item[{In addition to metaTrajIO.\_\_init\_\_ args,}] \leavevmode\begin{itemize}
\item {} 
\emph{Rfb} :               feedback resistance of amplifier

\item {} 
\emph{Cfb} :               feedback capacitance of amplifier

\item {} 
\emph{format} :    `V' for voltage or `pA' for current. Default is `V'

\end{itemize}

\end{description}

\item[{Returns}] \leavevmode
None

\item[{Errors}] \leavevmode\begin{itemize}
\item {} 
\emph{InsufficientArgumentsError} : if the mandatory arguments \code{Rfb} and \code{Cfb} are not set.

\end{itemize}

\end{description}\end{quote}
\index{readdata() (mosaic.qdfTrajIO.qdfTrajIO method)}

\begin{fulllineitems}
\phantomsection\label{api-doc/mosaic.traj:mosaic.qdfTrajIO.qdfTrajIO.readdata}\pysiglinewithargsret{\bfcode{readdata}}{\emph{fname}}{}
Read one or more files and append their data to the data pipeline.
Set a class attribute Fs with the sampling frequency in Hz.
\begin{quote}\begin{description}
\item[{Parameters}] \leavevmode\begin{itemize}
\item {} 
\emph{fname} :  list of data files to read

\end{itemize}

\item[{Returns}] \leavevmode
None

\item[{Errors}] \leavevmode\begin{itemize}
\item {} 
\emph{SamplingRateChangedError} : if the sampling rate for any data file differs from previous

\end{itemize}

\end{description}\end{quote}

\end{fulllineitems}


\end{fulllineitems}



\subsubsection{mosaic.binTrajIO module}
\label{api-doc/mosaic.traj:mosaic-bintrajio-module}\phantomsection\label{api-doc/mosaic.traj:module-mosaic.binTrajIO}\index{mosaic.binTrajIO (module)}
Binary file implementation of metaTrajIO. Read raw binary files with specified record sizes
\begin{quote}
\begin{quote}\begin{description}
\item[{Created}] \leavevmode
4/22/2013

\item[{Author}] \leavevmode
Arvind Balijepalli \textless{}\href{mailto:arvind.balijepalli@nist.gov}{arvind.balijepalli@nist.gov}\textgreater{}

\item[{License}] \leavevmode
See LICENSE.TXT

\item[{ChangeLog}] \leavevmode
\end{description}\end{quote}

\begin{DUlineblock}{0em}
\item[] 9/13/15         AB      Updated logging to use mosaicLog class
\item[] 3/28/15         AB      Updated file read code to match new metaTrajIO API.
\item[] 1/27/15         AB      Memory map files on read.
\item[] 1/26/15         AB      Refactored code to read interleaved binary data.
\item[] 7/27/14         AB      Update interface to specify python PythonStructCode instead of
\item[]
\begin{DUlineblock}{\DUlineblockindent}
\item[] RecordSize. This will allow any binary file to be decoded
\item[] The AmplifierScale and AmplifierOffset are set to 1 and 0
\item[] respectively if PythonStructCode is an integer or short.
\end{DUlineblock}
\item[] 4/22/13         AB      Initial version
\end{DUlineblock}
\end{quote}
\index{binTrajIO (class in mosaic.binTrajIO)}

\begin{fulllineitems}
\phantomsection\label{api-doc/mosaic.traj:mosaic.binTrajIO.binTrajIO}\pysiglinewithargsret{\strong{class }\code{mosaic.binTrajIO.}\bfcode{binTrajIO}}{\emph{**kwargs}}{}
Bases: {\hyperref[api\string-doc/mosaic.meta:mosaic.metaTrajIO.metaTrajIO]{\emph{\code{mosaic.metaTrajIO.metaTrajIO}}}}

Read a file that contains interleaved binary data, ordered by column. Only a single 
column that holds ionic current data is read. The current in pA 
is returned after scaling by the amplifier scale factor (\code{AmplifierScale}) and 
removing any offsets (\code{AmplifierOffset}) if provided.
\begin{quote}\begin{description}
\item[{Usage and Assumptions}] \leavevmode
Binary data is interleaved by column. For three columns (\emph{a}, \emph{b}, and \emph{c}) and \emph{N} rows, 
binary data is assumed to be of the form:
\begin{quote}

{[} a\_1, b\_1, c\_1, a\_2, b\_2, c\_2, ... ... ..., a\_N, b\_N, c\_N {]}
\end{quote}

The column layout is specified with the \code{ColumnTypes} parameter, which accepts a list of tuples. 
For the example above, if column \textbf{a} is the ionic current in a 64-bit floating point format, 
column \textbf{b} is the ionic current representation in 16-bit integer format and column \textbf{c} is 
an index in 16-bit integer format, the \code{ColumnTypes} paramter is a list with three 
tuples, one for each column, as shown below:
\begin{quote}

{[}(`curr\_pA', `float64'), (`AD\_V', `int16'), (`index', `int16'){]}
\end{quote}

The first element of each tuple is an arbitrary text label and the second element is 
a valid \href{http://docs.scipy.org/doc/numpy/user/basics.types.html}{Numpy type}.

Finally, the \code{IonicCurrentColumn} parameter holds the name (text label defined above) of the 
column that holds the ionic current time-series. Note that if an integer column is selected, 
the \code{AmplifierScale} and \code{AmplifierOffset} parameters can be used to convert the voltage from 
the A/D to a current.

Assuming that we use a floating point representation of the ionic current, and
a sampling rate of 50 kHz, a settings section that will read the binary file format 
defined above is:

\begin{Verbatim}[commandchars=\\\{\}]
\PYG{l+s+s2}{\PYGZdq{}binTrajIO\PYGZdq{}}\PYG{o}{:} \PYG{p}{\PYGZob{}}
        \PYG{l+s+s2}{\PYGZdq{}AmplifierScale\PYGZdq{}} \PYG{o}{:} \PYG{l+s+s2}{\PYGZdq{}1\PYGZdq{}}\PYG{p}{,}
        \PYG{l+s+s2}{\PYGZdq{}AmplifierOffset\PYGZdq{}} \PYG{o}{:} \PYG{l+s+s2}{\PYGZdq{}0\PYGZdq{}}\PYG{p}{,}
        \PYG{l+s+s2}{\PYGZdq{}SamplingFrequency\PYGZdq{}} \PYG{o}{:} \PYG{l+s+s2}{\PYGZdq{}50000\PYGZdq{}}\PYG{p}{,}
        \PYG{l+s+s2}{\PYGZdq{}ColumnTypes\PYGZdq{}} \PYG{o}{:} \PYG{l+s+s2}{\PYGZdq{}[(\PYGZsq{}curr\PYGZus{}pA\PYGZsq{}, \PYGZsq{}float64\PYGZsq{}), (\PYGZsq{}AD\PYGZus{}V\PYGZsq{}, \PYGZsq{}int16\PYGZsq{}), (\PYGZsq{}index\PYGZsq{}, \PYGZsq{}int16\PYGZsq{})]\PYGZdq{}}\PYG{p}{,}
        \PYG{l+s+s2}{\PYGZdq{}IonicCurrentColumn\PYGZdq{}} \PYG{o}{:} \PYG{l+s+s2}{\PYGZdq{}curr\PYGZus{}pA\PYGZdq{}}\PYG{p}{,}
        \PYG{l+s+s2}{\PYGZdq{}dcOffset\PYGZdq{}}\PYG{o}{:} \PYG{l+s+s2}{\PYGZdq{}0.0\PYGZdq{}}\PYG{p}{,} 
        \PYG{l+s+s2}{\PYGZdq{}filter\PYGZdq{}}\PYG{o}{:} \PYG{l+s+s2}{\PYGZdq{}*.bin\PYGZdq{}}\PYG{p}{,} 
        \PYG{l+s+s2}{\PYGZdq{}start\PYGZdq{}}\PYG{o}{:} \PYG{l+s+s2}{\PYGZdq{}0.0\PYGZdq{}}\PYG{p}{,}
        \PYG{l+s+s2}{\PYGZdq{}HeaderOffset\PYGZdq{}}\PYG{o}{:} \PYG{l+m+mi}{0} 
\PYG{p}{\PYGZcb{}}
\end{Verbatim}

\item[{Settings Examples}] \leavevmode
Read 16-bit signed integers (big endian) with a 512 byte header offset. Set the amplifier scale to 400 pA, sampling rate to 200 kHz.
\begin{quote}

\begin{Verbatim}[commandchars=\\\{\}]
\PYG{l+s+s2}{\PYGZdq{}binTrajIO\PYGZdq{}}\PYG{o}{:} \PYG{p}{\PYGZob{}}
        \PYG{l+s+s2}{\PYGZdq{}AmplifierOffset\PYGZdq{}}\PYG{o}{:} \PYG{l+s+s2}{\PYGZdq{}0.0\PYGZdq{}}\PYG{p}{,} 
        \PYG{l+s+s2}{\PYGZdq{}SamplingFrequency\PYGZdq{}}\PYG{o}{:} \PYG{l+m+mi}{200000}\PYG{p}{,} 
        \PYG{l+s+s2}{\PYGZdq{}AmplifierScale\PYGZdq{}}\PYG{o}{:} \PYG{l+s+s2}{\PYGZdq{}400./2**16\PYGZdq{}}\PYG{p}{,} 
        \PYG{l+s+s2}{\PYGZdq{}ColumnTypes\PYGZdq{}}\PYG{o}{:} \PYG{l+s+s2}{\PYGZdq{}[(\PYGZsq{}curr\PYGZus{}pA\PYGZsq{}, \PYGZsq{}\PYGZgt{}i2\PYGZsq{})]\PYGZdq{}}\PYG{p}{,} 
        \PYG{l+s+s2}{\PYGZdq{}dcOffset\PYGZdq{}}\PYG{o}{:} \PYG{l+m+mf}{0.0}\PYG{p}{,} 
        \PYG{l+s+s2}{\PYGZdq{}filter\PYGZdq{}}\PYG{o}{:} \PYG{l+s+s2}{\PYGZdq{}*.dat\PYGZdq{}}\PYG{p}{,} 
        \PYG{l+s+s2}{\PYGZdq{}start\PYGZdq{}}\PYG{o}{:} \PYG{l+m+mf}{0.0}\PYG{p}{,} 
        \PYG{l+s+s2}{\PYGZdq{}HeaderOffset\PYGZdq{}}\PYG{o}{:} \PYG{l+m+mi}{512}\PYG{p}{,} 
        \PYG{l+s+s2}{\PYGZdq{}IonicCurrentColumn\PYGZdq{}}\PYG{o}{:} \PYG{l+s+s2}{\PYGZdq{}curr\PYGZus{}pA\PYGZdq{}}
\PYG{p}{\PYGZcb{}}
\end{Verbatim}
\end{quote}

Read a two-column file: 64-bit floating point and 64-bit integers, and no header offset. Set the amplifier scale to 1 and sampling rate to 200 kHz.
\begin{quote}

\begin{Verbatim}[commandchars=\\\{\}]
\PYG{l+s+s2}{\PYGZdq{}binTrajIO\PYGZdq{}}\PYG{o}{:} \PYG{p}{\PYGZob{}}
        \PYG{l+s+s2}{\PYGZdq{}AmplifierOffset\PYGZdq{}}\PYG{o}{:} \PYG{l+s+s2}{\PYGZdq{}0.0\PYGZdq{}}\PYG{p}{,} 
        \PYG{l+s+s2}{\PYGZdq{}SamplingFrequency\PYGZdq{}}\PYG{o}{:} \PYG{l+m+mi}{200000}\PYG{p}{,} 
        \PYG{l+s+s2}{\PYGZdq{}AmplifierScale\PYGZdq{}}\PYG{o}{:} \PYG{l+s+s2}{\PYGZdq{}1.0\PYGZdq{}}\PYG{p}{,} 
        \PYG{l+s+s2}{\PYGZdq{}ColumnTypes\PYGZdq{}} \PYG{o}{:} \PYG{l+s+s2}{\PYGZdq{}[(\PYGZsq{}curr\PYGZus{}pA\PYGZsq{}, \PYGZsq{}float64\PYGZsq{}), (\PYGZsq{}AD\PYGZus{}V\PYGZsq{}, \PYGZsq{}int64\PYGZsq{})]\PYGZdq{}}\PYG{p}{,}
        \PYG{l+s+s2}{\PYGZdq{}dcOffset\PYGZdq{}}\PYG{o}{:} \PYG{l+m+mf}{0.0}\PYG{p}{,} 
        \PYG{l+s+s2}{\PYGZdq{}filter\PYGZdq{}}\PYG{o}{:} \PYG{l+s+s2}{\PYGZdq{}*.bin\PYGZdq{}}\PYG{p}{,} 
        \PYG{l+s+s2}{\PYGZdq{}start\PYGZdq{}}\PYG{o}{:} \PYG{l+m+mf}{0.0}\PYG{p}{,} 
        \PYG{l+s+s2}{\PYGZdq{}HeaderOffset\PYGZdq{}}\PYG{o}{:} \PYG{l+m+mi}{0}\PYG{p}{,} 
        \PYG{l+s+s2}{\PYGZdq{}IonicCurrentColumn\PYGZdq{}}\PYG{o}{:} \PYG{l+s+s2}{\PYGZdq{}curr\PYGZus{}pA\PYGZdq{}}
\PYG{p}{\PYGZcb{}}
\end{Verbatim}
\end{quote}

\item[{Parameters}] \leavevmode\begin{description}
\item[{In addition to {\hyperref[api\string-doc/mosaic.meta:mosaic.metaTrajIO.metaTrajIO]{\emph{\code{metaTrajIO}}}} args,}] \leavevmode\begin{itemize}
\item {} 
\emph{AmplifierScale} :            Full scale of amplifier (pA/2\textasciicircum{}nbits) that varies with the gain (default: 1.0).

\item {} 
\emph{AmplifierOffset} :           Current offset in the recorded data in pA  (default: 0.0).

\item {} 
\emph{SamplingFrequency} : Sampling rate of data in the file in Hz.

\item {} 
\emph{HeaderOffset} :              Ignore first \emph{n} bytes of the file for header (default: 0 bytes).

\item {} 
\emph{ColumnTypes} :       A list of tuples with column names and types (see \href{http://docs.scipy.org/doc/numpy/user/basics.types.html}{Numpy types}). Note only integer and floating point numbers are supported.

\item {} 
\emph{IonicCurrentColumn} : Column name that holds ionic current data.

\end{itemize}

\end{description}

\item[{Returns}] \leavevmode
None

\item[{Errors}] \leavevmode
None

\end{description}\end{quote}
\index{readdata() (mosaic.binTrajIO.binTrajIO method)}

\begin{fulllineitems}
\phantomsection\label{api-doc/mosaic.traj:mosaic.binTrajIO.binTrajIO.readdata}\pysiglinewithargsret{\bfcode{readdata}}{\emph{fname}}{}
Return raw data from a single data file. Set a class 
attribute Fs with the sampling frequency in Hz.
\begin{quote}\begin{description}
\item[{Parameters}] \leavevmode\begin{itemize}
\item {} 
\emph{fname} :  fileame to read

\end{itemize}

\item[{Returns}] \leavevmode\begin{itemize}
\item {} 
An array object that holds raw (unscaled) data from \emph{fname}

\end{itemize}

\item[{Errors}] \leavevmode
None

\end{description}\end{quote}

\end{fulllineitems}

\index{scaleData() (mosaic.binTrajIO.binTrajIO method)}

\begin{fulllineitems}
\phantomsection\label{api-doc/mosaic.traj:mosaic.binTrajIO.binTrajIO.scaleData}\pysiglinewithargsret{\bfcode{scaleData}}{\emph{data}}{}
See {\hyperref[api\string-doc/mosaic.meta:mosaic.metaTrajIO.metaTrajIO.scaleData]{\emph{\code{mosaic.metaTrajIO.metaTrajIO.scaleData()}}}}.

\end{fulllineitems}


\end{fulllineitems}



\subsubsection{mosaic.tsvTrajIO module}
\label{api-doc/mosaic.traj:mosaic-tsvtrajio-module}\phantomsection\label{api-doc/mosaic.traj:module-mosaic.tsvTrajIO}\index{mosaic.tsvTrajIO (module)}
An implementation of metaTrajIO that reads tab separated valued (TSV) files
\begin{quote}\begin{description}
\item[{Created}] \leavevmode
7/31/2012

\item[{Author}] \leavevmode
Arvind Balijepalli \textless{}\href{mailto:arvind.balijepalli@nist.gov}{arvind.balijepalli@nist.gov}\textgreater{}

\item[{License}] \leavevmode
See LICENSE.TXT

\item[{ChangeLog}] \leavevmode
\end{description}\end{quote}

\begin{DUlineblock}{0em}
\item[] 11/30/15        AB      Assumes \code{timeCol} is specified in seconds.
\item[] 11/30/15        AB      Added a new keyword \code{scale} to allow scaling TSV data.
\item[] 3/28/15         AB      Updated file read code to match new metaTrajIO API.
\item[] 6/30/13         AB      Added the `seprator' kwarg to the class initializer to allow any delimited
\item[]
\begin{DUlineblock}{\DUlineblockindent}
\item[] files to be read. e.g. `''``t' (default), `,', etc.
\end{DUlineblock}
\item[] 7/31/12         AB      Initial version
\end{DUlineblock}
\index{tsvTrajIO (class in mosaic.tsvTrajIO)}

\begin{fulllineitems}
\phantomsection\label{api-doc/mosaic.traj:mosaic.tsvTrajIO.tsvTrajIO}\pysiglinewithargsret{\strong{class }\code{mosaic.tsvTrajIO.}\bfcode{tsvTrajIO}}{\emph{**kwargs}}{}
Bases: {\hyperref[api\string-doc/mosaic.meta:mosaic.metaTrajIO.metaTrajIO]{\emph{\code{mosaic.metaTrajIO.metaTrajIO}}}}

Read tab separated valued (TSV) files.
\begin{quote}\begin{description}
\item[{Parameters}] \leavevmode\begin{description}
\item[{In addition to {\hyperref[api\string-doc/mosaic.meta:mosaic.metaTrajIO.metaTrajIO]{\emph{\code{metaTrajIO}}}} args,}] \leavevmode\begin{itemize}
\item {} 
\emph{headers} :           If True, the first row is ignored (default: True)

\item {} 
\emph{separator} : set the data separator (defualt: `''``t')

\item {} 
\emph{scale} : set the data scale (default: 1). For example to convert from  to pA set \code{scale=1e12}.

\end{itemize}
\begin{description}
\item[{Either:}] \leavevmode\begin{itemize}
\item {} 
\emph{Fs} :                        Sampling frequency in Hz. If set, all other options are ignored and the first column in the file is assumed to be the current in pA.

\end{itemize}

\item[{Or:}] \leavevmode\begin{itemize}
\item {} 
\emph{nCols} :             number of columns in TSV file (default:2, first column is time in ms and second is current in pA)

\item {} 
\emph{timeCol} :           explicitly set the time column (default: 0, first col)

\item {} 
\emph{currCol} :           explicitly set the position of the current column (default: 1)

\end{itemize}

\end{description}

\end{description}

If neither \code{Fs} nor \{\code{nCols}, \code{timeCol}, \code{currCol}\} are set then the latter 
is assumed with the listed default values.

\end{description}\end{quote}
\index{readdata() (mosaic.tsvTrajIO.tsvTrajIO method)}

\begin{fulllineitems}
\phantomsection\label{api-doc/mosaic.traj:mosaic.tsvTrajIO.tsvTrajIO.readdata}\pysiglinewithargsret{\bfcode{readdata}}{\emph{fname}}{}
Read a single TSV file and return raw (unscaled) data contained within it.
Set/update a class attribute Fs with the sampling frequency in Hz.
\begin{quote}\begin{description}
\item[{Parameters}] \leavevmode\begin{itemize}
\item {} 
\emph{fname} :  fileame to read

\end{itemize}

\item[{Returns}] \leavevmode\begin{itemize}
\item {} 
An array object that holds raw (unscaled) data from \emph{fname}

\end{itemize}

\item[{Errors}] \leavevmode\begin{itemize}
\item {} 
\emph{SamplingRateChangedError} : if the sampling rate for any data file differs from previous

\end{itemize}

\end{description}\end{quote}

\end{fulllineitems}


\end{fulllineitems}



\subsection{Time-Series Filters}
\label{api-doc/mosaic.filter::doc}\label{api-doc/mosaic.filter:time-series-filters}

\subsubsection{mosaic.waveletDenoiseFilter module}
\label{api-doc/mosaic.filter:mosaic-waveletdenoisefilter-module}\phantomsection\label{api-doc/mosaic.filter:module-mosaic.waveletDenoiseFilter}\index{mosaic.waveletDenoiseFilter (module)}
Implementation of a wavelet based denoising filter
\begin{quote}\begin{description}
\item[{Created}] \leavevmode
8/31/2014

\item[{Author}] \leavevmode
Arvind Balijepalli \textless{}\href{mailto:arvind.balijepalli@nist.gov}{arvind.balijepalli@nist.gov}\textgreater{}

\item[{License}] \leavevmode
See LICENSE.TXT

\item[{Author}] \leavevmode
Arvind Balijepalli

\item[{ChangeLog}] \leavevmode
\end{description}\end{quote}

\begin{DUlineblock}{0em}
\item[] 9/13/15         AB      Updated logging to use mosaicLog class
\item[] 8/31/14         AB      Initial version
\end{DUlineblock}
\index{waveletDenoiseFilter (class in mosaic.waveletDenoiseFilter)}

\begin{fulllineitems}
\phantomsection\label{api-doc/mosaic.filter:mosaic.waveletDenoiseFilter.waveletDenoiseFilter}\pysiglinewithargsret{\strong{class }\code{mosaic.waveletDenoiseFilter.}\bfcode{waveletDenoiseFilter}}{\emph{**kwargs}}{}
Bases: {\hyperref[api\string-doc/mosaic.meta:mosaic.metaIOFilter.metaIOFilter]{\emph{\code{mosaic.metaIOFilter.metaIOFilter}}}}
\begin{quote}\begin{description}
\item[{Keyword Args}] \leavevmode\begin{description}
\item[{In addition to metaIOFilter args,}] \leavevmode\begin{itemize}
\item {} 
\emph{wavelet} :           the type of wavelet

\item {} 
\emph{level} :             wavelet level

\item {} 
\emph{threshold} : threshold type

\end{itemize}

\end{description}

\end{description}\end{quote}
\index{filterData() (mosaic.waveletDenoiseFilter.waveletDenoiseFilter method)}

\begin{fulllineitems}
\phantomsection\label{api-doc/mosaic.filter:mosaic.waveletDenoiseFilter.waveletDenoiseFilter.filterData}\pysiglinewithargsret{\bfcode{filterData}}{\emph{icurr}, \emph{Fs}}{}
Denoise an ionic current time-series and store it in self.eventData
\begin{quote}\begin{description}
\item[{Parameters}] \leavevmode\begin{itemize}
\item {} 
\emph{icurr} :     ionic current in pA

\item {} 
\emph{Fs} :        original sampling frequency in Hz

\end{itemize}

\end{description}\end{quote}

\end{fulllineitems}

\index{formatsettings() (mosaic.waveletDenoiseFilter.waveletDenoiseFilter method)}

\begin{fulllineitems}
\phantomsection\label{api-doc/mosaic.filter:mosaic.waveletDenoiseFilter.waveletDenoiseFilter.formatsettings}\pysiglinewithargsret{\bfcode{formatsettings}}{}{}
Return a formatted string of filter settings

\end{fulllineitems}


\end{fulllineitems}



\subsubsection{mosaic.besselLowpassFilter module}
\label{api-doc/mosaic.filter:mosaic-bessellowpassfilter-module}\phantomsection\label{api-doc/mosaic.filter:module-mosaic.besselLowpassFilter}\index{mosaic.besselLowpassFilter (module)}
Implementation of an `N' order Bessel filter
\begin{quote}\begin{description}
\item[{Created}] \leavevmode
7/1/2013

\item[{Author}] \leavevmode
Arvind Balijepalli \textless{}\href{mailto:arvind.balijepalli@nist.gov}{arvind.balijepalli@nist.gov}\textgreater{}

\item[{License}] \leavevmode
See LICENSE.TXT

\item[{ChangeLog}] \leavevmode
\end{description}\end{quote}

\begin{DUlineblock}{0em}
\item[] 9/13/15         AB      Updated logging to use mosaicLog class
\item[] 7/1/13          AB      Initial version
\end{DUlineblock}
\index{besselLowpassFilter (class in mosaic.besselLowpassFilter)}

\begin{fulllineitems}
\phantomsection\label{api-doc/mosaic.filter:mosaic.besselLowpassFilter.besselLowpassFilter}\pysiglinewithargsret{\strong{class }\code{mosaic.besselLowpassFilter.}\bfcode{besselLowpassFilter}}{\emph{**kwargs}}{}
Bases: {\hyperref[api\string-doc/mosaic.meta:mosaic.metaIOFilter.metaIOFilter]{\emph{\code{mosaic.metaIOFilter.metaIOFilter}}}}
\begin{quote}\begin{description}
\item[{Keyword Args}] \leavevmode
\end{description}\end{quote}
\begin{description}
\item[{In addition to metaIOFilter.\_\_init\_\_ args,}] \leavevmode\begin{itemize}
\item {} 
\emph{filterOrder} :               the filter order

\item {} 
\emph{filterCutoff} :      filter cutoff frequency in Hz

\end{itemize}

\end{description}
\index{filterData() (mosaic.besselLowpassFilter.besselLowpassFilter method)}

\begin{fulllineitems}
\phantomsection\label{api-doc/mosaic.filter:mosaic.besselLowpassFilter.besselLowpassFilter.filterData}\pysiglinewithargsret{\bfcode{filterData}}{\emph{icurr}, \emph{Fs}}{}
Denoise an ionic current time-series and store it in self.eventData
\begin{quote}\begin{description}
\item[{Parameters}] \leavevmode\begin{itemize}
\item {} 
\emph{icurr} :     ionic current in pA

\item {} 
\emph{Fs} :        original sampling frequency in Hz

\end{itemize}

\end{description}\end{quote}

\end{fulllineitems}

\index{formatsettings() (mosaic.besselLowpassFilter.besselLowpassFilter method)}

\begin{fulllineitems}
\phantomsection\label{api-doc/mosaic.filter:mosaic.besselLowpassFilter.besselLowpassFilter.formatsettings}\pysiglinewithargsret{\bfcode{formatsettings}}{}{}
Populate \emph{logObject} with settings strings for display

\end{fulllineitems}


\end{fulllineitems}



\subsubsection{mosaic.convolutionFilter module}
\label{api-doc/mosaic.filter:mosaic-convolutionfilter-module}\phantomsection\label{api-doc/mosaic.filter:module-mosaic.convolutionFilter}\index{mosaic.convolutionFilter (module)}
Implementation of a weighted moving average (tap delay line) filter
\begin{quote}\begin{description}
\item[{Created}] \leavevmode
8/16/2013

\item[{Author}] \leavevmode
Arvind Balijepalli \textless{}\href{mailto:arvind.balijepalli@nist.gov}{arvind.balijepalli@nist.gov}\textgreater{}

\item[{License}] \leavevmode
See LICENSE.TXT

\item[{ChangeLog}] \leavevmode
\end{description}\end{quote}

\begin{DUlineblock}{0em}
\item[] 9/13/15         AB      Updated logging to use mosaicLog class
\item[] 8/16/13         AB      Initial version
\end{DUlineblock}
\index{convolutionFilter (class in mosaic.convolutionFilter)}

\begin{fulllineitems}
\phantomsection\label{api-doc/mosaic.filter:mosaic.convolutionFilter.convolutionFilter}\pysiglinewithargsret{\strong{class }\code{mosaic.convolutionFilter.}\bfcode{convolutionFilter}}{\emph{**kwargs}}{}
Bases: {\hyperref[api\string-doc/mosaic.meta:mosaic.metaIOFilter.metaIOFilter]{\emph{\code{mosaic.metaIOFilter.metaIOFilter}}}}
\begin{quote}\begin{description}
\item[{Keyword Args}] \leavevmode
\end{description}\end{quote}
\begin{description}
\item[{In addition to metaIOFilter.\_\_init\_\_ args,}] \leavevmode\begin{itemize}
\item {} 
\emph{filterCoeff} :               filter coefficients (default is a 10 point uniform moving average)

\end{itemize}

\end{description}
\index{filterData() (mosaic.convolutionFilter.convolutionFilter method)}

\begin{fulllineitems}
\phantomsection\label{api-doc/mosaic.filter:mosaic.convolutionFilter.convolutionFilter.filterData}\pysiglinewithargsret{\bfcode{filterData}}{\emph{icurr}, \emph{Fs}}{}
Denoise an ionic current time-series and store it in self.eventData
\begin{quote}\begin{description}
\item[{Parameters}] \leavevmode\begin{itemize}
\item {} 
\emph{icurr} :     ionic current in pA

\item {} 
\emph{Fs} :        original sampling frequency in Hz

\end{itemize}

\end{description}\end{quote}

\end{fulllineitems}

\index{formatsettings() (mosaic.convolutionFilter.convolutionFilter method)}

\begin{fulllineitems}
\phantomsection\label{api-doc/mosaic.filter:mosaic.convolutionFilter.convolutionFilter.formatsettings}\pysiglinewithargsret{\bfcode{formatsettings}}{}{}
Return a formatted string of filter settings

\end{fulllineitems}


\end{fulllineitems}



\subsection{Event Partition and Segment}
\label{api-doc/mosaic.partition:event-partition-and-segment}\label{api-doc/mosaic.partition::doc}

\subsubsection{mosaic.eventSegment module}
\label{api-doc/mosaic.partition:mosaic-eventsegment-module}\phantomsection\label{api-doc/mosaic.partition:module-mosaic.eventSegment}\index{mosaic.eventSegment (module)}
Partition a trajectory into individual events and pass each event 
to an implementation of eventProcessor
\begin{quote}\begin{description}
\item[{Created}] \leavevmode
7/17/2012

\item[{Author}] \leavevmode
Arvind Balijepalli \textless{}\href{mailto:arvind.balijepalli@nist.gov}{arvind.balijepalli@nist.gov}\textgreater{}

\item[{License}] \leavevmode
See LICENSE.TXT

\item[{ChangeLog}] \leavevmode
\end{description}\end{quote}

\begin{DUlineblock}{0em}
\item[] 5/17/14         AB  Delete plotting support
\item[] 5/17/14         AB  Add metaMDIO support for meta-data and time-series storage
\item[] 2/14/14         AB      Pass absdatidx argument to event processing to track absolute time of
\item[]
\begin{DUlineblock}{\DUlineblockindent}
\item[] event start for capture rate estimation.
\end{DUlineblock}
\item[] 6/22/13         AB      Use plotting hooks in metaEventPartition to plot blockade depth histogram in
\item[]
\begin{DUlineblock}{\DUlineblockindent}
\item[] real-time using matplotlib.
\end{DUlineblock}
\item[] 4/22/13         AB      Rewrote this class as an implementation of the base class metaEventPartition.
\item[]
\begin{DUlineblock}{\DUlineblockindent}
\item[] Included event processing parallelization using ZMQ.
\end{DUlineblock}
\item[] 9/26/12         AB  Allowed automatic open channel state calculation to be overridden.
\item[]
\begin{DUlineblock}{\DUlineblockindent}
\item[] To do this the settings ``meanOpenCurr'',''sdOpenCurr'' and ``slopeOpenCurr''
\item[] must be set manually. If all three settings are absent or
\item[] set to 01, they are autuomatically estimated.
\item[] Added ``writeEventTS'' boolean setting to control whether raw
\item[] events are written to file. Default is ON (1)
\end{DUlineblock}
\item[] 8/24/12         AB      Settings are now read from a settings file that
\item[]
\begin{DUlineblock}{\DUlineblockindent}
\item[] is located either with the data or in the working directory
\item[] that the program is run from. Each class that relies on the
\item[] settings file will fallback to default values if the file
\item[] is not found.
\end{DUlineblock}
\item[] 7/17/12         AB      Initial version
\end{DUlineblock}
\index{eventSegment (class in mosaic.eventSegment)}

\begin{fulllineitems}
\phantomsection\label{api-doc/mosaic.partition:mosaic.eventSegment.eventSegment}\pysiglinewithargsret{\strong{class }\code{mosaic.eventSegment.}\bfcode{eventSegment}}{\emph{trajDataObj}, \emph{eventProcHnd}, \emph{eventPartitionSettings}, \emph{eventProcSettings}, \emph{settingsString}}{}
Bases: {\hyperref[api\string-doc/mosaic.meta:mosaic.metaEventPartition.metaEventPartition]{\emph{\code{mosaic.metaEventPartition.metaEventPartition}}}}

Implement an event partitioning algorithm by sub-classing the metaEventPartition class
\begin{quote}\begin{description}
\item[{Settings}] \leavevmode
In addition to the parameters described in \code{metaEventPartition}, the following parameters from are read from the settings file (.settings in the data path or current working directory):
\begin{itemize}
\item {} \begin{description}
\item[{\emph{blockSizeSec}}] \leavevmode{[}Functions that perform block processing use this value to set the size of {]}
their windows in seconds. For example, open channel conductance is processed
for windows with a size specified by this parameter. (default: 1 second)

\end{description}

\item {} 
\emph{eventPad} :          Number of points to include before and after a detected event. (default: 500)

\item {} 
\emph{minEventLength} :    Minimum number points in the blocked state to qualify as an event (default: 5)

\item {} \begin{description}
\item[{\emph{eventThreshold}}] \leavevmode{[}Threshold, number of SD away from the open channel mean. If the abs(curr) is less{]}
than `abs(mean)-(eventThreshold*SD)' a new event is registered (default: 6)

\end{description}

\item {} \begin{description}
\item[{\emph{driftThreshold}}] \leavevmode{[}Trigger a drift warning when the mean open channel current deviates by `driftThreshold'*{]}
SD from the baseline open channel current (default: 2)

\end{description}

\item {} \begin{description}
\item[{\emph{maxDriftRate}}] \leavevmode{[}Trigger a warning when the open channel conductance changes at a rate faster {]}
than that specified. (default: 2 pA/s)

\end{description}

\item {} \begin{description}
\item[{\emph{meanOpenCurr}}] \leavevmode{[}Explicitly set mean open channel current. (pA) (default: -1, to {]}
calculate automatically)

\end{description}

\item {} \begin{description}
\item[{\emph{sdOpenCurr}}] \leavevmode{[}Explicitly set open channel current SD. (pA) (default: -1, to {]}
calculate automatically)

\end{description}

\item {} \begin{description}
\item[{\emph{slopeOpenCurr}}] \leavevmode{[}Explicitly set open channel current slope. (default: -1, to {]}
calculate automatically)

\end{description}

\end{itemize}

\end{description}\end{quote}
\index{formatsettings() (mosaic.eventSegment.eventSegment method)}

\begin{fulllineitems}
\phantomsection\label{api-doc/mosaic.partition:mosaic.eventSegment.eventSegment.formatsettings}\pysiglinewithargsret{\bfcode{formatsettings}}{}{}
Return a formatted string of settings for display in the output log.

\end{fulllineitems}

\index{formatstats() (mosaic.eventSegment.eventSegment method)}

\begin{fulllineitems}
\phantomsection\label{api-doc/mosaic.partition:mosaic.eventSegment.eventSegment.formatstats}\pysiglinewithargsret{\bfcode{formatstats}}{}{}
Return a formatted string of statistics for display in the output log.

\end{fulllineitems}


\end{fulllineitems}



\subsection{Event Processing}
\label{api-doc/mosaic.processing:event-processing}\label{api-doc/mosaic.processing::doc}

\subsubsection{mosaic.adept2State module}
\label{api-doc/mosaic.processing:mosaic-adept2state-module}\phantomsection\label{api-doc/mosaic.processing:module-mosaic.adept2State}\index{mosaic.adept2State (module)}
A class that extends metaEventProcessing to implement the step response algorithm from {[}Balijepalli:2014{]}
\begin{quote}\begin{description}
\item[{Created}] \leavevmode
4/18/2013

\item[{Author}] \leavevmode
Arvind Balijepalli \textless{}\href{mailto:arvind.balijepalli@nist.gov}{arvind.balijepalli@nist.gov}\textgreater{}

\item[{License}] \leavevmode
See LICENSE.TXT

\item[{ChangeLog}] \leavevmode
\end{description}\end{quote}

\begin{DUlineblock}{0em}
\item[] 03/30/16        AB      Change UnlinkRCConst to LinkRCConst to avoid double negatives.
\item[] 12/09/15        KB      Added Windows specific optimizations
\item[] 8/24/15         AB      Rename algorithm to ADEPT 2 State.
\item[] 7/23/15         JF  Added a new test to reject RC Constants \textless{}=0
\item[] 6/24/15         AB      Added an option to unlink the RC constants in stepResponseAnalysis.
\item[] 11/7/14         AB      Error codes describing event rejection are now more specific.
\item[] 11/5/14         AB      Fixed a bug in the event fitting logic that prevented
\item[]
\begin{DUlineblock}{\DUlineblockindent}
\item[] long events from being correctly analyzed.
\end{DUlineblock}
\item[] 5/17/14         AB  Modified md interface functions for metaMDIO support
\item[] 2/16/14         AB      Added new metadata field, `AbsEventStart' to track
\item[]
\begin{DUlineblock}{\DUlineblockindent}
\item[] global time of event start to allow capture rate estimation.
\end{DUlineblock}
\item[] 6/20/13         AB      Added an additional check to reject events
\item[]
\begin{DUlineblock}{\DUlineblockindent}
\item[] with blockade depths \textgreater{} BlockRejectRatio (default: 0.8)
\end{DUlineblock}
\item[] 4/18/13         AB      Initial version
\end{DUlineblock}
\index{adept2State (class in mosaic.adept2State)}

\begin{fulllineitems}
\phantomsection\label{api-doc/mosaic.processing:mosaic.adept2State.adept2State}\pysiglinewithargsret{\strong{class }\code{mosaic.adept2State.}\bfcode{adept2State}}{\emph{icurr}, \emph{Fs}, \emph{**kwargs}}{}
Bases: {\hyperref[api\string-doc/mosaic.meta:mosaic.metaEventProcessor.metaEventProcessor]{\emph{\code{mosaic.metaEventProcessor.metaEventProcessor}}}}

Analyze an event that is characteristic of PEG blockades. This method includes system 
information in the analysis, specifically the filtering effects (throught the RC constant)
of either amplifiers or the membrane/nanopore complex. The analysis generates several 
parameters that are stored as metadata including:
\begin{enumerate}
\item {} 
Blockade depth: the ratio of the open channel current to the blocked current

\item {} 
Residence time: the time the molecule spends inside the pore

\item {} 
Tau: the RC constant of the response to a step input (e.g. the entry or exit of the molecule into or out of the nanopore).

\end{enumerate}
\begin{quote}\begin{description}
\item[{Keyword Args}] \leavevmode\begin{description}
\item[{In addition to {\hyperref[api\string-doc/mosaic.meta:mosaic.metaEventProcessor.metaEventProcessor]{\emph{\code{metaEventProcessor}}}} args,}] \leavevmode\begin{itemize}
\item {} 
\emph{FitTol} :            Tolerance value for the least squares algorithm that controls the convergence of the fit (Default: \emph{1e-7}).

\item {} 
\emph{FitIters} :          Maximum number of iterations before terminating the fit (Default: \emph{50000}).

\item {} 
\emph{LinkRCConst} :       When True, the RC constants associated with each state in the fit function are varied together. (Default: \emph{True})

\end{itemize}

\end{description}

\item[{Errors}] \leavevmode
When an event cannot be analyzed, the blockade depth, residence time and rise time are set to -1.

\end{description}\end{quote}
\index{formatsettings() (mosaic.adept2State.adept2State method)}

\begin{fulllineitems}
\phantomsection\label{api-doc/mosaic.processing:mosaic.adept2State.adept2State.formatsettings}\pysiglinewithargsret{\bfcode{formatsettings}}{}{}
Return a formatted string of settings for display

\end{fulllineitems}

\index{mdAveragePropertiesList() (mosaic.adept2State.adept2State method)}

\begin{fulllineitems}
\phantomsection\label{api-doc/mosaic.processing:mosaic.adept2State.adept2State.mdAveragePropertiesList}\pysiglinewithargsret{\bfcode{mdAveragePropertiesList}}{}{}
Return a list of meta-data properties that will be averaged 
and displayed at the end of a run.

\end{fulllineitems}


\end{fulllineitems}

\index{datblock (class in mosaic.adept2State)}

\begin{fulllineitems}
\phantomsection\label{api-doc/mosaic.processing:mosaic.adept2State.datblock}\pysiglinewithargsret{\strong{class }\code{mosaic.adept2State.}\bfcode{datblock}}{\emph{dat}}{}
Smart data block that holds time-series data and keeps track
of its mean and SD.

\end{fulllineitems}



\subsubsection{mosaic.adept module}
\label{api-doc/mosaic.processing:mosaic-adept-module}\phantomsection\label{api-doc/mosaic.processing:module-mosaic.adept}\index{mosaic.adept (module)}
Analyze a multi-step event
\begin{quote}\begin{description}
\item[{Created}] \leavevmode
4/18/2013

\item[{Author}] \leavevmode
Arvind Balijepalli \textless{}\href{mailto:arvind.balijepalli@nist.gov}{arvind.balijepalli@nist.gov}\textgreater{}

\item[{License}] \leavevmode
See LICENSE.TXT

\item[{ChangeLog}] \leavevmode
\end{description}\end{quote}

\begin{DUlineblock}{0em}
\item[] 05/22/16    JF  Added new test to reject BD \textless{} 0 or BD \textgreater{} 1, improved readability of error tests.
\item[] 03/30/16        AB      Change UnlinkRCConst to LinkRCConst to avoid double negatives.
\item[] 3/16/16         AB      Migrate InitThreshold setting to CUSUM StepSize.
\item[] 2/22/16         AB      Use CUSUM to estimate intial guesses in ADEPT for long events.
\item[] 2/20/16         AB      Format settings log.
\item[] 12/09/15        KB      Added Windows specific optimizations
\item[] 8/24/15         AB      Rename algorithm to ADEPT.
\item[] 8/02/15         JF      Added a new test to reject RC Constants \textless{}=0
\item[] 4/12/15         AB      Refactored code to improve reusability.
\item[] 3/20/15         AB      Added a maximum event length setting (MaxEventLength) that automatically rejects events longer than the specified value.
\item[] 3/20/15         AB      Added a new metadata column (mdStateResTime) that saves the residence time of each state to the database.
\item[] 3/6/15          AB      Added a new test for negative event delays
\item[] 3/6/15          JF      Added MinStateLength to output log
\item[] 3/5/15          AB      Updated initial state determination to include a minumum state length parameter (MinStateLength).
\item[]
\begin{DUlineblock}{\DUlineblockindent}
\item[] Initial state estimates now utilize gradient information for improved state identification.
\end{DUlineblock}
\item[] 1/7/15          AB  Save the number of states in an event to the DB using the mdNStates column
\item[] 12/31/14        AB      Changed multi-state function to include a separate tau for
\item[]
\begin{DUlineblock}{\DUlineblockindent}
\item[] each state following Balijepalli et al, ACS Nano 2014.
\end{DUlineblock}
\item[] 12/30/14        JF      Removed min/max constraint on tau
\item[] 11/7/14         AB      Error codes describing event rejection are now more specific.
\item[] 11/6/14         AB      Fixed a bug in the event fitting logic that prevents the
\item[]
\begin{DUlineblock}{\DUlineblockindent}
\item[] analysis of long states.
\end{DUlineblock}
\item[] 8/21/14         AB      Added AbsEventStart and BlockDepth (constructed from mdCurrentStep
\item[]
\begin{DUlineblock}{\DUlineblockindent}
\item[] and mdOpenChCurrent) metadata.
\end{DUlineblock}
\item[] 5/17/14         AB  Modified md interface functions for metaMDIO support
\item[] 9/26/13         AB      Initial version
\end{DUlineblock}
\index{adept (class in mosaic.adept)}

\begin{fulllineitems}
\phantomsection\label{api-doc/mosaic.processing:mosaic.adept.adept}\pysiglinewithargsret{\strong{class }\code{mosaic.adept.}\bfcode{adept}}{\emph{icurr}, \emph{Fs}, \emph{**kwargs}}{}
Bases: {\hyperref[api\string-doc/mosaic.meta:mosaic.metaEventProcessor.metaEventProcessor]{\emph{\code{mosaic.metaEventProcessor.metaEventProcessor}}}}

Analyze a multi-step event that contains two or more states. This method includes system 
information in the analysis, specifically the filtering effects (through the RC constant)
of either amplifiers or the membrane/nanopore complex. The analysis generates several 
parameters that are stored as metadata including:
\begin{enumerate}
\item {} 
Blockade depth: the ratio of the open channel current to the blocked current

\item {} 
Residence time: the time the molecule spends inside the pore

\item {} 
Tau: the RC constant  of the response to a step input (e.g. the entry or exit of the molecule into or out of the nanopore).

\end{enumerate}
\begin{quote}\begin{description}
\item[{Keyword Args}] \leavevmode\begin{description}
\item[{In addition to {\hyperref[api\string-doc/mosaic.meta:mosaic.metaEventProcessor.metaEventProcessor]{\emph{\code{metaEventProcessor}}}} args,}] \leavevmode\begin{itemize}
\item {} 
\emph{StepSize} :                  The multiple of the standard deviations considered significant to dtecting an event (default: 3.0).

\item {} 
\emph{MinStateLength} :    minimum number of data points required to assign a state within an event (default: 4)

\item {} 
\emph{MaxEventLength} :    maximum length (in data points) of events that will be processed (default: 10000)

\item {} 
\emph{FitTol} :                    fit tolerance for convergence (default: 1.e-7)

\item {} 
\emph{FitIters} :                  maximum fit iterations (default: 5000)

\item {} 
\emph{LinkRCConst} :       When True, the RC constants associated with each state in the fit function are varied together. (Default: \emph{True})

\end{itemize}

\end{description}

\item[{Errors}] \leavevmode
When an event cannot be analyzed, all metadata are set to -1.

\end{description}\end{quote}
\index{formatsettings() (mosaic.adept.adept method)}

\begin{fulllineitems}
\phantomsection\label{api-doc/mosaic.processing:mosaic.adept.adept.formatsettings}\pysiglinewithargsret{\bfcode{formatsettings}}{}{}
Return a formatted string of settings for display

\end{fulllineitems}

\index{mdAveragePropertiesList() (mosaic.adept.adept method)}

\begin{fulllineitems}
\phantomsection\label{api-doc/mosaic.processing:mosaic.adept.adept.mdAveragePropertiesList}\pysiglinewithargsret{\bfcode{mdAveragePropertiesList}}{}{}
Return a list of meta-data properties that will be averaged 
and displayed at the end of a run.

\end{fulllineitems}


\end{fulllineitems}



\subsubsection{mosaic.cusumPlus module}
\label{api-doc/mosaic.processing:mosaic-cusumplus-module}\phantomsection\label{api-doc/mosaic.processing:module-mosaic.cusumPlus}\index{mosaic.cusumPlus (module)}
Analyze a multi-step event with the CUSUM+ algorithm
\begin{quote}\begin{description}
\item[{Created}] \leavevmode
2/10/2015

\item[{Author}] \leavevmode
Kyle Briggs \textless{}\href{mailto:kbrig035@uottawa.ca}{kbrig035@uottawa.ca}\textgreater{}

\item[{License}] \leavevmode
See LICENSE.TXT

\item[{ChangeLog}] \leavevmode
\end{description}\end{quote}

\begin{DUlineblock}{0em}
\item[] 8/24/15         AB      Rename algorithm to CUSUM+
\item[] 3/20/15         AB      Added a new metadata column (mdStateResTime) that saves the residence time of each state to the database.
\item[] 3/18/15         KB      Implemented rise time skipping
\item[] 3/17/15         KB      Implemented adaptive threshold
\item[] 2/12/15         AB      Updated metadata representation to be consistent with stepResponseAnalysis and multiStateAnalysis
\item[] 2/10/15         KB      Initial version
\end{DUlineblock}
\index{cusumPlus (class in mosaic.cusumPlus)}

\begin{fulllineitems}
\phantomsection\label{api-doc/mosaic.processing:mosaic.cusumPlus.cusumPlus}\pysiglinewithargsret{\strong{class }\code{mosaic.cusumPlus.}\bfcode{cusumPlus}}{\emph{icurr}, \emph{Fs}, \emph{**kwargs}}{}
Bases: {\hyperref[api\string-doc/mosaic.meta:mosaic.metaEventProcessor.metaEventProcessor]{\emph{\code{mosaic.metaEventProcessor.metaEventProcessor}}}}

Implements a modified version of the CUSUM algorithm (used by OpenNanopore for example) in MOSAIC. This approach sacrifices including system information in the analysis in favor of much faster fitting of single- and multi-level events.

CUSUM+ will detect jumps that are smaller than \emph{StepSize}, but they will have to be sustained longer. Threshold can be thought of, very roughly, as proportional to the length of time a subevent must be sustained for it to be detected. The algorithm will adjust the actual threshold used on a per-event basis in order to minimize false positive detection of current jumps This algorithm is based on code used in OpenNanopore, which you can read about here: \href{http://pubs.rsc.org/en/Content/ArticleLanding/2012/NR/c2nr30951c\#!divAbstract}{http://pubs.rsc.org/en/Content/ArticleLanding/2012/NR/c2nr30951c\#!divAbstract}

Some known issues with CUSUM+:
\begin{enumerate}
\item {} 
If the duration of a sub-event is shorter than than the MinLength parameter, CUSUM+ will be unable to detect it. CUSUM+ will not detect events within MinLength of a previous event.

\item {} 
CUSUM assumes an instantaneous transition between current states. As a result, if the RC rise time of the system is large, CUSUM+ can trigger and detect intermediate states during the change time. This can be avoided by choosing a number of samples to skip equal to about 2-5RC.

\item {} 
As a consequence of using a statistical t-test, CUSUM can have false positives. The algorithm has an adaptive threshold that tries to minimize the chances of this happening while maintaining good sensitivity (expected number of false positives within an event is less than 1).

\end{enumerate}

To use it requires four settings:
\begin{quote}

\begin{Verbatim}[commandchars=\\\{\}]
\PYG{l+s+s2}{\PYGZdq{}cusumPlus\PYGZdq{}}\PYG{o}{:} \PYG{p}{\PYGZob{}}
        \PYG{l+s+s2}{\PYGZdq{}StepSize\PYGZdq{}}\PYG{o}{:} \PYG{l+m+mf}{3.0}\PYG{p}{,} 
        \PYG{l+s+s2}{\PYGZdq{}MinThreshold\PYGZdq{}}\PYG{o}{:} \PYG{l+m+mf}{3.0}\PYG{p}{,}
        \PYG{l+s+s2}{\PYGZdq{}MaxThreshold\PYGZdq{}}\PYG{o}{:} \PYG{l+m+mf}{10.0}\PYG{p}{,}
        \PYG{l+s+s2}{\PYGZdq{}MinLength\PYGZdq{}} \PYG{o}{:} \PYG{l+m+mi}{10}\PYG{p}{,}
\PYG{p}{\PYGZcb{}}
\end{Verbatim}
\begin{quote}\begin{description}
\item[{Keyword Args}] \leavevmode\begin{description}
\item[{In addition to {\hyperref[api\string-doc/mosaic.meta:mosaic.metaEventProcessor.metaEventProcessor]{\emph{\code{metaEventProcessor}}}} args,}] \leavevmode\begin{itemize}
\item {} 
\emph{StepSize} :                  The number of baseline standard deviations are considered significant (3 is usually a good starting point).

\item {} 
\emph{MinThreshold} :              One of two sensitivity parameters (lower is more sensitive). A good starting point is to set \emph{MinThreshold} equal to \emph{StepSize}.

\item {} 
\emph{MaxThreshold} :              One of two sensitivity parameters (lower is more sensitive). Set \emph{MaxThreshold} about 3x higher than \emph{MinThreshold}.

\item {} 
\emph{MinLength} :                 The number of samples to skip after detecting a jump, in order to avoid triggering during the rise time and returning an artificially high number of states. This number of points is also skipped when averaging levels. About 4 times the RC constant of the system is a good starting value.

\end{itemize}

\end{description}

\item[{Errors}] \leavevmode
When an event cannot be analyzed, all metadata are set to -1.

\end{description}\end{quote}
\end{quote}
\index{formatsettings() (mosaic.cusumPlus.cusumPlus method)}

\begin{fulllineitems}
\phantomsection\label{api-doc/mosaic.processing:mosaic.cusumPlus.cusumPlus.formatsettings}\pysiglinewithargsret{\bfcode{formatsettings}}{}{}
Return a formatted string of settings for display

\end{fulllineitems}

\index{mdAveragePropertiesList() (mosaic.cusumPlus.cusumPlus method)}

\begin{fulllineitems}
\phantomsection\label{api-doc/mosaic.processing:mosaic.cusumPlus.cusumPlus.mdAveragePropertiesList}\pysiglinewithargsret{\bfcode{mdAveragePropertiesList}}{}{}
Return a list of meta-data properties that will be averaged 
and displayed at the end of a run.

\end{fulllineitems}


\end{fulllineitems}

\index{datblock (class in mosaic.cusumPlus)}

\begin{fulllineitems}
\phantomsection\label{api-doc/mosaic.processing:mosaic.cusumPlus.datblock}\pysiglinewithargsret{\strong{class }\code{mosaic.cusumPlus.}\bfcode{datblock}}{\emph{dat}}{}
Smart data block that holds a time-series of data and keeps track
of its mean and SD.

\end{fulllineitems}



\subsection{Data Output}
\label{api-doc/mosaic.output::doc}\label{api-doc/mosaic.output:data-output}

\subsubsection{mosaic.sqlite3MDIO module}
\label{api-doc/mosaic.output:mosaic-sqlite3mdio-module}\phantomsection\label{api-doc/mosaic.output:module-mosaic.sqlite3MDIO}\index{mosaic.sqlite3MDIO (module)}
A class that extends metaMDIO to implement SQLite support for metadata storage.
\begin{quote}\begin{description}
\item[{Created}] \leavevmode
9/28/2014

\item[{Author}] \leavevmode
Arvind Balijepalli \textless{}\href{mailto:arvind.balijepalli@nist.gov}{arvind.balijepalli@nist.gov}\textgreater{}

\item[{License}] \leavevmode
See LICENSE.TXT

\item[{ChangeLog}] \leavevmode
\end{description}\end{quote}

\begin{DUlineblock}{0em}
\item[] 12/6/15         AB      Add sampling frequency to analysis info table
\item[] 8/5/15          AB      Added a function to export database tables to CSV
\item[] 8/5/15          AB      Misc bug fixes
\item[] 4/1/15          AB      Added an estimate of data length to the DB
\item[] 3/23/15         AB      Added a raw query function that does not automatically decode column data.
\item[] 11/9/14         AB  Implemented the analysis log I/O interface for sqlite3 databases.
\item[] 9/28/14         AB      Initial version
\end{DUlineblock}
\index{data\_record (class in mosaic.sqlite3MDIO)}

\begin{fulllineitems}
\phantomsection\label{api-doc/mosaic.output:mosaic.sqlite3MDIO.data_record}\pysiglinewithargsret{\strong{class }\code{mosaic.sqlite3MDIO.}\bfcode{data\_record}}{\emph{data\_label}, \emph{data}, \emph{data\_t}}{}
Bases: \href{http://docs.python.org/library/stdtypes.html\#dict}{\code{dict}}

Smart data record structure that automatically encodes/decodes data for storage
in a sqlite3 DB.

\end{fulllineitems}

\index{sqlite3MDIO (class in mosaic.sqlite3MDIO)}

\begin{fulllineitems}
\phantomsection\label{api-doc/mosaic.output:mosaic.sqlite3MDIO.sqlite3MDIO}\pysigline{\strong{class }\code{mosaic.sqlite3MDIO.}\bfcode{sqlite3MDIO}}
Bases: {\hyperref[api\string-doc/mosaic.meta:mosaic.metaMDIO.metaMDIO]{\emph{\code{mosaic.metaMDIO.metaMDIO}}}}
\index{exportToCSV() (mosaic.sqlite3MDIO.sqlite3MDIO method)}

\begin{fulllineitems}
\phantomsection\label{api-doc/mosaic.output:mosaic.sqlite3MDIO.sqlite3MDIO.exportToCSV}\pysiglinewithargsret{\bfcode{exportToCSV}}{\emph{query}}{}
Export database records that match the specified query to a CSV flat file.

\end{fulllineitems}


\end{fulllineitems}



\subsection{Miscellaneous}
\label{api-doc/mosaic.misc::doc}\label{api-doc/mosaic.misc:miscellaneous}

\subsubsection{mosaic.settings module}
\label{api-doc/mosaic.misc:mosaic-settings-module}\label{api-doc/mosaic.misc:module-mosaic.settings}\index{mosaic.settings (module)}
Load analysis settings from a JSON file.
\begin{quote}\begin{description}
\item[{Created}] \leavevmode
8/24/2012

\item[{Author}] \leavevmode
Arvind Balijepalli \textless{}\href{mailto:arvind.balijepalli@nist.gov}{arvind.balijepalli@nist.gov}\textgreater{}

\item[{License}] \leavevmode
See LICENSE.TXT

\item[{ChangeLog}] \leavevmode
\end{description}\end{quote}

\begin{DUlineblock}{0em}
\item[] 3/16/16         AB      Replaced InitThreshold with StepSize in default settings for ADEPT and warn users when InitThreshold is used.
\item[] 8/24/15         AB      Updated algorithm names.
\item[] 6/24/15         AB      Added an option to unlink the RC constants in stepResponseAnalysis.
\item[] 3/20/15         AB      Added MaxEventLength to multiStateAnalysis settings
\item[] 3/6/15          JF      Corrected formatting on cusumLevelAnalysis and multiStateAnalysis dictionary file
\item[] 3/6/15          AB      Added MinStateLength parameter for multiStateAnalysis to dictionary
\item[] 2/14/15         AB      Added default settings for cusumLevelAnalysis.
\item[] 8/20/14         AB      Changed precedence of settings file search to datpath/.settings,
\item[]
\begin{DUlineblock}{\DUlineblockindent}
\item[] datpath/settings, coderoot/.settings and coderoot/settings
\end{DUlineblock}
\item[] 8/6/14          AB      Add a function to parse a settings string.
\item[] 9/5/13          AB      Check for either .settings or settings in data directory
\item[]
\begin{DUlineblock}{\DUlineblockindent}
\item[] and code root. Warn when using default settings
\end{DUlineblock}
\item[] 8/24/12         AB      Initial version
\end{DUlineblock}
\index{settings (class in mosaic.settings)}

\begin{fulllineitems}
\phantomsection\label{api-doc/mosaic.misc:mosaic.settings.settings}\pysiglinewithargsret{\strong{class }\code{mosaic.settings.}\bfcode{settings}}{\emph{datpath}, \emph{defaultwarn=True}}{}
Initialize a settings object.
\begin{quote}\begin{description}
\item[{Args}] \leavevmode\begin{itemize}
\item {} 
\emph{datpath} :   Specify the location of the settings file. If a settings file is not found, return default settings.

\item {} 
\emph{defaultwarn} :       If \emph{True} warn the user if a settings file was not found in the path specified by \emph{datpath}.

\end{itemize}

\end{description}\end{quote}
\index{getSettings() (mosaic.settings.settings method)}

\begin{fulllineitems}
\phantomsection\label{api-doc/mosaic.misc:mosaic.settings.settings.getSettings}\pysiglinewithargsret{\bfcode{getSettings}}{\emph{section}}{}
Return settings for a specified section as a Python dict.
\begin{quote}\begin{description}
\item[{Args}] \leavevmode\begin{itemize}
\item {} 
\emph{section} :   specifies the section for which settings are requested. Returns an empty dictionary if the settings file doesn't exist the section is not found.

\end{itemize}

\end{description}\end{quote}

\end{fulllineitems}


\end{fulllineitems}



\subsubsection{mosaic.utilities.ionic\_current\_stats module}
\label{api-doc/mosaic.misc:mosaic-utilities-ionic-current-stats-module}\phantomsection\label{api-doc/mosaic.misc:module-mosaic.utilities.ionic_current_stats}\index{mosaic.utilities.ionic\_current\_stats (module)}\begin{quote}\begin{description}
\item[{Created}] \leavevmode
10/30/2014

\item[{Author}] \leavevmode
Arvind Balijepalli \textless{}\href{mailto:arvind.balijepalli@nist.gov}{arvind.balijepalli@nist.gov}\textgreater{}

\item[{License}] \leavevmode
See LICENSE.TXT

\item[{ChangeLog}] \leavevmode
\end{description}\end{quote}

\begin{DUlineblock}{0em}
\item[] 10/30/14        AB      Initial version
\end{DUlineblock}
\index{OpenCurrentDist() (in module mosaic.utilities.ionic\_current\_stats)}

\begin{fulllineitems}
\phantomsection\label{api-doc/mosaic.misc:mosaic.utilities.ionic_current_stats.OpenCurrentDist}\pysiglinewithargsret{\code{mosaic.utilities.ionic\_current\_stats.}\bfcode{OpenCurrentDist}}{\emph{dat}, \emph{limit}}{}
Calculate the mean and standard deviation of a time-series.
\begin{quote}\begin{description}
\item[{Args}] \leavevmode\begin{itemize}
\item {} 
\emph{dat}         : time-series data

\item {} 
\emph{limit}       : limit the calculation to the top 50\% (+0.5) of the range, bottom 50\% (-0.5) or the entire range (0). Any other value of \emph{limit} will cause it to be reset to 0 (i.e. full range).

\end{itemize}

\end{description}\end{quote}

\end{fulllineitems}



\subsubsection{mosaic.utilities.util module}
\label{api-doc/mosaic.misc:mosaic-utilities-util-module}\phantomsection\label{api-doc/mosaic.misc:module-mosaic.utilities.util}\index{mosaic.utilities.util (module)}
A collection of utility functions
\index{avg() (in module mosaic.utilities.util)}

\begin{fulllineitems}
\phantomsection\label{api-doc/mosaic.misc:mosaic.utilities.util.avg}\pysiglinewithargsret{\code{mosaic.utilities.util.}\bfcode{avg}}{\emph{dat}}{}
Calculate the average of a list of reals

\end{fulllineitems}

\index{commonest() (in module mosaic.utilities.util)}

\begin{fulllineitems}
\phantomsection\label{api-doc/mosaic.misc:mosaic.utilities.util.commonest}\pysiglinewithargsret{\code{mosaic.utilities.util.}\bfcode{commonest}}{\emph{dat}}{}
Return the most common element in a list.

\end{fulllineitems}

\index{decimate() (in module mosaic.utilities.util)}

\begin{fulllineitems}
\phantomsection\label{api-doc/mosaic.misc:mosaic.utilities.util.decimate}\pysiglinewithargsret{\code{mosaic.utilities.util.}\bfcode{decimate}}{\emph{dat}, \emph{size}}{}
Decimate dat for a specified window size.

\end{fulllineitems}

\index{filter() (in module mosaic.utilities.util)}

\begin{fulllineitems}
\phantomsection\label{api-doc/mosaic.misc:mosaic.utilities.util.filter}\pysiglinewithargsret{\code{mosaic.utilities.util.}\bfcode{filter}}{\emph{dat}, \emph{windowSz}}{}
Filter the data using a convolution. Returns an
array of size len(dat)-windowSz+1 if dat is longer than 
windowSz. If len(dat) \textless{} windowSz, raise WindowSizeError

\end{fulllineitems}

\index{flat2() (in module mosaic.utilities.util)}

\begin{fulllineitems}
\phantomsection\label{api-doc/mosaic.misc:mosaic.utilities.util.flat2}\pysiglinewithargsret{\code{mosaic.utilities.util.}\bfcode{flat2}}{\emph{dat}}{}
Flatten a 2D array to a list

\end{fulllineitems}

\index{partition() (in module mosaic.utilities.util)}

\begin{fulllineitems}
\phantomsection\label{api-doc/mosaic.misc:mosaic.utilities.util.partition}\pysiglinewithargsret{\code{mosaic.utilities.util.}\bfcode{partition}}{\emph{dat}, \emph{size}}{}
Partition a list into sub-lists, each of length size. If the number of elements
in dat does not partition evenly, the last sub-list will have fewer elements.

\end{fulllineitems}

\index{sd() (in module mosaic.utilities.util)}

\begin{fulllineitems}
\phantomsection\label{api-doc/mosaic.misc:mosaic.utilities.util.sd}\pysiglinewithargsret{\code{mosaic.utilities.util.}\bfcode{sd}}{\emph{dat}}{}
Wrapper for numpy std

\end{fulllineitems}

\index{selectS() (in module mosaic.utilities.util)}

\begin{fulllineitems}
\phantomsection\label{api-doc/mosaic.misc:mosaic.utilities.util.selectS}\pysiglinewithargsret{\code{mosaic.utilities.util.}\bfcode{selectS}}{\emph{dat}, \emph{nSigma}, \emph{mu}, \emph{sd}}{}
Select and return data from a list that lie within
nSigma * SD of the mean.

\end{fulllineitems}



\subsubsection{mosaic.utilities.mosaicLog module}
\label{api-doc/mosaic.misc:mosaic-utilities-mosaiclog-module}\index{mosaicLog (class in mosaic.utilities.mosaicLog)}

\begin{fulllineitems}
\phantomsection\label{api-doc/mosaic.misc:mosaic.utilities.mosaicLog.mosaicLog}\pysigline{\strong{class }\code{mosaic.utilities.mosaicLog.}\bfcode{mosaicLog}}
Bases: \href{http://docs.python.org/library/stdtypes.html\#dict}{\code{dict}}
\index{addLogHeader() (mosaic.utilities.mosaicLog.mosaicLog method)}

\begin{fulllineitems}
\phantomsection\label{api-doc/mosaic.misc:mosaic.utilities.mosaicLog.mosaicLog.addLogHeader}\pysiglinewithargsret{\bfcode{addLogHeader}}{\emph{header}}{}
Add a section header to the output log.
\begin{quote}\begin{description}
\item[{Args}] \leavevmode\begin{itemize}
\item {} 
\emph{header} :    header text

\end{itemize}

\end{description}\end{quote}

\end{fulllineitems}

\index{addLogText() (mosaic.utilities.mosaicLog.mosaicLog method)}

\begin{fulllineitems}
\phantomsection\label{api-doc/mosaic.misc:mosaic.utilities.mosaicLog.mosaicLog.addLogText}\pysiglinewithargsret{\bfcode{addLogText}}{\emph{log}}{}
Add text to the output log.
\begin{quote}\begin{description}
\item[{Args}] \leavevmode\begin{itemize}
\item {} 
\emph{log} :       log text

\end{itemize}

\end{description}\end{quote}

\end{fulllineitems}


\end{fulllineitems}



\subsubsection{mosaic.utilities.mosaicTiming module}
\label{api-doc/mosaic.misc:mosaic-utilities-mosaictiming-module}\phantomsection\label{api-doc/mosaic.misc:module-mosaic.utilities.mosaicTiming}\index{mosaic.utilities.mosaicTiming (module)}
A class that provides platform independent timing and function profiling utilities.
\begin{quote}\begin{description}
\item[{Created}] \leavevmode
4/10/2016

\item[{Author}] \leavevmode
Arvind Balijepalli \textless{}\href{mailto:arvind.balijepalli@nist.gov}{arvind.balijepalli@nist.gov}\textgreater{}

\item[{License}] \leavevmode
See LICENSE.TXT

\item[{ChangeLog}] \leavevmode
\end{description}\end{quote}

\begin{DUlineblock}{0em}
\item[] 4/10/16         AB      Initial version
\end{DUlineblock}
\index{mosaicTiming (class in mosaic.utilities.mosaicTiming)}

\begin{fulllineitems}
\phantomsection\label{api-doc/mosaic.misc:mosaic.utilities.mosaicTiming.mosaicTiming}\pysiglinewithargsret{\strong{class }\code{mosaic.utilities.mosaicTiming.}\bfcode{mosaicTiming}}{\emph{logger=None}}{}
Profile code by attaching an instance of this class to any function. All the methods in this class are valid for the function being profiled.
\index{FunctionTiming() (mosaic.utilities.mosaicTiming.mosaicTiming method)}

\begin{fulllineitems}
\phantomsection\label{api-doc/mosaic.misc:mosaic.utilities.mosaicTiming.mosaicTiming.FunctionTiming}\pysiglinewithargsret{\bfcode{FunctionTiming}}{\emph{func}}{}
Pass the function to be profiled as an argument. Alternatively with python 2.4+, attach a decorator to the function being profiled
For example:
\begin{quote}

funcTimer=mosaicTiming.mosaicTiming()

@funcTimer.FunctionTiming
def someFunc():
\begin{quote}

print `doing something'
\end{quote}

\# summarize the profiling results for someFunc
funcTimer.PrintStatistics()
\end{quote}
\begin{quote}\begin{description}
\item[{Parameters}] \leavevmode\begin{itemize}
\item {} 
\emph{func} :    function to be profiled

\end{itemize}

\end{description}\end{quote}

\end{fulllineitems}

\index{PrintCurrentTime() (mosaic.utilities.mosaicTiming.mosaicTiming method)}

\begin{fulllineitems}
\phantomsection\label{api-doc/mosaic.misc:mosaic.utilities.mosaicTiming.mosaicTiming.PrintCurrentTime}\pysiglinewithargsret{\bfcode{PrintCurrentTime}}{}{}
Print timing results of the most recent function call

\end{fulllineitems}

\index{PrintStatistics() (mosaic.utilities.mosaicTiming.mosaicTiming method)}

\begin{fulllineitems}
\phantomsection\label{api-doc/mosaic.misc:mosaic.utilities.mosaicTiming.mosaicTiming.PrintStatistics}\pysiglinewithargsret{\bfcode{PrintStatistics}}{}{}
Print average timing results of the function call

\end{fulllineitems}

\index{Reset() (mosaic.utilities.mosaicTiming.mosaicTiming method)}

\begin{fulllineitems}
\phantomsection\label{api-doc/mosaic.misc:mosaic.utilities.mosaicTiming.mosaicTiming.Reset}\pysiglinewithargsret{\bfcode{Reset}}{}{}
Reset all profiling data collected for a function

\end{fulllineitems}

\index{time() (mosaic.utilities.mosaicTiming.mosaicTiming method)}

\begin{fulllineitems}
\phantomsection\label{api-doc/mosaic.misc:mosaic.utilities.mosaicTiming.mosaicTiming.time}\pysiglinewithargsret{\bfcode{time}}{}{}
Replace time.time() with a platform independent timing function

\end{fulllineitems}


\end{fulllineitems}



\subsubsection{mosaic.utilities.fit\_funcs module}
\label{api-doc/mosaic.misc:mosaic-utilities-fit-funcs-module}\phantomsection\label{api-doc/mosaic.misc:module-mosaic.utilities.fit_funcs}\index{mosaic.utilities.fit\_funcs (module)}
Fit functions used in processing algorithms.
\begin{quote}\begin{description}
\item[{Created}] \leavevmode
10/30/2014

\item[{Author}] \leavevmode
Arvind Balijepalli \textless{}\href{mailto:arvind.balijepalli@nist.gov}{arvind.balijepalli@nist.gov}\textgreater{}

\item[{License}] \leavevmode
See LICENSE.TXT

\item[{ChangeLog}] \leavevmode
\end{description}\end{quote}

\begin{DUlineblock}{0em}
\item[] 12/09/15        KB      Added a wrapper for multiStateFunc
\item[] 6/24/15         AB      Relaxed stepResponseFunc to include different RC constants
\item[]
\begin{DUlineblock}{\DUlineblockindent}
\item[] for up and down states.
\end{DUlineblock}
\item[] 12/31/14        AB      Changed multi-state function to include a separate tau for
\item[]
\begin{DUlineblock}{\DUlineblockindent}
\item[] each state following Balijepalli et al, ACS Nano 2014.
\end{DUlineblock}
\item[] 11/19/14        AB      Initial version
\end{DUlineblock}


\subsection{MOSAIC Script Repository}
\label{api-doc/mosaicscripts:mosaic-script-repository}\label{api-doc/mosaicscripts::doc}

\subsubsection{mosaicscripts.plots.timeseries module}
\label{api-doc/mosaicscripts:module-mosaicscripts.plots.timeseries}\label{api-doc/mosaicscripts:mosaicscripts-plots-timeseries-module}\index{mosaicscripts.plots.timeseries (module)}
Plot an ionic current time-series.
\begin{quote}\begin{description}
\item[{Created}] \leavevmode
11/19/2015

\item[{Author}] \leavevmode
Arvind Balijepalli \textless{}\href{mailto:arvind.balijepalli@nist.gov}{arvind.balijepalli@nist.gov}\textgreater{}

\item[{ChangeLog}] \leavevmode
\end{description}\end{quote}

\begin{DUlineblock}{0em}
\item[] 12/12/15        AB      Generalized plot function to allow different data types
\item[] 11/19/15        AB      Initial version
\end{DUlineblock}
\index{PlotTimeseries() (in module mosaicscripts.plots.timeseries)}

\begin{fulllineitems}
\phantomsection\label{api-doc/mosaicscripts:mosaicscripts.plots.timeseries.PlotTimeseries}\pysiglinewithargsret{\code{mosaicscripts.plots.timeseries.}\bfcode{PlotTimeseries}}{\emph{dir}, \emph{data\_type}, \emph{t0}, \emph{t1}, \emph{Fs}, \emph{**kwargs}}{}
Generate publication quality time-series plots.
\begin{quote}\begin{description}
\item[{Args}] \leavevmode\begin{itemize}
\item {} 
\emph{dir} :                       directory containing data files

\item {} 
\emph{data\_type} :         One of ``abf'', ``qdf'', ``bin'' or ``tsv''.

\item {} 
\emph{t0} :                        start time.

\item {} 
\emph{t1} :                        end time.

\item {} 
\emph{Fs} :                        Sampling rate in Hz.

\item {} 
\emph{labels} :            Axes text labels. For example \code{{}`{[}"t (s)", "-i (pA)"{]}{}`} for a current vs. time plot.

\end{itemize}

\item[{Keyword Args}] \leavevmode\begin{itemize}
\item {} 
\emph{data\_args} :         (optional) For ``qdf'', ``bin'' or ``tsv'', settings to read in data. See {\hyperref[doc/settingsFile:settings\string-page]{\emph{Settings File}}} for details.

\item {} 
\emph{axes} :                      (optional) Show axes (Default: True)

\item {} \begin{description}
\item[{\emph{highlights}}] \leavevmode{[}(optional) Highlight segments of the time-series with a different style (Default: None). For example: {]}\begin{description}
\item[{highlights={[}}] \leavevmode
{[}{[}0.282, 0.293{]}, \{`color' : `\#3F50A0', `marker' : `.', `markersize' : 0.1\}{]},
{[}{[}0.584, 0.597{]}, \{`color' : `\#D42324', `marker' : `.', `markersize' : 0.1\}{]},
{[}{[}0.685, 0.695{]}, \{`color' : `\#EB751A', `marker' : `.', `markersize' : 0.1\}{]}

\end{description}

{]}

\end{description}

\end{itemize}

Highlight three events at specied location (arg 1: start, end) with specified styles.
- \emph{plotopts} :          (optional) Specify plot style. See \href{http://matplotlib.org/api/pyplot\_api.html\#matplotlib.pyplot.plot}{http://matplotlib.org/api/pyplot\_api.html\#matplotlib.pyplot.plot} for details.
- \emph{figname} :                           (optional) figure name if saving an image. File extension determines format.
- \emph{dpi} :                                       (optional) figure resolution

\end{description}\end{quote}

\end{fulllineitems}



\subsubsection{mosaicscripts.plots.histogram module}
\label{api-doc/mosaicscripts:module-mosaicscripts.plots.histogram}\label{api-doc/mosaicscripts:mosaicscripts-plots-histogram-module}\index{mosaicscripts.plots.histogram (module)}
1-D Histogram plot.
\begin{quote}\begin{description}
\item[{Created}] \leavevmode
12/13/2015

\item[{Author}] \leavevmode
Arvind Balijepalli \textless{}\href{mailto:arvind.balijepalli@nist.gov}{arvind.balijepalli@nist.gov}\textgreater{}

\item[{License}] \leavevmode
See LICENSE.TXT

\item[{ChangeLog}] \leavevmode
\end{description}\end{quote}

\begin{DUlineblock}{0em}
\item[] 12/13/15                AB      Initial version
\end{DUlineblock}
\index{histogram\_plot() (in module mosaicscripts.plots.histogram)}

\begin{fulllineitems}
\phantomsection\label{api-doc/mosaicscripts:mosaicscripts.plots.histogram.histogram_plot}\pysiglinewithargsret{\code{mosaicscripts.plots.histogram.}\bfcode{histogram\_plot}}{\emph{dat}, \emph{nbins}, \emph{x\_range}, \emph{**kwargs}}{}~\begin{quote}

Generate publication quality contour plots using the \code{{}`contour\_plot{}`} function. The function expects a two-dimensional array of data (typically blockade depth and residence time) and several options as listed below:
\begin{quote}\begin{description}
\item[{Args}] \leavevmode\begin{itemize}
\item {} 
\emph{dat} :                           2-D array with format {[}{[}x1,y1{]}, {[}x2,y2{]}, ... ... ... {[}xn,yn{]}{]}

\item {} 
\emph{nbinx} :                             number of bins.

\item {} 
\emph{x\_range} :                   list with min and max in X. If \emph{None}, min and max values of the data set the range.

\end{itemize}

\item[{Keyword Args}] \leavevmode\begin{itemize}
\item {} 
\emph{density} :                   (optional) If True, display the probability density function. Default is \emph{False}

\item {} 
\emph{color} :                             (optional) Plot color. Default is \emph{\#4155A3}.

\item {} 
\emph{fill\_alpha} :                (optional) Fill transperancy. 0 turns off fill. Default is 0.25.

\item {} 
\emph{xticks} :                    (optional) specify ticks for the X-axis. List of format {[} (tick, label), ...{]}

\item {} 
\emph{yticks} :                    (optional) specify ticks for the X-axis. List of format {[} (tick, label), ...{]}

\item {} 
\emph{figname} :                   (optional) figure name if saving an image. File extension determines format.

\item {} 
\emph{dpi} :                               (optional) figure resolution.

\item {} 
\emph{show} :                              (optional) if True (default) call the show() function to display the plot.

\item {} 
\emph{return\_histogram} :  (optional) if True, return the histogram values and bins. Default is False.

\item {} 
\emph{advanced\_opts} :             (optional) a Python dictionary that supplies advanced plotting options. See {\color{red}\bfseries{}{}`}Matplotlib plot documentation

\end{itemize}

\end{description}\end{quote}
\end{quote}

\textless{}\href{http://matplotlib.org/api/pyplot\_api.html\#matplotlib.pyplot.plot}{http://matplotlib.org/api/pyplot\_api.html\#matplotlib.pyplot.plot}\textgreater{}{}`\_ for details.

\end{fulllineitems}



\subsubsection{mosaicscripts.plots.contour module}
\label{api-doc/mosaicscripts:module-mosaicscripts.plots.contour}\label{api-doc/mosaicscripts:mosaicscripts-plots-contour-module}\index{mosaicscripts.plots.contour (module)}
Contour plot overlaid on top of an image.
\begin{quote}\begin{description}
\item[{Created}] \leavevmode
11/11/2015

\item[{Author}] \leavevmode
Arvind Balijepalli \textless{}\href{mailto:arvind.balijepalli@nist.gov}{arvind.balijepalli@nist.gov}\textgreater{}

\item[{License}] \leavevmode
See LICENSE.TXT

\item[{ChangeLog}] \leavevmode
\end{description}\end{quote}

\begin{DUlineblock}{0em}
\item[] 02/05/16                AB      Add options to scale z-axis
\item[] 01/10/15                AB  Rename custom colormaps
\item[] 11/11/15                AB      Initial version
\end{DUlineblock}
\index{contour\_plot() (in module mosaicscripts.plots.contour)}

\begin{fulllineitems}
\phantomsection\label{api-doc/mosaicscripts:mosaicscripts.plots.contour.contour_plot}\pysiglinewithargsret{\code{mosaicscripts.plots.contour.}\bfcode{contour\_plot}}{\emph{dat2d}, \emph{x\_range}, \emph{y\_range}, \emph{bin\_size}, \emph{contours}, \emph{colormap}, \emph{img\_interpolation}, \emph{**kwargs}}{}
Generate publication quality contour plots using the \code{{}`contour\_plot{}`} function. The function expects a two-dimensional array of data (typically blockade depth and residence time) and several options as listed below:
\begin{quote}\begin{description}
\item[{Args}] \leavevmode\begin{itemize}
\item {} 
\emph{dat2d} :                         2-D array with format {[}{[}x1,y1{]}, {[}x2,y2{]}, ... ... ... {[}xn,yn{]}{]}

\item {} 
\emph{x\_range} :                           list with min and max in X

\item {} 
\emph{y\_range} :                           list with min and max in Y

\item {} 
\emph{bin\_size} :                          bin size

\item {} 
\emph{contours} :                          number of contours

\item {} 
\emph{colormap} :                          Colormap to use. Expects a colormap object. See \href{http://matplotlib.org/examples/color/colormaps\_reference.html}{http://matplotlib.org/examples/color/colormaps\_reference.html}.

\item {} 
\emph{img\_interpolation} :         interpolation to use for image

\end{itemize}

\item[{Keyword Args}] \leavevmode\begin{itemize}
\item {} 
\emph{zscale} :                            (optional) plot the probability density if set to \emph{density} or scale to the max count if set to \emph{unity}.

\item {} 
\emph{xticks} :                            (optional) specify ticks for the X-axis. List of format {[} (tick, label), ...{]}

\item {} 
\emph{yticks} :                            (optional) specify ticks for the X-axis. List of format {[} (tick, label), ...{]}

\item {} 
\emph{figname} :                           (optional) figure name if saving an image. File extension determines format.

\item {} 
\emph{dpi} :                                       (optional) figure resolution

\item {} 
\emph{colorbar\_num\_ticks} :        (optional) number of ticks in the colorbar

\item {} 
\emph{cb\_round\_digits} :           (optional) round colorbar ticks to multiple of cb\_round\_digits. For example, -2 rounds to 100. See python docs.

\item {} 
\emph{min\_count\_pct} :                     (optional) set bins with \textless{} min\_count\_pct of the maximum to 0

\item {} 
\emph{axes\_type} :                         (optional) set linear or log axis. Expects a list for X and Y. For example {[}'linear', `log'{]}.

\end{itemize}

\end{description}\end{quote}

\end{fulllineitems}



\chapter{Change Log}
\label{doc/ChangeLog:changelog-page}\label{doc/ChangeLog::doc}\label{doc/ChangeLog:change-log}
\textbf{v1.3}
\begin{itemize}
\item {} 
Added CUSUM algorithm (see pull requests \#34, \#43, \#45, and \#46)

\item {} 
Streamlined unit test framework. Added new tests for CUSUM.

\item {} 
{[}GUI{]} Performance optimization.

\item {} 
{[}GUI{]} Added CUSUM support to MOSAIC GUI.

\item {} 
{[}GUI{]} Fit window in MOSAIC GUI displays idealized pulses overlays.

\item {} 
{[}GUI{]} Added additional analysis statistics.

\item {} 
{[}Addons{]} Added CUSUM support to Mathematica addon (PlotEvents in MosaicUtils.m)

\item {} 
{[}Addons{]} Mathematica queries are handled through an external Python script.

\item {} 
{[}Addons{]} Added an option to limit PlotEvents in Mathematica addon to N events.

\item {} 
Updated MOSAIC dependencies to include newer packages. Run `python setup.py mosaic\_deps' to update.

\item {} 
Added a new metadata column (mdStateResTime) that saves the residence time of each state to the database. This affects multiStateAnalysis and cusumLevelAnalysis.

\item {} 
Removed mosaicgui from PyPi. `pip install mosaic-nist' only installs command line modules.

\item {} 
Top level ConvertToCSV supports arbitrary file extensions.

\item {} 
Fixes issues \#36, \#37, \#38, \#39 and \#47.

\item {} 
Known Issues: See \#8 and \#10.

\end{itemize}

\textbf{v1.2}
\begin{itemize}
\item {} 
Added support for arbitrary binary file formats (\#33)

\item {} 
{[}GUI{]} Included binary file support.

\item {} 
Documentation updates and bug fixes.

\item {} 
Known Issues: See \#8 and \#10.

\end{itemize}

\textbf{v1.1}
\begin{itemize}
\item {} 
{[}Addons{]} \href{http://www.wavemetrics.com/products/igorpro/igorpro.htm}{IGOR} support.

\item {} 
PyPi package automatically installs MOSAIC dependencies.

\item {} 
Miscellaneous bug fixes.

\item {} 
\emph{Known Issues:} See \#8 and \#10.

\end{itemize}

\textbf{v1.0}
\begin{itemize}
\item {} 
Fixed a bug in multistate code that constrained the RC constant resulting in systematic fitting errors (pull request \#25).

\item {} 
Updated multistate to include a separate RC constant for each state, to be consistent with functional form in Balijepalli et al., ACS Nano 2014.

\item {} 
Misc bug fixes in tsvTrajIO parsing.

\item {} 
The number of states is saved to the MDIO DB for multistate analysis (issue \#26).

\item {} 
Created a new package on PyPI (mosaic-nist) to allow installation with setuptools.

\item {} 
{[}GUI{]} Updated help link to point to Sphinx documentation on Github.

\item {} 
\emph{Known Issues:} See \#8 and \#10.

\end{itemize}

\textbf{v1.0b3.2}
\begin{itemize}
\item {} 
{[}GUI{]} Misc bug fixes

\item {} 
{[}Addons{]} Added code to import MOSAIC output into Matlab (pull requests \#18 and \#20)

\item {} 
{[}Addons{]} Updated \href{http://www.wolfram.com/mathematica/}{Mathematica} addons to automatically decode multi-state data.

\item {} 
Resolves issues \#16 and \#22

\end{itemize}

\textbf{v1.0b3.1}
\begin{itemize}
\item {} 
{[}GUI{]} Added multiState support to mosaicgui.

\item {} 
Analysis information such as alogirthms used, data type, etc. are now stored within a MDIO database.

\item {} 
{[}GUI{]} Autocomplete in mosaicgui only suggests database columns that are valid when used in a query.

\item {} 
Reorganized \href{http://www.wolfram.com/mathematica/}{Mathematica} addon code.

\end{itemize}

\textbf{v1.0b3}
\begin{itemize}
\item {} 
Fixed a bug that prevented events longer than \textasciitilde{}700 data points from being correctly analyzed.

\item {} 
Fixed a problem that prevented event data from being correctly padded before analysis.

\item {} 
Resolves \#2. TrajIO settings are now read in from the settings file.

\item {} 
{[}GUI{]} Resolves \#3. Threshold entry box in GUI becomes nonresponsive when meanOpenCurr is negative.

\item {} 
{[}GUI{]} Resolves \#4. Analysis fails when using wavletDenioseFilter from GUI.

\item {} 
{[}GUI{]} Histogram in BlockDepthViewer window can be saved to a CSV file from the File Menu.

\item {} 
Analysis log is saved to the MDIO database.

\item {} 
{[}GUI{]} ConsoleLogViwer displays the analysis log saved in the MDIO database.

\item {} 
{[}GUI{]} Added a new dialog that displays an experimental feature warning wavelet-based denoising is selected.

\item {} 
Updated error codes reported in database to be more descriptive of the failure.

\item {} 
Improved and expanded unit testing framework.

\item {} 
Moved installation and testing to setuptools.

\end{itemize}

\textbf{v1.0b2}
\begin{itemize}
\item {} 
{[}GUI{]} Fixed threshold update error from 1.0b1.

\item {} 
Considerably improved automatic open channel state detection.

\item {} 
The default settings string is now included within the source code.

\item {} 
Implemented new top-level class ConvertToCSV that allows conversion of data read by any TrajIO object to comma separated files.

\item {} 
Updated build system and unit testing framework.

\item {} 
{[}GUI{]} Misc UI updates.

\end{itemize}

\textbf{v1.0b1}
\begin{itemize}
\item {} 
{[}GUI{]} Added a menu option to save a settings file prior to starting the analysis.

\item {} 
{[}GUI{]} Current threshold is now defined by an ionic current. The trajectory viewer displays the deviation of the threshold from the mean current.

\item {} 
Analysis settings are saved within the analysissettings table of the sqlite database. When an analysis database is loaded into the GUI, settings are parsed from within the database.

\item {} 
When an analysis file is loaded, widgets in the main window remain enabled. This allows starting a new analysis run with the current settings.

\item {} 
{[}GUI{]} Implemented an analysis log viewer that displays the event processing log.

\item {} 
{[}GUI{]} Initial commit of wavelets based peak detection in blockdepthview.

\item {} 
{[}GUI{]} Added all points histogram to trajectory viewer.

\item {} 
\emph{Known Issues:} Selecting automatic baseline detection can sometimes cause the threshold in the trajectory viewer to change. Moving the slider will cause the settings and trajectory windows to synchronize.

\end{itemize}



\begin{thebibliography}{BEC+14}
\bibitem[BEC+14]{BEC+14}{\phantomsection\label{doc/References:balijepalli-2014ft} 
Arvind Balijepalli, Jessica Ettedgui, Andrew T Cornio, Joseph W F Robertson, Kin P Cheung, John J Kasianowicz, and Canute Vaz. Quantifying short-lived events in multistate ionic current measurements.. \emph{ACS Nano}, 8(2):1547–1553, February 2014.
}
\bibitem[RGG+12]{RGG+12}{\phantomsection\label{doc/References:raillon-2012is} 
C Raillon, P Granjon, M Graf, L J Steinbock, and A Radenovic. Fast and automatic processing of multi-level events in nanopore translocation experiments.. \emph{Nanoscale}, 4(16):4916–4924, August 2012.
}
\bibitem[RBR+12]{RBR+12}{\phantomsection\label{doc/References:reiner-2012bg} 
Joseph E Reiner, Arvind Balijepalli, Joseph W F Robertson, Jason Campbell, John Suehle, and John J Kasianowicz. Disease Detection and Management via Single Nanopore-Based Sensors.. \emph{Chem. Rev.}, 112:6432–6451, November 2012.
}
\end{thebibliography}


\renewcommand{\indexname}{Python Module Index}
\begin{theindex}
\def\bigletter#1{{\Large\sffamily#1}\nopagebreak\vspace{1mm}}
\bigletter{m}
\item {\texttt{mosaic.abfTrajIO}}, \pageref{api-doc/mosaic.traj:module-mosaic.abfTrajIO}
\item {\texttt{mosaic.adept}}, \pageref{api-doc/mosaic.processing:module-mosaic.adept}
\item {\texttt{mosaic.adept2State}}, \pageref{api-doc/mosaic.processing:module-mosaic.adept2State}
\item {\texttt{mosaic.besselLowpassFilter}}, \pageref{api-doc/mosaic.filter:module-mosaic.besselLowpassFilter}
\item {\texttt{mosaic.binTrajIO}}, \pageref{api-doc/mosaic.traj:module-mosaic.binTrajIO}
\item {\texttt{mosaic.ConvertToCSV}}, \pageref{api-doc/mosaic:module-mosaic.ConvertToCSV}
\item {\texttt{mosaic.convolutionFilter}}, \pageref{api-doc/mosaic.filter:module-mosaic.convolutionFilter}
\item {\texttt{mosaic.cusumPlus}}, \pageref{api-doc/mosaic.processing:module-mosaic.cusumPlus}
\item {\texttt{mosaic.eventSegment}}, \pageref{api-doc/mosaic.partition:module-mosaic.eventSegment}
\item {\texttt{mosaic.qdfTrajIO}}, \pageref{api-doc/mosaic.traj:module-mosaic.qdfTrajIO}
\item {\texttt{mosaic.settings}}, \pageref{api-doc/mosaic.misc:module-mosaic.settings}
\item {\texttt{mosaic.SingleChannelAnalysis}}, \pageref{api-doc/mosaic:module-mosaic.SingleChannelAnalysis}
\item {\texttt{mosaic.sqlite3MDIO}}, \pageref{api-doc/mosaic.output:module-mosaic.sqlite3MDIO}
\item {\texttt{mosaic.tsvTrajIO}}, \pageref{api-doc/mosaic.traj:module-mosaic.tsvTrajIO}
\item {\texttt{mosaic.utilities.fit\_funcs}}, \pageref{api-doc/mosaic.misc:module-mosaic.utilities.fit_funcs}
\item {\texttt{mosaic.utilities.ionic\_current\_stats}}, \pageref{api-doc/mosaic.misc:module-mosaic.utilities.ionic_current_stats}
\item {\texttt{mosaic.utilities.mosaicTiming}}, \pageref{api-doc/mosaic.misc:module-mosaic.utilities.mosaicTiming}
\item {\texttt{mosaic.utilities.util}}, \pageref{api-doc/mosaic.misc:module-mosaic.utilities.util}
\item {\texttt{mosaic.waveletDenoiseFilter}}, \pageref{api-doc/mosaic.filter:module-mosaic.waveletDenoiseFilter}
\item {\texttt{mosaicscripts.plots.contour}}, \pageref{api-doc/mosaicscripts:module-mosaicscripts.plots.contour}
\item {\texttt{mosaicscripts.plots.histogram}}, \pageref{api-doc/mosaicscripts:module-mosaicscripts.plots.histogram}
\item {\texttt{mosaicscripts.plots.timeseries}}, \pageref{api-doc/mosaicscripts:module-mosaicscripts.plots.timeseries}
\end{theindex}

\renewcommand{\indexname}{Index}
\printindex
\end{document}
